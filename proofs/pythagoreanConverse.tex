%\documentclass[handout]{ximera}
\documentclass[nooutcomes,noauthor]{ximera}

% For preamble materials

\usepackage{pgf,tikz}
\usepackage{mathrsfs}
\usetikzlibrary{arrows}
\usepackage{framed}
\usepackage{amsmath}
\pgfplotsset{compat=1.17}

\def\fixnote#1{\begin{framed}{\textcolor{red}{Fix note: #1}}\end{framed}}  % Allows insertion of red notes about needed edits
%\def\fixnote#1{}

\def\detail#1{{\textcolor{blue}{Detail: #1}}}   

\pdfOnly{\renewenvironment{image}[1][]{\begin{center}}{\end{center}}}

\graphicspath{
  {./}
  {chapter1/}
  {chapter2/}
  {chapter4/}
  {proofs/}
  {graphics/}
  {../graphics/}
}

\newenvironment{sectionOutcomes}{}{}


%%% This set of code is all of our user defined commands
\newcommand{\bysame}{\mbox{\rule{3em}{.4pt}}\,}
\newcommand{\N}{\mathbb N}
\newcommand{\C}{\mathbb C}
\newcommand{\W}{\mathbb W}
\newcommand{\Z}{\mathbb Z}
\newcommand{\Q}{\mathbb Q}
\newcommand{\R}{\mathbb R}
\newcommand{\A}{\mathbb A}
\newcommand{\D}{\mathcal D}
\newcommand{\F}{\mathcal F}
\newcommand{\ph}{\varphi}
\newcommand{\ep}{\varepsilon}
\newcommand{\aph}{\alpha}
\newcommand{\QM}{\begin{center}{\huge\textbf{?}}\end{center}}

\renewcommand{\le}{\leqslant}
\renewcommand{\ge}{\geqslant}
\renewcommand{\a}{\wedge}
\renewcommand{\v}{\vee}
\renewcommand{\l}{\ell}
\newcommand{\mat}{\mathsf}
\renewcommand{\vec}{\mathbf}
\renewcommand{\subset}{\subseteq}
\renewcommand{\supset}{\supseteq}
%\renewcommand{\emptyset}{\varnothing}
%\newcommand{\xto}{\xrightarrow}
%\renewcommand{\qedsymbol}{$\blacksquare$}
%\newcommand{\bibname}{References and Further Reading}
%\renewcommand{\bar}{\protect\overline}
%\renewcommand{\hat}{\protect\widehat}
%\renewcommand{\tilde}{\widetilde}
%\newcommand{\tri}{\triangle}
%\newcommand{\minipad}{\vspace{1ex}}
%\newcommand{\leftexp}[2]{{\vphantom{#2}}^{#1}{#2}}

%% More user defined commands
\renewcommand{\epsilon}{\varepsilon}
\renewcommand{\theta}{\vartheta} %% only for kmath
\renewcommand{\l}{\ell}
\renewcommand{\d}{\, d}
\newcommand{\ddx}{\frac{d}{dx}}
\newcommand{\dydx}{\frac{dy}{dx}}


\usepackage{bigstrut}


\title{Converse of the Pythagorean Theorem}
\author{Bart Snapp and Brad Findell}

\outcome{Learning outcome goes here.}

\begin{document}
\begin{abstract}
  We reason about the converse of the Pythagorean Theorem.
\end{abstract}
\maketitle


\begin{problem}

Suppose a triangle has side lengths $a$, $b$, and $c$.  
\begin{enumerate}
\item Pythagorean Theorem: If the triangle is a $\answer[format=string]{right}$ triangle with hypotenuse $c$, then $a^2+b^2=\answer{c^2}$.  

\item Converse of the Pythagorean Theorem:  If $a^2+b^2=c^2$, then the triangle is a $\answer[format=string]{right}$ triangle with hypotenuse $\answer{c}$.  

\item The converse of a true theorem is \wordChoice{\choice{always true}\choice[correct]{sometimes true}\choice{never true}}.  

\item Is the converse of the Pythagorean Theorem true?  $\answer[format=string]{Yes}$
\end{enumerate}
\end{problem}

\begin{problem}
\begin{proof}[Proof.] 
We have $\triangle ABC$ with side lengths $a$, $b$, and $c$, as shown.  

\begin{center}
\begin{image}
\definecolor{qqwuqq}{rgb}{0,0.39215686274509803,0}
\definecolor{xdxdff}{rgb}{0.49019607843137253,0.49019607843137253,1}
\definecolor{ududff}{rgb}{0.30196078431372547,0.30196078431372547,1}
\definecolor{uuuuuu}{rgb}{0.26666666666666666,0.26666666666666666,0.26666666666666666}
\begin{tikzpicture}[line cap=round,line join=round,>=triangle 45,x=1cm,y=1cm]
%\draw[line width=2pt,color=qqwuqq,fill=qqwuqq,fill opacity=0.10000000149011612] (3,0.28284271247461906) -- (2.717157287525381,0.28284271247461906) -- (2.717157287525381,0) -- (3,0) -- cycle; 
\draw [line width=1.2pt] (0,0)-- (3,2);
\draw [line width=1.2pt] (3,2)-- (3,0);
\draw [line width=1.2pt] (3,0)-- (0,0);
\begin{scriptsize}
\draw [fill=uuuuuu] (0,0) circle (1pt);
\draw[color=uuuuuu] (-0.29,0.12) node {$A$};
\draw [fill=ududff] (3,2) circle (1.5pt);
\draw[color=ududff] (3.15,2.3) node {$B$};
\draw[color=black] (1.39,1.29) node {$c$};
\draw [fill=xdxdff] (3,0) circle (1.5pt);
\draw[color=xdxdff] (3.2,-0.03) node {$C$};
\draw[color=black] (3.27,1.05) node {$a$};
\draw[color=black] (1.63,-0.20) node {$b$};
\draw [fill=white] (6,0) circle (0.5pt);
\draw [fill=white] (-3,0) circle (0.5pt);
\end{scriptsize}
\end{tikzpicture}
\end{image}
\end{center}

Suppose $a^2+b^2=c^2$.  We want to prove that the triangle must be $\answer[format=string]{right}$, and that the hypotenuse is $c$.  Until we show that $\angle C$ is a right angle, we must consider the possibility that it is not.  

Choose the \textbf{best} proof strategy from among the following: 
\begin{multipleChoice}
\choice{It is always true that $a^2+b^2 = c^2$.}
\choice{Make a different side the hypotenuse.}
\choice{Use the quadratic formula.}
\choice[correct]{Construct another triangle, known to be right, and show that it is congruent to the given triangle.}
\choice{Use different letters.}
\choice{Try it for an acute or obtuse triangle.} 
\end{multipleChoice}

\begin{problem}
Correct!  Construct a \textbf{right} triangle as shown: 

\begin{center}
\begin{image}
\definecolor{qqwuqq}{rgb}{0,0.39215686274509803,0}
\definecolor{xdxdff}{rgb}{0.49019607843137253,0.49019607843137253,1}
\definecolor{ududff}{rgb}{0.30196078431372547,0.30196078431372547,1}
\definecolor{uuuuuu}{rgb}{0.26666666666666666,0.26666666666666666,0.26666666666666666}
\begin{tikzpicture}[line cap=round,line join=round,>=triangle 45,x=1cm,y=1cm]
\draw[line width=1.2pt,color=qqwuqq,fill=qqwuqq,fill opacity=0.10000000149011612] (3,0.28284271247461906) -- (2.717157287525381,0.28284271247461906) -- (2.717157287525381,0) -- (3,0) -- cycle; 
\draw [line width=1.2pt] (0,0)-- (3,2);
\draw [line width=1.2pt] (3,2)-- (3,0);
\draw [line width=1.2pt] (3,0)-- (0,0);
\begin{scriptsize}
\draw [fill=uuuuuu] (0,0) circle (1pt);
\draw[color=uuuuuu] (-0.29,0.12) node {$X$};
\draw [fill=ududff] (3,2) circle (1.5pt);
\draw[color=ududff] (3.15,2.3) node {$Y$};
\draw[color=black] (1.39,1.29) node {$d$};
\draw [fill=xdxdff] (3,0) circle (1.5pt);
\draw[color=xdxdff] (3.2,-0.03) node {$Z$};
\draw[color=black] (3.27,1.05) node {$a$};
\draw[color=black] (1.63,-0.20) node {$b$};
\draw [fill=white] (6,0) circle (0.5pt);
\draw [fill=white] (-3,0) circle (0.5pt);
\end{scriptsize}
\end{tikzpicture}
\end{image}
\end{center}

Right $\triangle XYZ$, with right angle $\answer{Z}$, has legs of length $a$ and $\answer{b}$.  Opposite $\angle Z$ is a hypotenuse whose length is \wordChoice{\choice{equal to $c$}\choice{obvious}\choice[correct]{not yet known}}, so let's call it $\answer{d}$.  

From the (forward version of the) Pythagorean Theorem, it follows that $a^2+b^2=\answer{d^2}$, where $d$ is the hypotenuse of $\triangle \answer{XYZ}$, known to be 
\wordChoice{\choice{acute}\choice[correct]{right}\choice{obtuse}}.  From the two equations, it follows that $d^2=\answer{c^2}$.  

Algebraically, we see that $d=\pm c$, from which can conclude that $d=\answer{c}$ because the side lengths must be \wordChoice{\choice[correct]{positive}\choice{zero}\choice{negative}\choice{non-zero}}.   

Now we conclude that $\triangle ABC \cong \triangle XYZ$ by $\answer[format=string]{SSS}$ triangle congruence.  And from the congruent triangles, we see that $\angle C\cong\angle \answer{Z}$, which implies that $\triangle ABC$ is a $\answer[format=string]{right}$ triangle.  
\end{problem}
\end{proof}
\end{problem}


\end{document}