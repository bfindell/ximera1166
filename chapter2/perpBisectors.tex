\documentclass[nooutcomes]{ximera}
%\documentclass[space,handout,nooutcomes]{ximera}

% For preamble materials

\usepackage{pgf,tikz}
\usepackage{mathrsfs}
\usetikzlibrary{arrows}
\usepackage{framed}
\usepackage{amsmath}
\pgfplotsset{compat=1.17}

\def\fixnote#1{\begin{framed}{\textcolor{red}{Fix note: #1}}\end{framed}}  % Allows insertion of red notes about needed edits
%\def\fixnote#1{}

\def\detail#1{{\textcolor{blue}{Detail: #1}}}   

\pdfOnly{\renewenvironment{image}[1][]{\begin{center}}{\end{center}}}

\graphicspath{
  {./}
  {chapter1/}
  {chapter2/}
  {chapter4/}
  {proofs/}
  {graphics/}
  {../graphics/}
}

\newenvironment{sectionOutcomes}{}{}


%%% This set of code is all of our user defined commands
\newcommand{\bysame}{\mbox{\rule{3em}{.4pt}}\,}
\newcommand{\N}{\mathbb N}
\newcommand{\C}{\mathbb C}
\newcommand{\W}{\mathbb W}
\newcommand{\Z}{\mathbb Z}
\newcommand{\Q}{\mathbb Q}
\newcommand{\R}{\mathbb R}
\newcommand{\A}{\mathbb A}
\newcommand{\D}{\mathcal D}
\newcommand{\F}{\mathcal F}
\newcommand{\ph}{\varphi}
\newcommand{\ep}{\varepsilon}
\newcommand{\aph}{\alpha}
\newcommand{\QM}{\begin{center}{\huge\textbf{?}}\end{center}}

\renewcommand{\le}{\leqslant}
\renewcommand{\ge}{\geqslant}
\renewcommand{\a}{\wedge}
\renewcommand{\v}{\vee}
\renewcommand{\l}{\ell}
\newcommand{\mat}{\mathsf}
\renewcommand{\vec}{\mathbf}
\renewcommand{\subset}{\subseteq}
\renewcommand{\supset}{\supseteq}
%\renewcommand{\emptyset}{\varnothing}
%\newcommand{\xto}{\xrightarrow}
%\renewcommand{\qedsymbol}{$\blacksquare$}
%\newcommand{\bibname}{References and Further Reading}
%\renewcommand{\bar}{\protect\overline}
%\renewcommand{\hat}{\protect\widehat}
%\renewcommand{\tilde}{\widetilde}
%\newcommand{\tri}{\triangle}
%\newcommand{\minipad}{\vspace{1ex}}
%\newcommand{\leftexp}[2]{{\vphantom{#2}}^{#1}{#2}}

%% More user defined commands
\renewcommand{\epsilon}{\varepsilon}
\renewcommand{\theta}{\vartheta} %% only for kmath
\renewcommand{\l}{\ell}
\renewcommand{\d}{\, d}
\newcommand{\ddx}{\frac{d}{dx}}
\newcommand{\dydx}{\frac{dy}{dx}}


\usepackage{bigstrut}


\title{Perpendicular Bisectors}
\author{Bart Snapp and Brad Findell}
\begin{document}
\begin{abstract}
Short-answer problems about perpendicular bisectors. 
\end{abstract}
\maketitle


\begin{problem}
In the problems below, we use the following previously proven theorem: 
\begin{theorem}
Points on the $\answer[format=string]{perpendicular bisector}$ of a segment are exactly those 
that are equidistant from the endpoints of the segment.  
\end{theorem}
The phrase \emph{exactly those} means the theorem is equivalent to the following pair of statements: 
\begin{enumerate}
\item If a point is on the perpendicular bisector of a segment, then it is equidistant from the endpoints of the segment. 
\item If a point equidistant from the endpoints of the segment, then it is on the perpendicular bisector of that segment. 
\end{enumerate}
\end{problem}

\begin{problem}
\begin{theorem}
The perpendicular bisectors of a triangle are concurrent.  
\end{theorem}
Proof: Given $\triangle ABC$, let line $k$ be the perpendicular bisector of segment $\answer{AB}$; 
let line $j$ be the perpendicular bisector of segment $\answer{BC}$; 
and let $\answer{X}$ be their intersection, as shown in the figure below.  

\begin{image}
\definecolor{uuuuuu}{rgb}{0.26666666666666666,0.26666666666666666,0.26666666666666666}
\definecolor{qqqqff}{rgb}{0.,0.,1.}
\begin{tikzpicture}[line cap=round,line join=round,>=triangle 45,x=1.0cm,y=1.0cm]
\clip(-0.7,-1.) rectangle (6.2,3.5);
\draw [line width=0.8pt] (0.,1.)-- (4.,0.);
\draw [line width=0.8pt] (4.,0.)-- (5.,2.);
\draw [line width=0.8pt] (5.,2.)-- (0.,1.);
\draw [line width=0.8pt,domain=-0.7:6.6] plot(\x,{(-7.5--4.*\x)/1.});
\draw [line width=0.8pt,domain=-0.7:6.6] plot(\x,{(-6.5--1.*\x)/-2.});
%\draw [line width=0.8pt,dash pattern=on 2pt off 2pt] (2.39,2.06)-- (0.,1.);
%\draw [line width=0.8pt,dash pattern=on 2pt off 2pt] (2.39,2.06)-- (5.,2.);
%\draw [line width=0.8pt,dash pattern=on 2pt off 2pt] (2.39,2.06)-- (4.,0.);
%\draw [line width=0.8pt] (2.39,2.06) circle (2.612cm);
\begin{scriptsize}
\draw [fill=qqqqff] (0.,1.) circle (1.5pt);
\draw[color=qqqqff] (-0.2,0.85) node {$A$};
\draw [fill=qqqqff] (4.,0.) circle (1.5pt);
\draw[color=qqqqff] (4.2,-0.1) node {$B$};
\draw [fill=qqqqff] (5.,2.) circle (1.5pt);
\draw[color=qqqqff] (5.2,2.2) node {$C$};
\draw[color=black] (2.40,2.82) node {$k$};
\draw[color=black] (1.,2.5) node {$j$};
\draw [fill=uuuuuu] (2.3889,2.0555) circle (1.5pt);
\draw[color=uuuuuu] (2.6,2.26) node {$X$};
\end{scriptsize}
\end{tikzpicture}
\end{image}

\begin{enumerate}
\item First, $XA = XB$ because $X$ is on the perpendicular bisector of segment $\answer{AB}$, from Theorem 1, part \wordChoice{\choice[correct]{(a)}\choice{(b)}\choice{neither}}. 
\item Similarly, $XB = XC$ because $X$ is on the perpendicular bisector of segment$\answer{BC}$. 
\item Then, $XA = XC$ because they both equal $\answer{XB}$.  
\item Thus, X must be on the perpendicular bisector (not shown) of segment $\answer{AC}$, from Theorem 1, part \wordChoice{\choice{(a)}\choice[correct]{(b)}\choice{neither}}.  
\item Therefore $X$ is on all three perpendicular bisectors, which is to say they are concurrent.  
\end{enumerate}

Note that $X$ is the center of the $\answer[format=string]{circumcircle}$ of $\triangle ABC$, as shown below, because it is equidistant from its three vertices.  

\begin{image}
\definecolor{uuuuuu}{rgb}{0.26666666666666666,0.26666666666666666,0.26666666666666666}
\definecolor{qqqqff}{rgb}{0.,0.,1.}
\begin{tikzpicture}[line cap=round,line join=round,>=triangle 45,x=1.0cm,y=1.0cm]
\clip(-0.7,-1.) rectangle (6.2,5.);
\draw [line width=0.8pt] (0.,1.)-- (4.,0.);
\draw [line width=0.8pt] (4.,0.)-- (5.,2.);
\draw [line width=0.8pt] (5.,2.)-- (0.,1.);
\draw [line width=0.8pt,domain=-0.7:6.6] plot(\x,{(-7.5--4.*\x)/1.});
\draw [line width=0.8pt,domain=-0.7:6.6] plot(\x,{(-6.5--1.*\x)/-2.});
\draw [line width=0.8pt,dash pattern=on 2pt off 2pt] (2.39,2.06)-- (0.,1.);
\draw [line width=0.8pt,dash pattern=on 2pt off 2pt] (2.39,2.06)-- (5.,2.);
\draw [line width=0.8pt,dash pattern=on 2pt off 2pt] (2.39,2.06)-- (4.,0.);
\draw [line width=0.8pt] (2.39,2.06) circle (2.612cm);
\begin{scriptsize}
\draw [fill=qqqqff] (0.,1.) circle (1.5pt);
\draw[color=qqqqff] (-0.2,0.85) node {$A$};
\draw [fill=qqqqff] (4.,0.) circle (1.5pt);
\draw[color=qqqqff] (4.2,-0.1) node {$B$};
\draw [fill=qqqqff] (5.,2.) circle (1.5pt);
\draw[color=qqqqff] (5.2,2.2) node {$C$};
\draw[color=black] (2.40,2.82) node {$k$};
\draw[color=black] (1.,2.5) node {$j$};
\draw [fill=uuuuuu] (2.3889,2.0555) circle (1.5pt);
\draw[color=uuuuuu] (2.6,2.26) node {$X$};
\end{scriptsize}
\end{tikzpicture}
\end{image}

\end{problem}

\end{document}
