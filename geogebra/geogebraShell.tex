\documentclass[nooutcomes]{ximera}

% For preamble materials

\usepackage{pgf,tikz}
\usepackage{mathrsfs}
\usetikzlibrary{arrows}
\usepackage{framed}
\usepackage{amsmath}
\pgfplotsset{compat=1.17}

\def\fixnote#1{\begin{framed}{\textcolor{red}{Fix note: #1}}\end{framed}}  % Allows insertion of red notes about needed edits
%\def\fixnote#1{}

\def\detail#1{{\textcolor{blue}{Detail: #1}}}   

\pdfOnly{\renewenvironment{image}[1][]{\begin{center}}{\end{center}}}

\graphicspath{
  {./}
  {chapter1/}
  {chapter2/}
  {chapter4/}
  {proofs/}
  {graphics/}
  {../graphics/}
}

\newenvironment{sectionOutcomes}{}{}


%%% This set of code is all of our user defined commands
\newcommand{\bysame}{\mbox{\rule{3em}{.4pt}}\,}
\newcommand{\N}{\mathbb N}
\newcommand{\C}{\mathbb C}
\newcommand{\W}{\mathbb W}
\newcommand{\Z}{\mathbb Z}
\newcommand{\Q}{\mathbb Q}
\newcommand{\R}{\mathbb R}
\newcommand{\A}{\mathbb A}
\newcommand{\D}{\mathcal D}
\newcommand{\F}{\mathcal F}
\newcommand{\ph}{\varphi}
\newcommand{\ep}{\varepsilon}
\newcommand{\aph}{\alpha}
\newcommand{\QM}{\begin{center}{\huge\textbf{?}}\end{center}}

\renewcommand{\le}{\leqslant}
\renewcommand{\ge}{\geqslant}
\renewcommand{\a}{\wedge}
\renewcommand{\v}{\vee}
\renewcommand{\l}{\ell}
\newcommand{\mat}{\mathsf}
\renewcommand{\vec}{\mathbf}
\renewcommand{\subset}{\subseteq}
\renewcommand{\supset}{\supseteq}
%\renewcommand{\emptyset}{\varnothing}
%\newcommand{\xto}{\xrightarrow}
%\renewcommand{\qedsymbol}{$\blacksquare$}
%\newcommand{\bibname}{References and Further Reading}
%\renewcommand{\bar}{\protect\overline}
%\renewcommand{\hat}{\protect\widehat}
%\renewcommand{\tilde}{\widetilde}
%\newcommand{\tri}{\triangle}
%\newcommand{\minipad}{\vspace{1ex}}
%\newcommand{\leftexp}[2]{{\vphantom{#2}}^{#1}{#2}}

%% More user defined commands
\renewcommand{\epsilon}{\varepsilon}
\renewcommand{\theta}{\vartheta} %% only for kmath
\renewcommand{\l}{\ell}
\renewcommand{\d}{\, d}
\newcommand{\ddx}{\frac{d}{dx}}
\newcommand{\dydx}{\frac{dy}{dx}}


\usepackage{bigstrut}


\title{GeoGebra Demo}
\author{Brad Findell}
\begin{document}
\begin{abstract}
Testing ways to embed Euclid the Game. 
\end{abstract}
\maketitle


\begin{problem}
Complete the following GeoGebra activity.  

\ifxake
\Hcode{
<div>
<iframe scrolling="auto" src="geogebraDemo.html" width="720px" height="600px" border="1px"> </iframe>
</div>
}
\fi
What is your answer? $\answer{3}$.  Hint: 3. 

%
%<p class="noindent"><iframe scrolling="no" src="https://www.geogebra.org/material/iframe/id/mjscvuw3/width/600/height/320/border/888888/rc/false/ai/false/sdz/false/smb/false/stb/false/stbh/false/ld/false/sri/false/at/auto" width="600px" height="320px" style="border:0px;">


%Number of moves: 
%\begin{javascript}
%document.write(localStorage.Level4)
%\end{javascript}


\end{problem}

% Sample javascript with validator
%
%\begin{javascript}
%function isPrime(num) {
%  for(var i = 2; i < num; i++)
%    if(num % i === 0) return false;
%  return num > 1;
%}
%
%function isPrimeFactorization(x,y) {
%  var terms = x.split('*').map( function(t) { return parseInt(t) } );
%  return terms.every( isPrime ) &&
%    (terms.reduce( function(a,c) { return a*c; }, 1 )) == parseInt(y);
%}
%\end{javascript}
%
%Then $\answer[format=string,validator=isPrimeFactorization]{33}$ would
%accept 3*11 and 11*3.  This doesn't work with powers or signs.
%

\end{document}
