\documentclass[nooutcomes]{ximera}
%\documentclass[space,handout,nooutcomes]{ximera}

% For preamble materials

\graphicspath{
  {./}
  {chapter1/}
  {chapter2/}
  {chapter4/}
  {math1/}
  {math2/}
}

\usepackage{pgf,tikz}
\usepackage{mathrsfs}
\usetikzlibrary{arrows}
\pgfplotsset{compat=1.16}


\newcommand{\N}{\mathbb N}
\newcommand{\W}{\mathbb W}
\newcommand{\C}{\mathbb C}
\newcommand{\Z}{\mathbb Z}
\newcommand{\Q}{\mathbb Q}
\newcommand{\R}{\mathbb R}




%\usepackage{tikz}


\title{Dilation and Area}
\author{Brad Findell}
\begin{document}
\begin{abstract}
Area under a dilation. 
\end{abstract}
\maketitle


\begin{problem}
In a previous class, we made some measurements on sketch (below) of a dilation and used those measurements to determine the scale factor of the dilation. 
\begin{image}
\includegraphics[width=5in]{dilation.png}
\end{image}
Write three different expressions that would correctly compute the scale factor.  
\vfill
\end{problem}

\newpage
\begin{problem}
To determine what happens to the areas of the figures under scaling, Isabel suggested that we overlay a grid on the pictures to estimate areas. 
\begin{enumerate} 
\item Compare and contrast the following two approaches to implementing Isabel's idea.  
\begin{image}
\includegraphics[width=4.4in]{dilationGrid1.png}

\includegraphics[width=4in]{dilationGrid2.png}
\end{image}
\item Complete the approach that you prefer to justify a relationship between the areas of the two red figures.  
\vfill
\item What can you say about a relationship between the perimeters of the red figures?  
\vfill
\end{enumerate}
\end{problem}

\newpage
\begin{problem}
So far we have explored how lengths and areas of two-dimensional objects change under scaling.  How will these ideas generalize to three-dimensional objects?  Explain your reasoning.  
\vfill
\end{problem}

\end{document}
