\documentclass[nooutcomes]{ximera}
%\documentclass[space,handout,nooutcomes]{ximera}

% For preamble materials

\usepackage{pgf,tikz}
\usepackage{mathrsfs}
\usetikzlibrary{arrows}
\usepackage{framed}
\usepackage{amsmath}
\pgfplotsset{compat=1.17}

\def\fixnote#1{\begin{framed}{\textcolor{red}{Fix note: #1}}\end{framed}}  % Allows insertion of red notes about needed edits
%\def\fixnote#1{}

\def\detail#1{{\textcolor{blue}{Detail: #1}}}   

\pdfOnly{\renewenvironment{image}[1][]{\begin{center}}{\end{center}}}

\graphicspath{
  {./}
  {chapter1/}
  {chapter2/}
  {chapter4/}
  {proofs/}
  {graphics/}
  {../graphics/}
}

\newenvironment{sectionOutcomes}{}{}


%%% This set of code is all of our user defined commands
\newcommand{\bysame}{\mbox{\rule{3em}{.4pt}}\,}
\newcommand{\N}{\mathbb N}
\newcommand{\C}{\mathbb C}
\newcommand{\W}{\mathbb W}
\newcommand{\Z}{\mathbb Z}
\newcommand{\Q}{\mathbb Q}
\newcommand{\R}{\mathbb R}
\newcommand{\A}{\mathbb A}
\newcommand{\D}{\mathcal D}
\newcommand{\F}{\mathcal F}
\newcommand{\ph}{\varphi}
\newcommand{\ep}{\varepsilon}
\newcommand{\aph}{\alpha}
\newcommand{\QM}{\begin{center}{\huge\textbf{?}}\end{center}}

\renewcommand{\le}{\leqslant}
\renewcommand{\ge}{\geqslant}
\renewcommand{\a}{\wedge}
\renewcommand{\v}{\vee}
\renewcommand{\l}{\ell}
\newcommand{\mat}{\mathsf}
\renewcommand{\vec}{\mathbf}
\renewcommand{\subset}{\subseteq}
\renewcommand{\supset}{\supseteq}
%\renewcommand{\emptyset}{\varnothing}
%\newcommand{\xto}{\xrightarrow}
%\renewcommand{\qedsymbol}{$\blacksquare$}
%\newcommand{\bibname}{References and Further Reading}
%\renewcommand{\bar}{\protect\overline}
%\renewcommand{\hat}{\protect\widehat}
%\renewcommand{\tilde}{\widetilde}
%\newcommand{\tri}{\triangle}
%\newcommand{\minipad}{\vspace{1ex}}
%\newcommand{\leftexp}[2]{{\vphantom{#2}}^{#1}{#2}}

%% More user defined commands
\renewcommand{\epsilon}{\varepsilon}
\renewcommand{\theta}{\vartheta} %% only for kmath
\renewcommand{\l}{\ell}
\renewcommand{\d}{\, d}
\newcommand{\ddx}{\frac{d}{dx}}
\newcommand{\dydx}{\frac{dy}{dx}}


\usepackage{bigstrut}


\title{Rotate 180 Degrees}
\author{Brad Findell}
\begin{document}
\begin{abstract}
Proofs updated. 
\end{abstract}
\maketitle


%\begin{axiom}
%Parallel postulate (uniqueness of parallels):  Given a line and a point not on the line, there is exactly one line through the given point parallel to the given line.  
%\end{axiom}

\begin{theorem}

A $180^\circ$ rotation about a point on a line maps the line to 
$\answer[format=string]{itself}$. 

\begin{problem}
\begin{proof}
Suppose point $P$ is on line $k$.  The point cuts the line into two opposite $\answer[format=string]{rays}$.  Call them $\overrightarrow{PR}$ and $\overrightarrow{PQ}$.  
\begin{image}
\definecolor{qqqqff}{rgb}{0.,0.,1.}
\begin{tikzpicture}[line cap=round,line join=round,>=triangle 45,x=1.0cm,y=1.0cm]
\clip(0.8,8.0) rectangle (11.5,11.5);
\draw [line width=0.8pt,domain=0.8:11.5] plot(\x,{(16.0192+0.76*\x)/2.12});
\begin{small}
\draw [fill=qqqqff] (5.98,9.7) circle (1.5pt);
\draw[color=qqqqff] (6.12,9.45) node {$P$};
\draw [fill=qqqqff] (8.1,10.46) circle (1.5pt);
\draw[color=qqqqff] (8.24,10.2) node {$Q$};
\draw [fill=qqqqff] (4.3666,9.1216) circle (1.5pt);
\draw[color=qqqqff] (4.2,9.4) node {$R$};
\draw[color=black] (2.0,8.6) node {$k$};
\end{small}
\end{tikzpicture}
\end{image}

Because $\angle QPR$ is a $\answer[format=string]{straight angle}$ (two words), a $180^\circ$ rotation about $P$ maps $\overrightarrow{PR}$ onto \wordChoice{\choice{$\overrightarrow{RP}$}\choice[correct]{$\overrightarrow{PQ}$}\choice{$\overrightarrow{QP}$}\choice{$\overrightarrow{PR}$}}, and vice versa.  Swapping the two opposite rays maps the line onto itself.  
\end{proof}
\end{problem}
\end{theorem}


\begin{theorem}
A $180^\circ$ rotation about a point not on a line takes the line to 
\wordChoice{\choice{itself}\choice[correct]{a parallel line}
\choice{a perpendicular line}\choice{a slanted line}}.

\begin{problem}
\begin{proof}
Let $O$ be a point not on line $l$.  Let $P$ be an arbitrary point on $R(l)$, the rotated image of $l$.  
To show that $R(l)$ is parallel to $l$, 
it is sufficient to show that $P$ cannot lie also on $l$.  

\begin{image}
\definecolor{qqqqff}{rgb}{0.,0.,1.}
\begin{tikzpicture}[line cap=round,line join=round,>=triangle 45,x=1.0cm,y=1.0cm]
\clip(-0.8,6.0) rectangle (8.2,10.3);
\draw [line width=0.8pt,domain=-0.84:8.24] plot(\x,{(--43.8512--1.22*\x)/6.92});
\draw [line width=0.8pt,dash pattern=on 2pt off 2pt,domain=-0.84:8.24] plot(\x,{(-61.52-1.22*\x)/-6.92});
\draw [line width=0.6pt,dash pattern=on 2pt off 3pt,domain=-0.52:8.12] plot(\x,{(--38.6071-1.8626*\x)/3.91725});
\begin{scriptsize}
\draw[color=black] (-0.04,6.6) node {$l$};
\draw[color=black] (7.3,6.6) node {$k$};
\draw [fill=qqqqff] (3.44,8.22) circle (1.2pt);
\draw[color=qqqqff] (3.55,8.50) node {$O$};
\draw[color=black] (-0.16,8.6) node {$R(l)$};
\draw [fill=qqqqff] (1.48135,9.1513) circle (1.2pt);
\draw[color=qqqqff] (1.62,9.45) node {$P$};
\draw [fill=qqqqff] (5.3986,7.2887) circle (1.2pt);
\draw[color=qqqqff] (5.54,7.05) node {$Q$};
\end{scriptsize}
\end{tikzpicture}
\end{image}
%\[
%\includegraphics[scale=0.6]{../graphics/lineRotation.pdf}
%\]

Because $P$ is on $R(l)$, there is a point $Q$ on $l$ such that $P = R(Q)$.  The rotated image of 
$\overrightarrow{OQ}$ is \wordChoice{\choice{$\overrightarrow{QO}$}\choice[correct]{$\overrightarrow{OP}$}\choice{$\overrightarrow{QP}$}}, and because $\angle QOP$ is $180^\circ$, it 
follows that $Q$, $O$, and $P$ are $\answer[format=string]{collinear}$.  Call that line $k$.  We know line $k$ is distinct 
from $l$ because point $O$ is on line $\answer{k}$ but not on line $\answer{l}$.  Now, if $P$ were on $l$, then points $P$ and $Q$ 
would both be on two distinct lines, $k$ and $l$, contradicting the assumption (in Euclidean geometry) that on two distinct points there is a \wordChoice{\choice{parallel}\choice{perpendicular}\choice[correct]{unique}\choice{random}} line.  The theorem is proved.  
\end{proof}
\end{problem}
\end{theorem}

\end{document}
