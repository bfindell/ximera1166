\documentclass[handout,nooutcomes]{ximera}
%\documentclass[space,handout,nooutcomes]{ximera}


% For preamble materials

\usepackage{pgf,tikz}
\usepackage{mathrsfs}
\usetikzlibrary{arrows}
\usepackage{framed}
\usepackage{amsmath}
\pgfplotsset{compat=1.17}

\def\fixnote#1{\begin{framed}{\textcolor{red}{Fix note: #1}}\end{framed}}  % Allows insertion of red notes about needed edits
%\def\fixnote#1{}

\def\detail#1{{\textcolor{blue}{Detail: #1}}}   

\pdfOnly{\renewenvironment{image}[1][]{\begin{center}}{\end{center}}}

\graphicspath{
  {./}
  {chapter1/}
  {chapter2/}
  {chapter4/}
  {proofs/}
  {graphics/}
  {../graphics/}
}

\newenvironment{sectionOutcomes}{}{}


%%% This set of code is all of our user defined commands
\newcommand{\bysame}{\mbox{\rule{3em}{.4pt}}\,}
\newcommand{\N}{\mathbb N}
\newcommand{\C}{\mathbb C}
\newcommand{\W}{\mathbb W}
\newcommand{\Z}{\mathbb Z}
\newcommand{\Q}{\mathbb Q}
\newcommand{\R}{\mathbb R}
\newcommand{\A}{\mathbb A}
\newcommand{\D}{\mathcal D}
\newcommand{\F}{\mathcal F}
\newcommand{\ph}{\varphi}
\newcommand{\ep}{\varepsilon}
\newcommand{\aph}{\alpha}
\newcommand{\QM}{\begin{center}{\huge\textbf{?}}\end{center}}

\renewcommand{\le}{\leqslant}
\renewcommand{\ge}{\geqslant}
\renewcommand{\a}{\wedge}
\renewcommand{\v}{\vee}
\renewcommand{\l}{\ell}
\newcommand{\mat}{\mathsf}
\renewcommand{\vec}{\mathbf}
\renewcommand{\subset}{\subseteq}
\renewcommand{\supset}{\supseteq}
%\renewcommand{\emptyset}{\varnothing}
%\newcommand{\xto}{\xrightarrow}
%\renewcommand{\qedsymbol}{$\blacksquare$}
%\newcommand{\bibname}{References and Further Reading}
%\renewcommand{\bar}{\protect\overline}
%\renewcommand{\hat}{\protect\widehat}
%\renewcommand{\tilde}{\widetilde}
%\newcommand{\tri}{\triangle}
%\newcommand{\minipad}{\vspace{1ex}}
%\newcommand{\leftexp}[2]{{\vphantom{#2}}^{#1}{#2}}

%% More user defined commands
\renewcommand{\epsilon}{\varepsilon}
\renewcommand{\theta}{\vartheta} %% only for kmath
\renewcommand{\l}{\ell}
\renewcommand{\d}{\, d}
\newcommand{\ddx}{\frac{d}{dx}}
\newcommand{\dydx}{\frac{dy}{dx}}


\usepackage{bigstrut}

%\usetikzlibrary {arrows.meta}

\title{Inscribed Angle Proof}
\author{Brad Findell}
\begin{document}
\begin{abstract}
Proofs updated. 
\end{abstract}
\maketitle

\begin{problem}
In the figure below, $\overline{AC}$ is a diameter of a circle with center $P$. Prove that $\angle ABC$ is a right angle.  

\begin{image}
\definecolor{qqwuqq}{rgb}{0.,0.3926,0.}
\definecolor{uuuuuu}{rgb}{0.2667,0.2667,0.2667}
\definecolor{qqqqff}{rgb}{0.,0.,1.}
\begin{tikzpicture}[scale=0.7,line cap=round,line join=round,>=triangle 45,x=1.0cm,y=1.0cm]
\clip(-4.5,-3.8) rectangle (13.5,3.8);
%\draw [shift={(-3.6056,0.)},line width=0.8pt,color=qqwuqq,fill=qqwuqq,fill opacity=0.1] (0,0) -- (0.:1.115) arc (0.:28.155:1.115) -- cycle;
%\draw [shift={(2.,3.)},line width=0.8pt,color=qqwuqq,fill=qqwuqq,fill opacity=0.1] (0,0) -- (-151.845:1.115) arc (-151.845:-123.690:1.115) -- cycle;
%\draw [shift={(2.,3.)},line width=0.8pt,color=qqwuqq,fill=qqwuqq,fill opacity=0.1] (0,0) -- (-123.690:0.892) arc (-123.690:-61.845:0.892) -- cycle;
%\draw [shift={(2.,3.)},line width=0.8pt,color=qqwuqq] (-123.690:0.669) arc (-123.690:-61.85:0.669);
%\draw [shift={(3.6056,0.)},line width=0.8pt,color=qqwuqq,fill=qqwuqq,fill opacity=0.1] (0,0) -- (118.155:0.892) arc (118.155:180.:0.892) -- cycle;
%\draw [shift={(3.6056,0.)},line width=0.8pt,color=qqwuqq] (118.155:0.669) arc (118.155:180.:0.669);
\draw [line width=0.8pt] (0.,0.) circle (3.606cm);
\draw [line width=0.8pt] (0.,0.) circle (3.606cm);
\draw [line width=0.8pt] (-3.6056,0.)-- (2.,3.);
\draw [line width=0.8pt] (2.,3.)-- (3.6056,0.);
\draw [line width=0.8pt] (-3.6056,0.)-- (0.,0.);
%\draw [line width=0.8pt] (-1.8028,0.2006) -- (-1.8028,-0.2006);
\draw [line width=0.8pt] (0.,0.)-- (3.6056,0.);
%\draw [line width=0.8pt] (1.8028,0.2006) -- (1.8028,-0.2006);
%\draw [line width=0.8pt] (0.,0.)-- (2.,3.);
%\draw [line width=0.8pt] (0.833,1.611) -- (1.167,1.389);
%\draw (-2.2,0.35) node {$x$};
%\draw (0.95,2.1) node {$x$};
%\draw (1.95,1.85) node {$y$};
%\draw (2.6,0.6) node {$y$};
%\draw (0.55,0.3) node {$z$};
%\begin{scriptsize}
\draw [fill=qqqqff] (0.,0.) circle (1.2pt);
\draw[color=qqqqff] (-0.24,-0.28) node {$P$};
\draw [fill=qqqqff] (2.,3.) circle (1.2pt);
\draw[color=qqqqff] (2.25,3.3) node {$B$};
\draw [fill=qqqqff] (-3.60,0.) circle (1.2pt);
\draw[color=qqqqff] (-3.9,0) node {$A$};
\draw [fill=qqqqff] (3.60,0.) circle (1.2pt);
\draw[color=qqqqff] (3.90,0) node {$C$};
%\end{scriptsize}
%\end{tikzpicture}

\begin{scope}[shift={(9,0)}]
%\definecolor{qqwuqq}{rgb}{0.,0.3926,0.}
%\definecolor{qqqqff}{rgb}{0.,0.,1.}
%\begin{tikzpicture}[line cap=round,line join=round,>=triangle 45,x=1.0cm,y=1.0cm]
%\clip(-4.5,-3.8) rectangle (4.5,3.8);
%\draw [shift={(-3.6056,0.)},line width=0.8pt,color=qqwuqq,fill=qqwuqq,fill opacity=0.1] (0,0) -- (0.:1.115) arc (0.:28.155:1.115) -- cycle;
%\draw [shift={(2.,3.)},line width=0.8pt,color=qqwuqq,fill=qqwuqq,fill opacity=0.1] (0,0) -- (-151.845:1.115) arc (-151.845:-123.690:1.115) -- cycle;
%\draw [shift={(2.,3.)},line width=0.8pt,color=qqwuqq,fill=qqwuqq,fill opacity=0.1] (0,0) -- (-123.690:0.892) arc (-123.690:-61.845:0.892) -- cycle;
%\draw [shift={(2.,3.)},line width=0.8pt,color=qqwuqq] (-123.690:0.669) arc (-123.690:-61.85:0.669);
%\draw [shift={(3.6056,0.)},line width=0.8pt,color=qqwuqq,fill=qqwuqq,fill opacity=0.1] (0,0) -- (118.155:0.892) arc (118.155:180.:0.892) -- cycle;
%\draw [shift={(3.6056,0.)},line width=0.8pt,color=qqwuqq] (118.155:0.669) arc (118.155:180.:0.669);
\draw [line width=0.8pt] (0.,0.) circle (3.606cm);
\draw [line width=0.8pt] (-3.6056,0.)-- (2.,3.);
\draw [line width=0.8pt] (2.,3.)-- (3.6056,0.);
\draw [line width=0.8pt] (-3.6056,0.)-- (0.,0.);
\draw [line width=0.8pt] (-1.8028,0.2006) -- (-1.8028,-0.2006);
\draw [line width=0.8pt] (0.,0.)-- (3.6056,0.);
\draw [line width=0.8pt] (1.8028,0.2006) -- (1.8028,-0.2006);
\draw [line width=0.8pt] (0.,0.)-- (2.,3.);
\draw [line width=0.8pt] (0.833,1.611) -- (1.167,1.389);
%\draw (-2.2,0.35) node {$x$};
%\draw (0.95,2.1) node {$x$};
%\draw (1.95,1.85) node {$y$};
%\draw (2.6,0.6) node {$y$};
%\draw (0.55,0.3) node {$z$};
%\begin{scriptsize}
\draw [fill=qqqqff] (0.,0.) circle (1.2pt);
\draw[color=qqqqff] (-0.24,-0.28) node {$P$};
\draw [fill=qqqqff] (2.,3.) circle (1.2pt);
\draw[color=qqqqff] (2.25,3.3) node {$B$};
\draw [fill=qqqqff] (-3.60,0.) circle (1.2pt);
\draw[color=qqqqff] (-3.9,0) node {$A$};
\draw [fill=qqqqff] (3.60,0.) circle (1.2pt);
\draw[color=qqqqff] (3.90,0) node {$C$};
%\end{scriptsize}
\end{scope}
\end{tikzpicture}
\end{image}

\begin{enumerate}
\item Beginning with the diagram on the left, Natalia draws $\overline{PB}$ and marks the diagram to show segments that she knows to be congruent because each one is a $\answer[format=string]{radius}$ of the circle.  

\begin{image}
\definecolor{qqwuqq}{rgb}{0.,0.3926,0.}
\definecolor{qqqqff}{rgb}{0.,0.,1.}
\begin{tikzpicture}[scale=0.7,line cap=round,line join=round,>=triangle 45,x=1.0cm,y=1.0cm]
\clip(-4.5,-3.8) rectangle (13.5,3.8);
\draw [shift={(-3.6056,0.)},line width=0.8pt,color=qqwuqq,fill=qqwuqq,fill opacity=0.1] (0,0) -- (0.:1.115) arc (0.:28.155:1.115) -- cycle;
\draw [shift={(2.,3.)},line width=0.8pt,color=qqwuqq,fill=qqwuqq,fill opacity=0.1] (0,0) -- (-151.845:1.115) arc (-151.845:-123.690:1.115) -- cycle;
\draw [shift={(2.,3.)},line width=0.8pt,color=qqwuqq,fill=qqwuqq,fill opacity=0.1] (0,0) -- (-123.690:0.892) arc (-123.690:-61.845:0.892) -- cycle;
\draw [shift={(2.,3.)},line width=0.8pt,color=qqwuqq] (-123.690:0.669) arc (-123.690:-61.85:0.669);
\draw [shift={(3.6056,0.)},line width=0.8pt,color=qqwuqq,fill=qqwuqq,fill opacity=0.1] (0,0) -- (118.155:0.892) arc (118.155:180.:0.892) -- cycle;
\draw [shift={(3.6056,0.)},line width=0.8pt,color=qqwuqq] (118.155:0.669) arc (118.155:180.:0.669);
\draw [line width=0.8pt] (0.,0.) circle (3.606cm);
\draw [line width=0.8pt] (-3.6056,0.)-- (2.,3.);
\draw [line width=0.8pt] (2.,3.)-- (3.6056,0.);
\draw [line width=0.8pt] (-3.6056,0.)-- (0.,0.);
\draw [line width=0.8pt] (-1.8028,0.2006) -- (-1.8028,-0.2006);
\draw [line width=0.8pt] (0.,0.)-- (3.6056,0.);
\draw [line width=0.8pt] (1.8028,0.2006) -- (1.8028,-0.2006);
\draw [line width=0.8pt] (0.,0.)-- (2.,3.);
\draw [line width=0.8pt] (0.833,1.611) -- (1.167,1.389);
%\draw (-2.2,0.35) node {$x$};
%\draw (0.95,2.1) node {$x$};
%\draw (1.95,1.85) node {$y$};
%\draw (2.6,0.6) node {$y$};
%\draw (0.55,0.3) node {$z$};
%\begin{scriptsize}
\draw [fill=qqqqff] (0.,0.) circle (1.2pt);
\draw[color=qqqqff] (-0.24,-0.28) node {$P$};
\draw [fill=qqqqff] (2.,3.) circle (1.2pt);
\draw[color=qqqqff] (2.25,3.3) node {$B$};
\draw [fill=qqqqff] (-3.60,0.) circle (1.2pt);
\draw[color=qqqqff] (-3.9,0) node {$A$};
\draw [fill=qqqqff] (3.60,0.) circle (1.2pt);
\draw[color=qqqqff] (3.90,0) node {$C$};
%\end{scriptsize}
%\end{tikzpicture}

\begin{scope}[shift={(9,0)}]
%\definecolor{qqwuqq}{rgb}{0.,0.3926,0.}
%\definecolor{qqqqff}{rgb}{0.,0.,1.}
%\begin{tikzpicture}[line cap=round,line join=round,>=triangle 45,x=1.0cm,y=1.0cm]
%\clip(-4.5,-3.8) rectangle (4.5,3.8);
\draw [shift={(-3.6056,0.)},line width=0.8pt,color=qqwuqq,fill=qqwuqq,fill opacity=0.1] (0,0) -- (0.:1.115) arc (0.:28.155:1.115) -- cycle;
\draw [shift={(2.,3.)},line width=0.8pt,color=qqwuqq,fill=qqwuqq,fill opacity=0.1] (0,0) -- (-151.845:1.115) arc (-151.845:-123.690:1.115) -- cycle;
\draw [shift={(2.,3.)},line width=0.8pt,color=qqwuqq,fill=qqwuqq,fill opacity=0.1] (0,0) -- (-123.690:0.892) arc (-123.690:-61.845:0.892) -- cycle;
\draw [shift={(2.,3.)},line width=0.8pt,color=qqwuqq] (-123.690:0.669) arc (-123.690:-61.85:0.669);
\draw [shift={(3.6056,0.)},line width=0.8pt,color=qqwuqq,fill=qqwuqq,fill opacity=0.1] (0,0) -- (118.155:0.892) arc (118.155:180.:0.892) -- cycle;
\draw [shift={(3.6056,0.)},line width=0.8pt,color=qqwuqq] (118.155:0.669) arc (118.155:180.:0.669);
\draw [line width=0.8pt] (0.,0.) circle (3.606cm);
\draw [line width=0.8pt] (-3.6056,0.)-- (2.,3.);
\draw [line width=0.8pt] (2.,3.)-- (3.6056,0.);
\draw [line width=0.8pt] (-3.6056,0.)-- (0.,0.);
\draw [line width=0.8pt] (-1.8028,0.2006) -- (-1.8028,-0.2006);
\draw [line width=0.8pt] (0.,0.)-- (3.6056,0.);
\draw [line width=0.8pt] (1.8028,0.2006) -- (1.8028,-0.2006);
\draw [line width=0.8pt] (0.,0.)-- (2.,3.);
\draw [line width=0.8pt] (0.833,1.611) -- (1.167,1.389);
\draw (-2.2,0.35) node {$x$};
\draw (0.95,2.1) node {$x$};
\draw (1.95,1.85) node {$y$};
\draw (2.6,0.6) node {$y$};
%\draw (0.55,0.3) node {$z$};
%\begin{scriptsize}
\draw [fill=qqqqff] (0.,0.) circle (1.2pt);
\draw[color=qqqqff] (-0.24,-0.28) node {$P$};
\draw [fill=qqqqff] (2.,3.) circle (1.2pt);
\draw[color=qqqqff] (2.25,3.3) node {$B$};
\draw [fill=qqqqff] (-3.60,0.) circle (1.2pt);
\draw[color=qqqqff] (-3.9,0) node {$A$};
\draw [fill=qqqqff] (3.60,0.) circle (1.2pt);
\draw[color=qqqqff] (3.90,0) node {$C$};
%\end{scriptsize}
\end{scope}
\end{tikzpicture}
\end{image}


\item Natalia sees that $\triangle APB$ and $\triangle BPC$ are $\answer[format=string]{isosceles}$ triangles, so she marks the figure to show angles that must be congruent because of a theorem she knows about such triangles. 

\item In order to do some algebra with these congruent angles, Natalia labels their measures $x$ and $y$, as shown in the picture on the right.  

\item She writes an equation for the sum of the angles of $\triangle ABC$: 

\[
\answer{x+(x+y)+y} = 180^\circ
\]

\item She divides that equation by $\answer{2}$ to conclude that $m\angle ABC = x+y = \answer{90}$ degrees, confirming that $\angle ABC$ is a
$\answer[format=string]{right}$ angle.  

\end{enumerate}
\end{problem}


\begin{problem}
A special case of the relationship between an inscribed angle and the corresponding central angle.

In the figure below, $\overline{AC}$ is a diameter of a circle with center $P$. Prove that $z=\answer{2x}$.  

\begin{image}
\definecolor{qqwuqq}{rgb}{0.,0.3926,0.}
\definecolor{qqqqff}{rgb}{0.,0.,1.}
\begin{tikzpicture}[scale=0.7,line cap=round,line join=round,>=triangle 45,x=1.0cm,y=1.0cm]
\clip(-4.5,-3.8) rectangle (4.5,3.8);
\draw [shift={(-3.6056,0.)},line width=0.8pt,color=qqwuqq,fill=qqwuqq,fill opacity=0.1] (0,0) -- (0.:1.115) arc (0.:28.155:1.115) -- cycle;
\draw [shift={(2.,3.)},line width=0.8pt,color=qqwuqq,fill=qqwuqq,fill opacity=0.1] (0,0) -- (-151.845:1.115) arc (-151.845:-123.690:1.115) -- cycle;
%\draw [shift={(2.,3.)},line width=0.8pt,color=qqwuqq,fill=qqwuqq,fill opacity=0.1] (0,0) -- (-123.690:0.892) arc (-123.690:-61.845:0.892) -- cycle;
%\draw [shift={(2.,3.)},line width=0.8pt,color=qqwuqq] (-123.690:0.669) arc (-123.690:-61.85:0.669);
%\draw [shift={(3.6056,0.)},line width=0.8pt,color=qqwuqq,fill=qqwuqq,fill opacity=0.1] (0,0) -- (118.155:0.892) arc (118.155:180.:0.892) -- cycle;
%\draw [shift={(3.6056,0.)},line width=0.8pt,color=qqwuqq] (118.155:0.669) arc (118.155:180.:0.669);
\draw [line width=0.8pt] (0.,0.) circle (3.606cm);
\draw [line width=0.8pt] (-3.6056,0.)-- (2.,3.);
%\draw [line width=0.8pt] (2.,3.)-- (3.6056,0.);
\draw [line width=0.8pt] (-3.6056,0.)-- (0.,0.);
\draw [line width=0.8pt] (-1.8028,0.2006) -- (-1.8028,-0.2006);
\draw [line width=0.8pt] (0.,0.)-- (3.6056,0.);
\draw [line width=0.8pt] (1.8028,0.2006) -- (1.8028,-0.2006);
\draw [line width=0.8pt] (0.,0.)-- (2.,3.);
\draw [line width=0.8pt] (0.833,1.611) -- (1.167,1.389);
\draw (-2.2,0.35) node {$x$};
\draw (0.95,2.1) node {$x$};
%\draw (1.95,1.85) node {$y$};
%\draw (2.6,0.6) node {$y$};
\draw (0.55,0.3) node {$z$};
%\begin{scriptsize}
\draw [fill=qqqqff] (0.,0.) circle (1.2pt);
\draw[color=qqqqff] (-0.24,-0.28) node {$P$};
\draw [fill=qqqqff] (2.,3.) circle (1.2pt);
\draw[color=qqqqff] (2.25,3.3) node {$B$};
\draw [fill=qqqqff] (-3.60,0.) circle (1.2pt);
\draw[color=qqqqff] (-3.9,0) node {$A$};
\draw [fill=qqqqff] (3.60,0.) circle (1.2pt);
\draw[color=qqqqff] (3.90,0) node {$C$};
%\end{scriptsize}
\end{tikzpicture}
\end{image}

Because $z$ is the measure of an angle exterior to $\triangle \answer{APB}$, it is equal to the sum of the measures of the \wordChoice{\choice{opposite}\choice{adjacent}\choice[correct]{remote interior}\choice{alternate interior}} angles.  In other words $z=\answer{2x}$.  

Alternatively, without using the exterior angle theorem, one might proceed as follows:
\begin{enumerate}
\item $\angle APB + x + x = 180^\circ$ because of the angle sum in $\triangle \answer{ABP}$.
\item $\angle APB + z = 180^\circ$ because they form a linear pair. 
\item Then $z = \answer{2x}$ by comparing the two equations. 
\end{enumerate}


\begin{problem}

Correct!   Now that we have handled the special case in which the center of the circle lies on one side of the inscribed angle, we consider two other cases:  
\begin{enumerate}
\item When the center of the circle, $P$, is in the $\answer[format=string]{interior}$ of the inscribed angle (left figure); and 
\item When the center of the circle is exterior to the inscribed angle (right figure).  
\end{enumerate}

\begin{image}
\definecolor{qqwuqq}{rgb}{0.,0.39215686274509803,0.}
\definecolor{xdxdff}{rgb}{0.49019607843137253,0.49019607843137253,1.}
\definecolor{uuuuuu}{rgb}{0.26666666666666666,0.26666666666666666,0.26666666666666666}
\definecolor{qqqqff}{rgb}{0.,0.,1.}
\begin{tikzpicture}[scale=0.7,line cap=round,line join=round,x=1.0cm,y=1.0cm]
\clip(-4.5,-3.8) rectangle (13.5,3.8);
%\draw [shift={(-3.6055512754639896,0.)},line width=0.4pt,color=qqwuqq,fill=qqwuqq,fill opacity=0.10000000149011612] (0,0) -- (0.:1.0) arc (0.:28.154966237010104:1.0) -- cycle;
\draw [line width=0.8pt] (0.,0.) circle (3.6055512754639896cm);
\draw [line width=0.8pt] (-3.6055512754639896,0.)-- (2.,3.);
\draw [line width=0.8pt] (-3.6055512754639896,0.)-- (2.991121161538481,-2.0132546279585926);
\draw [line width=0.8pt] (0.,0.)-- (2.,3.);
\draw [line width=0.8pt] (0.,0.)-- (2.991121161538481,-2.0132546279585926);
%\draw (-2.5,0.6) node[anchor=north west] {$x$};
%\draw (-2.2, 0.05) node[anchor=north west] {$y$};
%\draw [shift={(-3.6055512754639896,0.)},->,line width=0.6pt,color=qqwuqq] (0.:1.0) arc (0.:28.154966237010104:1.0);
%\draw [line width=0.8pt] (-3.6055512754639896,0.)-- (3.371865460227506,1.2769194642281683);
%\draw [line width=0.8pt] (0.,0.)-- (3.371865460227506,1.2769194642281683);
%\draw (-1.85,0.45) node[anchor=north west] {$z$};
%\draw [line width=0.8pt] (-3.6055512754639896,0.)-- (0.,0.);
%\draw [line width=0.8pt] (0.,0.)-- (3.6055512754639896,0.);
%\begin{scriptsize}
\draw [fill=qqqqff] (0.,0.) circle (1.2pt);
\draw[color=qqqqff] (-0.24,-0.28) node {$P$};
\draw [fill=qqqqff] (2.,3.) circle (1.2pt);
\draw[color=qqqqff] (2.25,3.3) node {$B$};
\draw [fill=uuuuuu] (-3.6055512754639896,0.) circle (1.2pt);
\draw[color=qqqqff] (-3.9,0) node {$A$};
\draw [fill=xdxdff] (2.991121161538481,-2.0132546279585926) circle (1.2pt);
\draw[color=qqqqff] (3.25,-2.1) node {$D$};
%\draw [fill=uuuuuu] (3.6055512754639896,0.) circle (1.2pt);
%\draw[color=qqqqff] (3.90,0) node {$C$};
%\draw [fill=uuuuuu] (3.371865460227506,1.2769194642281683) circle (1.2pt);
%\draw[color=qqqqff] (3.63,1.53) node {$E$};
%\end{scriptsize}

\begin{scope}[shift={(9,0)}]
%\draw [shift={(-3.6055512754639896,0.)},line width=0.4pt,color=qqwuqq,fill=qqwuqq,fill opacity=0.10000000149011612] (0,0) -- (0.:1.0) arc (0.:28.154966237010104:1.0) -- cycle;
\draw [line width=0.8pt] (0.,0.) circle (3.6055512754639896cm);
\draw [line width=0.8pt] (-3.6055512754639896,0.)-- (2.,3.);
%\draw [line width=0.8pt] (-3.6055512754639896,0.)-- (2.991121161538481,-2.0132546279585926);
\draw [line width=0.8pt] (0.,0.)-- (2.,3.);
%\draw [line width=0.8pt] (0.,0.)-- (2.991121161538481,-2.0132546279585926);
%\draw (-2.5,0.6) node[anchor=north west] {$x$};
%\draw (-2.2, 0.05) node[anchor=north west] {$y$};
%\draw [shift={(-3.6055512754639896,0.)},->,line width=0.6pt,color=qqwuqq] (0.:1.0) arc (0.:28.154966237010104:1.0);
\draw [line width=0.8pt] (-3.6055512754639896,0.)-- (3.371865460227506,1.2769194642281683);
\draw [line width=0.8pt] (0.,0.)-- (3.371865460227506,1.2769194642281683);
%\draw (-1.85,0.45) node[anchor=north west] {$z$};
%\draw [line width=0.8pt] (-3.6055512754639896,0.)-- (0.,0.);
%\draw [line width=0.8pt] (0.,0.)-- (3.6055512754639896,0.);
%\begin{scriptsize}
\draw [fill=qqqqff] (0.,0.) circle (1.2pt);
\draw[color=qqqqff] (-0.24,-0.28) node {$P$};
\draw [fill=qqqqff] (2.,3.) circle (1.2pt);
\draw[color=qqqqff] (2.25,3.3) node {$B$};
\draw [fill=uuuuuu] (-3.6055512754639896,0.) circle (1.2pt);
\draw[color=qqqqff] (-3.9,0) node {$A$};
%\draw [fill=xdxdff] (2.991121161538481,-2.0132546279585926) circle (1.2pt);
%\draw[color=qqqqff] (3.25,-2.1) node {$D$};
%\draw [fill=uuuuuu] (3.6055512754639896,0.) circle (1.2pt);
%\draw[color=qqqqff] (3.90,0) node {$C$};
\draw [fill=uuuuuu] (3.371865460227506,1.2769194642281683) circle (1.2pt);
\draw[color=qqqqff] (3.63,1.53) node {$E$};
%\end{scriptsize}
\end{scope}

\end{tikzpicture}
\end{image}

\begin{problem}
Correct!  To prove these cases, we add segments and labels to the figures so that we can make use of the previous result: 

\begin{image}
\definecolor{qqwuqq}{rgb}{0.,0.39215686274509803,0.}
\definecolor{xdxdff}{rgb}{0.49019607843137253,0.49019607843137253,1.}
\definecolor{uuuuuu}{rgb}{0.26666666666666666,0.26666666666666666,0.26666666666666666}
\definecolor{qqqqff}{rgb}{0.,0.,1.}
\begin{tikzpicture}[scale=0.7,line cap=round,line join=round,x=1.0cm,y=1.0cm]
\clip(-4.5,-3.8) rectangle (13.5,3.8);
%\draw [shift={(-3.6055512754639896,0.)},line width=0.4pt,color=qqwuqq,fill=qqwuqq,fill opacity=0.10000000149011612] (0,0) -- (0.:1.0) arc (0.:28.154966237010104:1.0) -- cycle;
\draw [line width=0.8pt] (0.,0.) circle (3.6055512754639896cm);
\draw [line width=0.8pt] (-3.6055512754639896,0.)-- (2.,3.);
\draw [line width=0.8pt] (-3.6055512754639896,0.)-- (2.991121161538481,-2.0132546279585926);
\draw [line width=0.8pt] (0.,0.)-- (2.,3.);
\draw [line width=0.8pt] (0.,0.)-- (2.991121161538481,-2.0132546279585926);
\draw (-2.5,0.6) node[anchor=north west] {$x$};
\draw (-2.2, 0.05) node[anchor=north west] {$y$};
%\draw [shift={(-3.6055512754639896,0.)},->,line width=0.6pt,color=qqwuqq] (0.:1.0) arc (0.:28.154966237010104:1.0);
%\draw [line width=0.8pt] (-3.6055512754639896,0.)-- (3.371865460227506,1.2769194642281683);
%\draw [line width=0.8pt] (0.,0.)-- (3.371865460227506,1.2769194642281683);
%\draw (-1.85,0.45) node[anchor=north west] {$z$};
\draw [line width=0.8pt] (-3.6055512754639896,0.)-- (0.,0.);
\draw [line width=0.8pt] (0.,0.)-- (3.6055512754639896,0.);
%\begin{scriptsize}
\draw [fill=qqqqff] (0.,0.) circle (1.2pt);
\draw[color=qqqqff] (-0.24,-0.28) node {$P$};
\draw [fill=qqqqff] (2.,3.) circle (1.2pt);
\draw[color=qqqqff] (2.25,3.3) node {$B$};
\draw [fill=uuuuuu] (-3.6055512754639896,0.) circle (1.2pt);
\draw[color=qqqqff] (-3.9,0) node {$A$};
\draw [fill=xdxdff] (2.991121161538481,-2.0132546279585926) circle (1.2pt);
\draw[color=qqqqff] (3.25,-2.1) node {$D$};
\draw [fill=uuuuuu] (3.6055512754639896,0.) circle (1.2pt);
\draw[color=qqqqff] (3.90,0) node {$C$};
%\draw [fill=uuuuuu] (3.371865460227506,1.2769194642281683) circle (1.2pt);
%\draw[color=qqqqff] (3.63,1.53) node {$E$};
%\end{scriptsize}

\begin{scope}[shift={(9,0)}]
\draw [shift={(-3.6055512754639896,0.)},line width=0.4pt,color=qqwuqq,fill=qqwuqq,fill opacity=0.10000000149011612] (0,0) -- (0.:1.0) arc (0.:28.154966237010104:1.0) -- cycle;
\draw [line width=0.8pt] (0.,0.) circle (3.6055512754639896cm);
\draw [line width=0.8pt] (-3.6055512754639896,0.)-- (2.,3.);
%\draw [line width=0.8pt] (-3.6055512754639896,0.)-- (2.991121161538481,-2.0132546279585926);
\draw [line width=0.8pt] (0.,0.)-- (2.,3.);
%\draw [line width=0.8pt] (0.,0.)-- (2.991121161538481,-2.0132546279585926);
\draw (-2.5,0.6) node[anchor=north west] {$x$};
%\draw (-2.2, 0.05) node[anchor=north west] {$y$};
\draw [shift={(-3.6055512754639896,0.)},->,line width=0.6pt,color=qqwuqq] (0.:1.0) arc (0.:28.154966237010104:1.0);
\draw [line width=0.8pt] (-3.6055512754639896,0.)-- (3.371865460227506,1.2769194642281683);
\draw [line width=0.8pt] (0.,0.)-- (3.371865460227506,1.2769194642281683);
\draw (-1.8,0.45) node[anchor=north west] {$z$};
\draw [line width=0.8pt] (-3.6055512754639896,0.)-- (0.,0.);
\draw [line width=0.8pt] (0.,0.)-- (3.6055512754639896,0.);
%\begin{scriptsize}
\draw [fill=qqqqff] (0.,0.) circle (1.2pt);
\draw[color=qqqqff] (-0.24,-0.28) node {$P$};
\draw [fill=qqqqff] (2.,3.) circle (1.2pt);
\draw[color=qqqqff] (2.25,3.3) node {$B$};
\draw [fill=uuuuuu] (-3.6055512754639896,0.) circle (1.2pt);
\draw[color=qqqqff] (-3.9,0) node {$A$};
%\draw [fill=xdxdff] (2.991121161538481,-2.0132546279585926) circle (1.2pt);
%\draw[color=qqqqff] (3.25,-2.1) node {$D$};
\draw [fill=uuuuuu] (3.6055512754639896,0.) circle (1.2pt);
\draw[color=qqqqff] (3.90,0) node {$C$};
\draw [fill=uuuuuu] (3.371865460227506,1.2769194642281683) circle (1.2pt);
\draw[color=qqqqff] (3.63,1.53) node {$E$};
%\end{scriptsize}
\end{scope}

\end{tikzpicture}
\end{image}

\begin{warning}
In the following sentences, there are two correct ways of naming the angles, but only one of them is accepted by Ximera. Apologies. 
\end{warning}

In both figures, $\angle BAC = \answer{x}$, so $\angle \answer[format=string]{BPC} = \answer{2x}$.  

On the left, $\angle CAD = \answer{y}$, so $\angle \answer[format=string]{CPD} = \answer{2y}$. 

On the right, $\angle CAE = \answer{z}$, so $\angle \answer[format=string]{CPE} = \answer{2z}$.  

Putting these results together:

 $\angle BPD = \angle BPC$ \wordChoice{\choice[correct]{$+$}\choice{$-$}\choice{$\times$}\choice{$\div$}}$ \angle \answer[format=string]{CPD} 
 = \answer{2x + 2y} = 2 \angle \answer[format=string]{BAD}$, and 

 $\angle BPD = \angle BPC$ \wordChoice{\choice{$+$}\choice[correct]{$-$}\choice{$\times$}\choice{$\div$}} $\angle \answer[format=string]{CPE} 
 = \answer{2x - 2z} = 2 \angle \answer[format=string]{BAE}$.  

In summary, given an inscribed angle and a corresponding central angle, we have shown, in three mutually exclusive cases, that the central angle is 
\wordChoice{\choice{equal to}\choice[correct]{twice}\choice{half}\choice{less than}} the inscribed angle.  

\end{problem}
\end{problem}
 

\end{problem}

\end{document}
