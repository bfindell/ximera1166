\documentclass[nooutcomes]{ximera}
%\documentclass[space,handout,nooutcomes]{ximera}

% For preamble materials

\graphicspath{
  {./}
  {chapter1/}
  {chapter2/}
  {chapter4/}
  {math1/}
  {math2/}
}

\usepackage{pgf,tikz}
\usepackage{mathrsfs}
\usetikzlibrary{arrows}
\pgfplotsset{compat=1.16}


\newcommand{\N}{\mathbb N}
\newcommand{\W}{\mathbb W}
\newcommand{\C}{\mathbb C}
\newcommand{\Z}{\mathbb Z}
\newcommand{\Q}{\mathbb Q}
\newcommand{\R}{\mathbb R}




\title{Set Theory Problems}
\author{Bart Snapp and Brad Findell}
\begin{document}
\begin{abstract}
Short-answer problems about sets. 
\end{abstract}
\maketitle

\begin{problem}
Given two sets $X$ and $Y$, $X\cup Y$ is the set of elements that are
\begin{multipleChoice}
\choice{in $X$ or in $Y$ (but not in both).}
\choice[correct]{in $X$ or in $Y$ (or both, as the ``or'' is inclusive).}  
\choice{in $X$ and in $Y$.}
\choice{in $X$ but not in $Y$.}
\choice{in $Y$ but not in $X$.} 
\end{multipleChoice}
\end{problem}

\begin{problem}
Given two sets $X$ and $Y$, $X\cap Y$ is the set of elements that are 
\begin{multipleChoice}
\choice{in $X$ or in $Y$ (but not in both).}
\choice{in $X$ or in $Y$ (or both, as the ``or'' is inclusive).}  
\choice[correct]{in $X$ and in $Y$.}
\choice{in $X$ but not in $Y$.}
\choice{in $Y$ but not in $X$.} 
\end{multipleChoice}
\end{problem}

\begin{problem}
Given two sets $X$ and $Y$, $X - Y$ is the set of elements that are 
\begin{multipleChoice}
\choice{in $X$ or in $Y$ (but not in both).}
\choice{in $X$ or in $Y$ (or both, as the ``or'' is inclusive).}  
\choice{in $X$ and in $Y$.}
\choice[correct]{in $X$ but not in $Y$.}
\choice{in $Y$ but not in $X$.} 
\end{multipleChoice}
\end{problem}

\begin{problem}
Explain the difference between the symbols $\in$ and $\subset$.
\begin{freeResponse}
\begin{hint}
The notation $X \in Y$ means that $X$ is a single element in the set $Y$.  In this case, $X$ is probably not a set.  The notation $X \subset Y$ requires that both $X$ and $Y$ are sets and, furthermore, that every element of $X$ is also in $Y$.
\end{hint}
\end{freeResponse}
\end{problem}

\begin{problem}
How is $\{\emptyset\}$ different from $\emptyset$?  
\begin{freeResponse}
\begin{hint}
The empty set, $\emptyset$, is a set that contains no elements.  That is, $\emptyset = \{\}$.  The set $\{\emptyset\}$ contains one element that is itself a set---and that element happens to be the empty set.  We could instead write $\{\{\}\}$, but that looks ugly.
\end{hint}
\end{freeResponse}
\end{problem}

\begin{problem}
Draw a Venn diagram for the set of elements that are in $X$ or $Y$ \emph{but not both}. 
How does it differ from the Venn diagram for $X\cup Y$?  
\begin{freeResponse}
\begin{hint}
Same as the Venn diagram for $X\cup Y$, except that the $X\cap Y$ part is not shaded.  
\end{hint}
\end{freeResponse}
\end{problem}

\begin{problem}
If we let $X$ be the set of ``right triangles'' and we let $Y$ be the set of ``equilateral triangles'' does the picture below show the relationship between these two sets?
\begin{image}
  \includegraphics{set4.png}
\end{image}
\begin{multipleChoice}
\choice[correct]{Yes.}
\choice{No.}
\choice{Not enough information.}
\end{multipleChoice}

Explain your reasoning.
\begin{freeResponse}
\begin{hint}
Yes.  The picture is accurate because no right triangles are also equilateral triangles.  
\end{hint}
\end{freeResponse}
\end{problem}

\begin{problem}
If $X = \{1,2,3,4,5\}$ and $Y = \{3,4,5,6\}$ find the following: (List elements in ascending order, separated by commas, with no spaces.)
\begin{enumerate}
\item $X\cup Y = \{\answer[format=string]{1,2,3,4,5,6}\}$
\item $X\cap Y = \{\answer[format=string]{3,4,5}\}$
\item $X-Y = \{\answer[format=string]{1,2}\}$
\item $Y-X = \{\answer[format=string]{6}\}$
\end{enumerate}
\end{problem}

\begin{problem}
Let $n\Z$ represent the integer multiples of $n$. So for example:
\[
3\Z = \{\dots,-12,-9,-6,-3,0,3,6,9,12,\dots\}
\]
Compute the following (use capital $Z$ for $\Z$):
\begin{enumerate}
\item $3\Z\cap 4\Z = \answer{12Z}$ 
\item $2\Z\cap 5\Z = \answer{10Z}$
\item $3\Z\cap 6\Z = \answer{6Z}$
\item $4\Z\cap 6\Z = \answer{12Z}$
\item $4\Z\cap 10\Z = \answer{20Z}$
\end{enumerate}
\end{problem}

\begin{problem}
Make a general rule for intersecting sets of the form $n\Z$ and
  $m\Z$. Explain why your rule works.
\begin{freeResponse}
\begin{hint}
The intersection of two sets is what they have in \emph{common}.  The intersection of the set of multiples of $n$ and the set of multiples of $m$ are called \emph{common multiples} (surprise!), and they are all multiples of the least common multiple of $n$ and $m$.  
\end{hint}
\end{freeResponse}
\end{problem}

\begin{problem}
If $X\cup Y = X$, what can we say about the relationship between the sets $X$ and $Y$? Explain your reasoning.

% $Y\subset X$ because every element of $Y$ must already be in $X$.  
\wordChoice{\choice{$X\subset Y$}\choice{$X=Y$}\choice[correct]{$Y\subset X$}\choice{$X=\emptyset$}}
because every element of \wordChoice{\choice{$X$}\choice[correct]{$Y$}} must be in \wordChoice{\choice[correct]{$X$}\choice{$Y$}}.

\end{problem}

\begin{problem}
If $X\cap Y = X$, what can we say about the relationship between the sets $X$ and $Y$? Explain your reasoning.

% $X\subset Y$ because every element of $X$ must already be in $Y$. 
\wordChoice{\choice[correct]{$X\subset Y$}\choice{$X=Y$}\choice{$Y\subset X$}\choice{$X=\emptyset$}}
because every element of \wordChoice{\choice[correct]{$X$}\choice{$Y$}} must be in \wordChoice{\choice{$X$}\choice[correct]{$Y$}}.
\end{problem}

\begin{problem}
If $X-Y =\emptyset$, what can we say about the relationship between the sets $X$ and $Y$? Explain your reasoning.

% $X\subset Y$ because that would mean $X$ contains no elements that are not also in $Y$.
\wordChoice{\choice[correct]{$X\subset Y$}\choice{$X=Y$}\choice{$Y\subset X$}\choice{$X=\emptyset$}}
because every element of \wordChoice{\choice[correct]{$X$}\choice{$Y$}} must be in \wordChoice{\choice{$X$}\choice[correct]{$Y$}}.
\end{problem}



\end{document}
