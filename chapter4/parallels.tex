\documentclass[nooutcomes]{ximera}
%\documentclass[space,handout,nooutcomes]{ximera}

% For preamble materials

\graphicspath{
  {./}
  {chapter1/}
  {chapter2/}
  {chapter4/}
  {math1/}
  {math2/}
}

\usepackage{pgf,tikz}
\usepackage{mathrsfs}
\usetikzlibrary{arrows}
\pgfplotsset{compat=1.16}


\newcommand{\N}{\mathbb N}
\newcommand{\W}{\mathbb W}
\newcommand{\C}{\mathbb C}
\newcommand{\Z}{\mathbb Z}
\newcommand{\Q}{\mathbb Q}
\newcommand{\R}{\mathbb R}




%\usepackage{tikz}


\title{Transformations}
\author{Bart Snapp and Brad Findell}
\begin{document}
\begin{abstract}
Short-answer problems about transformations. 
\end{abstract}
\maketitle



%\begin{question}
%What is required to specify a translation?  
%\begin{freeResponse}
%\begin{hint}
%A vector.  Or (equivalently) a magnitude and a direction.  
%\end{hint}
%\end{freeResponse}
%\end{question}

%\begin{question}
%What is required to specify a rotation? 
%\begin{freeResponse}
%\begin{hint}
%A center and an angle (assuming an agreement about the direction of rotation).  
%\end{hint}
%\end{freeResponse}
%\end{question}

%\begin{question}
%What is required to specify a reflection?  
%\begin{freeResponse}
%\begin{hint}
%A line.  
%\end{hint}
%\end{freeResponse}
%\end{question}

\begin{question}
Use adjacent angles to prove that vertical angles are equal.    
\end{question}

\begin{question}
Now use rotations to prove that vertical angles are equal.

\end{question}

\begin{question}
Prove that alternate interior angles and corresponding angles of a transversal with respect to a pair of parallel lines are equal.
\end{question}

\begin{question}
Prove that the sum of the interior angles of a triangle is $180^\circ$.
\end{question}

\begin{question}
Prove: If a pair of alternate interior angles or a pair of corresponding angles of a transversal with respect to two lines are equal, then the lines are parallel.
\begin{freeResponse}
\begin{hint}
\end{hint}
\end{freeResponse}
\end{question}

\begin{question}
Prove: If two parallel lines are cut by a transversal, then alternate interior angles are congruent.  
\begin{freeResponse}
\begin{hint}
Let the two parallel lines be $\ell$ and $k$.  And let $P$ and $Q$ be the intersections between the transversal and $\ell$ and $k$, respectively.  Let $M$ be the midpoint of \overline{PQ}.  Rotate $\ell$ and the transversal $180^\circ$ about $M$.  $R(Q)=P$.  We proved previously that such a rotation will map $\ell$ onto a parallel line.  That lines contains $P$.  And because the parallel to $k$ through $P$ is unique, $R(k)=\ell$.  Because 
\end{hint}
\end{freeResponse}
\end{question}


\end{document}
