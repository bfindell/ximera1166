\documentclass[nooutcomes]{ximera}
%\documentclass[space,handout,nooutcomes]{ximera}

% Section 2.3.  Trickier Constructions

% For preamble materials

\usepackage{pgf,tikz}
\usepackage{mathrsfs}
\usetikzlibrary{arrows}
\usepackage{framed}
\usepackage{amsmath}
\pgfplotsset{compat=1.17}

\def\fixnote#1{\begin{framed}{\textcolor{red}{Fix note: #1}}\end{framed}}  % Allows insertion of red notes about needed edits
%\def\fixnote#1{}

\def\detail#1{{\textcolor{blue}{Detail: #1}}}   

\pdfOnly{\renewenvironment{image}[1][]{\begin{center}}{\end{center}}}

\graphicspath{
  {./}
  {chapter1/}
  {chapter2/}
  {chapter4/}
  {proofs/}
  {graphics/}
  {../graphics/}
}

\newenvironment{sectionOutcomes}{}{}


%%% This set of code is all of our user defined commands
\newcommand{\bysame}{\mbox{\rule{3em}{.4pt}}\,}
\newcommand{\N}{\mathbb N}
\newcommand{\C}{\mathbb C}
\newcommand{\W}{\mathbb W}
\newcommand{\Z}{\mathbb Z}
\newcommand{\Q}{\mathbb Q}
\newcommand{\R}{\mathbb R}
\newcommand{\A}{\mathbb A}
\newcommand{\D}{\mathcal D}
\newcommand{\F}{\mathcal F}
\newcommand{\ph}{\varphi}
\newcommand{\ep}{\varepsilon}
\newcommand{\aph}{\alpha}
\newcommand{\QM}{\begin{center}{\huge\textbf{?}}\end{center}}

\renewcommand{\le}{\leqslant}
\renewcommand{\ge}{\geqslant}
\renewcommand{\a}{\wedge}
\renewcommand{\v}{\vee}
\renewcommand{\l}{\ell}
\newcommand{\mat}{\mathsf}
\renewcommand{\vec}{\mathbf}
\renewcommand{\subset}{\subseteq}
\renewcommand{\supset}{\supseteq}
%\renewcommand{\emptyset}{\varnothing}
%\newcommand{\xto}{\xrightarrow}
%\renewcommand{\qedsymbol}{$\blacksquare$}
%\newcommand{\bibname}{References and Further Reading}
%\renewcommand{\bar}{\protect\overline}
%\renewcommand{\hat}{\protect\widehat}
%\renewcommand{\tilde}{\widetilde}
%\newcommand{\tri}{\triangle}
%\newcommand{\minipad}{\vspace{1ex}}
%\newcommand{\leftexp}[2]{{\vphantom{#2}}^{#1}{#2}}

%% More user defined commands
\renewcommand{\epsilon}{\varepsilon}
\renewcommand{\theta}{\vartheta} %% only for kmath
\renewcommand{\l}{\ell}
\renewcommand{\d}{\, d}
\newcommand{\ddx}{\frac{d}{dx}}
\newcommand{\dydx}{\frac{dy}{dx}}


\usepackage{bigstrut}


\title{Trickier Constuctions}
\author{Bart Snapp and Brad Findell}
\begin{document}
\begin{abstract}
Short-answer questions about tricky constructions. 
\end{abstract}
\maketitle


%%% In Polya's book ``Mathematical Discovery'' he has some interesting
%%% notation for problems like the ones we give below. It may be good to
%%% use something like that in the future.

\begin{problem}
Construct a square. Explain the steps in your construction.
\end{problem}

\begin{problem}
Construct a regular hexagon. Explain the steps in your construction.
\end{problem}

\begin{problem}
Your friend Margy is building a clock. She needs to know how to align
the twelve numbers on her clock so that they are equally spaced on a
circle. Explain how to use a compass and straightedge construction to
help her out. Illustrate your answer with a construction and explain
the steps in your construction.
\end{problem}

\begin{problem}
Construct a triangle given two sides of a triangle and the angle
  between them. Explain the steps in your construction.
\end{problem}

\begin{problem}
State the SAS Theorem.
\end{problem}

\begin{problem}
Construct a triangle given three sides of a triangle. Explain
  the steps in your construction.
\end{problem}

\begin{problem}
State the SSS Theorem.
\end{problem}

\begin{problem}
Construct a triangle given a side and two angles where one of
  the angles does not touch the given side. Explain the steps in your
  construction.
\end{problem}

\begin{problem}
State the SAA Theorem.
\end{problem}

\begin{problem}
Construct a triangle given a side between two given
  angles. Explain the steps in your construction.
\end{problem}

\begin{problem}
State the ASA Theorem.
\end{problem}

\begin{problem}
Explain why when given an isosceles triangle, that two
  of its angles have equal measure. Hint: Use the SAS Theorem.
\end{problem}

\begin{problem}
Construct a figure showing that a triangle cannot always be
  uniquely determined when given an angle, a side adjacent to that
  angle, and the side opposite the angle. Explain the steps in your
  construction and explain how your figure shows what is
  desired. Explain what this says about the possibility of a SSA
  theorem.  Hint: Draw many pictures to help yourself out.
\end{problem}

\begin{problem}
Give a construction showing that a triangle is uniquely
  determined if you are given a right-angle, a side touching that
  angle, and another side not touching the angle. Explain the steps in
  your construction and explain how your figure shows what is desired.
\end{problem}

\begin{problem}
Construct a triangle given two adjacent sides of a triangle and
  a median to one of the given sides. Explain the steps in your
  construction.
\end{problem}

\begin{problem}
Construct a triangle given two sides and the altitude to the
  third side. Explain the steps in your construction.
\end{problem}

\begin{problem}
Construct a triangle given a side, the median to the side, and
  the angle opposite to the side. Explain the steps in your
  construction.
\end{problem}

%\item Construct a triangle given two altitudes and an angle touching
%  one of them. Explain the steps in your construction. %% TANGENTS NEEDED

\begin{problem}
Construct a triangle given an altitude, and two angles not
  touching the altitude. Explain the steps in your construction.
\end{problem}

\begin{problem}
Construct a triangle given the length of one side, the length of
  the the median to that side, and the length of the altitude of the
  opposite angle. Explain the steps in your construction.
\end{problem}

\begin{problem}
Construct a triangle, given one angle, the length of an adjacent
  side and the altitude to that side. Explain the steps in your
  construction.
\end{problem}

\begin{problem}
Construct a circle with a given radius tangent to two other
  given circles. Explain the steps in your construction.
\end{problem}

\begin{problem}
Does a given angle and a given opposite side uniquely determine
  a triangle? Explain your answer.
\end{problem}

\begin{problem}
You are on the bank of a river. There is a tree directly in
  front of you on the other side of the river. Directly left of you is
  a friend a known distance away. Your friend knows the angle starting
  with them, going to the tree, and ending with you. How wide is the
  river? Explain your work.
\end{problem}

\begin{problem}
You are on a boat at night. You can see three lighthouses, and
  you know their position on a map.  Also you know the angles of the
  light rays from the lighthouses.  How do you figure out where you
  are? Explain your work.
\end{problem}

\begin{problem}
Construct a triangle given an angle, the length of a side
  adjacent to the given angle, and the length of the angle's bisector
  to the opposite side. Explain the steps in your construction.
\end{problem}

\begin{problem}
Construct a triangle given an angle, the length of the opposite
  side, and the length of the altitude of the given angle. Explain the
  steps in your construction.
\end{problem}

\begin{problem}
Construct a triangle given one side, the length of the altitude
  of the opposite angle, and the radius of the circumcircle. Explain
  the steps in your construction.
\end{problem}

\begin{problem}
Construct a triangle given one side, the length of the altitude
  of an adjacent angle, and the radius of the circumcircle. Explain
  the steps in your construction.
\end{problem}

\begin{problem}
Construct a triangle given one side, the length of the median
  connecting that side to the opposite angle, and the radius of the
  circumcircle. Explain the steps in your construction.  
\end{problem}

\begin{problem}
Construct a triangle given one angle and the lengths of the
  altitudes to the two other angles. Explain the steps in your
  construction.
\end{problem}

\begin{problem}
Construct a circle with a given radius tangent to two given
  intersecting lines. Explain the steps in your construction.
\end{problem}

\begin{problem}
Given a circle and a line, construct another circle of a given
  radius that is tangent to both the original circle and line. Explain
  the steps in your construction.
\end{problem}

\begin{problem}
Construct a circle with three smaller circles of equal size
  inside such that each smaller circle is tangent to the other two and
  the larger outside circle. Explain the steps in your construction.
\end{problem}

\end{document}
