\documentclass[nooutcomes]{ximera}
%\documentclass[space,handout,nooutcomes]{ximera}

% For preamble materials

\usepackage{pgf,tikz}
\usepackage{mathrsfs}
\usetikzlibrary{arrows}
\usepackage{framed}
\usepackage{amsmath}
\pgfplotsset{compat=1.17}

\def\fixnote#1{\begin{framed}{\textcolor{red}{Fix note: #1}}\end{framed}}  % Allows insertion of red notes about needed edits
%\def\fixnote#1{}

\def\detail#1{{\textcolor{blue}{Detail: #1}}}   

\pdfOnly{\renewenvironment{image}[1][]{\begin{center}}{\end{center}}}

\graphicspath{
  {./}
  {chapter1/}
  {chapter2/}
  {chapter4/}
  {proofs/}
  {graphics/}
  {../graphics/}
}

\newenvironment{sectionOutcomes}{}{}


%%% This set of code is all of our user defined commands
\newcommand{\bysame}{\mbox{\rule{3em}{.4pt}}\,}
\newcommand{\N}{\mathbb N}
\newcommand{\C}{\mathbb C}
\newcommand{\W}{\mathbb W}
\newcommand{\Z}{\mathbb Z}
\newcommand{\Q}{\mathbb Q}
\newcommand{\R}{\mathbb R}
\newcommand{\A}{\mathbb A}
\newcommand{\D}{\mathcal D}
\newcommand{\F}{\mathcal F}
\newcommand{\ph}{\varphi}
\newcommand{\ep}{\varepsilon}
\newcommand{\aph}{\alpha}
\newcommand{\QM}{\begin{center}{\huge\textbf{?}}\end{center}}

\renewcommand{\le}{\leqslant}
\renewcommand{\ge}{\geqslant}
\renewcommand{\a}{\wedge}
\renewcommand{\v}{\vee}
\renewcommand{\l}{\ell}
\newcommand{\mat}{\mathsf}
\renewcommand{\vec}{\mathbf}
\renewcommand{\subset}{\subseteq}
\renewcommand{\supset}{\supseteq}
%\renewcommand{\emptyset}{\varnothing}
%\newcommand{\xto}{\xrightarrow}
%\renewcommand{\qedsymbol}{$\blacksquare$}
%\newcommand{\bibname}{References and Further Reading}
%\renewcommand{\bar}{\protect\overline}
%\renewcommand{\hat}{\protect\widehat}
%\renewcommand{\tilde}{\widetilde}
%\newcommand{\tri}{\triangle}
%\newcommand{\minipad}{\vspace{1ex}}
%\newcommand{\leftexp}[2]{{\vphantom{#2}}^{#1}{#2}}

%% More user defined commands
\renewcommand{\epsilon}{\varepsilon}
\renewcommand{\theta}{\vartheta} %% only for kmath
\renewcommand{\l}{\ell}
\renewcommand{\d}{\, d}
\newcommand{\ddx}{\frac{d}{dx}}
\newcommand{\dydx}{\frac{dy}{dx}}


\usepackage{bigstrut}


\title{Isosceles Triangle Theorem}
\author{Brad Findell}
\begin{document}
\begin{abstract}
Proofs updated. 
\end{abstract}
\maketitle

% How to translate part of a TikZ image
%
%\begin{scope}[shift={(2,0)}]
%  ... insert graphic here
%\end{scope}
%

\textbf{Important Notes:}
\begin{itemize}
\item Below are several different proofs, along with some that are not proofs.  Please consider them separately.
\item Most of the proofs could be extended to conclude that, in the case of an isosceles triangle, the perpendicular bisector, angle bisector, median, and altitude all lie on the same line.  But for now, let's imagine we don't know this yet.
\end{itemize}

\begin{problem}
Prove that the base angles of an isosceles triangle are congruent.   
\begin{image}
\definecolor{qqqqff}{rgb}{0.,0.,1.}
\definecolor{qqwuqq}{rgb}{0.,0.39215,0.}
\begin{tikzpicture}[line width=0.8pt,line cap=round,line join=round,>=triangle 45,x=1.0cm,y=1.0cm]
%\clip(-0.4,-0.4) rectangle (2.4,2.6);
%\clip(-0.4,-0.4) rectangle (5.4,2.6);
%\draw [shift={(1.,2.2)},color=qqwuqq,fill=qqwuqq,fill opacity=0.1] (0,0) -- (-114.444:0.404) arc (-114.444:-65.556:0.404) -- cycle;  % Mark angle C
%\draw [shift={(1.,2.2)},color=qqwuqq,fill=qqwuqq,fill opacity=0.1] (0,0) -- (-114.444:0.55) arc (-114.444:-90.:0.55) -- cycle;  % Mark angle DCA
%\draw [shift={(1.,2.2)},color=qqwuqq,fill=qqwuqq,fill opacity=0.1] (0,0) -- (-90.:0.48) arc (-90.:-65.556:0.48) -- cycle;  % Mark angle DCB
%\draw[line width=0.8pt,color=qqwuqq,fill=qqwuqq,fill opacity=0.1] (1.,0.2) -- (0.8,0.2) -- (0.8,0.) -- (1.,0.) -- cycle; 
%\draw[line width=0.8pt,color=qqwuqq,fill=qqwuqq,fill opacity=0.1] (1.2,0.) -- (1.2,0.2) -- (1.,0.2) -- (1.,0.) -- cycle; 
\draw (0.,0.)-- (1.,2.2)-- (2.,0.)-- cycle;
\draw (0.4171,1.1376) -- (0.5828,1.0623);
\draw (1.5828,1.1376) -- (1.4171,1.0623);
%\draw (1.,2.2)-- (1.,0.); % segment CD
%\draw (1.0910,1.1354) -- (0.9090,1.1354);  %  Mark on CD
%\draw (1.0910,1.0646) -- (0.9090,1.0646);  %  Mark on CD
%\draw (0.4292,0.0910) -- (0.4292,-0.0910); % Mark on AD
%\draw (0.5,0.0910) -- (0.5,-0.0910);       % Mark on AD
%\draw (0.5708,0.0910) -- (0.5708,-0.0910); % Mark on AD
%\draw (1.4292,0.0910) -- (1.4292,-0.0910); % Mark on DB
%\draw (1.5,0.0910) -- (1.5,-0.0910);       % Mark on DB
%\draw (1.5708,0.0910) -- (1.5708,-0.0910); % Mark on DB
\begin{scriptsize}
\draw [fill=qqqqff] (0.,0.) circle (1.2pt);
\draw[color=qqqqff] (-0.18,-0.13) node {$A$};
\draw [fill=qqqqff] (2.,0.) circle (1.2pt);
\draw[color=qqqqff] (2.18,-0.13) node {$B$};
\draw [fill=qqqqff] (1.,2.2) circle (1.2pt);
\draw[color=qqqqff] (1.14,2.45) node {$C$};
%\draw [fill=qqqqff] (1.,0.) circle (1.2pt);
%\draw[color=qqqqff] (0.97,-0.2) node {$D$};
\draw (1,-0.6) node[align=center] {Given figure};
\end{scriptsize}
%\end{tikzpicture}

\begin{scope}[shift={(4,0)}]
% Isosceles Triangle with Angle Bisector
%\definecolor{qqqqff}{rgb}{0.,0.,1.}
%\definecolor{qqwuqq}{rgb}{0.,0.39215,0.}
%\begin{tikzpicture}[line width=0.8pt,line cap=round,line join=round,>=triangle 45,x=1.0cm,y=1.0cm]
%\clip(-0.4,-0.4) rectangle (2.4,2.6);
%\draw [shift={(1.,2.2)},color=qqwuqq,fill=qqwuqq,fill opacity=0.1] (0,0) -- (-114.444:0.404) arc (-114.444:-65.556:0.404) -- cycle;  % Mark angle C
\draw [shift={(1.,2.2)},color=qqwuqq,fill=qqwuqq,fill opacity=0.1] (0,0) -- (-114.444:0.55) arc (-114.444:-90.:0.55) -- cycle;  % Mark angle DCA
\draw [shift={(1.,2.2)},color=qqwuqq,fill=qqwuqq,fill opacity=0.1] (0,0) -- (-90.:0.48) arc (-90.:-65.556:0.48) -- cycle;  % Mark angle DCB
%\draw[line width=0.8pt,color=qqwuqq,fill=qqwuqq,fill opacity=0.1] (1.,0.2) -- (0.8,0.2) -- (0.8,0.) -- (1.,0.) -- cycle; 
%\draw[line width=0.8pt,color=qqwuqq,fill=qqwuqq,fill opacity=0.1] (1.2,0.) -- (1.2,0.2) -- (1.,0.2) -- (1.,0.) -- cycle; 
\draw (0.,0.)-- (1.,2.2)-- (2.,0.)-- cycle;
\draw (0.4171,1.1376) -- (0.5828,1.0623);
\draw (1.5828,1.1376) -- (1.4171,1.0623);
\draw (1.,2.2)-- (1.,0.); % segment CD
\draw (1.0910,1.1354) -- (0.9090,1.1354);  %  Mark on CD
\draw (1.0910,1.0646) -- (0.9090,1.0646);  %  Mark on CD
%\draw (0.4292,0.0910) -- (0.4292,-0.0910); % Mark on AD
%\draw (0.5,0.0910) -- (0.5,-0.0910);       % Mark on AD
%\draw (0.5708,0.0910) -- (0.5708,-0.0910); % Mark on AD
%\draw (1.4292,0.0910) -- (1.4292,-0.0910); % Mark on DB
%\draw (1.5,0.0910) -- (1.5,-0.0910);       % Mark on DB
%\draw (1.5708,0.0910) -- (1.5708,-0.0910); % Mark on DB
\begin{scriptsize}
\draw [fill=qqqqff] (0.,0.) circle (1.2pt);
\draw[color=qqqqff] (-0.18,-0.13) node {$A$};
\draw [fill=qqqqff] (2.,0.) circle (1.2pt);
\draw[color=qqqqff] (2.18,-0.13) node {$B$};
\draw [fill=qqqqff] (1.,2.2) circle (1.2pt);
\draw[color=qqqqff] (1.14,2.45) node {$C$};
\draw [fill=qqqqff] (1.,0.) circle (1.2pt);
\draw[color=qqqqff] (0.97,-0.2) node {$D$};
\draw (1,-0.6) node[align=center] {Morgan's figure};
\end{scriptsize}
\end{scope}
\end{tikzpicture}
\end{image}

\begin{enumerate}
\item Beginning with the given figure on the left, Morgan draws $\overline{CD}$ and marks the figure intending that this new segment is a(n) \wordChoice{\choice{median}\choice[correct]{angle bisector}\choice{perpendicular bisector}\choice{altitude}}.

\item Based on the marked figure, Morgan claims that the $\triangle ACD\cong \triangle\answer[format=string]{BCD}$ by \wordChoice{\choice[correct]{SAS}\choice{SSS}\choice{SSA}\choice{ASA}\choice{HL}}. 

\item Finally, Morgan concludes that $\angle A\cong \angle\answer[format=string]{B}$, as they are corresponding parts of congruent triangles. 
\end{enumerate}
\end{problem}


\begin{problem}
Prove that the base angles of an isosceles triangle are congruent.   
\begin{image}
\definecolor{qqqqff}{rgb}{0.,0.,1.}
\definecolor{qqwuqq}{rgb}{0.,0.3921,0.}
\begin{tikzpicture}[line width=0.8pt,line cap=round,line join=round,>=triangle 45,x=1.0cm,y=1.0cm]
%\clip(-0.4,-0.4) rectangle (2.4,2.6);
%\clip(-0.4,-0.4) rectangle (5.4,2.6);
%\draw [shift={(1.,2.2)},color=qqwuqq,fill=qqwuqq,fill opacity=0.1] (0,0) -- (-114.444:0.404) arc (-114.444:-65.556:0.404) -- cycle;  % Mark angle C
%\draw [shift={(1.,2.2)},color=qqwuqq,fill=qqwuqq,fill opacity=0.1] (0,0) -- (-114.444:0.55) arc (-114.444:-90.:0.55) -- cycle;  % Mark angle DCA
%\draw [shift={(1.,2.2)},color=qqwuqq,fill=qqwuqq,fill opacity=0.1] (0,0) -- (-90.:0.48) arc (-90.:-65.556:0.48) -- cycle;  % Mark angle DCB
%\draw[line width=0.8pt,color=qqwuqq,fill=qqwuqq,fill opacity=0.1] (1.,0.2) -- (0.8,0.2) -- (0.8,0.) -- (1.,0.) -- cycle; 
%\draw[line width=0.8pt,color=qqwuqq,fill=qqwuqq,fill opacity=0.1] (1.2,0.) -- (1.2,0.2) -- (1.,0.2) -- (1.,0.) -- cycle; 
\draw (0.,0.)-- (1.,2.2)-- (2.,0.)-- cycle;
\draw (0.4171,1.1376) -- (0.5828,1.0623);
\draw (1.5828,1.1376) -- (1.4171,1.0623);
%\draw (1.,2.2)-- (1.,0.); % segment CD
%\draw (1.0910,1.1354) -- (0.9090,1.1354);  %  Mark on CD
%\draw (1.0910,1.0646) -- (0.9090,1.0646);  %  Mark on CD
%\draw (0.4292,0.0910) -- (0.4292,-0.0910); % Mark on AD
%\draw (0.5,0.0910) -- (0.5,-0.0910);       % Mark on AD
%\draw (0.5708,0.0910) -- (0.5708,-0.0910); % Mark on AD
%\draw (1.4292,0.0910) -- (1.4292,-0.0910); % Mark on DB
%\draw (1.5,0.0910) -- (1.5,-0.0910);       % Mark on DB
%\draw (1.5708,0.0910) -- (1.5708,-0.0910); % Mark on DB
\begin{scriptsize}
\draw [fill=qqqqff] (0.,0.) circle (1.2pt);
\draw[color=qqqqff] (-0.18,-0.13) node {$A$};
\draw [fill=qqqqff] (2.,0.) circle (1.2pt);
\draw[color=qqqqff] (2.18,-0.13) node {$B$};
\draw [fill=qqqqff] (1.,2.2) circle (1.2pt);
\draw[color=qqqqff] (1.14,2.45) node {$C$};
%\draw [fill=qqqqff] (1.,0.) circle (1.2pt);
%\draw[color=qqqqff] (0.97,-0.2) node {$D$};
\draw (1,-0.6) node[align=center] {Given figure};
\end{scriptsize}
%\end{tikzpicture}

\begin{scope}[shift={(4,0)}]
% Isosceles triangle with median
%\definecolor{qqqqff}{rgb}{0.,0.,1.}
%\definecolor{qqwuqq}{rgb}{0.,0.39215,0.}
%\begin{tikzpicture}[line width=0.8pt,line cap=round,line join=round,>=triangle 45,x=1.0cm,y=1.0cm]
%\clip(-0.4,-0.4) rectangle (2.4,2.6);
%\draw [shift={(1.,2.2)},color=qqwuqq,fill=qqwuqq,fill opacity=0.1] (0,0) -- (-114.444:0.404) arc (-114.444:-65.556:0.404) -- cycle;  % Mark angle C
%\draw [shift={(1.,2.2)},color=qqwuqq,fill=qqwuqq,fill opacity=0.1] (0,0) -- (-114.444:0.55) arc (-114.444:-90.:0.55) -- cycle;  % Mark angle DCA
%\draw [shift={(1.,2.2)},color=qqwuqq,fill=qqwuqq,fill opacity=0.1] (0,0) -- (-90.:0.48) arc (-90.:-65.556:0.48) -- cycle;  % Mark angle DCB
%\draw[line width=0.8pt,color=qqwuqq,fill=qqwuqq,fill opacity=0.1] (1.,0.2) -- (0.8,0.2) -- (0.8,0.) -- (1.,0.) -- cycle; 
%\draw[line width=0.8pt,color=qqwuqq,fill=qqwuqq,fill opacity=0.1] (1.2,0.) -- (1.2,0.2) -- (1.,0.2) -- (1.,0.) -- cycle; 
\draw (0.,0.)-- (1.,2.2)-- (2.,0.)-- cycle;
\draw (0.4171,1.1376) -- (0.5828,1.0623);
\draw (1.5828,1.1376) -- (1.4171,1.0623);
\draw (1.,2.2)-- (1.,0.); % segment CD
\draw (1.0910,1.1354) -- (0.9090,1.1354);  %  Mark on CD
\draw (1.0910,1.0646) -- (0.9090,1.0646);  %  Mark on CD
\draw (0.4292,0.0910) -- (0.4292,-0.0910); % Mark on AD
\draw (0.5,0.0910) -- (0.5,-0.0910);       % Mark on AD
\draw (0.5708,0.0910) -- (0.5708,-0.0910); % Mark on AD
\draw (1.4292,0.0910) -- (1.4292,-0.0910); % Mark on DB
\draw (1.5,0.0910) -- (1.5,-0.0910);       % Mark on DB
\draw (1.5708,0.0910) -- (1.5708,-0.0910); % Mark on DB
\begin{scriptsize}
\draw [fill=qqqqff] (0.,0.) circle (1.2pt);
\draw[color=qqqqff] (-0.18,-0.13) node {$A$};
\draw [fill=qqqqff] (2.,0.) circle (1.2pt);
\draw[color=qqqqff] (2.18,-0.13) node {$B$};
\draw [fill=qqqqff] (1.,2.2) circle (1.2pt);
\draw[color=qqqqff] (1.14,2.45) node {$C$};
\draw [fill=qqqqff] (1.,0.) circle (1.2pt);
\draw[color=qqqqff] (0.97,-0.2) node {$D$};
\draw (1,-0.6) node[align=center] {Deja's figure};
\end{scriptsize}
\end{scope}
\end{tikzpicture}

\end{image}

\begin{enumerate}
\item Beginning with the given figure on the left, Deja draws $\overline{CD}$ and marks the figure intending that this new segment is a(n) \wordChoice{\choice[correct]{median}\choice{angle bisector}\choice{perpendicular bisector}\choice{altitude}}.

\item Based on the marked figure, Deja claims that the $\triangle ACD\cong \triangle\answer[format=string]{BCD}$ by \wordChoice{\choice{SAS}\choice[correct]{SSS}\choice{SSA}\choice{ASA}\choice{HL}}. 

\item Finally, Deja concludes that $\angle A\cong \angle\answer[format=string]{B}$, as they are corresponding parts of congruent triangles. 
\end{enumerate}
\end{problem}


\begin{problem}
Prove that the base angles of an isosceles triangle are congruent.   
\begin{image}
\definecolor{qqqqff}{rgb}{0.,0.,1.}
\definecolor{qqwuqq}{rgb}{0.,0.39215,0.}
\begin{tikzpicture}[line width=0.8pt,line cap=round,line join=round,>=triangle 45,x=1.0cm,y=1.0cm]
%\clip(-0.4,-0.4) rectangle (2.4,2.6);
%\clip(-0.4,-0.4) rectangle (5.4,2.6);
%\draw [shift={(1.,2.2)},color=qqwuqq,fill=qqwuqq,fill opacity=0.1] (0,0) -- (-114.444:0.404) arc (-114.444:-65.556:0.404) -- cycle;  % Mark angle C
%\draw [shift={(1.,2.2)},color=qqwuqq,fill=qqwuqq,fill opacity=0.1] (0,0) -- (-114.444:0.55) arc (-114.444:-90.:0.55) -- cycle;  % Mark angle DCA
%\draw [shift={(1.,2.2)},color=qqwuqq,fill=qqwuqq,fill opacity=0.1] (0,0) -- (-90.:0.48) arc (-90.:-65.556:0.48) -- cycle;  % Mark angle DCB
%\draw[line width=0.8pt,color=qqwuqq,fill=qqwuqq,fill opacity=0.1] (1.,0.2) -- (0.8,0.2) -- (0.8,0.) -- (1.,0.) -- cycle; 
%\draw[line width=0.8pt,color=qqwuqq,fill=qqwuqq,fill opacity=0.1] (1.2,0.) -- (1.2,0.2) -- (1.,0.2) -- (1.,0.) -- cycle; 
\draw (0.,0.)-- (1.,2.2)-- (2.,0.)-- cycle;
\draw (0.4171,1.1376) -- (0.5828,1.0623);
\draw (1.5828,1.1376) -- (1.4171,1.0623);
%\draw (1.,2.2)-- (1.,0.); % segment CD
%\draw (1.0910,1.1354) -- (0.9090,1.1354);  %  Mark on CD
%\draw (1.0910,1.0646) -- (0.9090,1.0646);  %  Mark on CD
%\draw (0.4292,0.0910) -- (0.4292,-0.0910); % Mark on AD
%\draw (0.5,0.0910) -- (0.5,-0.0910);       % Mark on AD
%\draw (0.5708,0.0910) -- (0.5708,-0.0910); % Mark on AD
%\draw (1.4292,0.0910) -- (1.4292,-0.0910); % Mark on DB
%\draw (1.5,0.0910) -- (1.5,-0.0910);       % Mark on DB
%\draw (1.5708,0.0910) -- (1.5708,-0.0910); % Mark on DB
\begin{scriptsize}
\draw [fill=qqqqff] (0.,0.) circle (1.2pt);
\draw[color=qqqqff] (-0.18,-0.13) node {$A$};
\draw [fill=qqqqff] (2.,0.) circle (1.2pt);
\draw[color=qqqqff] (2.18,-0.13) node {$B$};
\draw [fill=qqqqff] (1.,2.2) circle (1.2pt);
\draw[color=qqqqff] (1.14,2.45) node {$C$};
%\draw [fill=qqqqff] (1.,0.) circle (1.2pt);
%\draw[color=qqqqff] (0.97,-0.2) node {$D$};
\draw (1,-0.6) node[align=center] {Given figure};
\end{scriptsize}
%\end{tikzpicture}

\begin{scope}[shift={(4,0)}]
% Isosceles triangle with altitude
%\definecolor{qqqqff}{rgb}{0.,0.,1.}
%\definecolor{qqwuqq}{rgb}{0.,0.39215,0.}
%\begin{tikzpicture}[line width=0.8pt,line cap=round,line join=round,>=triangle 45,x=1.0cm,y=1.0cm]
%\clip(-0.4,-0.4) rectangle (2.4,2.6);
%\draw [shift={(1.,2.2)},color=qqwuqq,fill=qqwuqq,fill opacity=0.1] (0,0) -- (-114.444:0.404) arc (-114.444:-65.556:0.404) -- cycle;  % Mark angle C
%\draw [shift={(1.,2.2)},color=qqwuqq,fill=qqwuqq,fill opacity=0.1] (0,0) -- (-114.444:0.55) arc (-114.444:-90.:0.55) -- cycle;  % Mark angle DCA
%\draw [shift={(1.,2.2)},color=qqwuqq,fill=qqwuqq,fill opacity=0.1] (0,0) -- (-90.:0.48) arc (-90.:-65.556:0.48) -- cycle;  % Mark angle DCB
\draw[line width=0.8pt,color=qqwuqq,fill=qqwuqq,fill opacity=0.1] (1.,0.2) -- (0.8,0.2) -- (0.8,0.) -- (1.,0.) -- cycle; 
\draw[line width=0.8pt,color=qqwuqq,fill=qqwuqq,fill opacity=0.1] (1.2,0.) -- (1.2,0.2) -- (1.,0.2) -- (1.,0.) -- cycle; 
\draw (0.,0.)-- (1.,2.2)-- (2.,0.)-- cycle;
\draw (0.4171,1.1376) -- (0.5828,1.0623);
\draw (1.5828,1.1376) -- (1.4171,1.0623);
\draw (1.,2.2)-- (1.,0.); % segment CD
\draw (1.0910,1.1354) -- (0.9090,1.1354);  %  Mark on CD
\draw (1.0910,1.0646) -- (0.9090,1.0646);  %  Mark on CD
%\draw (0.4292,0.0910) -- (0.4292,-0.0910); % Mark on AD
%\draw (0.5,0.0910) -- (0.5,-0.0910);       % Mark on AD
%\draw (0.5708,0.0910) -- (0.5708,-0.0910); % Mark on AD
%\draw (1.4292,0.0910) -- (1.4292,-0.0910); % Mark on DB
%\draw (1.5,0.0910) -- (1.5,-0.0910);       % Mark on DB
%\draw (1.5708,0.0910) -- (1.5708,-0.0910); % Mark on DB
\begin{scriptsize}
\draw [fill=qqqqff] (0.,0.) circle (1.2pt);
\draw[color=qqqqff] (-0.18,-0.13) node {$A$};
\draw [fill=qqqqff] (2.,0.) circle (1.2pt);
\draw[color=qqqqff] (2.18,-0.13) node {$B$};
\draw [fill=qqqqff] (1.,2.2) circle (1.2pt);
\draw[color=qqqqff] (1.14,2.45) node {$C$};
\draw [fill=qqqqff] (1.,0.) circle (1.2pt);
\draw[color=qqqqff] (0.97,-0.2) node {$D$};
\draw (1,-0.6) node[align=center] {Elle's figure};
\end{scriptsize}
\end{scope}
\end{tikzpicture}
\end{image}

\begin{enumerate}
\item Beginning with the given figure on the left, Elle draws $\overline{CD}$ and marks the figure intending that this new segment is a(n) \wordChoice{\choice{median}\choice{angle bisector}\choice{perpendicular bisector}\choice[correct]{altitude}}.

\item Based on the marked figure, Elle claims that the $\triangle ACD\cong \triangle\answer[format=string]{BCD}$ by \wordChoice{\choice{SAS}\choice{SSS}\choice{SSA}\choice{ASA}\choice[correct]{HL}}. 

\item Finally, Elle concludes that $\angle A\cong \angle\answer[format=string]{B}$, as they are corresponding parts of congruent triangles. 
\end{enumerate}
\end{problem}


\begin{problem}
Simon and Taylor are trying to prove that the base angles of an isosceles triangle are congruent.   
\begin{image}
\definecolor{qqqqff}{rgb}{0.,0.,1.}
\definecolor{qqwuqq}{rgb}{0.,0.39215,0.}
\begin{tikzpicture}[line width=0.8pt,line cap=round,line join=round,>=triangle 45,x=1.0cm,y=1.0cm]
%\clip(-0.4,-0.4) rectangle (2.4,2.6);
%\clip(-0.4,-0.75) rectangle (8.4,2.6);
%\draw [shift={(1.,2.2)},color=qqwuqq,fill=qqwuqq,fill opacity=0.1] (0,0) -- (-114.444:0.404) arc (-114.444:-65.556:0.404) -- cycle;  % Mark angle C
%\draw [shift={(1.,2.2)},color=qqwuqq,fill=qqwuqq,fill opacity=0.1] (0,0) -- (-114.444:0.55) arc (-114.444:-90.:0.55) -- cycle;  % Mark angle DCA
%\draw [shift={(1.,2.2)},color=qqwuqq,fill=qqwuqq,fill opacity=0.1] (0,0) -- (-90.:0.48) arc (-90.:-65.556:0.48) -- cycle;  % Mark angle DCB
%\draw[line width=0.8pt,color=qqwuqq,fill=qqwuqq,fill opacity=0.1] (1.,0.2) -- (0.8,0.2) -- (0.8,0.) -- (1.,0.) -- cycle; 
%\draw[line width=0.8pt,color=qqwuqq,fill=qqwuqq,fill opacity=0.1] (1.2,0.) -- (1.2,0.2) -- (1.,0.2) -- (1.,0.) -- cycle; 
\draw (0.,0.)-- (1.,2.2)-- (2.,0.)-- cycle;
\draw (0.4171,1.1376) -- (0.5828,1.0623);
\draw (1.5828,1.1376) -- (1.4171,1.0623);
%\draw (1.,2.2)-- (1.,0.); % segment CD
%\draw (1.0910,1.1354) -- (0.9090,1.1354);  %  Mark on CD
%\draw (1.0910,1.0646) -- (0.9090,1.0646);  %  Mark on CD
%\draw (0.4292,0.0910) -- (0.4292,-0.0910); % Mark on AD
%\draw (0.5,0.0910) -- (0.5,-0.0910);       % Mark on AD
%\draw (0.5708,0.0910) -- (0.5708,-0.0910); % Mark on AD
%\draw (1.4292,0.0910) -- (1.4292,-0.0910); % Mark on DB
%\draw (1.5,0.0910) -- (1.5,-0.0910);       % Mark on DB
%\draw (1.5708,0.0910) -- (1.5708,-0.0910); % Mark on DB
\begin{scriptsize}
\draw [fill=qqqqff] (0.,0.) circle (1.2pt);
\draw[color=qqqqff] (-0.18,-0.13) node {$A$};
\draw [fill=qqqqff] (2.,0.) circle (1.2pt);
\draw[color=qqqqff] (2.18,-0.13) node {$B$};
\draw [fill=qqqqff] (1.,2.2) circle (1.2pt);
\draw[color=qqqqff] (1.14,2.45) node {$C$};
%\draw [fill=qqqqff] (1.,0.) circle (1.2pt);
%\draw[color=qqqqff] (0.97,-0.2) node {$D$};
\draw (1,-0.6) node[align=center] {Given figure};
\end{scriptsize}
%\end{tikzpicture}

\begin{scope}[shift={(3,0)}]
% Isosceles triangle with overspecified perpendicular bisector
%\definecolor{qqqqff}{rgb}{0.,0.,1.}
%\definecolor{qqwuqq}{rgb}{0.,0.39215,0.}
%\begin{tikzpicture}[line width=0.8pt,line cap=round,line join=round,>=triangle 45,x=1.0cm,y=1.0cm]
%\clip(-0.4,-0.4) rectangle (2.4,2.6);
%\draw [shift={(1.,2.2)},color=qqwuqq,fill=qqwuqq,fill opacity=0.1] (0,0) -- (-114.444:0.404) arc (-114.444:-65.556:0.404) -- cycle;  % Mark angle C
%\draw [shift={(1.,2.2)},color=qqwuqq,fill=qqwuqq,fill opacity=0.1] (0,0) -- (-114.444:0.55) arc (-114.444:-90.:0.55) -- cycle;  % Mark angle DCA
%\draw [shift={(1.,2.2)},color=qqwuqq,fill=qqwuqq,fill opacity=0.1] (0,0) -- (-90.:0.48) arc (-90.:-65.556:0.48) -- cycle;  % Mark angle DCB
\draw[line width=0.8pt,color=qqwuqq,fill=qqwuqq,fill opacity=0.1] (1.,0.2) -- (0.8,0.2) -- (0.8,0.) -- (1.,0.) -- cycle; 
\draw[line width=0.8pt,color=qqwuqq,fill=qqwuqq,fill opacity=0.1] (1.2,0.) -- (1.2,0.2) -- (1.,0.2) -- (1.,0.) -- cycle; 
\draw (0.,0.)-- (1.,2.2)-- (2.,0.)-- cycle;
\draw (0.4171,1.1376) -- (0.5828,1.0623);
\draw (1.5828,1.1376) -- (1.4171,1.0623);
\draw (1.,2.2)-- (1.,0.); % segment CD
\draw (1.0910,1.1354) -- (0.9090,1.1354);  %  Mark on CD
\draw (1.0910,1.0646) -- (0.9090,1.0646);  %  Mark on CD
\draw (0.4292,0.0910) -- (0.4292,-0.0910); % Mark on AD
\draw (0.5,0.0910) -- (0.5,-0.0910);       % Mark on AD
\draw (0.5708,0.0910) -- (0.5708,-0.0910); % Mark on AD
\draw (1.4292,0.0910) -- (1.4292,-0.0910); % Mark on DB
\draw (1.5,0.0910) -- (1.5,-0.0910);       % Mark on DB
\draw (1.5708,0.0910) -- (1.5708,-0.0910); % Mark on DB
\begin{scriptsize}
\draw [fill=qqqqff] (0.,0.) circle (1.2pt);
\draw[color=qqqqff] (-0.18,-0.13) node {$A$};
\draw [fill=qqqqff] (2.,0.) circle (1.2pt);
\draw[color=qqqqff] (2.18,-0.13) node {$B$};
\draw [fill=qqqqff] (1.,2.2) circle (1.2pt);
\draw[color=qqqqff] (1.14,2.45) node {$C$};
\draw [fill=qqqqff] (1.,0.) circle (1.2pt);
\draw[color=qqqqff] (0.97,-0.2) node {$D$};
\draw (1,-0.6) node[align=center] {Simon's figure};
\end{scriptsize}
\end{scope}
%\end{tikzpicture}

\begin{scope}[shift={(6,0)}]
% Isosceles triangle with a perpendicular bisector that misses vertex
%\definecolor{qqqqff}{rgb}{0.,0.,1.}
%\definecolor{qqwuqq}{rgb}{0.,0.39215,0.}
%\begin{tikzpicture}[line width=0.8pt,line cap=round,line join=round,>=triangle 45,x=1.0cm,y=1.0cm]
%\clip(-0.4,-0.4) rectangle (2.4,2.6);
%\draw [shift={(1.,2.2)},color=qqwuqq,fill=qqwuqq,fill opacity=0.1] (0,0) -- (-114.444:0.404) arc (-114.444:-65.556:0.404) -- cycle;  % Mark angle C
%\draw [shift={(1.,2.2)},color=qqwuqq,fill=qqwuqq,fill opacity=0.1] (0,0) -- (-114.444:0.55) arc (-114.444:-90.:0.55) -- cycle;  % Mark angle DCA
%\draw [shift={(1.,2.2)},color=qqwuqq,fill=qqwuqq,fill opacity=0.1] (0,0) -- (-90.:0.48) arc (-90.:-65.556:0.48) -- cycle;  % Mark angle DCB
\draw[line width=0.8pt,color=qqwuqq,fill=qqwuqq,fill opacity=0.1] (1.,0.2) -- (0.8,0.2) -- (0.8,0.) -- (1.,0.) -- cycle; 
\draw[line width=0.8pt,color=qqwuqq,fill=qqwuqq,fill opacity=0.1] (1.2,0.) -- (1.2,0.2) -- (1.,0.2) -- (1.,0.) -- cycle; 
\draw (0.,0.)-- (1.,2.2)-- (2.,0.)-- cycle;
\draw (0.4171,1.1376) -- (0.5828,1.0623);
\draw (1.5828,1.1376) -- (1.4171,1.0623);
\draw (0.95,2.4)-- (1.,0.); % segment CD
%\draw (1.0910,1.1354) -- (0.9090,1.1354);  %  Mark on CD
%\draw (1.0910,1.0646) -- (0.9090,1.0646);  %  Mark on CD
\draw (0.4292,0.0910) -- (0.4292,-0.0910); % Mark on AD
\draw (0.5,0.0910) -- (0.5,-0.0910);       % Mark on AD
\draw (0.5708,0.0910) -- (0.5708,-0.0910); % Mark on AD
\draw (1.4292,0.0910) -- (1.4292,-0.0910); % Mark on DB
\draw (1.5,0.0910) -- (1.5,-0.0910);       % Mark on DB
\draw (1.5708,0.0910) -- (1.5708,-0.0910); % Mark on DB
\begin{scriptsize}
\draw [fill=qqqqff] (0.,0.) circle (1.2pt);
\draw[color=qqqqff] (-0.18,-0.13) node {$A$};
\draw [fill=qqqqff] (2.,0.) circle (1.2pt);
\draw[color=qqqqff] (2.18,-0.13) node {$B$};
\draw [fill=qqqqff] (1.00,2.2) circle (1.2pt);
\draw[color=qqqqff] (1.24,2.45) node {$C$};
\draw [fill=qqqqff] (1.,0.) circle (1.2pt);
\draw[color=qqqqff] (0.97,-0.2) node {$D$};
\draw (1,-0.6) node[align=center] {Taylor's figure};
\end{scriptsize}
\end{scope}
\end{tikzpicture}
\end{image}

Beginning with the given figure on the left, Simon draws $\overline{CD}$ and marks the second figure intending that this new segment is a perpendicular bisector of $\overline{AB}$.

Taylor presents the third figure and asks, ``How do we know the perpendicular bisector will go through point $C$?  Maybe the figure looks like this!'' 

Without using other facts about isosceles triangles or perpendicular bisectors, choose the best judgment regarding their disagreement: 

\begin{multipleChoice}
\choice{Simon is correct, and $\triangle ACD\cong \triangle BCD$ by SAS.} 
\choice{Simon is correct, and $\triangle ACD\cong \triangle BCD$ by SSS}
\choice[correct]{Taylor is correct because a perpendicular bisector of a side of a triangle usually misses the opposite vertex.}
\choice{Neither of them are correct.}  
\end{multipleChoice} 

\begin{problem}
From Simon's figure, a claim that $\triangle ACD\cong \triangle BCD$ by SAS ignores what information?
\begin{multipleChoice}
\choice{$\angle A\cong \angle B$}
\choice{$\overline{CD}\perp \overline{AB}$}
\choice[correct]{$\overline{AC}\cong \overline{BC}$}
\choice{$\overline{AD}\cong \overline{BD}$}
\end{multipleChoice}
\begin{feedback}[correct]
\textbf{That's right!} Ignoring the given (i.e., that the triangle is isosceles), would be like trying to prove the base angles of \textbf{any} triangle are congruent.
\end{feedback}

\begin{problem}
From Simon's figure, a claim that $\triangle ACD\cong \triangle BCD$ by SSS ignores what information?
\begin{multipleChoice}
\choice{$\angle A\cong \angle B$}
\choice[correct]{$\overline{CD}\perp \overline{AB}$}
\choice{$\overline{AC}\cong \overline{BC}$}
\choice{$\overline{AD}\cong \overline{BD}$}
\end{multipleChoice}
\begin{feedback}[correct]
\textbf{Right again!} Ignoring the marking that $\overline{CD}\perp \overline{AB}$ would mean that $\overline{CD}$ is merely a median---\textbf{not} a perpendicular bisector.  Then the proof may be completed as above.  
\end{feedback}  
\end{problem}
\end{problem}
\end{problem}

\begin{problem}
Let's summarize the work so far:  When trying to prove that the base angles of an isosceles triangle are congruent, it is often useful to draw an additional line (or segment) in the figure.  

Which of the following lines can lead to a correct proof? [Select all.]
\begin{selectAll}
\choice[correct]{median}
\choice[correct]{angle bisector}
\choice{perpendicular bisector}
\choice[correct]{altitude}
\end{selectAll}
\end{problem}

Next we consider a proof that does not involve drawing a new line (or segment). 

\begin{problem}
Prove that the base angles of an isosceles triangle are congruent.   
\begin{image}
\definecolor{qqqqff}{rgb}{0.,0.,1.}
\definecolor{qqwuqq}{rgb}{0.,0.39215,0.}
\begin{tikzpicture}[line width=0.8pt,line cap=round,line join=round,>=triangle 45,x=1.0cm,y=1.0cm]
%\clip(-0.4,-0.4) rectangle (2.4,2.6);
%\clip(-0.4,-0.75) rectangle (5.4,2.6);
%\draw [shift={(1.,2.2)},color=qqwuqq,fill=qqwuqq,fill opacity=0.1] (0,0) -- (-114.444:0.404) arc (-114.444:-65.556:0.404) -- cycle;  % Mark angle C
%\draw [shift={(1.,2.2)},color=qqwuqq,fill=qqwuqq,fill opacity=0.1] (0,0) -- (-114.444:0.55) arc (-114.444:-90.:0.55) -- cycle;  % Mark angle DCA
%\draw [shift={(1.,2.2)},color=qqwuqq,fill=qqwuqq,fill opacity=0.1] (0,0) -- (-90.:0.48) arc (-90.:-65.556:0.48) -- cycle;  % Mark angle DCB
%\draw[line width=0.8pt,color=qqwuqq,fill=qqwuqq,fill opacity=0.1] (1.,0.2) -- (0.8,0.2) -- (0.8,0.) -- (1.,0.) -- cycle; 
%\draw[line width=0.8pt,color=qqwuqq,fill=qqwuqq,fill opacity=0.1] (1.2,0.) -- (1.2,0.2) -- (1.,0.2) -- (1.,0.) -- cycle; 
\draw (0.,0.)-- (1.,2.2)-- (2.,0.)-- cycle;
\draw (0.4171,1.1376) -- (0.5828,1.0623);
\draw (1.5828,1.1376) -- (1.4171,1.0623);
%\draw (1.,2.2)-- (1.,0.); % segment CD
%\draw (1.0910,1.1354) -- (0.9090,1.1354);  %  Mark on CD
%\draw (1.0910,1.0646) -- (0.9090,1.0646);  %  Mark on CD
%\draw (0.4292,0.0910) -- (0.4292,-0.0910); % Mark on AD
%\draw (0.5,0.0910) -- (0.5,-0.0910);       % Mark on AD
%\draw (0.5708,0.0910) -- (0.5708,-0.0910); % Mark on AD
%\draw (1.4292,0.0910) -- (1.4292,-0.0910); % Mark on DB
%\draw (1.5,0.0910) -- (1.5,-0.0910);       % Mark on DB
%\draw (1.5708,0.0910) -- (1.5708,-0.0910); % Mark on DB
\begin{scriptsize}
\draw [fill=qqqqff] (0.,0.) circle (1.2pt);
\draw[color=qqqqff] (-0.18,-0.13) node {$A$};
\draw [fill=qqqqff] (2.,0.) circle (1.2pt);
\draw[color=qqqqff] (2.18,-0.13) node {$B$};
\draw [fill=qqqqff] (1.,2.2) circle (1.2pt);
\draw[color=qqqqff] (1.14,2.45) node {$C$};
%\draw [fill=qqqqff] (1.,0.) circle (1.2pt);
%\draw[color=qqqqff] (0.97,-0.2) node {$D$};
\draw (1,-0.6) node[align=center] {Given figure};
\end{scriptsize}
%\end{tikzpicture}

\begin{scope}[shift={(4,0)}]
% Isosceles triangle using symmetry 
%\definecolor{qqqqff}{rgb}{0.,0.,1.}
%\definecolor{qqwuqq}{rgb}{0.,0.39215,0.}
%\begin{tikzpicture}[line width=0.8pt,line cap=round,line join=round,>=triangle 45,x=1.0cm,y=1.0cm]
%\clip(-0.4,-0.4) rectangle (2.4,2.6);
\draw [shift={(1.,2.2)},color=qqwuqq,fill=qqwuqq,fill opacity=0.1] (0,0) -- (-114.444:0.404) arc (-114.444:-65.556:0.404) -- cycle;  % Mark angle C
%\draw [shift={(1.,2.2)},color=qqwuqq,fill=qqwuqq,fill opacity=0.1] (0,0) -- (-114.444:0.55) arc (-114.444:-90.:0.55) -- cycle;  % Mark angle DCA
%\draw [shift={(1.,2.2)},color=qqwuqq,fill=qqwuqq,fill opacity=0.1] (0,0) -- (-90.:0.48) arc (-90.:-65.556:0.48) -- cycle;  % Mark angle DCB
%\draw[line width=0.8pt,color=qqwuqq,fill=qqwuqq,fill opacity=0.1] (1.,0.2) -- (0.8,0.2) -- (0.8,0.) -- (1.,0.) -- cycle; 
%\draw[line width=0.8pt,color=qqwuqq,fill=qqwuqq,fill opacity=0.1] (1.2,0.) -- (1.2,0.2) -- (1.,0.2) -- (1.,0.) -- cycle; 
\draw (0.,0.)-- (1.,2.2)-- (2.,0.)-- cycle;
\draw (0.4171,1.1376) -- (0.5828,1.0623);
\draw (1.5828,1.1376) -- (1.4171,1.0623);
%\draw (1.,2.2)-- (1.,0.); % segment CD
%\draw (1.0910,1.1354) -- (0.9090,1.1354);  %  Mark on CD
%\draw (1.0910,1.0646) -- (0.9090,1.0646);  %  Mark on CD
%\draw (0.4292,0.0910) -- (0.4292,-0.0910); % Mark on AD
%\draw (0.5,0.0910) -- (0.5,-0.0910);       % Mark on AD
%\draw (0.5708,0.0910) -- (0.5708,-0.0910); % Mark on AD
%\draw (1.4292,0.0910) -- (1.4292,-0.0910); % Mark on DB
%\draw (1.5,0.0910) -- (1.5,-0.0910);       % Mark on DB
%\draw (1.5708,0.0910) -- (1.5708,-0.0910); % Mark on DB
\begin{scriptsize}
\draw [fill=qqqqff] (0.,0.) circle (1.2pt);
\draw[color=qqqqff] (-0.18,-0.13) node {$A$};
\draw [fill=qqqqff] (2.,0.) circle (1.2pt);
\draw[color=qqqqff] (2.18,-0.13) node {$B$};
\draw [fill=qqqqff] (1.,2.2) circle (1.2pt);
\draw[color=qqqqff] (1.14,2.45) node {$C$};
%\draw [fill=qqqqff] (1.,0.) circle (1.2pt);
%\draw[color=qqqqff] (0.97,-0.2) node {$D$};
\draw (1,-0.6) node[align=center] {Lissy's figure};
\end{scriptsize}
\end{scope}
\end{tikzpicture}
\end{image}

\begin{enumerate}
\item Examining the given figure on the left, Lissy notices symmetry in the triangle and claims that the triangle is congruent to itself by a \wordChoice{\choice{translation}\choice[correct]{reflection}\choice{rotation}}.  

\item Based on the marked figure, Lissy claims that the $\triangle ACB\cong \triangle\answer[format=string]{BCA}$ by \wordChoice{\choice[correct]{SAS}\choice{SSS}\choice{SSA}\choice{ASA}\choice{HL}}. 

\item Finally, Lissy concludes that $\angle A\cong \angle\answer[format=string]{B}$, as they are corresponding parts of congruent triangles. 
\end{enumerate}

\end{problem}



\end{document}
