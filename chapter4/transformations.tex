\documentclass[nooutcomes]{ximera}
%\documentclass[space,handout,nooutcomes]{ximera}

% For preamble materials

\graphicspath{
  {./}
  {chapter1/}
  {chapter2/}
  {chapter4/}
  {math1/}
  {math2/}
}

\usepackage{pgf,tikz}
\usepackage{mathrsfs}
\usetikzlibrary{arrows}
\pgfplotsset{compat=1.16}


\newcommand{\N}{\mathbb N}
\newcommand{\W}{\mathbb W}
\newcommand{\C}{\mathbb C}
\newcommand{\Z}{\mathbb Z}
\newcommand{\Q}{\mathbb Q}
\newcommand{\R}{\mathbb R}




%\usepackage{tikz}


\title{Transformations}
\author{Bart Snapp and Brad Findell}
\begin{document}
\begin{abstract}
Short-answer problems about transformations. 
\end{abstract}
\maketitle


\begin{question}
To specify a translation, we need a $\answer[format=string]{vector}$.  Equivalently, we need a magnitude (or length or $\answer[format=string]{distance}$) and a $\answer[format=string]{direction}$.  
\end{question}

\begin{question}
To specify a rotation, we need a $\answer[format=string]{center}$ and an $\answer[format=string]{angle}$ (assuming an agreement about the direction of rotation).  
\end{question}

\begin{question}
To specify a reflection, we need a $\answer[format=string]{line}$.  
\end{question}

\begin{question}
A transformation that does nothing is call the $\answer[format=string]{identity transformation}$.  (Hint: Two words.)
\end{question}

Sometimes a sequence of transformations can be described as a single translation, rotation, or reflection.  

\begin{question}
What kind of transformation is a translation followed by a translation?  Explain.  Be sure to consider any special cases.  

\textbf{Answer:} Usually a $\answer[format=string]{translation}$.  
\begin{feedback}[correct]
\textbf{Correct!} The resulting translation vector is the vector sum of the two given translation vectors.  (Put the vectors head to tail.)
\end{feedback}
\begin{question}
If the vectors are opposites of each other, the result 
is the $\answer[format=string]{identity transformation}$ (two words).  
\begin{feedback}[correct]
\textbf{Correct!} The identity transformation can be thought of as a translation by a vector of magnitude $0$ (in any direction).
\end{feedback}
\end{question}
\end{question}

\begin{question}
What kind of transformation is a rotation followed by a rotation?  Explain.  Be sure to consider any special cases.   

\textbf{Answer:} Usually a $\answer[format=string]{rotation}$.  
\begin{feedback}[correct]
\textbf{Correct!} The resulting angle of rotation is the sum of the two given angles of rotation.  
\end{feedback}
\begin{question}
If the angles sum to a multiple of $360^\circ$ and the centers are different, then the result is a $\answer[format=string]{translation}$.  
\begin{question}
If the centers of rotation are the same \textbf{and} the angles sum to a multiple of $360^\circ$, the result is the $\answer[format=string]{identity transformation}$ (two words).  
\begin{feedback}[correct]
\textbf{Correct!} The identity transformation can be thought of as a rotation by an angle of $0^\circ$ (about any center).
\end{feedback}
\end{question}
\end{question}
\end{question}

\begin{question}
What kind of transformation is a reflection followed by another reflection?    

\textbf{Answer:} If the reflection lines intersect, the result is a $\answer[format=string]{rotation}$.  
\begin{feedback}[correct]
\textbf{Correct!} The $\answer{center}$ of the resulting rotation is the point of intersection of the reflection lines.  What is the angle?
\end{feedback}
\begin{question}
If the reflection lines are parallel, the result is a $\answer[format=string]{translation}$. 
\begin{feedback}[correct]
\textbf{Correct!} The resulting translation vector is $\answer{perpendicular}$ to the two reflection lines. What is the distance?
\end{feedback}
\begin{question}
If the reflection lines are the same line, the result is the $\answer[format=string]{identity transformation}$.
\begin{feedback}[correct]
\textbf{Good thinking!} Reflecting twice about the same line ``brings you back to where you started.'' 
\end{feedback}
\end{question}
\end{question}
\end{question}

\begin{question}
Will the letter F look like an F after a reflection?  What about after a sequence of two reflections?  What about after a sequence of 73 or 124 reflections?  Explain your reasoning.  
\begin{freeResponse}
\begin{hint}
Ignoring which direction is up, after a reflection the F will look like a ``backwards F''.  More generally, after an odd number of reflections, the F will look like a backwards F.  After an even number of reflections, the F will look like a typical F.  
\end{hint}
\end{freeResponse}
\end{question}

\begin{question}
How will your answer to the previous problem change if you use a capital D?  Explain.  
\begin{freeResponse}
\begin{hint}
Ignoring which side is up, the D will always look like a D.  Because of its line symmetry, a reflection doesn't appear to reverse its ``orientation.''  
\end{hint}
\end{freeResponse}
\end{question}

\begin{question}
Given a figure and its image after a translation, how do find the direction and distance of the translation?    How many points and images do you need?  
\begin{freeResponse}
\begin{hint}
Draw a vector from any point to its image.  The vector provides both the direction and the magnitude.  Any point and its image will do.  
\end{hint}
\end{freeResponse}
\end{question}

\begin{question}
Given a figure and its image after a reflection, how do you find the line of reflection?  How many points and images do you need?  
\begin{freeResponse}
\begin{hint}
Draw a segment from a point to its image.  The perpendicular bisector of that segment is the line of reflection.  Any point and its image will do, (as long the point moves under the reflection).  
\end{hint}
\end{freeResponse}
\end{question}

\begin{question}
Given a figure and its image after a rotation, how do you find the center and the angle of the rotation?  How many points and images do you need?  
\begin{freeResponse}
\begin{hint}
Draw a segment from a point $P$ to its image $P'$.  The center of rotation is somewhere on the perpendicular bisector of that segment.  Draw a segment from a second point $Q$ to its image $Q'$.  The center of rotation is also somewhere on the the perpendicular bisector of that segment.  As long as the segments $\overline{PP'}$ and $\overline{QQ'}$ are not parallel, the two perpendicular bisectors will intersect at a point $C$, which is the unique center of the rotation.  

To find the angle of rotation, measure $\angle PCP'$ or $\angle QCQ'$.

Two points and their images are enough, (as long as the segments $\overline{PP'}$ and $\overline{QQ'}$ are not parallel). 
\end{hint}
\end{freeResponse}
\end{question}


\end{document}
