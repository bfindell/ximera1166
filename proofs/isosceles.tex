\documentclass[nooutcomes]{ximera}
%\documentclass[space,handout,nooutcomes]{ximera}

% For preamble materials

\usepackage{pgf,tikz}
\usepackage{mathrsfs}
\usetikzlibrary{arrows}
\usepackage{framed}
\usepackage{amsmath}
\pgfplotsset{compat=1.17}

\def\fixnote#1{\begin{framed}{\textcolor{red}{Fix note: #1}}\end{framed}}  % Allows insertion of red notes about needed edits
%\def\fixnote#1{}

\def\detail#1{{\textcolor{blue}{Detail: #1}}}   

\pdfOnly{\renewenvironment{image}[1][]{\begin{center}}{\end{center}}}

\graphicspath{
  {./}
  {chapter1/}
  {chapter2/}
  {chapter4/}
  {proofs/}
  {graphics/}
  {../graphics/}
}

\newenvironment{sectionOutcomes}{}{}


%%% This set of code is all of our user defined commands
\newcommand{\bysame}{\mbox{\rule{3em}{.4pt}}\,}
\newcommand{\N}{\mathbb N}
\newcommand{\C}{\mathbb C}
\newcommand{\W}{\mathbb W}
\newcommand{\Z}{\mathbb Z}
\newcommand{\Q}{\mathbb Q}
\newcommand{\R}{\mathbb R}
\newcommand{\A}{\mathbb A}
\newcommand{\D}{\mathcal D}
\newcommand{\F}{\mathcal F}
\newcommand{\ph}{\varphi}
\newcommand{\ep}{\varepsilon}
\newcommand{\aph}{\alpha}
\newcommand{\QM}{\begin{center}{\huge\textbf{?}}\end{center}}

\renewcommand{\le}{\leqslant}
\renewcommand{\ge}{\geqslant}
\renewcommand{\a}{\wedge}
\renewcommand{\v}{\vee}
\renewcommand{\l}{\ell}
\newcommand{\mat}{\mathsf}
\renewcommand{\vec}{\mathbf}
\renewcommand{\subset}{\subseteq}
\renewcommand{\supset}{\supseteq}
%\renewcommand{\emptyset}{\varnothing}
%\newcommand{\xto}{\xrightarrow}
%\renewcommand{\qedsymbol}{$\blacksquare$}
%\newcommand{\bibname}{References and Further Reading}
%\renewcommand{\bar}{\protect\overline}
%\renewcommand{\hat}{\protect\widehat}
%\renewcommand{\tilde}{\widetilde}
%\newcommand{\tri}{\triangle}
%\newcommand{\minipad}{\vspace{1ex}}
%\newcommand{\leftexp}[2]{{\vphantom{#2}}^{#1}{#2}}

%% More user defined commands
\renewcommand{\epsilon}{\varepsilon}
\renewcommand{\theta}{\vartheta} %% only for kmath
\renewcommand{\l}{\ell}
\renewcommand{\d}{\, d}
\newcommand{\ddx}{\frac{d}{dx}}
\newcommand{\dydx}{\frac{dy}{dx}}


\usepackage{bigstrut}


\title{The Isosceles Triangle Theorem}
\author{Brad Findell}
\begin{document}
\begin{abstract}
Proofs updated. 
\end{abstract}
\maketitle

% How to translate part of a TikZ image
%
%\begin{scope}[shift={(2,0)}]
%  ... insert graphic here
%\end{scope}
%
% This figure shows an $\answer[format=string]{angle bisector}$ 


%\begin{image}
%% Isosceles triangle ABC, marked
%\definecolor{qqqqff}{rgb}{0.,0.,1.}
%\definecolor{qqwuqq}{rgb}{0.,0.39215,0.}
%\begin{tikzpicture}[line cap=round,line join=round,>=triangle 45,x=1.0cm,y=1.0cm]
%%\clip(-0.4,-0.4) rectangle (2.4,2.6);
%\clip(-1.4,-0.4) rectangle (3.4,2.6);
%%\draw [shift={(1.,2.2)},line width=0.8pt,color=qqwuqq,fill=qqwuqq,fill opacity=0.1] (0,0) -- (-114.444:0.404) arc (-114.444:-65.556:0.404) -- cycle;  % Mark angle C
%%\draw [shift={(1.,2.2)},line width=0.8pt,color=qqwuqq,fill=qqwuqq,fill opacity=0.1] (0,0) -- (-114.444:0.55) arc (-114.444:-90.:0.55) -- cycle;  % Mark angle DCA
%%\draw [shift={(1.,2.2)},line width=0.8pt,color=qqwuqq,fill=qqwuqq,fill opacity=0.1] (0,0) -- (-90.:0.48) arc (-90.:-65.556:0.48) -- cycle;  % Mark angle DCB
%%\draw[line width=0.8pt,color=qqwuqq,fill=qqwuqq,fill opacity=0.1] (1.,0.2) -- (0.8,0.2) -- (0.8,0.) -- (1.,0.) -- cycle; 
%%\draw[line width=0.8pt,color=qqwuqq,fill=qqwuqq,fill opacity=0.1] (1.2,0.) -- (1.2,0.2) -- (1.,0.2) -- (1.,0.) -- cycle; 
%\draw [line width=0.8pt] (0.,0.)-- (1.,2.2);
%\draw [line width=0.8pt] (0.4171,1.1376) -- (0.5828,1.0623);
%\draw [line width=0.8pt] (1.,2.2)-- (2.,0.);
%\draw [line width=0.8pt] (1.5828,1.1376) -- (1.4171,1.0623);
%\draw [line width=0.8pt] (0.,0.)-- (2.,0.);
%%\draw [line width=0.8pt] (1.,2.2)-- (1.,0.); % segment CD
%%\draw [line width=0.8pt] (1.0910,1.1354) -- (0.9090,1.1354);  %  Mark on CD
%%\draw [line width=0.8pt] (1.0910,1.0646) -- (0.9090,1.0646);  %  Mark on CD
%%\draw [line width=0.8pt] (0.4292,0.0910) -- (0.4292,-0.0910); % Mark on AD
%%\draw [line width=0.8pt] (0.5,0.0910) -- (0.5,-0.0910);       % Mark on AD
%%\draw [line width=0.8pt] (0.5708,0.0910) -- (0.5708,-0.0910); % Mark on AD
%%\draw [line width=0.8pt] (1.4292,0.0910) -- (1.4292,-0.0910); % Mark on DB
%%\draw [line width=0.8pt] (1.5,0.0910) -- (1.5,-0.0910);       % Mark on DB
%%\draw [line width=0.8pt] (1.5708,0.0910) -- (1.5708,-0.0910); % Mark on DB
%\begin{scriptsize}
%\draw [fill=qqqqff] (0.,0.) circle (1.2pt);
%\draw[color=qqqqff] (-0.18,-0.13) node {$A$};
%\draw [fill=qqqqff] (2.,0.) circle (1.2pt);
%\draw[color=qqqqff] (2.18,-0.13) node {$B$};
%\draw [fill=qqqqff] (1.,2.2) circle (1.2pt);
%\draw[color=qqqqff] (1.14,2.45) node {$C$};
%%\draw [fill=qqqqff] (1.,0.) circle (1.2pt);
%%\draw[color=qqqqff] (0.97,-0.2) node {$D$};
%\end{scriptsize}
%\end{tikzpicture}
%\end{image}


\begin{problem}
Prove that the base angles of an isosceles triangle are congruent.   

\begin{image}
\definecolor{qqqqff}{rgb}{0.,0.,1.}
\definecolor{qqwuqq}{rgb}{0.,0.39215,0.}
\begin{tikzpicture}[line cap=round,line join=round,>=triangle 45,x=1.0cm,y=1.0cm]
%\clip(-0.4,-0.4) rectangle (2.4,2.6);
\clip(-0.4,-0.4) rectangle (5.4,2.6);
%\draw [shift={(1.,2.2)},line width=0.8pt,color=qqwuqq,fill=qqwuqq,fill opacity=0.1] (0,0) -- (-114.444:0.404) arc (-114.444:-65.556:0.404) -- cycle;  % Mark angle C
%\draw [shift={(1.,2.2)},line width=0.8pt,color=qqwuqq,fill=qqwuqq,fill opacity=0.1] (0,0) -- (-114.444:0.55) arc (-114.444:-90.:0.55) -- cycle;  % Mark angle DCA
%\draw [shift={(1.,2.2)},line width=0.8pt,color=qqwuqq,fill=qqwuqq,fill opacity=0.1] (0,0) -- (-90.:0.48) arc (-90.:-65.556:0.48) -- cycle;  % Mark angle DCB
%\draw[line width=0.8pt,color=qqwuqq,fill=qqwuqq,fill opacity=0.1] (1.,0.2) -- (0.8,0.2) -- (0.8,0.) -- (1.,0.) -- cycle; 
%\draw[line width=0.8pt,color=qqwuqq,fill=qqwuqq,fill opacity=0.1] (1.2,0.) -- (1.2,0.2) -- (1.,0.2) -- (1.,0.) -- cycle; 
\draw [line width=0.8pt] (0.,0.)-- (1.,2.2);
\draw [line width=0.8pt] (0.4171,1.1376) -- (0.5828,1.0623);
\draw [line width=0.8pt] (1.,2.2)-- (2.,0.);
\draw [line width=0.8pt] (1.5828,1.1376) -- (1.4171,1.0623);
\draw [line width=0.8pt] (0.,0.)-- (2.,0.);
%\draw [line width=0.8pt] (1.,2.2)-- (1.,0.); % segment CD
%\draw [line width=0.8pt] (1.0910,1.1354) -- (0.9090,1.1354);  %  Mark on CD
%\draw [line width=0.8pt] (1.0910,1.0646) -- (0.9090,1.0646);  %  Mark on CD
%\draw [line width=0.8pt] (0.4292,0.0910) -- (0.4292,-0.0910); % Mark on AD
%\draw [line width=0.8pt] (0.5,0.0910) -- (0.5,-0.0910);       % Mark on AD
%\draw [line width=0.8pt] (0.5708,0.0910) -- (0.5708,-0.0910); % Mark on AD
%\draw [line width=0.8pt] (1.4292,0.0910) -- (1.4292,-0.0910); % Mark on DB
%\draw [line width=0.8pt] (1.5,0.0910) -- (1.5,-0.0910);       % Mark on DB
%\draw [line width=0.8pt] (1.5708,0.0910) -- (1.5708,-0.0910); % Mark on DB
\begin{scriptsize}
\draw [fill=qqqqff] (0.,0.) circle (1.2pt);
\draw[color=qqqqff] (-0.18,-0.13) node {$A$};
\draw [fill=qqqqff] (2.,0.) circle (1.2pt);
\draw[color=qqqqff] (2.18,-0.13) node {$B$};
\draw [fill=qqqqff] (1.,2.2) circle (1.2pt);
\draw[color=qqqqff] (1.14,2.45) node {$C$};
%\draw [fill=qqqqff] (1.,0.) circle (1.2pt);
%\draw[color=qqqqff] (0.97,-0.2) node {$D$};
\end{scriptsize}
%\end{tikzpicture}

\begin{scope}[shift={(3,0)}]
% Isosceles Triangle with Angle Bisector
%\definecolor{qqqqff}{rgb}{0.,0.,1.}
%\definecolor{qqwuqq}{rgb}{0.,0.39215,0.}
%\begin{tikzpicture}[line cap=round,line join=round,>=triangle 45,x=1.0cm,y=1.0cm]
%\clip(-0.4,-0.4) rectangle (2.4,2.6);
%\draw [shift={(1.,2.2)},line width=0.8pt,color=qqwuqq,fill=qqwuqq,fill opacity=0.1] (0,0) -- (-114.444:0.404) arc (-114.444:-65.556:0.404) -- cycle;  % Mark angle C
\draw [shift={(1.,2.2)},line width=0.8pt,color=qqwuqq,fill=qqwuqq,fill opacity=0.1] (0,0) -- (-114.444:0.55) arc (-114.444:-90.:0.55) -- cycle;  % Mark angle DCA
\draw [shift={(1.,2.2)},line width=0.8pt,color=qqwuqq,fill=qqwuqq,fill opacity=0.1] (0,0) -- (-90.:0.48) arc (-90.:-65.556:0.48) -- cycle;  % Mark angle DCB
%\draw[line width=0.8pt,color=qqwuqq,fill=qqwuqq,fill opacity=0.1] (1.,0.2) -- (0.8,0.2) -- (0.8,0.) -- (1.,0.) -- cycle; 
%\draw[line width=0.8pt,color=qqwuqq,fill=qqwuqq,fill opacity=0.1] (1.2,0.) -- (1.2,0.2) -- (1.,0.2) -- (1.,0.) -- cycle; 
\draw [line width=0.8pt] (0.,0.)-- (1.,2.2);
\draw [line width=0.8pt] (0.4171,1.1376) -- (0.5828,1.0623);
\draw [line width=0.8pt] (1.,2.2)-- (2.,0.);
\draw [line width=0.8pt] (1.5828,1.1376) -- (1.4171,1.0623);
\draw [line width=0.8pt] (0.,0.)-- (2.,0.);
\draw [line width=0.8pt] (1.,2.2)-- (1.,0.); % segment CD
\draw [line width=0.8pt] (1.0910,1.1354) -- (0.9090,1.1354);  %  Mark on CD
\draw [line width=0.8pt] (1.0910,1.0646) -- (0.9090,1.0646);  %  Mark on CD
%\draw [line width=0.8pt] (0.4292,0.0910) -- (0.4292,-0.0910); % Mark on AD
%\draw [line width=0.8pt] (0.5,0.0910) -- (0.5,-0.0910);       % Mark on AD
%\draw [line width=0.8pt] (0.5708,0.0910) -- (0.5708,-0.0910); % Mark on AD
%\draw [line width=0.8pt] (1.4292,0.0910) -- (1.4292,-0.0910); % Mark on DB
%\draw [line width=0.8pt] (1.5,0.0910) -- (1.5,-0.0910);       % Mark on DB
%\draw [line width=0.8pt] (1.5708,0.0910) -- (1.5708,-0.0910); % Mark on DB
\begin{scriptsize}
\draw [fill=qqqqff] (0.,0.) circle (1.2pt);
\draw[color=qqqqff] (-0.18,-0.13) node {$A$};
\draw [fill=qqqqff] (2.,0.) circle (1.2pt);
\draw[color=qqqqff] (2.18,-0.13) node {$B$};
\draw [fill=qqqqff] (1.,2.2) circle (1.2pt);
\draw[color=qqqqff] (1.14,2.45) node {$C$};
\draw [fill=qqqqff] (1.,0.) circle (1.2pt);
\draw[color=qqqqff] (0.97,-0.2) node {$D$};
\end{scriptsize}
\end{scope}
\end{tikzpicture}
\end{image}

\begin{enumerate}
\item Beginning with the given figure on the left, Morgan draws $\overline{CD}$ and marks the figure intending that this new segment is a(n) \wordChoice{\choice{median}\choice[correct]{angle bisector}\choice{perpendicular bisector}\choice{altitude}}.

\item Based on the marked figure, Morgan claims that the $\triangle ACD\cong \triangle\answer[format=string]{BCD}$ by \wordChoice{\choice[correct]{SAS}\choice{SSS}\choice{SSA}\choice{ASA}\choice{HL}}. 

\item Finally, Morgan concludes that $\angle A\cong \angle\answer[format=string]{B}$, as they are corresponding parts of congruent triangles. 
\end{enumerate}

\end{problem}

\begin{problem}
Prove that the base angles of an isosceles triangle are congruent.   

\begin{image}
\definecolor{qqqqff}{rgb}{0.,0.,1.}
\definecolor{qqwuqq}{rgb}{0.,0.39215,0.}
\begin{tikzpicture}[line cap=round,line join=round,>=triangle 45,x=1.0cm,y=1.0cm]
%\clip(-0.4,-0.4) rectangle (2.4,2.6);
\clip(-0.4,-0.4) rectangle (5.4,2.6);
%\draw [shift={(1.,2.2)},line width=0.8pt,color=qqwuqq,fill=qqwuqq,fill opacity=0.1] (0,0) -- (-114.444:0.404) arc (-114.444:-65.556:0.404) -- cycle;  % Mark angle C
%\draw [shift={(1.,2.2)},line width=0.8pt,color=qqwuqq,fill=qqwuqq,fill opacity=0.1] (0,0) -- (-114.444:0.55) arc (-114.444:-90.:0.55) -- cycle;  % Mark angle DCA
%\draw [shift={(1.,2.2)},line width=0.8pt,color=qqwuqq,fill=qqwuqq,fill opacity=0.1] (0,0) -- (-90.:0.48) arc (-90.:-65.556:0.48) -- cycle;  % Mark angle DCB
%\draw[line width=0.8pt,color=qqwuqq,fill=qqwuqq,fill opacity=0.1] (1.,0.2) -- (0.8,0.2) -- (0.8,0.) -- (1.,0.) -- cycle; 
%\draw[line width=0.8pt,color=qqwuqq,fill=qqwuqq,fill opacity=0.1] (1.2,0.) -- (1.2,0.2) -- (1.,0.2) -- (1.,0.) -- cycle; 
\draw [line width=0.8pt] (0.,0.)-- (1.,2.2);
\draw [line width=0.8pt] (0.4171,1.1376) -- (0.5828,1.0623);
\draw [line width=0.8pt] (1.,2.2)-- (2.,0.);
\draw [line width=0.8pt] (1.5828,1.1376) -- (1.4171,1.0623);
\draw [line width=0.8pt] (0.,0.)-- (2.,0.);
%\draw [line width=0.8pt] (1.,2.2)-- (1.,0.); % segment CD
%\draw [line width=0.8pt] (1.0910,1.1354) -- (0.9090,1.1354);  %  Mark on CD
%\draw [line width=0.8pt] (1.0910,1.0646) -- (0.9090,1.0646);  %  Mark on CD
%\draw [line width=0.8pt] (0.4292,0.0910) -- (0.4292,-0.0910); % Mark on AD
%\draw [line width=0.8pt] (0.5,0.0910) -- (0.5,-0.0910);       % Mark on AD
%\draw [line width=0.8pt] (0.5708,0.0910) -- (0.5708,-0.0910); % Mark on AD
%\draw [line width=0.8pt] (1.4292,0.0910) -- (1.4292,-0.0910); % Mark on DB
%\draw [line width=0.8pt] (1.5,0.0910) -- (1.5,-0.0910);       % Mark on DB
%\draw [line width=0.8pt] (1.5708,0.0910) -- (1.5708,-0.0910); % Mark on DB
\begin{scriptsize}
\draw [fill=qqqqff] (0.,0.) circle (1.2pt);
\draw[color=qqqqff] (-0.18,-0.13) node {$A$};
\draw [fill=qqqqff] (2.,0.) circle (1.2pt);
\draw[color=qqqqff] (2.18,-0.13) node {$B$};
\draw [fill=qqqqff] (1.,2.2) circle (1.2pt);
\draw[color=qqqqff] (1.14,2.45) node {$C$};
%\draw [fill=qqqqff] (1.,0.) circle (1.2pt);
%\draw[color=qqqqff] (0.97,-0.2) node {$D$};
\end{scriptsize}
%\end{tikzpicture}

\begin{scope}[shift={(3,0)}]
% Isosceles triangle with median
%\definecolor{qqqqff}{rgb}{0.,0.,1.}
%\definecolor{qqwuqq}{rgb}{0.,0.39215,0.}
%\begin{tikzpicture}[line cap=round,line join=round,>=triangle 45,x=1.0cm,y=1.0cm]
%\clip(-0.4,-0.4) rectangle (2.4,2.6);
%\draw [shift={(1.,2.2)},line width=0.8pt,color=qqwuqq,fill=qqwuqq,fill opacity=0.1] (0,0) -- (-114.444:0.404) arc (-114.444:-65.556:0.404) -- cycle;  % Mark angle C
%\draw [shift={(1.,2.2)},line width=0.8pt,color=qqwuqq,fill=qqwuqq,fill opacity=0.1] (0,0) -- (-114.444:0.55) arc (-114.444:-90.:0.55) -- cycle;  % Mark angle DCA
%\draw [shift={(1.,2.2)},line width=0.8pt,color=qqwuqq,fill=qqwuqq,fill opacity=0.1] (0,0) -- (-90.:0.48) arc (-90.:-65.556:0.48) -- cycle;  % Mark angle DCB
%\draw[line width=0.8pt,color=qqwuqq,fill=qqwuqq,fill opacity=0.1] (1.,0.2) -- (0.8,0.2) -- (0.8,0.) -- (1.,0.) -- cycle; 
%\draw[line width=0.8pt,color=qqwuqq,fill=qqwuqq,fill opacity=0.1] (1.2,0.) -- (1.2,0.2) -- (1.,0.2) -- (1.,0.) -- cycle; 
\draw [line width=0.8pt] (0.,0.)-- (1.,2.2);
\draw [line width=0.8pt] (0.4171,1.1376) -- (0.5828,1.0623);
\draw [line width=0.8pt] (1.,2.2)-- (2.,0.);
\draw [line width=0.8pt] (1.5828,1.1376) -- (1.4171,1.0623);
\draw [line width=0.8pt] (0.,0.)-- (2.,0.);
\draw [line width=0.8pt] (1.,2.2)-- (1.,0.); % segment CD
\draw [line width=0.8pt] (1.0910,1.1354) -- (0.9090,1.1354);  %  Mark on CD
\draw [line width=0.8pt] (1.0910,1.0646) -- (0.9090,1.0646);  %  Mark on CD
\draw [line width=0.8pt] (0.4292,0.0910) -- (0.4292,-0.0910); % Mark on AD
\draw [line width=0.8pt] (0.5,0.0910) -- (0.5,-0.0910);       % Mark on AD
\draw [line width=0.8pt] (0.5708,0.0910) -- (0.5708,-0.0910); % Mark on AD
\draw [line width=0.8pt] (1.4292,0.0910) -- (1.4292,-0.0910); % Mark on DB
\draw [line width=0.8pt] (1.5,0.0910) -- (1.5,-0.0910);       % Mark on DB
\draw [line width=0.8pt] (1.5708,0.0910) -- (1.5708,-0.0910); % Mark on DB
\begin{scriptsize}
\draw [fill=qqqqff] (0.,0.) circle (1.2pt);
\draw[color=qqqqff] (-0.18,-0.13) node {$A$};
\draw [fill=qqqqff] (2.,0.) circle (1.2pt);
\draw[color=qqqqff] (2.18,-0.13) node {$B$};
\draw [fill=qqqqff] (1.,2.2) circle (1.2pt);
\draw[color=qqqqff] (1.14,2.45) node {$C$};
\draw [fill=qqqqff] (1.,0.) circle (1.2pt);
\draw[color=qqqqff] (0.97,-0.2) node {$D$};
\end{scriptsize}
\end{scope}
\end{tikzpicture}

\end{image}

\begin{enumerate}
\item Beginning with the given figure on the left, Deja draws $\overline{CD}$ and marks the figure intending that this new segment is a(n) \wordChoice{\choice[correct]{median}\choice{angle bisector}\choice{perpendicular bisector}\choice{altitude}}.

\item Based on the marked figure, Deja claims that the $\triangle ACD\cong \triangle\answer[format=string]{BCD}$ by \wordChoice{\choice{SAS}\choice[correct]{SSS}\choice{SSA}\choice{ASA}\choice{HL}}. 

\item Finally, Deja concludes that $\angle A\cong \angle\answer[format=string]{B}$, as they are corresponding parts of congruent triangles. 
\end{enumerate}

\end{problem}

\begin{problem}
Prove that the base angles of an isosceles triangle are congruent.   

\begin{image}
\definecolor{qqqqff}{rgb}{0.,0.,1.}
\definecolor{qqwuqq}{rgb}{0.,0.39215,0.}
\begin{tikzpicture}[line cap=round,line join=round,>=triangle 45,x=1.0cm,y=1.0cm]
%\clip(-0.4,-0.4) rectangle (2.4,2.6);
\clip(-0.4,-0.4) rectangle (5.4,2.6);
%\draw [shift={(1.,2.2)},line width=0.8pt,color=qqwuqq,fill=qqwuqq,fill opacity=0.1] (0,0) -- (-114.444:0.404) arc (-114.444:-65.556:0.404) -- cycle;  % Mark angle C
%\draw [shift={(1.,2.2)},line width=0.8pt,color=qqwuqq,fill=qqwuqq,fill opacity=0.1] (0,0) -- (-114.444:0.55) arc (-114.444:-90.:0.55) -- cycle;  % Mark angle DCA
%\draw [shift={(1.,2.2)},line width=0.8pt,color=qqwuqq,fill=qqwuqq,fill opacity=0.1] (0,0) -- (-90.:0.48) arc (-90.:-65.556:0.48) -- cycle;  % Mark angle DCB
%\draw[line width=0.8pt,color=qqwuqq,fill=qqwuqq,fill opacity=0.1] (1.,0.2) -- (0.8,0.2) -- (0.8,0.) -- (1.,0.) -- cycle; 
%\draw[line width=0.8pt,color=qqwuqq,fill=qqwuqq,fill opacity=0.1] (1.2,0.) -- (1.2,0.2) -- (1.,0.2) -- (1.,0.) -- cycle; 
\draw [line width=0.8pt] (0.,0.)-- (1.,2.2);
\draw [line width=0.8pt] (0.4171,1.1376) -- (0.5828,1.0623);
\draw [line width=0.8pt] (1.,2.2)-- (2.,0.);
\draw [line width=0.8pt] (1.5828,1.1376) -- (1.4171,1.0623);
\draw [line width=0.8pt] (0.,0.)-- (2.,0.);
%\draw [line width=0.8pt] (1.,2.2)-- (1.,0.); % segment CD
%\draw [line width=0.8pt] (1.0910,1.1354) -- (0.9090,1.1354);  %  Mark on CD
%\draw [line width=0.8pt] (1.0910,1.0646) -- (0.9090,1.0646);  %  Mark on CD
%\draw [line width=0.8pt] (0.4292,0.0910) -- (0.4292,-0.0910); % Mark on AD
%\draw [line width=0.8pt] (0.5,0.0910) -- (0.5,-0.0910);       % Mark on AD
%\draw [line width=0.8pt] (0.5708,0.0910) -- (0.5708,-0.0910); % Mark on AD
%\draw [line width=0.8pt] (1.4292,0.0910) -- (1.4292,-0.0910); % Mark on DB
%\draw [line width=0.8pt] (1.5,0.0910) -- (1.5,-0.0910);       % Mark on DB
%\draw [line width=0.8pt] (1.5708,0.0910) -- (1.5708,-0.0910); % Mark on DB
\begin{scriptsize}
\draw [fill=qqqqff] (0.,0.) circle (1.2pt);
\draw[color=qqqqff] (-0.18,-0.13) node {$A$};
\draw [fill=qqqqff] (2.,0.) circle (1.2pt);
\draw[color=qqqqff] (2.18,-0.13) node {$B$};
\draw [fill=qqqqff] (1.,2.2) circle (1.2pt);
\draw[color=qqqqff] (1.14,2.45) node {$C$};
%\draw [fill=qqqqff] (1.,0.) circle (1.2pt);
%\draw[color=qqqqff] (0.97,-0.2) node {$D$};
\end{scriptsize}
%\end{tikzpicture}

\begin{scope}[shift={(3,0)}]
% Isosceles triangle with altitude
%\definecolor{qqqqff}{rgb}{0.,0.,1.}
%\definecolor{qqwuqq}{rgb}{0.,0.39215,0.}
%\begin{tikzpicture}[line cap=round,line join=round,>=triangle 45,x=1.0cm,y=1.0cm]
%\clip(-0.4,-0.4) rectangle (2.4,2.6);
%\draw [shift={(1.,2.2)},line width=0.8pt,color=qqwuqq,fill=qqwuqq,fill opacity=0.1] (0,0) -- (-114.444:0.404) arc (-114.444:-65.556:0.404) -- cycle;  % Mark angle C
%\draw [shift={(1.,2.2)},line width=0.8pt,color=qqwuqq,fill=qqwuqq,fill opacity=0.1] (0,0) -- (-114.444:0.55) arc (-114.444:-90.:0.55) -- cycle;  % Mark angle DCA
%\draw [shift={(1.,2.2)},line width=0.8pt,color=qqwuqq,fill=qqwuqq,fill opacity=0.1] (0,0) -- (-90.:0.48) arc (-90.:-65.556:0.48) -- cycle;  % Mark angle DCB
\draw[line width=0.8pt,color=qqwuqq,fill=qqwuqq,fill opacity=0.1] (1.,0.2) -- (0.8,0.2) -- (0.8,0.) -- (1.,0.) -- cycle; 
\draw[line width=0.8pt,color=qqwuqq,fill=qqwuqq,fill opacity=0.1] (1.2,0.) -- (1.2,0.2) -- (1.,0.2) -- (1.,0.) -- cycle; 
\draw [line width=0.8pt] (0.,0.)-- (1.,2.2);
\draw [line width=0.8pt] (0.4171,1.1376) -- (0.5828,1.0623);
\draw [line width=0.8pt] (1.,2.2)-- (2.,0.);
\draw [line width=0.8pt] (1.5828,1.1376) -- (1.4171,1.0623);
\draw [line width=0.8pt] (0.,0.)-- (2.,0.);
\draw [line width=0.8pt] (1.,2.2)-- (1.,0.); % segment CD
\draw [line width=0.8pt] (1.0910,1.1354) -- (0.9090,1.1354);  %  Mark on CD
\draw [line width=0.8pt] (1.0910,1.0646) -- (0.9090,1.0646);  %  Mark on CD
%\draw [line width=0.8pt] (0.4292,0.0910) -- (0.4292,-0.0910); % Mark on AD
%\draw [line width=0.8pt] (0.5,0.0910) -- (0.5,-0.0910);       % Mark on AD
%\draw [line width=0.8pt] (0.5708,0.0910) -- (0.5708,-0.0910); % Mark on AD
%\draw [line width=0.8pt] (1.4292,0.0910) -- (1.4292,-0.0910); % Mark on DB
%\draw [line width=0.8pt] (1.5,0.0910) -- (1.5,-0.0910);       % Mark on DB
%\draw [line width=0.8pt] (1.5708,0.0910) -- (1.5708,-0.0910); % Mark on DB
\begin{scriptsize}
\draw [fill=qqqqff] (0.,0.) circle (1.2pt);
\draw[color=qqqqff] (-0.18,-0.13) node {$A$};
\draw [fill=qqqqff] (2.,0.) circle (1.2pt);
\draw[color=qqqqff] (2.18,-0.13) node {$B$};
\draw [fill=qqqqff] (1.,2.2) circle (1.2pt);
\draw[color=qqqqff] (1.14,2.45) node {$C$};
\draw [fill=qqqqff] (1.,0.) circle (1.2pt);
\draw[color=qqqqff] (0.97,-0.2) node {$D$};
\end{scriptsize}
\end{scope}
\end{tikzpicture}
\end{image}

\begin{enumerate}
\item Beginning with the given figure on the left, Elle draws $\overline{CD}$ and marks the figure intending that this new segment is a(n) \wordChoice{\choice{median}\choice{angle bisector}\choice{perpendicular bisector}\choice[correct]{altitude}}.

\item Based on the marked figure, Deja claims that the $\triangle ACD\cong \triangle\answer[format=string]{BCD}$ by \wordChoice{\choice{SAS}\choice{SSS}\choice{SSA}\choice{ASA}\choice[correct]{HL}}. 

\item Finally, Deja concludes that $\angle A\cong \angle\answer[format=string]{B}$, as they are corresponding parts of congruent triangles. 
\end{enumerate}

\end{problem}


\begin{problem}
Simon and Taylor are trying to prove that the base angles of an isosceles triangle are congruent.   

\begin{image}
\definecolor{qqqqff}{rgb}{0.,0.,1.}
\definecolor{qqwuqq}{rgb}{0.,0.39215,0.}
\begin{tikzpicture}[line cap=round,line join=round,>=triangle 45,x=1.0cm,y=1.0cm]
%\clip(-0.4,-0.4) rectangle (2.4,2.6);
\clip(-0.4,-0.75) rectangle (8.4,2.6);
%\draw [shift={(1.,2.2)},line width=0.8pt,color=qqwuqq,fill=qqwuqq,fill opacity=0.1] (0,0) -- (-114.444:0.404) arc (-114.444:-65.556:0.404) -- cycle;  % Mark angle C
%\draw [shift={(1.,2.2)},line width=0.8pt,color=qqwuqq,fill=qqwuqq,fill opacity=0.1] (0,0) -- (-114.444:0.55) arc (-114.444:-90.:0.55) -- cycle;  % Mark angle DCA
%\draw [shift={(1.,2.2)},line width=0.8pt,color=qqwuqq,fill=qqwuqq,fill opacity=0.1] (0,0) -- (-90.:0.48) arc (-90.:-65.556:0.48) -- cycle;  % Mark angle DCB
%\draw[line width=0.8pt,color=qqwuqq,fill=qqwuqq,fill opacity=0.1] (1.,0.2) -- (0.8,0.2) -- (0.8,0.) -- (1.,0.) -- cycle; 
%\draw[line width=0.8pt,color=qqwuqq,fill=qqwuqq,fill opacity=0.1] (1.2,0.) -- (1.2,0.2) -- (1.,0.2) -- (1.,0.) -- cycle; 
\draw [line width=0.8pt] (0.,0.)-- (1.,2.2);
\draw [line width=0.8pt] (0.4171,1.1376) -- (0.5828,1.0623);
\draw [line width=0.8pt] (1.,2.2)-- (2.,0.);
\draw [line width=0.8pt] (1.5828,1.1376) -- (1.4171,1.0623);
\draw [line width=0.8pt] (0.,0.)-- (2.,0.);
%\draw [line width=0.8pt] (1.,2.2)-- (1.,0.); % segment CD
%\draw [line width=0.8pt] (1.0910,1.1354) -- (0.9090,1.1354);  %  Mark on CD
%\draw [line width=0.8pt] (1.0910,1.0646) -- (0.9090,1.0646);  %  Mark on CD
%\draw [line width=0.8pt] (0.4292,0.0910) -- (0.4292,-0.0910); % Mark on AD
%\draw [line width=0.8pt] (0.5,0.0910) -- (0.5,-0.0910);       % Mark on AD
%\draw [line width=0.8pt] (0.5708,0.0910) -- (0.5708,-0.0910); % Mark on AD
%\draw [line width=0.8pt] (1.4292,0.0910) -- (1.4292,-0.0910); % Mark on DB
%\draw [line width=0.8pt] (1.5,0.0910) -- (1.5,-0.0910);       % Mark on DB
%\draw [line width=0.8pt] (1.5708,0.0910) -- (1.5708,-0.0910); % Mark on DB
\begin{scriptsize}
\draw [fill=qqqqff] (0.,0.) circle (1.2pt);
\draw[color=qqqqff] (-0.18,-0.13) node {$A$};
\draw [fill=qqqqff] (2.,0.) circle (1.2pt);
\draw[color=qqqqff] (2.18,-0.13) node {$B$};
\draw [fill=qqqqff] (1.,2.2) circle (1.2pt);
\draw[color=qqqqff] (1.14,2.45) node {$C$};
\draw (1,-0.6) node[align=center] {Given figure};
%\draw [fill=qqqqff] (1.,0.) circle (1.2pt);
%\draw[color=qqqqff] (0.97,-0.2) node {$D$};
\end{scriptsize}
%\end{tikzpicture}

\begin{scope}[shift={(3,0)}]
% Isosceles triangle with overspecified perpendicular bisector
%\definecolor{qqqqff}{rgb}{0.,0.,1.}
%\definecolor{qqwuqq}{rgb}{0.,0.39215,0.}
%\begin{tikzpicture}[line cap=round,line join=round,>=triangle 45,x=1.0cm,y=1.0cm]
%\clip(-0.4,-0.4) rectangle (2.4,2.6);
%\draw [shift={(1.,2.2)},line width=0.8pt,color=qqwuqq,fill=qqwuqq,fill opacity=0.1] (0,0) -- (-114.444:0.404) arc (-114.444:-65.556:0.404) -- cycle;  % Mark angle C
%\draw [shift={(1.,2.2)},line width=0.8pt,color=qqwuqq,fill=qqwuqq,fill opacity=0.1] (0,0) -- (-114.444:0.55) arc (-114.444:-90.:0.55) -- cycle;  % Mark angle DCA
%\draw [shift={(1.,2.2)},line width=0.8pt,color=qqwuqq,fill=qqwuqq,fill opacity=0.1] (0,0) -- (-90.:0.48) arc (-90.:-65.556:0.48) -- cycle;  % Mark angle DCB
\draw[line width=0.8pt,color=qqwuqq,fill=qqwuqq,fill opacity=0.1] (1.,0.2) -- (0.8,0.2) -- (0.8,0.) -- (1.,0.) -- cycle; 
\draw[line width=0.8pt,color=qqwuqq,fill=qqwuqq,fill opacity=0.1] (1.2,0.) -- (1.2,0.2) -- (1.,0.2) -- (1.,0.) -- cycle; 
\draw [line width=0.8pt] (0.,0.)-- (1.,2.2);
\draw [line width=0.8pt] (0.4171,1.1376) -- (0.5828,1.0623);
\draw [line width=0.8pt] (1.,2.2)-- (2.,0.);
\draw [line width=0.8pt] (1.5828,1.1376) -- (1.4171,1.0623);
\draw [line width=0.8pt] (0.,0.)-- (2.,0.);
\draw [line width=0.8pt] (1.,2.2)-- (1.,0.); % segment CD
\draw [line width=0.8pt] (1.0910,1.1354) -- (0.9090,1.1354);  %  Mark on CD
\draw [line width=0.8pt] (1.0910,1.0646) -- (0.9090,1.0646);  %  Mark on CD
\draw [line width=0.8pt] (0.4292,0.0910) -- (0.4292,-0.0910); % Mark on AD
\draw [line width=0.8pt] (0.5,0.0910) -- (0.5,-0.0910);       % Mark on AD
\draw [line width=0.8pt] (0.5708,0.0910) -- (0.5708,-0.0910); % Mark on AD
\draw [line width=0.8pt] (1.4292,0.0910) -- (1.4292,-0.0910); % Mark on DB
\draw [line width=0.8pt] (1.5,0.0910) -- (1.5,-0.0910);       % Mark on DB
\draw [line width=0.8pt] (1.5708,0.0910) -- (1.5708,-0.0910); % Mark on DB
\begin{scriptsize}
\draw [fill=qqqqff] (0.,0.) circle (1.2pt);
\draw[color=qqqqff] (-0.18,-0.13) node {$A$};
\draw [fill=qqqqff] (2.,0.) circle (1.2pt);
\draw[color=qqqqff] (2.18,-0.13) node {$B$};
\draw [fill=qqqqff] (1.,2.2) circle (1.2pt);
\draw[color=qqqqff] (1.14,2.45) node {$C$};
\draw [fill=qqqqff] (1.,0.) circle (1.2pt);
\draw[color=qqqqff] (0.97,-0.2) node {$D$};
\draw (1,-0.6) node[align=center] {Simon's figure};
\end{scriptsize}
\end{scope}
%\end{tikzpicture}

\begin{scope}[shift={(6,0)}]
% Isosceles triangle with a perpendicular bisector that misses vertex
%\definecolor{qqqqff}{rgb}{0.,0.,1.}
%\definecolor{qqwuqq}{rgb}{0.,0.39215,0.}
%\begin{tikzpicture}[line cap=round,line join=round,>=triangle 45,x=1.0cm,y=1.0cm]
%\clip(-0.4,-0.4) rectangle (2.4,2.6);
%\draw [shift={(1.,2.2)},line width=0.8pt,color=qqwuqq,fill=qqwuqq,fill opacity=0.1] (0,0) -- (-114.444:0.404) arc (-114.444:-65.556:0.404) -- cycle;  % Mark angle C
%\draw [shift={(1.,2.2)},line width=0.8pt,color=qqwuqq,fill=qqwuqq,fill opacity=0.1] (0,0) -- (-114.444:0.55) arc (-114.444:-90.:0.55) -- cycle;  % Mark angle DCA
%\draw [shift={(1.,2.2)},line width=0.8pt,color=qqwuqq,fill=qqwuqq,fill opacity=0.1] (0,0) -- (-90.:0.48) arc (-90.:-65.556:0.48) -- cycle;  % Mark angle DCB
\draw[line width=0.8pt,color=qqwuqq,fill=qqwuqq,fill opacity=0.1] (1.,0.2) -- (0.8,0.2) -- (0.8,0.) -- (1.,0.) -- cycle; 
\draw[line width=0.8pt,color=qqwuqq,fill=qqwuqq,fill opacity=0.1] (1.2,0.) -- (1.2,0.2) -- (1.,0.2) -- (1.,0.) -- cycle; 
\draw [line width=0.8pt] (0.,0.)-- (1.06,2.2);
\draw [line width=0.8pt] (0.4471,1.1376) -- (0.6128,1.0623);
\draw [line width=0.8pt] (1.06,2.2)-- (2.,0.);
\draw [line width=0.8pt] (1.6128,1.1376) -- (1.4471,1.0623);
\draw [line width=0.8pt] (0.,0.)-- (2.,0.);
\draw [line width=0.8pt] (0.99,2.4)-- (1.,0.); % segment CD
%\draw [line width=0.8pt] (1.0910,1.1354) -- (0.9090,1.1354);  %  Mark on CD
%\draw [line width=0.8pt] (1.0910,1.0646) -- (0.9090,1.0646);  %  Mark on CD
\draw [line width=0.8pt] (0.4292,0.0910) -- (0.4292,-0.0910); % Mark on AD
\draw [line width=0.8pt] (0.5,0.0910) -- (0.5,-0.0910);       % Mark on AD
\draw [line width=0.8pt] (0.5708,0.0910) -- (0.5708,-0.0910); % Mark on AD
\draw [line width=0.8pt] (1.4292,0.0910) -- (1.4292,-0.0910); % Mark on DB
\draw [line width=0.8pt] (1.5,0.0910) -- (1.5,-0.0910);       % Mark on DB
\draw [line width=0.8pt] (1.5708,0.0910) -- (1.5708,-0.0910); % Mark on DB
\begin{scriptsize}
\draw [fill=qqqqff] (0.,0.) circle (1.2pt);
\draw[color=qqqqff] (-0.18,-0.13) node {$A$};
\draw [fill=qqqqff] (2.,0.) circle (1.2pt);
\draw[color=qqqqff] (2.18,-0.13) node {$B$};
\draw [fill=qqqqff] (1.06,2.2) circle (1.2pt);
\draw[color=qqqqff] (1.24,2.45) node {$C$};
\draw [fill=qqqqff] (1.,0.) circle (1.2pt);
\draw[color=qqqqff] (0.97,-0.2) node {$D$};
\draw (1,-0.6) node[align=center] {Taylor's figure};
\end{scriptsize}
\end{scope}
\end{tikzpicture}
\end{image}

Beginning with the given figure on the left, Simon draws $\overline{CD}$ and marks the second figure intending that this new segment is a perpendicular bisector of $\overline{AB}$.

Taylor claims that a perpendicular bisector of a side of a triangle usually misses the opposite vertex.  So without using other properties of isosceles triangles or perpendicular bisectors, the figure should allow for that possibility.  

Choose the best response to their argument: 
\begin{multipleChoice}
\choice{Simon is correct, and $\triangle ACD\cong \triangle BCD$ by SAS.} 
\choice{Simon is correct, and $\triangle ACD\cong \triangle BCD$ by SSS}
\choice[correct]{Taylor is correct, and the perpendicular bisector cannot be used to complete this proof.}
\choice{Neither of them are correct.}  
\end{multipleChoice} 
\end{problem}


\begin{problem}
Prove that the base angles of an isosceles triangle are congruent.   

\begin{image}
\definecolor{qqqqff}{rgb}{0.,0.,1.}
\definecolor{qqwuqq}{rgb}{0.,0.39215,0.}
\begin{tikzpicture}[line cap=round,line join=round,>=triangle 45,x=1.0cm,y=1.0cm]
%\clip(-0.4,-0.4) rectangle (2.4,2.6);
\clip(-0.4,-0.75) rectangle (5.4,2.6);
%\draw [shift={(1.,2.2)},line width=0.8pt,color=qqwuqq,fill=qqwuqq,fill opacity=0.1] (0,0) -- (-114.444:0.404) arc (-114.444:-65.556:0.404) -- cycle;  % Mark angle C
%\draw [shift={(1.,2.2)},line width=0.8pt,color=qqwuqq,fill=qqwuqq,fill opacity=0.1] (0,0) -- (-114.444:0.55) arc (-114.444:-90.:0.55) -- cycle;  % Mark angle DCA
%\draw [shift={(1.,2.2)},line width=0.8pt,color=qqwuqq,fill=qqwuqq,fill opacity=0.1] (0,0) -- (-90.:0.48) arc (-90.:-65.556:0.48) -- cycle;  % Mark angle DCB
%\draw[line width=0.8pt,color=qqwuqq,fill=qqwuqq,fill opacity=0.1] (1.,0.2) -- (0.8,0.2) -- (0.8,0.) -- (1.,0.) -- cycle; 
%\draw[line width=0.8pt,color=qqwuqq,fill=qqwuqq,fill opacity=0.1] (1.2,0.) -- (1.2,0.2) -- (1.,0.2) -- (1.,0.) -- cycle; 
\draw [line width=0.8pt] (0.,0.)-- (1.,2.2);
\draw [line width=0.8pt] (0.4171,1.1376) -- (0.5828,1.0623);
\draw [line width=0.8pt] (1.,2.2)-- (2.,0.);
\draw [line width=0.8pt] (1.5828,1.1376) -- (1.4171,1.0623);
\draw [line width=0.8pt] (0.,0.)-- (2.,0.);
%\draw [line width=0.8pt] (1.,2.2)-- (1.,0.); % segment CD
%\draw [line width=0.8pt] (1.0910,1.1354) -- (0.9090,1.1354);  %  Mark on CD
%\draw [line width=0.8pt] (1.0910,1.0646) -- (0.9090,1.0646);  %  Mark on CD
%\draw [line width=0.8pt] (0.4292,0.0910) -- (0.4292,-0.0910); % Mark on AD
%\draw [line width=0.8pt] (0.5,0.0910) -- (0.5,-0.0910);       % Mark on AD
%\draw [line width=0.8pt] (0.5708,0.0910) -- (0.5708,-0.0910); % Mark on AD
%\draw [line width=0.8pt] (1.4292,0.0910) -- (1.4292,-0.0910); % Mark on DB
%\draw [line width=0.8pt] (1.5,0.0910) -- (1.5,-0.0910);       % Mark on DB
%\draw [line width=0.8pt] (1.5708,0.0910) -- (1.5708,-0.0910); % Mark on DB
\begin{scriptsize}
\draw [fill=qqqqff] (0.,0.) circle (1.2pt);
\draw[color=qqqqff] (-0.18,-0.13) node {$A$};
\draw [fill=qqqqff] (2.,0.) circle (1.2pt);
\draw[color=qqqqff] (2.18,-0.13) node {$B$};
\draw [fill=qqqqff] (1.,2.2) circle (1.2pt);
\draw[color=qqqqff] (1.14,2.45) node {$C$};
\draw (1,-0.6) node[align=center] {Given figure};
%\draw [fill=qqqqff] (1.,0.) circle (1.2pt);
%\draw[color=qqqqff] (0.97,-0.2) node {$D$};
\end{scriptsize}
%\end{tikzpicture}

\begin{scope}[shift={(3,0)}]
% Isosceles triangle using symmetry 
%\definecolor{qqqqff}{rgb}{0.,0.,1.}
%\definecolor{qqwuqq}{rgb}{0.,0.39215,0.}
%\begin{tikzpicture}[line cap=round,line join=round,>=triangle 45,x=1.0cm,y=1.0cm]
%\clip(-0.4,-0.4) rectangle (2.4,2.6);
\draw [shift={(1.,2.2)},line width=0.8pt,color=qqwuqq,fill=qqwuqq,fill opacity=0.1] (0,0) -- (-114.444:0.404) arc (-114.444:-65.556:0.404) -- cycle;  % Mark angle C
%\draw [shift={(1.,2.2)},line width=0.8pt,color=qqwuqq,fill=qqwuqq,fill opacity=0.1] (0,0) -- (-114.444:0.55) arc (-114.444:-90.:0.55) -- cycle;  % Mark angle DCA
%\draw [shift={(1.,2.2)},line width=0.8pt,color=qqwuqq,fill=qqwuqq,fill opacity=0.1] (0,0) -- (-90.:0.48) arc (-90.:-65.556:0.48) -- cycle;  % Mark angle DCB
%\draw[line width=0.8pt,color=qqwuqq,fill=qqwuqq,fill opacity=0.1] (1.,0.2) -- (0.8,0.2) -- (0.8,0.) -- (1.,0.) -- cycle; 
%\draw[line width=0.8pt,color=qqwuqq,fill=qqwuqq,fill opacity=0.1] (1.2,0.) -- (1.2,0.2) -- (1.,0.2) -- (1.,0.) -- cycle; 
\draw [line width=0.8pt] (0.,0.)-- (1.,2.2);
\draw [line width=0.8pt] (0.4171,1.1376) -- (0.5828,1.0623);
\draw [line width=0.8pt] (1.,2.2)-- (2.,0.);
\draw [line width=0.8pt] (1.5828,1.1376) -- (1.4171,1.0623);
\draw [line width=0.8pt] (0.,0.)-- (2.,0.);
%\draw [line width=0.8pt] (1.,2.2)-- (1.,0.); % segment CD
%\draw [line width=0.8pt] (1.0910,1.1354) -- (0.9090,1.1354);  %  Mark on CD
%\draw [line width=0.8pt] (1.0910,1.0646) -- (0.9090,1.0646);  %  Mark on CD
%\draw [line width=0.8pt] (0.4292,0.0910) -- (0.4292,-0.0910); % Mark on AD
%\draw [line width=0.8pt] (0.5,0.0910) -- (0.5,-0.0910);       % Mark on AD
%\draw [line width=0.8pt] (0.5708,0.0910) -- (0.5708,-0.0910); % Mark on AD
%\draw [line width=0.8pt] (1.4292,0.0910) -- (1.4292,-0.0910); % Mark on DB
%\draw [line width=0.8pt] (1.5,0.0910) -- (1.5,-0.0910);       % Mark on DB
%\draw [line width=0.8pt] (1.5708,0.0910) -- (1.5708,-0.0910); % Mark on DB
\begin{scriptsize}
\draw [fill=qqqqff] (0.,0.) circle (1.2pt);
\draw[color=qqqqff] (-0.18,-0.13) node {$A$};
\draw [fill=qqqqff] (2.,0.) circle (1.2pt);
\draw[color=qqqqff] (2.18,-0.13) node {$B$};
\draw [fill=qqqqff] (1.,2.2) circle (1.2pt);
\draw[color=qqqqff] (1.14,2.45) node {$C$};
%\draw [fill=qqqqff] (1.,0.) circle (1.2pt);
%\draw[color=qqqqff] (0.97,-0.2) node {$D$};
\end{scriptsize}
\end{scope}
\end{tikzpicture}
\end{image}

\begin{enumerate}
\item Examining the given figure on the left, Lissy notices symmetry in the triangle and claims that the triangle is congruent to itself by a \wordChoice{\choice{translation}\choice[correct]{reflection}\choice{rotation}}.  

\item Based on the marked figure, Lissy claims that the $\triangle ACB\cong \triangle\answer[format=string]{BCA}$ by \wordChoice{\choice[correct]{SAS}\choice{SSS}\choice{SSA}\choice{ASA}\choice{HL}}. 

\item Finally, Lissy concludes that $\angle A\cong \angle\answer[format=string]{B}$, as they are corresponding parts of congruent triangles. 
\end{enumerate}

\end{problem}



\end{document}
