% Section 2.2 Anatomy of Figures

\documentclass[nooutcomes]{ximera}
%\documentclass[space,handout,nooutcomes]{ximera}

% For preamble materials

\graphicspath{
  {./}
  {chapter1/}
  {chapter2/}
  {chapter4/}
  {math1/}
  {math2/}
}

\usepackage{pgf,tikz}
\usepackage{mathrsfs}
\usetikzlibrary{arrows}
\pgfplotsset{compat=1.16}


\newcommand{\N}{\mathbb N}
\newcommand{\W}{\mathbb W}
\newcommand{\C}{\mathbb C}
\newcommand{\Z}{\mathbb Z}
\newcommand{\Q}{\mathbb Q}
\newcommand{\R}{\mathbb R}




\title{Vocabulary Review}
\author{Bart Snapp and Brad Findell}
\begin{document}
\begin{abstract}
Short-answer, multiple-choice, and select-all questions about key vocabulary. 
\end{abstract}
\maketitle

%Useful questions: 
%
%What is regular quadrilateral? 
%Definition of ? 
%Write the Pythagorean theorem. 
%Measure angles. 
%Angle sum in a triangle. 
%Triangulate a figure 


\begin{question}  
An \textbf{equilateral quadrilateral} is called a $\answer[format=string]{rhombus}$.
\end{question}

\begin{question}  
An \textbf{equiangular quadrilateral} is called a $\answer[format=string]{rectangle}$. 
\end{question}

\begin{question}  
An \textbf{regular quadrilateral} is called a $\answer[format=string]{square}$. 
\end{question}

%\begin{question}  
%An line segment between two points on a circle is called a $\answer[format=string]{chord}$ of the circle.  
%\end{question}

\begin{question}  
A $\answer[format=string]{straight angle}$ measures $180^\circ$.  (Hint: Answer with two words.)
\end{question}

\begin{question}  
Two angles whose measures sum to $180^\circ$ are said to be $\answer[format=string]{supplementary}$.  
\end{question}

\begin{question}  
Two angles whose measures sum to $90^\circ$ are said to be $\answer[format=string]{complementary}$.  
\end{question}

\begin{question}  
Three (or more) points that lie on the same line are said to be $\answer[format=string]{collinear}$.  
\end{question}

\begin{question}  
Three (or more) lines that lie on the same point are said to be $\answer[format=string]{concurrent}$.  
\end{question}

%\begin{question}  
%When three (or more) points all lie on the same line, we say they are \dots
%\begin{multipleChoice}  
%\choice{coplanar.}  
%\choice[correct]{collinear.}  
%\choice{conjoined.}
%\choice{concurrent.}  
%\choice{none of these.}
%\end{multipleChoice}  
%\end{question}
%
%\begin{question}  
%When three (or more) lines all lie on the same point, we say they are \dots
%\begin{multipleChoice}  
%\choice{coplanar.}  
%\choice{collinear.}  
%\choice{conjoined.}
%\choice[correct]{concurrent.}  
%\choice{none of these.}
%\end{multipleChoice}  
%\end{question}


\end{document}

