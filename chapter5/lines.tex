
\documentclass[nooutcomes]{ximera}
%\documentclass[space,handout,nooutcomes]{ximera}

% For preamble materials

\usepackage{pgf,tikz}
\usepackage{mathrsfs}
\usetikzlibrary{arrows}
\usepackage{framed}
\usepackage{amsmath}
\pgfplotsset{compat=1.17}

\def\fixnote#1{\begin{framed}{\textcolor{red}{Fix note: #1}}\end{framed}}  % Allows insertion of red notes about needed edits
%\def\fixnote#1{}

\def\detail#1{{\textcolor{blue}{Detail: #1}}}   

\pdfOnly{\renewenvironment{image}[1][]{\begin{center}}{\end{center}}}

\graphicspath{
  {./}
  {chapter1/}
  {chapter2/}
  {chapter4/}
  {proofs/}
  {graphics/}
  {../graphics/}
}

\newenvironment{sectionOutcomes}{}{}


%%% This set of code is all of our user defined commands
\newcommand{\bysame}{\mbox{\rule{3em}{.4pt}}\,}
\newcommand{\N}{\mathbb N}
\newcommand{\C}{\mathbb C}
\newcommand{\W}{\mathbb W}
\newcommand{\Z}{\mathbb Z}
\newcommand{\Q}{\mathbb Q}
\newcommand{\R}{\mathbb R}
\newcommand{\A}{\mathbb A}
\newcommand{\D}{\mathcal D}
\newcommand{\F}{\mathcal F}
\newcommand{\ph}{\varphi}
\newcommand{\ep}{\varepsilon}
\newcommand{\aph}{\alpha}
\newcommand{\QM}{\begin{center}{\huge\textbf{?}}\end{center}}

\renewcommand{\le}{\leqslant}
\renewcommand{\ge}{\geqslant}
\renewcommand{\a}{\wedge}
\renewcommand{\v}{\vee}
\renewcommand{\l}{\ell}
\newcommand{\mat}{\mathsf}
\renewcommand{\vec}{\mathbf}
\renewcommand{\subset}{\subseteq}
\renewcommand{\supset}{\supseteq}
%\renewcommand{\emptyset}{\varnothing}
%\newcommand{\xto}{\xrightarrow}
%\renewcommand{\qedsymbol}{$\blacksquare$}
%\newcommand{\bibname}{References and Further Reading}
%\renewcommand{\bar}{\protect\overline}
%\renewcommand{\hat}{\protect\widehat}
%\renewcommand{\tilde}{\widetilde}
%\newcommand{\tri}{\triangle}
%\newcommand{\minipad}{\vspace{1ex}}
%\newcommand{\leftexp}[2]{{\vphantom{#2}}^{#1}{#2}}

%% More user defined commands
\renewcommand{\epsilon}{\varepsilon}
\renewcommand{\theta}{\vartheta} %% only for kmath
\renewcommand{\l}{\ell}
\renewcommand{\d}{\, d}
\newcommand{\ddx}{\frac{d}{dx}}
\newcommand{\dydx}{\frac{dy}{dx}}


\usepackage{bigstrut}


\title{Lines}
\author{Bart Snapp and Brad Findell}
\begin{document}
\begin{abstract}
These problems provide practice writing and solving equations of lines. 
\end{abstract}
\maketitle


\begin{problem}
Equations of lines are usually written in one of the following three forms.  (Use hyphens for two-word forms.) 
\begin{itemize}
\item $ax+by=c$ is called $\answer[format=string]{standard}$ form. 
\item $y-y_1=m(x-x_1)$ is called $\answer[format=string]{point-slope}$ form.  
\item $y=mx+b$ is called $\answer[format=string]{slope-intercept}$ form. 
\end{itemize}
\begin{problem}
Correct!  And each of the forms has advantages and disadvantages.  Here are some advantages:  
\begin{itemize}
\item In $\answer[format=string]{slope-intercept}$ form, the equation is unique.  
\item Any line in the plane can be represented in $\answer[format=string]{standard}$ form. 
\item In $\answer[format=string]{point-slope}$ form, an equation can often be written quickly.  
\item In $\answer[format=string]{standard}$ form, it is easy to compute both the $x$- and $y$-intercepts.  
\end{itemize}
\end{problem}
\end{problem}

% In point-slope form, it is easy to see the slope. 
% In slope-intercept form, it is easy to see the slope and the $y$-intercept.  

\begin{problem}
Write an equation of the line through $(2,3)$ and $(5,1)$.  

\begin{prompt}
$y = \answer{3+(1-3)/(5-2)(x-2)}$
\end{prompt}
\end{problem}


\begin{problem}
Write an equation (in standard form) of the line through $(2,4)$ parallel to $5x-3y=1$.  

My equation begins: 
\begin{multipleChoice}
\choice{$y=$}
\choice{$5x+3y=$}
\choice{$3x+5y=$}
\choice{$3x-5y=$}
\choice[correct]{$5x-3y=$}
\end{multipleChoice}
\begin{problem}
That can work!  Your equation is
\[
5x-3y=\answer{5\cdot2-3\cdot4}
\]
\end{problem}
\end{problem}

\begin{problem}
Write an equation (in standard form) of the line through $(p,q)$ parallel to $5x-3y=1$.  

My equation begins: 
\begin{multipleChoice}
\choice{$y=$}
\choice{$5x+3y=$}
\choice{$3x+5y=$}
\choice{$3x-5y=$}
\choice[correct]{$5x-3y=$}
\end{multipleChoice}
\begin{problem}
That can work!  Your equation is
\[
5x-3y=\answer{5p-3q}
\]
\end{problem}
\end{problem}


\begin{problem}
Now write an equation (in standard form) of the line through $(p,q)$ parallel to $ax+by=c$. 

\begin{prompt}
\[
\answer{ax+by} = \answer{ap+bq}
\]
\end{prompt}
\end{problem}


\begin{problem}
Write an equation (in standard form) of the line through $(2,4)$ perpendicular to $5x-3y=1$.  

My equation begins: 
\begin{multipleChoice}
\choice{$y=$}
\choice{$5x+3y=$}
\choice[correct]{$3x+5y=$}
\choice{$3x-5y=$}
\choice{$5x-3y=$}
\end{multipleChoice}
\begin{problem}
That can work!  Your equation is
\[
3x+5y=\answer{3\cdot2+5\cdot4}
\]
\end{problem}
\end{problem}

\begin{problem}
Write an equation (in standard form) of the line through $(p,q)$ perpendicular to $5x-3y=1$.  

My equation begins: 
\begin{multipleChoice}
\choice{$y=$}
\choice{$5x+3y=$}
\choice[correct]{$3x+5y=$}
\choice{$3x-5y=$}
\choice{$5x-3y=$}
\end{multipleChoice}
\begin{problem}
That can work!  Your equation is
\[
3x+5y=\answer{3p+5q}
\]
\end{problem}
\end{problem}


\begin{problem}
Now write an equation (in standard form) of the line through $(p,q)$ perpendicular to $ax+by=c$. 

My equation begins: 
\begin{multipleChoice}
\choice{$y=$}
\choice{$ax+by=$}
\choice{$ax-by=$}
\choice[correct]{$bx-ay=$}
\choice{$bx+ay=$}
\end{multipleChoice}
\begin{problem}
That can work! Your equation is
\begin{prompt}
\[
bx-ay = \answer{bp-aq}
\]
\end{prompt}
\end{problem}
\end{problem}

\begin{problem}
With two equations in two unknowns, two common solutions methods are called (in alphabetical order)   $\answer[format=string]{elimination}$ and $\answer[format=string]{substitution}$. 
\end{problem}

\begin{problem}
The simultaneous equations  $2x-3y=5$ and $x+4y=19$ have
\begin{multipleChoice}
\choice{no solution}
\choice[correct]{a unique solution}
\choice{more than one solution}
\end{multipleChoice}
\begin{problem}
%Correct, because the equations are $\answer[format=string]{equivalent}$.
%Correct, because the lines are $\answer[format=string]{parallel}$.
Correct!  And the solution is $x=\answer{7}$, $y = \answer{3}$. 
\end{problem}
\end{problem}

\begin{problem}
The simultaneous equations  $2x-3y=5$ and $-4x+6y=7$ have
\begin{multipleChoice}
\choice[correct]{no solution}
\choice{a unique solution}
\choice{more than one solution}
\end{multipleChoice}
\begin{problem}
%Correct, because the equations are $\answer[format=string]{equivalent}$.
Correct, because the lines are $\answer[format=string]{parallel}$.
%Correct!  And the solution is $x=\answer{7}
\end{problem}
\end{problem}

\begin{problem}
The simultaneous equations  $2x-3y=5$ and $-4x+6y=-10$ have
\begin{multipleChoice}
\choice{no solution}
\choice{a unique solution}
\choice[correct]{more than one solution}
\end{multipleChoice}
\begin{problem}
Correct, because the equations are $\answer[format=string]{equivalent}$.
%Correct, because the lines are $\answer[format=string]{parallel}$.
%Correct!  And the solution is $x=\answer{7}
\end{problem}
\end{problem}

  
%\item Find the intersection of the lines $2x-3y=4$ and $3x-5y=3$.  
%\item Find the intersection of the lines $2x-3y=4$ and $-4x+6y=-8$.
%\item Find the intersection of the lines $2x-3y=4$ and $-4x+6y=5$.


\end{document}


