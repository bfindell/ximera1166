\documentclass[nooutcomes]{ximera}
%\documentclass[space,handout,nooutcomes]{ximera}

% For preamble materials

\usepackage{pgf,tikz}
\usepackage{mathrsfs}
\usetikzlibrary{arrows}
\usepackage{framed}
\usepackage{amsmath}
\pgfplotsset{compat=1.17}

\def\fixnote#1{\begin{framed}{\textcolor{red}{Fix note: #1}}\end{framed}}  % Allows insertion of red notes about needed edits
%\def\fixnote#1{}

\def\detail#1{{\textcolor{blue}{Detail: #1}}}   

\pdfOnly{\renewenvironment{image}[1][]{\begin{center}}{\end{center}}}

\graphicspath{
  {./}
  {chapter1/}
  {chapter2/}
  {chapter4/}
  {proofs/}
  {graphics/}
  {../graphics/}
}

\newenvironment{sectionOutcomes}{}{}


%%% This set of code is all of our user defined commands
\newcommand{\bysame}{\mbox{\rule{3em}{.4pt}}\,}
\newcommand{\N}{\mathbb N}
\newcommand{\C}{\mathbb C}
\newcommand{\W}{\mathbb W}
\newcommand{\Z}{\mathbb Z}
\newcommand{\Q}{\mathbb Q}
\newcommand{\R}{\mathbb R}
\newcommand{\A}{\mathbb A}
\newcommand{\D}{\mathcal D}
\newcommand{\F}{\mathcal F}
\newcommand{\ph}{\varphi}
\newcommand{\ep}{\varepsilon}
\newcommand{\aph}{\alpha}
\newcommand{\QM}{\begin{center}{\huge\textbf{?}}\end{center}}

\renewcommand{\le}{\leqslant}
\renewcommand{\ge}{\geqslant}
\renewcommand{\a}{\wedge}
\renewcommand{\v}{\vee}
\renewcommand{\l}{\ell}
\newcommand{\mat}{\mathsf}
\renewcommand{\vec}{\mathbf}
\renewcommand{\subset}{\subseteq}
\renewcommand{\supset}{\supseteq}
%\renewcommand{\emptyset}{\varnothing}
%\newcommand{\xto}{\xrightarrow}
%\renewcommand{\qedsymbol}{$\blacksquare$}
%\newcommand{\bibname}{References and Further Reading}
%\renewcommand{\bar}{\protect\overline}
%\renewcommand{\hat}{\protect\widehat}
%\renewcommand{\tilde}{\widetilde}
%\newcommand{\tri}{\triangle}
%\newcommand{\minipad}{\vspace{1ex}}
%\newcommand{\leftexp}[2]{{\vphantom{#2}}^{#1}{#2}}

%% More user defined commands
\renewcommand{\epsilon}{\varepsilon}
\renewcommand{\theta}{\vartheta} %% only for kmath
\renewcommand{\l}{\ell}
\renewcommand{\d}{\, d}
\newcommand{\ddx}{\frac{d}{dx}}
\newcommand{\dydx}{\frac{dy}{dx}}


\usepackage{bigstrut}


%\usepackage{tikz}


\title{Transformations}
\author{Bart Snapp and Brad Findell}
\begin{document}
\begin{abstract}
Short-answer problems about transformations. 
\end{abstract}
\maketitle



%\begin{question}
%What is required to specify a translation?  
%\begin{freeResponse}
%\begin{hint}
%A vector.  Or (equivalently) a magnitude and a direction.  
%\end{hint}
%\end{freeResponse}
%\end{question}

%\begin{question}
%What is required to specify a rotation? 
%\begin{freeResponse}
%\begin{hint}
%A center and an angle (assuming an agreement about the direction of rotation).  
%\end{hint}
%\end{freeResponse}
%\end{question}

%\begin{question}
%What is required to specify a reflection?  
%\begin{freeResponse}
%\begin{hint}
%A line.  
%\end{hint}
%\end{freeResponse}
%\end{question}

\begin{question}
Use adjacent angles to prove that vertical angles are equal.    
\end{question}

\begin{question}
Now use rotations to prove that vertical angles are equal.

\end{question}

\begin{question}
Prove that alternate interior angles and corresponding angles of a transversal with respect to a pair of parallel lines are equal.
\end{question}

\begin{question}
Prove that the sum of the interior angles of a triangle is $180^\circ$.
\end{question}

\begin{question}
Prove: If a pair of alternate interior angles or a pair of corresponding angles of a transversal with respect to two lines are equal, then the lines are parallel.
\begin{freeResponse}
\begin{hint}
\end{hint}
\end{freeResponse}
\end{question}

\begin{question}
Prove: If two parallel lines are cut by a transversal, then alternate interior angles are congruent.  
\begin{freeResponse}
\begin{hint}
Let the two parallel lines be $\ell$ and $k$.  And let $P$ and $Q$ be the intersections between the transversal and $\ell$ and $k$, respectively.  Let $M$ be the midpoint of $\overline{PQ}$.  Rotate $\ell$ and the transversal $180^\circ$ about $M$.  $R(Q)=P$.  We proved previously that such a rotation will map $\ell$ onto a parallel line.  That lines contains $P$.  And because the parallel to $k$ through $P$ is unique, $R(k)=\ell$.  % Need to finish. 
\end{hint}
\end{freeResponse}
\end{question}


\end{document}
