% Section 1.1 Anatomy of Figures

\documentclass[nooutcomes]{ximera}
%\documentclass[space,handout,nooutcomes]{ximera}

% For preamble materials

\graphicspath{
  {./}
  {chapter1/}
  {chapter2/}
  {chapter4/}
  {math1/}
  {math2/}
}

\usepackage{pgf,tikz}
\usepackage{mathrsfs}
\usetikzlibrary{arrows}
\pgfplotsset{compat=1.16}


\newcommand{\N}{\mathbb N}
\newcommand{\W}{\mathbb W}
\newcommand{\C}{\mathbb C}
\newcommand{\Z}{\mathbb Z}
\newcommand{\Q}{\mathbb Q}
\newcommand{\R}{\mathbb R}




\title{Lines in a Triangle}
\author{Brad Findell}
\begin{document}
\begin{abstract}
Short-answer questions about lines in a triangle. 
\end{abstract}
\maketitle

Adjust the figures to fit the given conditions within \textbf{eyeball accuracy}.  Then enter the requested measurements.  


% Lines in a triangle
\begin{problem}
\begin{center}  
\geogebra{q32gyaud}{500}{390}  
\end{center}
In $\triangle ABC$ above, move point $D$ to make the following measurements.  \textbf{Enter -1 if it is not possible.}   
\begin{enumerate}
\item When $\overline{BD}$ is a median, $AD=\answer[tolerance=.08]{2.25}$.
\begin{hint}
A median is drawn from a vertex to the midpoint of the opposite side.
\end{hint}
\item When $\overline{BD}$ is a angle bisector, $AD=\answer[tolerance=.08]{2.78}$.
\begin{hint}
An angle bisector cuts an angle in half.  Focus near the vertex of the angle rather than near $D$.
\end{hint}
\item When $\overline{BD}$ is a perpendicular bisector, $AD=\answer{-1}$.
\begin{hint}
An perpendicular bisector cuts an segment in half and is perpendicular to it. \textbf{Enter -1 if it is not possible.} 
\end{hint}
\item When $\overline{BD}$ is an altitude, $AD=\answer[tolerance=.11]{6.46}$.
\begin{hint}
An altitude contains a vertex and is perpendicular to the line containing the opposite side. \textbf{Enter -1 if it is not possible.} 
\end{hint}
\end{enumerate}
\end{problem}



\end{document}

