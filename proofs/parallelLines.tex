\documentclass[nooutcomes]{ximera}
%\documentclass[space,handout,nooutcomes]{ximera}

% For preamble materials

\graphicspath{
  {./}
  {chapter1/}
  {chapter2/}
  {chapter4/}
  {math1/}
  {math2/}
}

\usepackage{pgf,tikz}
\usepackage{mathrsfs}
\usetikzlibrary{arrows}
\pgfplotsset{compat=1.16}


\newcommand{\N}{\mathbb N}
\newcommand{\W}{\mathbb W}
\newcommand{\C}{\mathbb C}
\newcommand{\Z}{\mathbb Z}
\newcommand{\Q}{\mathbb Q}
\newcommand{\R}{\mathbb R}




\title{Parallel Lines}
\author{Brad Findell}
\begin{document}
\begin{abstract}
Proofs updated.
\end{abstract}
\maketitle

\begin{axiom}
Parallel postulate (uniqueness of parallels):  Given a line and a point not on the line, there is exactly one line through the given point parallel to the given line.  
\end{axiom}


\begin{theorem}
If two parallel lines are cut by a transversal, alternate interior angles and corresponding angles are congruent.
\end{theorem}

\begin{problem}
\begin{proof}
Given that parallel lines $m$ and $n$ are cut by transversal $k$, prove that alternate interior angles are congruent.  

\begin{image}
\definecolor{uuuuuu}{rgb}{0.2667,0.2667,0.2667}
\definecolor{qqqqff}{rgb}{0.,0.,1.}
\begin{tikzpicture}[line cap=round,line join=round,>=triangle 45,x=1.0cm,y=1.0cm]
\clip(-0.2,1.7) rectangle (7.8,6.9);
\draw [line width=0.8pt,domain=-0.22:7.8] plot(\x,{(-0.--4.*\x)/2.});
\draw [line width=0.8pt,domain=-0.22:7.8] plot(\x,{(-10.--4.*\x)/2.});
\draw [line width=0.8pt,domain=-0.22:7.8] plot(\x,{(--10.--1.*\x)/3.});
\begin{scriptsize}
\draw [fill=qqqqff] (0.,0.) circle (1.5pt);
\draw [fill=qqqqff] (2.,4.) circle (1.5pt);
\draw[color=qqqqff] (2.1,3.8) node {$B$};
\draw[color=qqqqff] (1.7,3.7) node {$1$};
\draw[color=qqqqff] (2.35,4.3) node {$2$};
\draw[color=black] (0.77,2.21) node {$m$};
\draw [fill=qqqqff] (5.,5.) circle (1.5pt);
\draw[color=qqqqff] (5.1,4.8) node {$C$};
\draw[color=qqqqff] (4.65,4.7) node {$3$};
\draw[color=qqqqff] (4.8,5.2) node {$4$};
\draw[color=black] (3.8,2.19) node {$n$};
\draw[color=black] (0.82,3.91) node {$k$};
\draw [fill=qqqqff] (3.5,4.5) circle (1.5pt);
\draw[color=qqqqff] (3.64,4.79) node {$P$};
\end{scriptsize}
\end{tikzpicture}
\end{image}

Let $B$ and $C$ be the intersections of transversal $k$ with lines $m$ and $n$, respectively. Let $P$ be the midpoint of $\overline{BC}$.  

\begin{enumerate}
\item Rotate $180^\circ$ about $P$, which takes $k$ to \wordChoice{\choice[correct]{itself}\choice{$m$}\choice{$n$}}.  
\item The rotation maps $B$ to $\answer{C}$ because $PB = PC$ and the rotation preserves distances.  
\item Because $P$ is not on $m$, the rotation maps $m$ to a parallel line through $C$, which must be \wordChoice{\choice{$k$}\choice{$m$}\choice[correct]{$n$}} by the uniqueness of parallels.
%\item The rotation maps $n$ to \wordChoice{\choice{$k$}\choice[correct]{$m$}\choice{$n$}} by the same reasoning.
\item Thus, the rotation maps $\angle 2$ to \wordChoice{\choice{$\angle 1$}\choice{$\angle 2$}\choice[correct]{$\angle 3$}\choice{$\angle 4$}}.  These \wordChoice{\choice{corresponding}\choice[correct]{alternate interior}\choice{supplementary}\choice{vertical}\choice{opposite}}
angles must be congruent because the rotation preserves angle measures. 
\end{enumerate}
\end{proof}

\textbf{Note}: The congruence of corresponding angles now follows from the congruence of 
vertical angles.  But the next problem takes another approach, using a translation.  
\end{problem}

\begin{problem}
\begin{proof}
Given that parallel lines $m$ and $n$ are cut by transversal $k$, prove that corresponding angles are congruent.  

\begin{image}
\definecolor{uuuuuu}{rgb}{0.2667,0.2667,0.2667}
\definecolor{qqqqff}{rgb}{0.,0.,1.}
\begin{tikzpicture}[line cap=round,line join=round,>=triangle 45,x=1.0cm,y=1.0cm]
\clip(-0.2,1.7) rectangle (7.8,6.9);
\draw [line width=0.8pt,domain=-0.22:7.8] plot(\x,{(-0.--4.*\x)/2.});
\draw [line width=0.8pt,domain=-0.22:7.8] plot(\x,{(-10.--4.*\x)/2.});
\draw [line width=0.8pt,domain=-0.22:7.8] plot(\x,{(--10.--1.*\x)/3.});
\begin{scriptsize}
\draw [fill=qqqqff] (0.,0.) circle (1.5pt);
\draw [fill=qqqqff] (2.,4.) circle (1.5pt);
\draw[color=qqqqff] (2.1,3.8) node {$B$};
\draw[color=qqqqff] (1.7,3.7) node {$1$};
\draw[color=qqqqff] (2.35,4.3) node {$2$};
\draw[color=black] (0.77,2.21) node {$m$};
\draw [fill=qqqqff] (5.,5.) circle (1.5pt);
\draw[color=qqqqff] (5.1,4.8) node {$C$};
\draw[color=qqqqff] (4.65,4.7) node {$3$};
\draw[color=qqqqff] (4.8,5.2) node {$4$};
\draw[color=black] (3.8,2.19) node {$n$};
\draw[color=black] (0.82,3.91) node {$k$};
%\draw [fill=qqqqff] (3.5,4.5) circle (1.5pt);
%\draw[color=qqqqff] (3.64,4.79) node {$P$};
\end{scriptsize}
\end{tikzpicture}
\end{image}

Let $B$ and $C$ be the intersections of transversal $k$ with lines $m$ and $n$, respectively. 

\begin{enumerate}
\item Translate to the right along line $k$ by distance $BC$, which takes $k$ to \wordChoice{\choice[correct]{itself}\choice{$m$}\choice{$n$}}.
\item The translation maps $B$ to $\answer{C}$, and it maps $m$ to \wordChoice{\choice{$k$}\choice{$m$}\choice[correct]{$n$}} because the translation maintains parallels, and there is a unique parallel to $m$ through $C$.
\item The translation maps $\angle 1$ to \wordChoice{\choice{$\angle 1$}\choice{$\angle 2$}\choice[correct]{$\angle 3$}\choice{$\angle 4$}}.  These \wordChoice{\choice[correct]{corresponding}\choice{alternate interior}\choice{supplementary}\choice{vertical}\choice{opposite}}
angles must be congruent because the translation preserves angle measures. 
\end{enumerate}
\end{proof}
\end{problem}

\begin{theorem}
If two lines are cut by a transversal so that alternate interior angles are congruent, then the lines are parallel. 

\begin{question}
This theorem is the $\answer[format=string]{converse}$ of the previous theorem about alternate interior angles.
\end{question}
\end{theorem}

\begin{problem}
\begin{proof}
Given that $m$ and $n$ are cut by transversal $k$ with alternate interior angles congruent, prove that lines $m$ and $n$ are parallel.  

\begin{image}
\definecolor{uuuuuu}{rgb}{0.2667,0.2667,0.2667}
\definecolor{qqqqff}{rgb}{0.,0.,1.}
\begin{tikzpicture}[line cap=round,line join=round,>=triangle 45,x=1.0cm,y=1.0cm]
\clip(-0.2,1.7) rectangle (7.8,6.9);
\draw [line width=0.8pt,domain=-0.22:7.8] plot(\x,{(-0.--4.*\x)/2.});
\draw [line width=0.8pt,domain=-0.22:7.8] plot(\x,{(-10.--4.*\x)/2.});
\draw [line width=0.8pt,domain=-0.22:7.8] plot(\x,{(--10.--1.*\x)/3.});
\begin{scriptsize}
\draw [fill=qqqqff] (0.,0.) circle (1.5pt);
\draw [fill=qqqqff] (2.,4.) circle (1.5pt);
\draw[color=qqqqff] (2.1,3.8) node {$B$};
\draw[color=qqqqff] (1.7,3.7) node {$1$};
\draw[color=qqqqff] (2.35,4.3) node {$2$};
\draw[color=black] (0.77,2.21) node {$m$};
\draw [fill=qqqqff] (5.,5.) circle (1.5pt);
\draw[color=qqqqff] (5.1,4.8) node {$C$};
\draw[color=qqqqff] (4.65,4.7) node {$3$};
\draw[color=qqqqff] (4.8,5.2) node {$4$};
\draw[color=black] (3.8,2.19) node {$n$};
\draw[color=black] (0.82,3.91) node {$k$};
\draw [fill=qqqqff] (3.5,4.5) circle (1.5pt);
\draw[color=qqqqff] (3.64,4.79) node {$P$};
\end{scriptsize}
\end{tikzpicture}
\end{image}

Let $B$ and $C$ be the intersections of transversal $k$ with lines $m$ and $n$, respectively. Let $P$ be the midpoint of $\overline{BC}$. 

\begin{enumerate}
\item Rotate $180^\circ$ about $P$, which takes $k$ to \wordChoice{\choice[correct]{itself}\choice{$m$}\choice{$n$}}, and which swaps $B$ and $\answer{C}$ because distances are preserved.  
\item Because $\angle 2 \cong \angle 3$ and because a side of $\angle 2$ (i.e., $\overrightarrow{BP}$) is mapped to a side of $\angle 3$ (i.e., \wordChoice{\choice[correct]{$\overrightarrow{CP}$}\choice{$\overrightarrow{PC}$}\choice{$\overrightarrow{BP}$}}), it must be that the other side of $\angle 2$ (which lies on $m$) is mapped to the other side of $\angle 3$ (which lies on line $\answer{n}$).  Thus, $n$ is the image of $m$.  
\item Because $P$ is not on $m$, the $180^\circ$ rotation maps $m$ to a 
\wordChoice{\choice{transversal}\choice[correct]{parallel}\choice{vertical}\choice{perpendicular}} line through $C$.  Thus, $n$ must be parallel to $m$.  
\end{enumerate}
\end{proof}
\end{problem}

\end{document}
