\documentclass[nooutcomes]{ximera}
%\documentclass[space,handout,nooutcomes]{ximera}

% For preamble materials

\graphicspath{
  {./}
  {chapter1/}
  {chapter2/}
  {chapter4/}
  {math1/}
  {math2/}
}

\usepackage{pgf,tikz}
\usepackage{mathrsfs}
\usetikzlibrary{arrows}
\pgfplotsset{compat=1.16}


\newcommand{\N}{\mathbb N}
\newcommand{\W}{\mathbb W}
\newcommand{\C}{\mathbb C}
\newcommand{\Z}{\mathbb Z}
\newcommand{\Q}{\mathbb Q}
\newcommand{\R}{\mathbb R}




\title{Chapter 5 Review}
\author{Bart Snapp and Brad Findell}
\begin{document}
\begin{abstract}
Review problems for sections 5.1 and 5.2.
\end{abstract}
\maketitle


%\begin{problem}
%Right triangle trigonometry.  
%\begin{enumerate}
%\item Suppose $\alpha$ is in the second quadrant and $\sin\alpha = 3/4$.  Find $\cos\alpha$ and $\tan\alpha$.
%\item Suppose $\beta$ is in the third quadrant and $\cos\beta = -2/3$.  Find $\sin\beta$ and $\tan\beta$.
%\item Suppose $\gamma$ is in the fourth quadrant and $\tan\gamma = -2/3$.  Find $\sin\gamma$ and $\cos\gamma$.
%\end{enumerate}
%\end{problem}


\begin{problem}
Write an equation of the line through $(2,4)$ parallel to $5x-3y=1$.  
Now write an equation of the line through $(x_1,y_1)$ parallel to $ax+by=c$. 
\end{problem}

\begin{problem}
Write an equation of the line through $(2,4)$ perpendicular to $5x-3y=1$.  
Now write an equation of the line through $(x_1,y_1)$ perpendicular to $ax+by=c$. 
\end{problem}

\begin{problem}
Intersections of lines.  
\begin{enumerate}
\item Find the intersection of the lines $2x-3y=4$ and $3x-5y=3$.  
\item Find the intersection of the lines $2x-3y=4$ and $-4x+6y=-8$.
\item Find the intersection of the lines $2x-3y=4$ and $-4x+6y=5$.
\item How might you have predicted in advance how many solutions to expect for each previous system of equations?
\item Use algebra to help explain why lines intersect in zero, one, or infinitely many points.  (You know this geometrically, of course.  Here you demonstrate how algebra gives the same result.)  Indicate clearly the algebraic conditions
for which you get zero, one, or infinitely many points.  
\end{enumerate}
\end{problem}

\begin{problem}
Suppose you have a rectangle with vertices at $(0,0)$, $(a,0)$,
$(a,b)$ and $(0,b)$. Use algebra to prove that the diagonals have the
same length.
\end{problem}



\end{document}



