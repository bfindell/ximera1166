\documentclass[nooutcomes]{ximera}
%\documentclass[space,handout,nooutcomes]{ximera}

% For preamble materials

\usepackage{pgf,tikz}
\usepackage{mathrsfs}
\usetikzlibrary{arrows}
\usepackage{framed}
\usepackage{amsmath}
\pgfplotsset{compat=1.17}

\def\fixnote#1{\begin{framed}{\textcolor{red}{Fix note: #1}}\end{framed}}  % Allows insertion of red notes about needed edits
%\def\fixnote#1{}

\def\detail#1{{\textcolor{blue}{Detail: #1}}}   

\pdfOnly{\renewenvironment{image}[1][]{\begin{center}}{\end{center}}}

\graphicspath{
  {./}
  {chapter1/}
  {chapter2/}
  {chapter4/}
  {proofs/}
  {graphics/}
  {../graphics/}
}

\newenvironment{sectionOutcomes}{}{}


%%% This set of code is all of our user defined commands
\newcommand{\bysame}{\mbox{\rule{3em}{.4pt}}\,}
\newcommand{\N}{\mathbb N}
\newcommand{\C}{\mathbb C}
\newcommand{\W}{\mathbb W}
\newcommand{\Z}{\mathbb Z}
\newcommand{\Q}{\mathbb Q}
\newcommand{\R}{\mathbb R}
\newcommand{\A}{\mathbb A}
\newcommand{\D}{\mathcal D}
\newcommand{\F}{\mathcal F}
\newcommand{\ph}{\varphi}
\newcommand{\ep}{\varepsilon}
\newcommand{\aph}{\alpha}
\newcommand{\QM}{\begin{center}{\huge\textbf{?}}\end{center}}

\renewcommand{\le}{\leqslant}
\renewcommand{\ge}{\geqslant}
\renewcommand{\a}{\wedge}
\renewcommand{\v}{\vee}
\renewcommand{\l}{\ell}
\newcommand{\mat}{\mathsf}
\renewcommand{\vec}{\mathbf}
\renewcommand{\subset}{\subseteq}
\renewcommand{\supset}{\supseteq}
%\renewcommand{\emptyset}{\varnothing}
%\newcommand{\xto}{\xrightarrow}
%\renewcommand{\qedsymbol}{$\blacksquare$}
%\newcommand{\bibname}{References and Further Reading}
%\renewcommand{\bar}{\protect\overline}
%\renewcommand{\hat}{\protect\widehat}
%\renewcommand{\tilde}{\widetilde}
%\newcommand{\tri}{\triangle}
%\newcommand{\minipad}{\vspace{1ex}}
%\newcommand{\leftexp}[2]{{\vphantom{#2}}^{#1}{#2}}

%% More user defined commands
\renewcommand{\epsilon}{\varepsilon}
\renewcommand{\theta}{\vartheta} %% only for kmath
\renewcommand{\l}{\ell}
\renewcommand{\d}{\, d}
\newcommand{\ddx}{\frac{d}{dx}}
\newcommand{\dydx}{\frac{dy}{dx}}


\usepackage{bigstrut}


%\usepackage{tikz}


\title{Scaling in 3D}
\author{Brad Findell}
\begin{document}
\begin{abstract}
Short-answer problems about scaling in 3 dimensions. 
\end{abstract}
\maketitle



%\textbf{Definition.} Under a \textbf{dilation} about center $O$ and scale factor $r>0$, the image of $P$ is 
%a point $Q$ so that $Q$ lies on \wordChoice{\choice{segment}\choice[correct]{ray}\choice{line}} 
%$\answer[format=string]{OP}$ % $\overrightarrow{OP}$ 
%and $OQ=\answer[format=string]{rOP}$.  The image of $O$ is $\answer[format=string]{O}$. 

\section{Length, Area, and Volume Under Scaling}
In this section, we explore what happens to length, area, volume, and other measures under scaling.  
 
To explore how measures of figures change under scaling and non-scaling transformations, here are some useful strategies: 
\begin{itemize}
\item Cutting the figures and rearranging the pieces.  
\item Using ``rep-tiles.'' 
\item Using known formulas for perimeters, areas, volumes, or surface areas. 
\item Approximating with segments, squares, or cubes.  
\item Shearing. 
\end{itemize}

% What about angles, perimeter, surface area, weight, and temperature
% Scaling versus stretching (other examples of non-scaling)
% Aspect ratio

\begin{question}
Fiona's fudge is shipped in boxes that are right rectangular prisms, all similar to 3 cm by 4 cm by 5 cm.  Compute the following measures: 
\begin{enumerate}
\item The volume of the box = $\answer{60}$ cubic centimeters.
\item The volume of a box scaled by $k$ = $\answer{60k^3}$ cubic centimeters.
\item The surface area of the box = $\answer{12+12+15+15+20+20}$ square centimeters. 
\item The surface area of the box a box scaled by $k$ = $\answer{(12+12+15+15+20+20)k^2}$ square centimeters. 
\item The length + girth of the box = $\answer{5 + 3 + 3 + 4 + 4}$ cm.
\item The length + girth of a box scaled by $k$ = $\answer{(5 + 3 + 3 + 4 + 4)k}$ cm. 
\begin{hint}The U.S. postal services measures packages by ``length + girth.''  For a right rectangular prism, the length is the longest measure; the girth is the distance around the other two measures.  \end{hint}
\end{enumerate}
\end{question}

\begin{question}
Given a cylinder of radius $5$ and height $8$, how will the following volumes compare?  
\begin{enumerate}
\item A cylinder with $k$ times the radius and the same height will have  $\answer{k^2}$ times the volume.  
\item A cylinder with the same radius and $k$ times the height will have  $\answer{k}$ times the volume. 
\item A cylinder with $k$ times the radius and $k$ times the height will have  $\answer{k^3}$ times the volume. 
\end{enumerate}
\end{question}

\begin{question}
Which of the following cylinders are similar to the given cylinder of radius 5 and height 8?  
\begin{selectAll}
\choice{A cylinder with $k$ times the radius and the same height?}
\choice{A cylinder with the same radius and $k$ times the height?}
\choice[correct]{A cylinder with $k$ times the radius and $k$ times the height?}
\choice{None of the above}
\end{selectAll}

The similar cylinder will have surface area $\answer{k^2}$ times the surface area of the original cylinder.  
\end{question}

\begin{problem}
Some drugs work best when dosages are proportional to body surface area.  Other drugs work best when dosages are proportional to blood volume.  A typical adult male (5 ft 10 in, 175 lbs)  has a body surface area of about 2 square meters and about 5 liters of blood.  Scale these values up to estimate LeBron's body surface area and blood volume.  (Reminder:  LeBron is 6 ft 8 in, 250 lbs.)

LeBron's body surface area = $\answer{2(8/7)^2}$ square meters. 

LeBron's blood volume = $\answer{5(8/7)^3}$ liters. 
\end{problem}

%\item LeBron James wears size 16 shoe.  Bart says a 5 ft 10 in version of LeBron would have a foot 7/8 as long.  Does this make sense?  Explain why or why not.  If not, give a better estimate.  
%\answer{This is fine because foot length is proportional to height.}
%\item Bart says that the scaled-down LeBron would wear size 14 shoe because 14 is 7/8 of 16.  Does this make sense?  Explain why or why not.  If not, give a better estimate of the scaled version's shoe size.  
%\answer{This is wrong because shoe size is not proportional to foot length.  Somehow suggest that students use Internet data to create a linear function relating foot length to shoe size, and they should get a result close to size 10, which is more sensible.} 

\begin{problem}
Consider a version of LeBron that is $d$ times as tall.  How would following quantities compare between the scaled version and the real LeBron:   leather in the sole of a shoe, shoe size, inseam, fabric in a T-shirt, lung capacity, neck circumference, and hat size?  Explain briefly.  

Cool fact:  The size of a hat is the diameter (in inches) of the hat when it is reshaped into a circle.  Most adults have hat sizes between $6\frac{3}{4}$ and $8$.
\begin{freeResponse}
\begin{hint}
\begin{itemize}
\item Inseam and neck circumference are lengths, so they will be $d$ times as much.  
\item Hat size is proportional to head circumference, which is a length, so it will be $d$ times as much.  
\item Shoe sizes are not really lengths, and men's, women's, and children's shoe sizes are all different.  All three vary linearly with foot length but none are proportional to actual length measurements.   
\item Leather in the sole of the shoe and fabric in the T-shirt are both about area because their thicknesses stay constant, so they will be $d^2$ times as much. 
\item Lung capacity is about volume, so it will be $d^3$ times as much.  
\end{itemize}
\end{hint}
\end{freeResponse}
\end{problem}


\begin{problem}
A typical adult male gorilla is about 5.5 feet tall and weighs about 400 pounds. Suppose King Kong was about 22 feet tall and proportioned like a typical adult male gorilla.
\begin{enumerate}
\item What is the scale factor between from the typical gorilla to King Kong?  $\answer{22/5.5}$
\item King Kong's weight $\approx \answer{(400)4^3}$  pounds.

Briefly explain your reasoning.  \begin{hint}Weight is proportional to volume.\end{hint}
\item The circumference of the neck of a typical adult male gorilla is 36 inches. King Kong's neck circumference $\approx \answer{(36)4}$ inches. 

Briefly explain your reasoning.  \begin{hint}Circumference is a length.\end{hint}
\item Suppose an Ohio State sweatshirt for a typical adult male gorilla requires 3 square yards of fabric.  An Ohio State sweatshirt for 
King Kong would need $\approx \answer{(3)4^2}$ square yards of fabric. 

Briefly explain your reasoning.  \begin{hint}Fabric is about area.\end{hint}
\end{enumerate}
\end{problem}



\end{document}
