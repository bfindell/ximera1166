% Section 2.2 Anatomy of Figures

\documentclass[nooutcomes]{ximera}
%\documentclass[space,handout,nooutcomes]{ximera}

% For preamble materials

\usepackage{pgf,tikz}
\usepackage{mathrsfs}
\usetikzlibrary{arrows}
\usepackage{framed}
\usepackage{amsmath}
\pgfplotsset{compat=1.17}

\def\fixnote#1{\begin{framed}{\textcolor{red}{Fix note: #1}}\end{framed}}  % Allows insertion of red notes about needed edits
%\def\fixnote#1{}

\def\detail#1{{\textcolor{blue}{Detail: #1}}}   

\pdfOnly{\renewenvironment{image}[1][]{\begin{center}}{\end{center}}}

\graphicspath{
  {./}
  {chapter1/}
  {chapter2/}
  {chapter4/}
  {proofs/}
  {graphics/}
  {../graphics/}
}

\newenvironment{sectionOutcomes}{}{}


%%% This set of code is all of our user defined commands
\newcommand{\bysame}{\mbox{\rule{3em}{.4pt}}\,}
\newcommand{\N}{\mathbb N}
\newcommand{\C}{\mathbb C}
\newcommand{\W}{\mathbb W}
\newcommand{\Z}{\mathbb Z}
\newcommand{\Q}{\mathbb Q}
\newcommand{\R}{\mathbb R}
\newcommand{\A}{\mathbb A}
\newcommand{\D}{\mathcal D}
\newcommand{\F}{\mathcal F}
\newcommand{\ph}{\varphi}
\newcommand{\ep}{\varepsilon}
\newcommand{\aph}{\alpha}
\newcommand{\QM}{\begin{center}{\huge\textbf{?}}\end{center}}

\renewcommand{\le}{\leqslant}
\renewcommand{\ge}{\geqslant}
\renewcommand{\a}{\wedge}
\renewcommand{\v}{\vee}
\renewcommand{\l}{\ell}
\newcommand{\mat}{\mathsf}
\renewcommand{\vec}{\mathbf}
\renewcommand{\subset}{\subseteq}
\renewcommand{\supset}{\supseteq}
%\renewcommand{\emptyset}{\varnothing}
%\newcommand{\xto}{\xrightarrow}
%\renewcommand{\qedsymbol}{$\blacksquare$}
%\newcommand{\bibname}{References and Further Reading}
%\renewcommand{\bar}{\protect\overline}
%\renewcommand{\hat}{\protect\widehat}
%\renewcommand{\tilde}{\widetilde}
%\newcommand{\tri}{\triangle}
%\newcommand{\minipad}{\vspace{1ex}}
%\newcommand{\leftexp}[2]{{\vphantom{#2}}^{#1}{#2}}

%% More user defined commands
\renewcommand{\epsilon}{\varepsilon}
\renewcommand{\theta}{\vartheta} %% only for kmath
\renewcommand{\l}{\ell}
\renewcommand{\d}{\, d}
\newcommand{\ddx}{\frac{d}{dx}}
\newcommand{\dydx}{\frac{dy}{dx}}


\usepackage{bigstrut}


\title{Anatomy of Figures}
\author{Bart Snapp and Brad Findell}
\begin{document}
\begin{abstract}
Short-answer problems about centers of triangles. 
\end{abstract}
\maketitle



\begin{problem}
Compare and contrast the idea of ``intersecting sets'' with the
  idea of ``intersecting lines.''
\begin{freeResponse}
\begin{hint}
Geometric figures are sets of points.  The intersection of two geometric figures is the set of point(s) that the figures have in common.  Two lines either intersect in a single point, say $A$, or they do not intersect.  As sets, we would say the intersection is $\{A\}$ or $\{\}$, respectively. 
\end{hint}
\end{freeResponse}
\end{problem}

\begin{problem}
Place three points in the plane. Give a detailed discussion
  explaining how they may or may not be on a line.
\begin{freeResponse}
\begin{hint}
If the three points are distinct, then there are two possibilities:
\begin{itemize}
\item The points are collinear (i.e., they all lie on the same line). 
\item The points are not collinear.  Any two of the points determine a line that does not contain the third point.  For arbitrary points, this is the more likely situation.
\end{itemize}
\end{hint}
\end{freeResponse}
\end{problem}

\begin{problem}
Place three lines in the plane. Give a detailed discussion explaining
  how they may or may not intersect.
\begin{freeResponse}
\begin{hint}
If the three lines are distinct, then there are several possibilities: 
\begin{itemize}
\item The three lines are all parallel. 
\item Two of the three lines are parallel and the third line intersects the first two. 
\item The three lines are concurrent (i.e., they all lie on the same point).
\item The three lines are not parallel and not concurrent.  Any two of the lines intersect in a point that is not on the third line.  For arbitrary lines, this is the most likely situation. 
\end{itemize}
\end{hint}
\end{freeResponse}
\end{problem}

\begin{problem}
Explain how a perpendicular bisector is different from an
  altitude. Draw an example to illustrate the difference.
\begin{freeResponse}
\begin{hint}
A perpendicular bisector goes through the midpoint of a segment and is perpendicular to it.  An altitude to a segment typically does not go through the midpoint of the segment---and it might not intersect the segment at all.  Instead, the altitude goes through another vertex of the figure and is perpendicular to the (extended) line containing the segment.    
\end{hint}
\end{freeResponse}
\end{problem}

\begin{problem}
Explain how a median is different from an angle bisector.  Draw an
  example to illustrate the difference.
\begin{freeResponse}
\begin{hint}
A median extends from a vertex of a triangle to the midpoint of the opposite side (thereby bisecting that side).  The angle bisector, um, bisects the angle.
\end{hint}
\end{freeResponse}
\end{problem}

\begin{problem}
What is the name of the point that is the same distance from all
  three sides of a triangle? Explain your reasoning.
\begin{freeResponse}
\begin{hint}
The points on an angle bisector are equidistant from the sides of the angle.  So the point of concurrency of the angle bisectors of a triangle is equidistant from all three sides of the triangle.  That point of concurrency is called the incenter, as it is the center of the incircle.  
\end{hint}
\end{freeResponse}
\end{problem}

\begin{problem}
What is the name of the point that is the same distance from all
  three vertexes of a triangle? Explain your reasoning.
\begin{freeResponse}
\begin{hint}
The points on an perpendicular bisector are equidistant from the endpoints of the segment.  So the point of concurrency of the perpendicular bisectors of a triangle is equidistant from all three vertices of the triangle.  That point of concurrency is called the circumcenter, as it is the center of the circumcircle.  
\end{hint}
\end{freeResponse}
\end{problem}

\begin{problem}
Could the circumcenter be outside the triangle? If so, draw a
  picture and explain. If not, explain why not using pictures as
  necessary.
\begin{freeResponse}
\begin{hint}
Yes.  Try an obtuse triangle. 
\end{hint}
\end{freeResponse}
\end{problem}

\begin{problem}
Could the orthocenter be outside the triangle? If so, draw a
  picture and explain. If not, explain why not using pictures as
  necessary.
\begin{freeResponse}
\begin{hint}
Yes.  Try an obtuse triangle. 
\end{hint}
\end{freeResponse}
\end{problem}

\begin{problem}
Could the incenter be outside the triangle? If so, draw a
  picture and explain. If not, explain why not using pictures as
  necessary.
\begin{freeResponse}
\begin{hint}
No.  The incenter is the center of the incircle, which is entirely inside the triangle.  
\end{hint}
\end{freeResponse}
\end{problem}

\begin{problem}
Could the centroid be outside the triangle? If so, draw a
  picture and explain. If not, explain why not using pictures as
  necessary.
\begin{freeResponse}
\begin{hint}
No. The centroid is the center of gravity of the triangle, which must lie inside the triangle.  
\end{hint}
\end{freeResponse}
\end{problem}

\begin{problem}
Are there shapes that do not contain their centroid? If so, draw
  a picture and explain. If not, explain why not using pictures as
  necessary.
\begin{freeResponse}
\begin{hint}
Think of figures with holes in the ``middle,'' where the center of gravity might be.  
\end{hint}
\end{freeResponse}
\end{problem}

\begin{problem}
Draw an equilateral triangle. Now draw the lines containing the
  altitudes of this triangle. How many orthocenters do you have as
  intersections of lines in your drawing? Hints:
\begin{enumerate}
\item More than one.
\item How many triangles are in the picture you drew?
\end{enumerate}
\begin{freeResponse}
\begin{hint}
Consider four points: the three original points and their orthocenter.  Any of these points is the orthocenter of the triangle created by the other three points.  (This is quite subtle.  Examine your figure closely.)
\end{hint}
\end{freeResponse}
\end{problem}

\begin{problem}
Given a triangle, construct the circumcenter. Explain the steps
  in your construction.\index{circumcenter}
\begin{freeResponse}
\begin{hint}
The circumcenter is the point of concurrency of the perpendicular bisectors.  Because the perpendicular bisectors are concurrent, it is enough to find the intersection of two of them.  
\end{hint}
\end{freeResponse}
\end{problem}

\begin{problem}
Given a triangle, construct the orthocenter. Explain the steps
  in your construction.
\begin{freeResponse}
\begin{hint}
The orthocenter is the point of concurrency of the altitudes.  Because the altitudes are concurrent, it is enough to find the intersection of two of them.  
\end{hint}
\end{freeResponse}
\end{problem}

\begin{problem}
Given a triangle, construct the incenter. Explain the steps in
  your construction.
\begin{freeResponse}
\begin{hint}
The incenter is the point of concurrency of the angle bisectors.  Because the angle bisectors are concurrent, it is enough to find the intersection of two of them.  
\end{hint}
\end{freeResponse}
\end{problem}

\begin{problem}
Given a triangle, construct the centroid. Explain the steps in
  your construction.
\begin{freeResponse}
\begin{hint}
The centroid is the point of concurrency of the medians.  Because the medians are concurrent, it is enough to find the intersection of two of them.  
\end{hint}
\end{freeResponse}
\end{problem}

\begin{problem}
Given a triangle, construct the incircle. Explain the steps in
  your construction.
\begin{freeResponse}
\begin{hint}
The center of the incircle is the point of concurrency of the angle bisectors.  As stated above, the incenter is equidistant from the sides of the triangle.  To find that distance, which is the radius of the incircle, construct a perpendicular from the incenter to one side of the triangle, and then ``measure'' along that perpendicular bisector. 
\end{hint}
\end{freeResponse}
\end{problem}

\begin{problem}
Given a triangle, construct the circumcircle. Explain the steps
  in your construction.
\begin{freeResponse}
\begin{hint}
The center of the circumcircle is the point of concurrency of the perpendicular bisectors.  As stated above, the circumcenter is equidistant from the vertices of the triangle.  So use that distance as the radius of the circumcircle.  
\end{hint}
\end{freeResponse}
\end{problem}

\begin{problem}
Given a circle, give a construction that finds its center. 
\begin{freeResponse}
\begin{hint}
Any three points on the circle form a triangle.  The circumcenter of that triangle will be the center of the circle.  
\end{hint}
\end{freeResponse}
\end{problem}

\begin{problem}
Where is the circumcenter of a right triangle? Explain your
  reasoning.
\begin{freeResponse}
\begin{hint}
The circumcenter is at the midpoint of the hypotenuse.  The perpendicular bisector of the hypotenuse clearly contains this midpoint.  And it is just a tad harder to see that the perpendicular bisectors of the legs also contains this point.  (To see this, note that a midsegment to a leg lies on the perpendicular bisector of that leg.)
\end{hint}
\end{freeResponse}
\end{problem}

\begin{problem}
Where is the orthocenter of a right triangle? Explain your
  reasoning.
\begin{freeResponse}
\begin{hint}
The orthocenter of a right triangle is the vertex of the right angle.  The altitude to the hypotenuse goes through the vertex of the right angle.  Both legs of the right triangle are also altitudes of that right triangle, so they also go through the vertex of the right angle.  
\end{hint}
\end{freeResponse}
\end{problem}

\begin{problem}
Can you draw a triangle where the circumcenter, orthocenter,
  incenter, and centroid are all the same point?  If so, draw a
  picture and explain. If not, explain why not using pictures as
  necessary.
\begin{freeResponse}
\begin{hint}
Try an equilateral triangle.  
\end{hint}
\end{freeResponse}
\end{problem}

\begin{problem}
True or False: Explain your conclusions.
\begin{enumerate}
\item An altitude of a triangle is always perpendicular to a line
  containing some side of the triangle.
\item An altitude of a triangle always bisects some side of the
  triangle.
\item The incenter is always inside the triangle.
\item The circumcenter, the centroid, and the orthocenter always lie in a line.
\item The circumcenter can be outside the triangle.
\item The orthocenter is always inside the triangle.
\item The centroid is always inside the incircle.
\end{enumerate}
\begin{freeResponse}
\begin{hint}
Don't worry about (d) and (g):
\begin{enumerate}
\item True.  This is part of the definition of an altitude.  %An altitude of a triangle is always perpendicular to a line containing some side of the triangle.
\item False.  Any scalene triangle will do.  % An altitude of a triangle always bisects some side of the triangle.
\item True.  The incenter is the center of the incircle, which lies entirely inside the triangle.  %The incenter is always inside the triangle.
\item True.  An amazing fact.  Try it.  % The circumcenter, the centroid, and the orthocenter always lie in a line.
\item True.  Try an obtuse triangle.  % The circumcenter can be outside the triangle.
\item False.  Try an obtuse triangle.  % The orthocenter is always inside the triangle.
\item False.  Try a thin triangle.  % The centroid is always inside the incircle.
\end{enumerate}
\end{hint}
\end{freeResponse}
\end{problem}

\begin{problem}
Given 3 distinct points not all in a line, construct a circle
  that passes through all three points. Explain the steps in your
  construction.
\begin{freeResponse}
\begin{hint}
The three points form a triangle.  Construct the circumcircle of that triangle.  
\end{hint}
\end{freeResponse}
\end{problem}



\end{document}