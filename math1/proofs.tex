\documentclass[nooutcomes]{ximera}
%\documentclass[space,handout,nooutcomes]{ximera}

% For preamble materials

\graphicspath{
  {./}
  {chapter1/}
  {chapter2/}
  {chapter4/}
  {math1/}
  {math2/}
}

\usepackage{pgf,tikz}
\usepackage{mathrsfs}
\usetikzlibrary{arrows}
\pgfplotsset{compat=1.16}


\newcommand{\N}{\mathbb N}
\newcommand{\W}{\mathbb W}
\newcommand{\C}{\mathbb C}
\newcommand{\Z}{\mathbb Z}
\newcommand{\Q}{\mathbb Q}
\newcommand{\R}{\mathbb R}




\title{Proofs for Math 1}
\author{Brad Findell}
\begin{document}
\begin{abstract}
Proofs. 
\end{abstract}
\maketitle


\begin{problem}
Revision of 2017 Geometry released item 13. 

Two pairs of parallel lines intersect to form a parallelogram as shown.  
\begin{image}
\includegraphics{Q13.png}
\end{image}
Complete the proof that opposite angles of a parallelogram are congruent. 

Proof: 

\begin{enumerate}
\item $\angle 1 \cong \angle 2$ as \wordChoice{\choice{opposite angles}\choice[correct]{alternate interior angles}\choice{corresponding angles}}
for parallel lines \wordChoice{\choice[correct]{$m$ and $n$}\choice{$k$ and $l$}}.

\item $\angle 3 \cong \angle 2$ as \wordChoice{\choice{opposite angles}\choice{alternate interior angles}\choice[correct]{corresponding angles}}for parallel lines \wordChoice{\choice{$m$ and $n$}\choice[correct]{$k$ and $l$}}.
\item Then $\angle 1 \cong \angle 3$ because they are both congruent 
to $\angle 2$. 
\end{enumerate}
\end{problem}

\end{document}
