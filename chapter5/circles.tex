
\documentclass[nooutcomes]{ximera}
%\documentclass[space,handout,nooutcomes]{ximera}

% For preamble materials

\usepackage{pgf,tikz}
\usepackage{mathrsfs}
\usetikzlibrary{arrows}
\usepackage{framed}
\usepackage{amsmath}
\pgfplotsset{compat=1.17}

\def\fixnote#1{\begin{framed}{\textcolor{red}{Fix note: #1}}\end{framed}}  % Allows insertion of red notes about needed edits
%\def\fixnote#1{}

\def\detail#1{{\textcolor{blue}{Detail: #1}}}   

\pdfOnly{\renewenvironment{image}[1][]{\begin{center}}{\end{center}}}

\graphicspath{
  {./}
  {chapter1/}
  {chapter2/}
  {chapter4/}
  {proofs/}
  {graphics/}
  {../graphics/}
}

\newenvironment{sectionOutcomes}{}{}


%%% This set of code is all of our user defined commands
\newcommand{\bysame}{\mbox{\rule{3em}{.4pt}}\,}
\newcommand{\N}{\mathbb N}
\newcommand{\C}{\mathbb C}
\newcommand{\W}{\mathbb W}
\newcommand{\Z}{\mathbb Z}
\newcommand{\Q}{\mathbb Q}
\newcommand{\R}{\mathbb R}
\newcommand{\A}{\mathbb A}
\newcommand{\D}{\mathcal D}
\newcommand{\F}{\mathcal F}
\newcommand{\ph}{\varphi}
\newcommand{\ep}{\varepsilon}
\newcommand{\aph}{\alpha}
\newcommand{\QM}{\begin{center}{\huge\textbf{?}}\end{center}}

\renewcommand{\le}{\leqslant}
\renewcommand{\ge}{\geqslant}
\renewcommand{\a}{\wedge}
\renewcommand{\v}{\vee}
\renewcommand{\l}{\ell}
\newcommand{\mat}{\mathsf}
\renewcommand{\vec}{\mathbf}
\renewcommand{\subset}{\subseteq}
\renewcommand{\supset}{\supseteq}
%\renewcommand{\emptyset}{\varnothing}
%\newcommand{\xto}{\xrightarrow}
%\renewcommand{\qedsymbol}{$\blacksquare$}
%\newcommand{\bibname}{References and Further Reading}
%\renewcommand{\bar}{\protect\overline}
%\renewcommand{\hat}{\protect\widehat}
%\renewcommand{\tilde}{\widetilde}
%\newcommand{\tri}{\triangle}
%\newcommand{\minipad}{\vspace{1ex}}
%\newcommand{\leftexp}[2]{{\vphantom{#2}}^{#1}{#2}}

%% More user defined commands
\renewcommand{\epsilon}{\varepsilon}
\renewcommand{\theta}{\vartheta} %% only for kmath
\renewcommand{\l}{\ell}
\renewcommand{\d}{\, d}
\newcommand{\ddx}{\frac{d}{dx}}
\newcommand{\dydx}{\frac{dy}{dx}}


\usepackage{bigstrut}


\title{Lines, Circles, and Parabolas}
\author{Bart Snapp and Brad Findell}
\begin{document}
\begin{abstract}
These problems provide practice writing and solving equations of lines, circles, and parabolas. 
\end{abstract}
\maketitle


%\begin{problem}
%Write an equation of the line through (1,2) and parallel to $2x-3y=1$.  
%
%\begin{hint} Hint. \end{hint}
%\end{problem}
%
%Write an equation of the line through (1,2) and perpendicular to $2x-3y=1$.  
%
%circle given center 
%complete the square to find center
%
%parabola given focus and directrix (horizontal)
%parabola given focus and directrix (vertical)
%
%Intersection of two lines
%
%Intersection of two circles

With two equations in two unknowns, two common solutions methods are substitution and elimination.  

\section{A Line and a Circle}

When the two equations are of a circle and of a line, a common approach is to solve the line for one of the variables (say, $y$) and then to substituted into the equation of the circle.  This approach yields a \wordChoice{\choice{linear},\choice[correct]{quadratic},\choice{exponential}} equation in \wordChoice{\choice{0},\choice[correct]{1},\choice{2}} variable(s), which can be solved with the $\answer[format=string]{quadratic}$ formula.  

\begin{problem}
Use substitution to solve the following equations simultaneously.  Interpret the solutions.  
\[
(x-3)^2+(y-2)^2 = 14, \qquad  x - y = -4
\]
\[
x  = \answer{1/2} \pm \answer{\sqrt{3}/2}, y = \answer{9/2} \pm \answer{\sqrt{3}/2}
\]
\begin{hint}
The two distinct solutions indicate that line intersects the circle twice in the $xy$-plane.  
\end{hint}
\end{problem}


\begin{problem}
Use substitution to solve the following equations simultaneously.  Interpret the solutions.  
\[
(x-3)^2+(y-3)^2 = 8, \qquad x - y = -4
\]
\[
x  = \answer{1} \pm \answer{0}, y = \answer{5} \pm \answer{0}
\]
\begin{hint}
The single solution indicates that line is tangent to the circle in the $xy$-plane.  
\end{hint}
\end{problem}


\begin{problem}
Solve the following equations simultaneously.  Interpret the solutions.  
\[
(x-3)^2+(y-2)^2 = 12, \qquad x - y = -4
\]
\[
x  = \answer{1/2} \pm \answer{i/2}, y = \answer{9/2} \pm \answer{i/2}
\]
\begin{hint}
The complex solutions indicate that the line does not intersect the circle in the $xy$-plane.  
\end{hint}
\end{problem}

\section{Two Circles}
Given the equations of two circles, to find their intersections, first use the method of elimination.  Subtract one equation from the other, and the result will (usually) be an equation of a line.  Then proceed as above.  


\begin{problem}
Solve the following equations simultaneously.  Interpret the solutions.  
\[
(x-3)^2+(y-2)^2 = 4, \qquad (x-1)^2+(y-1)^2 = 9
\]
\[
x  = \answer{3} \pm \answer{2/\sqrt{5}}, y = \answer{2} \pm \answer{4/\sqrt{5}}
\]
\begin{hint}
The two distinct solutions indicate that circles intersect twice in the $xy$-plane.  
\end{hint}
\end{problem}


\begin{problem}
Solve the following equations simultaneously.  Interpret the solutions.  
\[
(x-3)^2+(y-2)^2 = 4, \qquad (x-1)^2+(y-1)^2 = 19
\]
\[
x  = \answer{5} \pm \answer{i/\sqrt{5}}, y = \answer{3} \pm \answer{2i/\sqrt{5}}
\]
\begin{hint}
The two complex solutions indicate that circles do not intersect in the $xy$-plane.  
\end{hint}
\end{problem}


%\begin{problem}
%Solve the following equations simultaneously
%$$(x-3)^2+(y+2)^2 = 4$$
%$$(x+1)^2+(y-2)^2 = 9$$
%\end{problem}


When finding the intersection of a line and a circle. 

0 points
1 point
2 points

We solve the equation of the line for one of the variables.  Then substitute into the equation of the circle.  This gives a quadratic equation in one variable, which has 0, 1, or 2 roots.  

Because it is a quadratic equation, it always has 2 complex roots, if a double root is counted twice.  [Real roots are also complex roots.]. 

If there are 0 real solutions, then the line and the circle did not intersect. 
If there is 1 real root, then the line is tangent to the circle at a single point.  
If there are 2 real roots, then the line intersects the circle in two points.  



When finding the intersection of two circles

0 points
1 point
2 points
infinitely many points



\end{document}


