\documentclass[nooutcomes]{ximera}
%\documentclass[space,handout,nooutcomes]{ximera}

% For preamble materials

\usepackage{pgf,tikz}
\usepackage{mathrsfs}
\usetikzlibrary{arrows}
\usepackage{framed}
\usepackage{amsmath}
\pgfplotsset{compat=1.17}

\def\fixnote#1{\begin{framed}{\textcolor{red}{Fix note: #1}}\end{framed}}  % Allows insertion of red notes about needed edits
%\def\fixnote#1{}

\def\detail#1{{\textcolor{blue}{Detail: #1}}}   

\pdfOnly{\renewenvironment{image}[1][]{\begin{center}}{\end{center}}}

\graphicspath{
  {./}
  {chapter1/}
  {chapter2/}
  {chapter4/}
  {proofs/}
  {graphics/}
  {../graphics/}
}

\newenvironment{sectionOutcomes}{}{}


%%% This set of code is all of our user defined commands
\newcommand{\bysame}{\mbox{\rule{3em}{.4pt}}\,}
\newcommand{\N}{\mathbb N}
\newcommand{\C}{\mathbb C}
\newcommand{\W}{\mathbb W}
\newcommand{\Z}{\mathbb Z}
\newcommand{\Q}{\mathbb Q}
\newcommand{\R}{\mathbb R}
\newcommand{\A}{\mathbb A}
\newcommand{\D}{\mathcal D}
\newcommand{\F}{\mathcal F}
\newcommand{\ph}{\varphi}
\newcommand{\ep}{\varepsilon}
\newcommand{\aph}{\alpha}
\newcommand{\QM}{\begin{center}{\huge\textbf{?}}\end{center}}

\renewcommand{\le}{\leqslant}
\renewcommand{\ge}{\geqslant}
\renewcommand{\a}{\wedge}
\renewcommand{\v}{\vee}
\renewcommand{\l}{\ell}
\newcommand{\mat}{\mathsf}
\renewcommand{\vec}{\mathbf}
\renewcommand{\subset}{\subseteq}
\renewcommand{\supset}{\supseteq}
%\renewcommand{\emptyset}{\varnothing}
%\newcommand{\xto}{\xrightarrow}
%\renewcommand{\qedsymbol}{$\blacksquare$}
%\newcommand{\bibname}{References and Further Reading}
%\renewcommand{\bar}{\protect\overline}
%\renewcommand{\hat}{\protect\widehat}
%\renewcommand{\tilde}{\widetilde}
%\newcommand{\tri}{\triangle}
%\newcommand{\minipad}{\vspace{1ex}}
%\newcommand{\leftexp}[2]{{\vphantom{#2}}^{#1}{#2}}

%% More user defined commands
\renewcommand{\epsilon}{\varepsilon}
\renewcommand{\theta}{\vartheta} %% only for kmath
\renewcommand{\l}{\ell}
\renewcommand{\d}{\, d}
\newcommand{\ddx}{\frac{d}{dx}}
\newcommand{\dydx}{\frac{dy}{dx}}


\usepackage{bigstrut}


\renewcommand{\Z}{\mathbb Z}

\title{Extra Set Theory Problems}
\author{Bart Snapp and Brad Findell}
\begin{document}
\begin{abstract}
More short-answer problems. 
\end{abstract}
\maketitle

\subsection*{Reminders}
\begin{itemize}
\item Sets are collections of objects such as numbers or points.  The objects are called \emph{elements} of the set, and the order elements are listed is not important.  

\item The notation $\{7, 3\}$ means ``The set containing 7 and 3.''  

\item Note that $\{8\}$ is not the same as the number 8 but rather is a set that contains one element that happens to be a number. 

\item The set containing zero elements, sometimes call the \emph{empty set} is denoted $\{\}$ or $\emptyset$.  

\item The elements of a set can themselves be sets.  

\end{itemize}


\begin{problem}
Indicate the number of elements in each set: 
\begin{enumerate}
\item The set $\{3, 5, 6, 9, 10\}$ has $\answer{5}$ element(s).
\item The set $\{ \{3,2,7\}, \{4,5\}, \{2\}, \emptyset \}$ has $\answer{4}$ element(s).
\item The set $\{ \{ \} \}$ has $\answer{1}$ element(s).
\item The set $\{\}$ has $\answer{0}$ element(s).
\item The set $\emptyset$ has $\answer{0}$ element(s).
\item The set $\{ \emptyset \}$ has $\answer{1}$ element(s).
\end{enumerate}

\end{problem}

\begin{problem}
Indicate whether each statement is true or false: 
\begin{enumerate}
\item $2\in \{3, 2, 5\}$. \wordChoice{\choice[correct]{True}\choice{False}}
\item $2\subseteq \{3, 2, 5\}$. \wordChoice{\choice{True}\choice[correct]{False}}
\item $\{2\}\in \{3, 2, 5\}$.  \wordChoice{\choice{True}\choice[correct]{False}}
\item $\{2\}\subseteq \{3, 2, 5\}$.  \wordChoice{\choice[correct]{True}\choice{False}}
\item $\{ \} \subseteq \{3, 2, 5\}$.  \wordChoice{\choice[correct]{True}\choice{False}}
\end{enumerate}
\begin{feedback}[correct]
Correct.  And the empty set is a subset of \textbf{any} set!  
\end{feedback}
\end{problem}

\begin{problem}
Indicate whether each statement is true or false: 
\begin{enumerate}
\item $\emptyset = \{ \}$. \wordChoice{\choice[correct]{True}\choice{False}}
\item $\emptyset = \{ \emptyset \} $. \wordChoice{\choice{True}\choice[correct]{False}}
\item $\{ \emptyset \} = \{\{\}\}$. \wordChoice{\choice[correct]{True}\choice{False}}
\item $\emptyset \in \{ \emptyset \} $. \wordChoice{\choice[correct]{True}\choice{False}}
\item $\emptyset \subseteq \{ \emptyset \} $. \wordChoice{\choice[correct]{True}\choice{False}}
\item $2 \in \{ \{3,2,7\}, \{4,5\}, \{2\}, \emptyset \}$. \wordChoice{\choice{True}\choice[correct]{False}}
\item $2 \subseteq \{ \{3,2,7\}, \{4,5\}, \{2\}, \emptyset \}$. \wordChoice{\choice{True}\choice[correct]{False}}
\item $\{2\} \in \{ \{3,2,7\}, \{4,5\}, \{2\}, \emptyset \}$. \wordChoice{\choice[correct]{True}\choice{False}}
\item $\{2\} \subseteq \{ \{3,2,7\}, \{4,5\}, \{2\}, \emptyset \}$. \wordChoice{\choice{True}\choice[correct]{False}}
\item $\{\{2\}\} \in \{ \{3,2,7\}, \{4,5\}, \{2\}, \emptyset \}$. \wordChoice{\choice{True}\choice[correct]{False}}
\item $\{\{2\}\} \subseteq \{ \{3,2,7\}, \{4,5\}, \{2\}, \emptyset \}$. \wordChoice{\choice[correct]{True}\choice{False}}
\end{enumerate}

\end{problem}

\begin{problem}
Explain the difference between the symbols $\in$ and $\subseteq$.
\begin{freeResponse}
\begin{hint}
The symbol $\in$ means ``is an element of,'' whereas $\subseteq$ means ``is a subset of.'' 
The notation $X \in Y$ means that $X$ is a single element in the set $Y$.  In this case, $X$ is typically not a set.  The notation $X \subseteq Y$, in contrast, requires that both $X$ and $Y$ are sets and, furthermore, that every element of $X$ is also in $Y$.
\end{hint}
\end{freeResponse}
\end{problem}

\begin{problem}
How is $\{\emptyset\}$ different from $\emptyset$?  
\begin{freeResponse}
\begin{hint}
The empty set, $\emptyset$, is a set that contains no elements.  That is, $\emptyset = \{\}$.  The set $\{\emptyset\}$ contains one element that is itself a set---and that element happens to be the empty set.  We could instead write $\{\{\}\}$, but that looks ugly.
\end{hint}
\end{freeResponse}
\end{problem}



\end{document}
