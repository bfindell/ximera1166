
\documentclass[nooutcomes]{ximera}
%\documentclass[space,handout,nooutcomes]{ximera}

% For preamble materials

\usepackage{pgf,tikz}
\usepackage{mathrsfs}
\usetikzlibrary{arrows}
\usepackage{framed}
\usepackage{amsmath}
\pgfplotsset{compat=1.17}

\def\fixnote#1{\begin{framed}{\textcolor{red}{Fix note: #1}}\end{framed}}  % Allows insertion of red notes about needed edits
%\def\fixnote#1{}

\def\detail#1{{\textcolor{blue}{Detail: #1}}}   

\pdfOnly{\renewenvironment{image}[1][]{\begin{center}}{\end{center}}}

\graphicspath{
  {./}
  {chapter1/}
  {chapter2/}
  {chapter4/}
  {proofs/}
  {graphics/}
  {../graphics/}
}

\newenvironment{sectionOutcomes}{}{}


%%% This set of code is all of our user defined commands
\newcommand{\bysame}{\mbox{\rule{3em}{.4pt}}\,}
\newcommand{\N}{\mathbb N}
\newcommand{\C}{\mathbb C}
\newcommand{\W}{\mathbb W}
\newcommand{\Z}{\mathbb Z}
\newcommand{\Q}{\mathbb Q}
\newcommand{\R}{\mathbb R}
\newcommand{\A}{\mathbb A}
\newcommand{\D}{\mathcal D}
\newcommand{\F}{\mathcal F}
\newcommand{\ph}{\varphi}
\newcommand{\ep}{\varepsilon}
\newcommand{\aph}{\alpha}
\newcommand{\QM}{\begin{center}{\huge\textbf{?}}\end{center}}

\renewcommand{\le}{\leqslant}
\renewcommand{\ge}{\geqslant}
\renewcommand{\a}{\wedge}
\renewcommand{\v}{\vee}
\renewcommand{\l}{\ell}
\newcommand{\mat}{\mathsf}
\renewcommand{\vec}{\mathbf}
\renewcommand{\subset}{\subseteq}
\renewcommand{\supset}{\supseteq}
%\renewcommand{\emptyset}{\varnothing}
%\newcommand{\xto}{\xrightarrow}
%\renewcommand{\qedsymbol}{$\blacksquare$}
%\newcommand{\bibname}{References and Further Reading}
%\renewcommand{\bar}{\protect\overline}
%\renewcommand{\hat}{\protect\widehat}
%\renewcommand{\tilde}{\widetilde}
%\newcommand{\tri}{\triangle}
%\newcommand{\minipad}{\vspace{1ex}}
%\newcommand{\leftexp}[2]{{\vphantom{#2}}^{#1}{#2}}

%% More user defined commands
\renewcommand{\epsilon}{\varepsilon}
\renewcommand{\theta}{\vartheta} %% only for kmath
\renewcommand{\l}{\ell}
\renewcommand{\d}{\, d}
\newcommand{\ddx}{\frac{d}{dx}}
\newcommand{\dydx}{\frac{dy}{dx}}


\usepackage{bigstrut}


\title{Lines}
\author{Bart Snapp and Brad Findell}
\begin{document}
\begin{abstract}
These problems provide practice writing and solving equations of lines. 
\end{abstract}
\maketitle

\begin{problem}
Write an equation of the line through $(2,3)$ and $(5,1)$.  

\begin{prompt}
$y = \answer{3+(1-3)/(5-2)(x-2)}$

Enter integer: \answer[format=integer]{2}

Enter number: \answer[format=number]{3}

Enter real: \answer[format=real]{4}

Enter rational: \answer[format=rational]{5}

Enter complex: \answer[format=complex]{6}

Enter string: \answer[format=string]{7}
\end{prompt}
\end{problem}


\begin{problem}
Write an equation of the line through $(2,4)$ parallel to $5x-3y=1$.  

  \begin{prompt}
    \begin{validator}[(-3*ua==5*ub) && (-3*uc==(5*2-3*4)*ub) && (ub != 0)]
      \(
         \answer[format=expression,id=ua]{5}x + \answer[format=number,id=ub]{-3}y = 
         \answer[format=real,id=uc]{5\cdot2-3\cdot4}
      \)
    \end{validator}
  \end{prompt}
\end{problem}

%\item[(a)] Find a vector $u$ normal to the plane $2x+2y+z=3$.
%  \begin{prompt}
%    \begin{validator}[(ux==uy) && (uy==2*uz) && (uy != 0)]
%      \(
%        u = \left( \answer[format=integer,id=ux]{2}, \answer[format=integer,id=uy]{2}, \answer[format=integer,id=uz]{1} \right)
%      \)
%    \end{validator}
%  \end{prompt}


\begin{problem}
Now write an equation of the line through $(p,q)$ parallel to $ax+by=c$. 

\begin{prompt}
$y = \answer{3+(1-3)/(5-2)(x-2)}$
\end{prompt}
\end{problem}

\begin{problem}
Write an equation of the line through $(2,4)$ perpendicular to $5x-3y=1$.  
Now write an equation of the line through $(p,q)$ perpendicular to $ax+by=c$. 
\end{problem}


%\begin{problem}
%Write an equation of the line through (1,2) and parallel to $2x-3y=1$.  
%
%\begin{hint} Hint. \end{hint}
%\end{problem}
%
%Write an equation of the line through (1,2) and perpendicular to $2x-3y=1$.  
%
%
With two equations in two unknowns, two common solutions methods are substitution and elimination.  



\end{document}


