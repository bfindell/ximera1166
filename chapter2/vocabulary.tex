% Section 2.2 Anatomy of Figures

\documentclass[nooutcomes]{ximera}
%\documentclass[space,handout,nooutcomes]{ximera}

% For preamble materials

\usepackage{pgf,tikz}
\usepackage{mathrsfs}
\usetikzlibrary{arrows}
\usepackage{framed}
\usepackage{amsmath}
\pgfplotsset{compat=1.17}

\def\fixnote#1{\begin{framed}{\textcolor{red}{Fix note: #1}}\end{framed}}  % Allows insertion of red notes about needed edits
%\def\fixnote#1{}

\def\detail#1{{\textcolor{blue}{Detail: #1}}}   

\pdfOnly{\renewenvironment{image}[1][]{\begin{center}}{\end{center}}}

\graphicspath{
  {./}
  {chapter1/}
  {chapter2/}
  {chapter4/}
  {proofs/}
  {graphics/}
  {../graphics/}
}

\newenvironment{sectionOutcomes}{}{}


%%% This set of code is all of our user defined commands
\newcommand{\bysame}{\mbox{\rule{3em}{.4pt}}\,}
\newcommand{\N}{\mathbb N}
\newcommand{\C}{\mathbb C}
\newcommand{\W}{\mathbb W}
\newcommand{\Z}{\mathbb Z}
\newcommand{\Q}{\mathbb Q}
\newcommand{\R}{\mathbb R}
\newcommand{\A}{\mathbb A}
\newcommand{\D}{\mathcal D}
\newcommand{\F}{\mathcal F}
\newcommand{\ph}{\varphi}
\newcommand{\ep}{\varepsilon}
\newcommand{\aph}{\alpha}
\newcommand{\QM}{\begin{center}{\huge\textbf{?}}\end{center}}

\renewcommand{\le}{\leqslant}
\renewcommand{\ge}{\geqslant}
\renewcommand{\a}{\wedge}
\renewcommand{\v}{\vee}
\renewcommand{\l}{\ell}
\newcommand{\mat}{\mathsf}
\renewcommand{\vec}{\mathbf}
\renewcommand{\subset}{\subseteq}
\renewcommand{\supset}{\supseteq}
%\renewcommand{\emptyset}{\varnothing}
%\newcommand{\xto}{\xrightarrow}
%\renewcommand{\qedsymbol}{$\blacksquare$}
%\newcommand{\bibname}{References and Further Reading}
%\renewcommand{\bar}{\protect\overline}
%\renewcommand{\hat}{\protect\widehat}
%\renewcommand{\tilde}{\widetilde}
%\newcommand{\tri}{\triangle}
%\newcommand{\minipad}{\vspace{1ex}}
%\newcommand{\leftexp}[2]{{\vphantom{#2}}^{#1}{#2}}

%% More user defined commands
\renewcommand{\epsilon}{\varepsilon}
\renewcommand{\theta}{\vartheta} %% only for kmath
\renewcommand{\l}{\ell}
\renewcommand{\d}{\, d}
\newcommand{\ddx}{\frac{d}{dx}}
\newcommand{\dydx}{\frac{dy}{dx}}


\usepackage{bigstrut}


\title{Vocabulary Review}
\author{Bart Snapp and Brad Findell}
\begin{document}
\begin{abstract}
Short-answer, multiple-choice, and select-all questions about key vocabulary. 
\end{abstract}
\maketitle

%Useful questions: 
%
%What is regular quadrilateral? 
%Definition of ? 
%Write the Pythagorean theorem. 
%Measure angles. 
%Angle sum in a triangle. 
%Triangulate a figure 



\begin{question}  
A \textbf{median} in a triangle always \dots
\begin{selectAll}  
\choice[correct]{contains the midpoint of a side of the triangle}
\choice{is perpendicular to a side of the triangle}
\choice[correct]{contains a vertex of the triangle}
\choice{is perpendicular to the line containing a side of the triangle}  
\choice{bisects an angle of the triangle}  
\end{selectAll}  
\end{question}

\begin{question}  
An \textbf{altitude} in a triangle always \dots
\begin{selectAll}  
\choice{contains the midpoint of a side of the triangle}
\choice{is perpendicular to a side of the triangle}
\choice[correct]{contains a vertex of the triangle}
\choice[correct]{is perpendicular to the line containing a side of the triangle}  
\choice{bisects an angle of the triangle}  
\end{selectAll}  
\end{question}

\begin{question}  
An \textbf{angle bisector} in a triangle always \dots
\begin{selectAll}  
\choice{contains the midpoint of a side of the triangle}
\choice{is perpendicular to a side of the triangle}
\choice[correct]{contains a vertex of the triangle}
\choice{is perpendicular to the line containing a side of the triangle}  
\choice[correct]{bisects an angle of the triangle}  
\end{selectAll}  
\end{question}

\begin{question}  
A \textbf{perpendicular bisector} in a triangle always \dots
\begin{selectAll}  
\choice[correct]{contains the midpoint of a side of the triangle}
\choice[correct]{is perpendicular to a side of the triangle}
\choice{contains a vertex of the triangle}
\choice{is perpendicular to the line containing a side of the triangle}  
\choice{bisects an angle of the triangle}  
\end{selectAll}  
\end{question}


\begin{question}  
The \textbf{circumcenter} of a triangle is \dots [select all]
  \begin{selectAll}  
    \choice{the point of concurrency of the medians.}  
    \choice{the point of concurrency of the angle bisectors.}  
    \choice[correct]{the point of concurrency of the perpendicular bisectors.}  
    \choice{the point of concurrency of the altitudes.}  
    \choice{the balance point for the triangle.}
    \choice{the center in the triangle.}
    \choice{the center of the incircle.}
    \choice[correct]{the center of the circumcircle.}
    \choice{equidistant from the sides of the triangle.}
    \choice[correct]{equidistant from the vertices of the triangle.}    
  \end{selectAll}  
\end{question}

\begin{question}  
  The \textbf{incenter} of a triangle is \dots [select all]
  \begin{selectAll}  
    \choice{the point of concurrency of the medians.}  
    \choice[correct]{the point of concurrency of the angle bisectors.}  
    \choice{the point of concurrency of the perpendicular bisectors.}  
    \choice{the point of concurrency of the altitudes.}  
    \choice{the balance point for the triangle.}
    \choice{the center in the triangle.}
    \choice[correct]{the center of the incircle.}
    \choice{the center of the circumcircle.}
    \choice[correct]{equidistant from the sides of the triangle.}
    \choice{equidistant from the vertices of the triangle.}    
  \end{selectAll}  
\end{question}

\begin{question}  
  The \textbf{centroid} of a triangle is \dots [select all]
  \begin{selectAll}  
    \choice[correct]{the point of concurrency of the medians.}  
    \choice{the point of concurrency of the angle bisectors.}  
    \choice{the point of concurrency of the perpendicular bisectors.}  
    \choice{the point of concurrency of the altitudes.}  
    \choice[correct]{the balance point for the triangle.}
    \choice{the center in the triangle.}
    \choice{the center of the incircle.}
    \choice{the center of the circumcircle.}
    \choice{equidistant from the sides of the triangle.}
    \choice{equidistant from the vertices of the triangle.}    
  \end{selectAll}  
\end{question}

\begin{question}  
  The \textbf{orthocenter} of a triangle is \dots [select all]
  \begin{selectAll}  
    \choice{the point of concurrency of the medians.}  
    \choice{the point of concurrency of the angle bisectors.}  
    \choice{the point of concurrency of the perpendicular bisectors.}  
    \choice[correct]{the point of concurrency of the altitudes.}  
    \choice{the balance point for the triangle.}
    \choice{the center in the triangle.}
    \choice{the center of the incircle.}
    \choice{the center of the circumcircle.}
    \choice{equidistant from the sides of the triangle.}
    \choice{equidistant from the vertices of the triangle.}    
  \end{selectAll}  
\end{question}

%\begin{question}  
%  A \textbf{midsegment} in a triangle is \dots [select all]
%  \begin{selectAll}  
%    \choice{a segment in the middle.}
%    \choice[correct]{a segment connecting the midpoints of two sides.}
%    \choice[correct]{parallel to a side of the triangle.}
%    \choice{perpendicular to a side of the triangle.}
%    \choice{also called a median.}
%  \end{selectAll}  
%\end{question}


\end{document}

%  \begin{explanation}\hfil
%    \begin{itemize}
%    \item Curve $A$ is defined by an
%      \wordChoice{\choice{even}\choice[correct]{odd}} degree
%      polynomial with a \wordChoice{\choice[correct]{positive}\choice{negative}}
%      leading term.
%    \item Curve $B$ is defined by an
%      \wordChoice{\choice{even}\choice[correct]{odd}} degree
%      polynomial with a
%      \wordChoice{\choice{positive}\choice[correct]{negative}} leading
%      term.
%    \item Curve $C$ is defined by an
%      \wordChoice{\choice[correct]{even}\choice{odd}} degree
%      polynomial with a \wordChoice{\choice[correct]{positive}\choice{negative}}
%      leading term.
%    \item Curve $D$ is defined by an
%      \wordChoice{\choice[correct]{even}\choice{odd}} degree
%      polynomial with a \wordChoice{\choice{positive}\choice[correct]{negative}}
%      leading term.
%    \end{itemize}
%  \end{explanation}
