\documentclass[nooutcomes]{ximera}
%\documentclass[space,handout,nooutcomes]{ximera}

% For preamble materials

\usepackage{pgf,tikz}
\usepackage{mathrsfs}
\usetikzlibrary{arrows}
\usepackage{framed}
\usepackage{amsmath}
\pgfplotsset{compat=1.17}

\def\fixnote#1{\begin{framed}{\textcolor{red}{Fix note: #1}}\end{framed}}  % Allows insertion of red notes about needed edits
%\def\fixnote#1{}

\def\detail#1{{\textcolor{blue}{Detail: #1}}}   

\pdfOnly{\renewenvironment{image}[1][]{\begin{center}}{\end{center}}}

\graphicspath{
  {./}
  {chapter1/}
  {chapter2/}
  {chapter4/}
  {proofs/}
  {graphics/}
  {../graphics/}
}

\newenvironment{sectionOutcomes}{}{}


%%% This set of code is all of our user defined commands
\newcommand{\bysame}{\mbox{\rule{3em}{.4pt}}\,}
\newcommand{\N}{\mathbb N}
\newcommand{\C}{\mathbb C}
\newcommand{\W}{\mathbb W}
\newcommand{\Z}{\mathbb Z}
\newcommand{\Q}{\mathbb Q}
\newcommand{\R}{\mathbb R}
\newcommand{\A}{\mathbb A}
\newcommand{\D}{\mathcal D}
\newcommand{\F}{\mathcal F}
\newcommand{\ph}{\varphi}
\newcommand{\ep}{\varepsilon}
\newcommand{\aph}{\alpha}
\newcommand{\QM}{\begin{center}{\huge\textbf{?}}\end{center}}

\renewcommand{\le}{\leqslant}
\renewcommand{\ge}{\geqslant}
\renewcommand{\a}{\wedge}
\renewcommand{\v}{\vee}
\renewcommand{\l}{\ell}
\newcommand{\mat}{\mathsf}
\renewcommand{\vec}{\mathbf}
\renewcommand{\subset}{\subseteq}
\renewcommand{\supset}{\supseteq}
%\renewcommand{\emptyset}{\varnothing}
%\newcommand{\xto}{\xrightarrow}
%\renewcommand{\qedsymbol}{$\blacksquare$}
%\newcommand{\bibname}{References and Further Reading}
%\renewcommand{\bar}{\protect\overline}
%\renewcommand{\hat}{\protect\widehat}
%\renewcommand{\tilde}{\widetilde}
%\newcommand{\tri}{\triangle}
%\newcommand{\minipad}{\vspace{1ex}}
%\newcommand{\leftexp}[2]{{\vphantom{#2}}^{#1}{#2}}

%% More user defined commands
\renewcommand{\epsilon}{\varepsilon}
\renewcommand{\theta}{\vartheta} %% only for kmath
\renewcommand{\l}{\ell}
\renewcommand{\d}{\, d}
\newcommand{\ddx}{\frac{d}{dx}}
\newcommand{\dydx}{\frac{dy}{dx}}


\usepackage{bigstrut}


\title{Proof by Picture}
\author{Bart Snapp and Brad Findell}
\begin{document}
\begin{abstract}
Short-answer proofs by pictures. 
\end{abstract}
\maketitle

\

\begin{problem}
What are the rules for folding and tracing constructions?
\end{problem}

\begin{problem}
Use folding and tracing to bisect a given line segment. Explain the steps in
  your construction.
\end{problem}

\begin{problem}
Given a line segment with a point on it, use folding and tracing to
  construct a line perpendicular to the segment that passes through
  the given point. Explain the steps in your construction.
\end{problem}

\begin{problem}
Use folding and tracing to bisect a given angle. Explain the steps in your
  construction.
\end{problem}

\begin{problem}
Given a point and line, use folding and tracing to construct a line parallel
  to the given line that passes through the given point. Explain the
  steps in your construction.
\end{problem}

\begin{problem}
Given a point and line, use folding and tracing to construct a line
  perpendicular to the given line that passes through the given
  point. Explain the steps in your construction.
\end{problem}

\begin{problem}
Given a circle (a center and a point on the circle) and line,
  use folding and tracing to construct the intersection. Explain the steps in your
  construction.
\end{problem}

\begin{problem}
Given a line segment, use folding and tracing to construct an equilateral
  triangle whose edge has the length of the given segment. Explain the
  steps in your construction.
\end{problem}

\begin{problem}
Explain how to use folding and tracing to transfer a segment.
\end{problem}

\begin{problem}
Given an angle and some point, use folding and tracing to copy the angle so
  that the new angle has as its vertex the given point. Explain the
  steps in your construction.
\end{problem}

\begin{problem}
Explain how to use folding and tracing to construct envelope of tangents for
  a parabola.
\end{problem}

\begin{problem}
Explain how to use folding and tracing to trisect a given angle.
\end{problem}

\begin{problem}
Use folding and tracing to construct a square. Explain the steps in your construction.
\end{problem}

\begin{problem}
Use folding and tracing to construct a regular hexagon. Explain the steps in
  your construction.
\end{problem}

%\begin{problem}
%Morley's Theorem states:\index{Morley's Theorem}
%If you trisect the angles of any triangle with lines, then those lines
%form a new equilateral triangle inside the original triangle.\index{equilateral triangle}\index{trisecting the angle}
%\[
%\includegraphics[scale=.5]{../graphics/morley.pdf}
%\]
%Give a folding and tracing construction illustrating Morley's Theorem. Explain the
%steps in your construction.
%\end{problem}

\begin{problem}
Given a length of $1$, construct a triangle whose perimeter is a
  multiple of $6$. Explain the steps in your construction.
\end{problem}

\begin{problem}
Construct a $30$-$60$-$90$ right triangle. Explain the steps in your
  construction.
\end{problem}

\begin{problem}
Given a length of $1$, construct a triangle with a perimeter of
  $3 + \sqrt{5}$. Explain the steps in your construction.
\end{problem}


\end{document}
