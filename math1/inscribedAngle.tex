\documentclass[nooutcomes]{ximera}
%\documentclass[space,handout,nooutcomes]{ximera}

% For preamble materials

\usepackage{pgf,tikz}
\usepackage{mathrsfs}
\usetikzlibrary{arrows}
\usepackage{framed}
\usepackage{amsmath}
\pgfplotsset{compat=1.17}

\def\fixnote#1{\begin{framed}{\textcolor{red}{Fix note: #1}}\end{framed}}  % Allows insertion of red notes about needed edits
%\def\fixnote#1{}

\def\detail#1{{\textcolor{blue}{Detail: #1}}}   

\pdfOnly{\renewenvironment{image}[1][]{\begin{center}}{\end{center}}}

\graphicspath{
  {./}
  {chapter1/}
  {chapter2/}
  {chapter4/}
  {proofs/}
  {graphics/}
  {../graphics/}
}

\newenvironment{sectionOutcomes}{}{}


%%% This set of code is all of our user defined commands
\newcommand{\bysame}{\mbox{\rule{3em}{.4pt}}\,}
\newcommand{\N}{\mathbb N}
\newcommand{\C}{\mathbb C}
\newcommand{\W}{\mathbb W}
\newcommand{\Z}{\mathbb Z}
\newcommand{\Q}{\mathbb Q}
\newcommand{\R}{\mathbb R}
\newcommand{\A}{\mathbb A}
\newcommand{\D}{\mathcal D}
\newcommand{\F}{\mathcal F}
\newcommand{\ph}{\varphi}
\newcommand{\ep}{\varepsilon}
\newcommand{\aph}{\alpha}
\newcommand{\QM}{\begin{center}{\huge\textbf{?}}\end{center}}

\renewcommand{\le}{\leqslant}
\renewcommand{\ge}{\geqslant}
\renewcommand{\a}{\wedge}
\renewcommand{\v}{\vee}
\renewcommand{\l}{\ell}
\newcommand{\mat}{\mathsf}
\renewcommand{\vec}{\mathbf}
\renewcommand{\subset}{\subseteq}
\renewcommand{\supset}{\supseteq}
%\renewcommand{\emptyset}{\varnothing}
%\newcommand{\xto}{\xrightarrow}
%\renewcommand{\qedsymbol}{$\blacksquare$}
%\newcommand{\bibname}{References and Further Reading}
%\renewcommand{\bar}{\protect\overline}
%\renewcommand{\hat}{\protect\widehat}
%\renewcommand{\tilde}{\widetilde}
%\newcommand{\tri}{\triangle}
%\newcommand{\minipad}{\vspace{1ex}}
%\newcommand{\leftexp}[2]{{\vphantom{#2}}^{#1}{#2}}

%% More user defined commands
\renewcommand{\epsilon}{\varepsilon}
\renewcommand{\theta}{\vartheta} %% only for kmath
\renewcommand{\l}{\ell}
\renewcommand{\d}{\, d}
\newcommand{\ddx}{\frac{d}{dx}}
\newcommand{\dydx}{\frac{dy}{dx}}


\usepackage{bigstrut}


\title{Inscribed Angles}
\author{Brad Findell}
\begin{document}
\begin{abstract}
Proofs. 
\end{abstract}
\maketitle

\begin{problem}
In the figure below, $\overline{AB}$ is a diameter of a circle with center $P$. Prove that $\angle B$ is a right angle.  

\begin{image}
\definecolor{qqwuqq}{rgb}{0.,0.3926,0.}
\definecolor{uuuuuu}{rgb}{0.2667,0.2667,0.2667}
\definecolor{qqqqff}{rgb}{0.,0.,1.}
\begin{tikzpicture}[line cap=round,line join=round,>=triangle 45,x=1.0cm,y=1.0cm]
\clip(-4.5,-3.8) rectangle (13.5,3.8);
%\draw [shift={(-3.6056,0.)},line width=0.8pt,color=qqwuqq,fill=qqwuqq,fill opacity=0.1] (0,0) -- (0.:1.115) arc (0.:28.155:1.115) -- cycle;
%\draw [shift={(2.,3.)},line width=0.8pt,color=qqwuqq,fill=qqwuqq,fill opacity=0.1] (0,0) -- (-151.845:1.115) arc (-151.845:-123.690:1.115) -- cycle;
%\draw [shift={(2.,3.)},line width=0.8pt,color=qqwuqq,fill=qqwuqq,fill opacity=0.1] (0,0) -- (-123.690:0.892) arc (-123.690:-61.845:0.892) -- cycle;
%\draw [shift={(2.,3.)},line width=0.8pt,color=qqwuqq] (-123.690:0.669) arc (-123.690:-61.85:0.669);
%\draw [shift={(3.6056,0.)},line width=0.8pt,color=qqwuqq,fill=qqwuqq,fill opacity=0.1] (0,0) -- (118.155:0.892) arc (118.155:180.:0.892) -- cycle;
%\draw [shift={(3.6056,0.)},line width=0.8pt,color=qqwuqq] (118.155:0.669) arc (118.155:180.:0.669);
\draw [line width=0.8pt] (0.,0.) circle (3.606cm);
\draw [line width=0.8pt] (0.,0.) circle (3.606cm);
\draw [line width=0.8pt] (-3.6056,0.)-- (2.,3.);
\draw [line width=0.8pt] (2.,3.)-- (3.6056,0.);
\draw [line width=0.8pt] (-3.6056,0.)-- (0.,0.);
%\draw [line width=0.8pt] (-1.8028,0.2006) -- (-1.8028,-0.2006);
\draw [line width=0.8pt] (0.,0.)-- (3.6056,0.);
%\draw [line width=0.8pt] (1.8028,0.2006) -- (1.8028,-0.2006);
%\draw [line width=0.8pt] (0.,0.)-- (2.,3.);
%\draw [line width=0.8pt] (0.833,1.611) -- (1.167,1.389);
%\draw (-2.2,0.35) node {$x$};
%\draw (0.95,2.1) node {$x$};
%\draw (1.95,1.85) node {$y$};
%\draw (2.6,0.6) node {$y$};
%\draw (0.55,0.3) node {$z$};
%\begin{scriptsize}
\draw [fill=qqqqff] (0.,0.) circle (1.2pt);
\draw[color=qqqqff] (-0.24,-0.28) node {$P$};
\draw [fill=qqqqff] (2.,3.) circle (1.2pt);
\draw[color=qqqqff] (2.25,3.3) node {$B$};
\draw [fill=qqqqff] (-3.60,0.) circle (1.2pt);
\draw[color=qqqqff] (-3.9,0) node {$A$};
\draw [fill=qqqqff] (3.60,0.) circle (1.2pt);
\draw[color=qqqqff] (3.90,0) node {$C$};
%\end{scriptsize}
%\end{tikzpicture}

\begin{scope}[shift={(9,0)}]
%\definecolor{qqwuqq}{rgb}{0.,0.3926,0.}
%\definecolor{qqqqff}{rgb}{0.,0.,1.}
%\begin{tikzpicture}[line cap=round,line join=round,>=triangle 45,x=1.0cm,y=1.0cm]
%\clip(-4.5,-3.8) rectangle (4.5,3.8);
%\draw [shift={(-3.6056,0.)},line width=0.8pt,color=qqwuqq,fill=qqwuqq,fill opacity=0.1] (0,0) -- (0.:1.115) arc (0.:28.155:1.115) -- cycle;
%\draw [shift={(2.,3.)},line width=0.8pt,color=qqwuqq,fill=qqwuqq,fill opacity=0.1] (0,0) -- (-151.845:1.115) arc (-151.845:-123.690:1.115) -- cycle;
%\draw [shift={(2.,3.)},line width=0.8pt,color=qqwuqq,fill=qqwuqq,fill opacity=0.1] (0,0) -- (-123.690:0.892) arc (-123.690:-61.845:0.892) -- cycle;
%\draw [shift={(2.,3.)},line width=0.8pt,color=qqwuqq] (-123.690:0.669) arc (-123.690:-61.85:0.669);
%\draw [shift={(3.6056,0.)},line width=0.8pt,color=qqwuqq,fill=qqwuqq,fill opacity=0.1] (0,0) -- (118.155:0.892) arc (118.155:180.:0.892) -- cycle;
%\draw [shift={(3.6056,0.)},line width=0.8pt,color=qqwuqq] (118.155:0.669) arc (118.155:180.:0.669);
\draw [line width=0.8pt] (0.,0.) circle (3.606cm);
\draw [line width=0.8pt] (-3.6056,0.)-- (2.,3.);
\draw [line width=0.8pt] (2.,3.)-- (3.6056,0.);
\draw [line width=0.8pt] (-3.6056,0.)-- (0.,0.);
\draw [line width=0.8pt] (-1.8028,0.2006) -- (-1.8028,-0.2006);
\draw [line width=0.8pt] (0.,0.)-- (3.6056,0.);
\draw [line width=0.8pt] (1.8028,0.2006) -- (1.8028,-0.2006);
\draw [line width=0.8pt] (0.,0.)-- (2.,3.);
\draw [line width=0.8pt] (0.833,1.611) -- (1.167,1.389);
%\draw (-2.2,0.35) node {$x$};
%\draw (0.95,2.1) node {$x$};
%\draw (1.95,1.85) node {$y$};
%\draw (2.6,0.6) node {$y$};
%\draw (0.55,0.3) node {$z$};
%\begin{scriptsize}
\draw [fill=qqqqff] (0.,0.) circle (1.2pt);
\draw[color=qqqqff] (-0.24,-0.28) node {$P$};
\draw [fill=qqqqff] (2.,3.) circle (1.2pt);
\draw[color=qqqqff] (2.25,3.3) node {$B$};
\draw [fill=qqqqff] (-3.60,0.) circle (1.2pt);
\draw[color=qqqqff] (-3.9,0) node {$A$};
\draw [fill=qqqqff] (3.60,0.) circle (1.2pt);
\draw[color=qqqqff] (3.90,0) node {$C$};
%\end{scriptsize}
\end{scope}
\end{tikzpicture}
\end{image}

\begin{enumerate}
\item Beginning with the diagram on the left, Natalia draws $\overline{PB}$ and marks the diagram to show segments that she knows to be congruent because each one is a $\answer[format=string]{radius}$ of the circle.  

\begin{image}
\definecolor{qqwuqq}{rgb}{0.,0.3926,0.}
\definecolor{qqqqff}{rgb}{0.,0.,1.}
\begin{tikzpicture}[line cap=round,line join=round,>=triangle 45,x=1.0cm,y=1.0cm]
\clip(-4.5,-3.8) rectangle (13.5,3.8);
\draw [shift={(-3.6056,0.)},line width=0.8pt,color=qqwuqq,fill=qqwuqq,fill opacity=0.1] (0,0) -- (0.:1.115) arc (0.:28.155:1.115) -- cycle;
\draw [shift={(2.,3.)},line width=0.8pt,color=qqwuqq,fill=qqwuqq,fill opacity=0.1] (0,0) -- (-151.845:1.115) arc (-151.845:-123.690:1.115) -- cycle;
\draw [shift={(2.,3.)},line width=0.8pt,color=qqwuqq,fill=qqwuqq,fill opacity=0.1] (0,0) -- (-123.690:0.892) arc (-123.690:-61.845:0.892) -- cycle;
\draw [shift={(2.,3.)},line width=0.8pt,color=qqwuqq] (-123.690:0.669) arc (-123.690:-61.85:0.669);
\draw [shift={(3.6056,0.)},line width=0.8pt,color=qqwuqq,fill=qqwuqq,fill opacity=0.1] (0,0) -- (118.155:0.892) arc (118.155:180.:0.892) -- cycle;
\draw [shift={(3.6056,0.)},line width=0.8pt,color=qqwuqq] (118.155:0.669) arc (118.155:180.:0.669);
\draw [line width=0.8pt] (0.,0.) circle (3.606cm);
\draw [line width=0.8pt] (-3.6056,0.)-- (2.,3.);
\draw [line width=0.8pt] (2.,3.)-- (3.6056,0.);
\draw [line width=0.8pt] (-3.6056,0.)-- (0.,0.);
\draw [line width=0.8pt] (-1.8028,0.2006) -- (-1.8028,-0.2006);
\draw [line width=0.8pt] (0.,0.)-- (3.6056,0.);
\draw [line width=0.8pt] (1.8028,0.2006) -- (1.8028,-0.2006);
\draw [line width=0.8pt] (0.,0.)-- (2.,3.);
\draw [line width=0.8pt] (0.833,1.611) -- (1.167,1.389);
%\draw (-2.2,0.35) node {$x$};
%\draw (0.95,2.1) node {$x$};
%\draw (1.95,1.85) node {$y$};
%\draw (2.6,0.6) node {$y$};
%\draw (0.55,0.3) node {$z$};
%\begin{scriptsize}
\draw [fill=qqqqff] (0.,0.) circle (1.2pt);
\draw[color=qqqqff] (-0.24,-0.28) node {$P$};
\draw [fill=qqqqff] (2.,3.) circle (1.2pt);
\draw[color=qqqqff] (2.25,3.3) node {$B$};
\draw [fill=qqqqff] (-3.60,0.) circle (1.2pt);
\draw[color=qqqqff] (-3.9,0) node {$A$};
\draw [fill=qqqqff] (3.60,0.) circle (1.2pt);
\draw[color=qqqqff] (3.90,0) node {$C$};
%\end{scriptsize}
%\end{tikzpicture}

\begin{scope}[shift={(9,0)}]
%\definecolor{qqwuqq}{rgb}{0.,0.3926,0.}
%\definecolor{qqqqff}{rgb}{0.,0.,1.}
%\begin{tikzpicture}[line cap=round,line join=round,>=triangle 45,x=1.0cm,y=1.0cm]
%\clip(-4.5,-3.8) rectangle (4.5,3.8);
\draw [shift={(-3.6056,0.)},line width=0.8pt,color=qqwuqq,fill=qqwuqq,fill opacity=0.1] (0,0) -- (0.:1.115) arc (0.:28.155:1.115) -- cycle;
\draw [shift={(2.,3.)},line width=0.8pt,color=qqwuqq,fill=qqwuqq,fill opacity=0.1] (0,0) -- (-151.845:1.115) arc (-151.845:-123.690:1.115) -- cycle;
\draw [shift={(2.,3.)},line width=0.8pt,color=qqwuqq,fill=qqwuqq,fill opacity=0.1] (0,0) -- (-123.690:0.892) arc (-123.690:-61.845:0.892) -- cycle;
\draw [shift={(2.,3.)},line width=0.8pt,color=qqwuqq] (-123.690:0.669) arc (-123.690:-61.85:0.669);
\draw [shift={(3.6056,0.)},line width=0.8pt,color=qqwuqq,fill=qqwuqq,fill opacity=0.1] (0,0) -- (118.155:0.892) arc (118.155:180.:0.892) -- cycle;
\draw [shift={(3.6056,0.)},line width=0.8pt,color=qqwuqq] (118.155:0.669) arc (118.155:180.:0.669);
\draw [line width=0.8pt] (0.,0.) circle (3.606cm);
\draw [line width=0.8pt] (-3.6056,0.)-- (2.,3.);
\draw [line width=0.8pt] (2.,3.)-- (3.6056,0.);
\draw [line width=0.8pt] (-3.6056,0.)-- (0.,0.);
\draw [line width=0.8pt] (-1.8028,0.2006) -- (-1.8028,-0.2006);
\draw [line width=0.8pt] (0.,0.)-- (3.6056,0.);
\draw [line width=0.8pt] (1.8028,0.2006) -- (1.8028,-0.2006);
\draw [line width=0.8pt] (0.,0.)-- (2.,3.);
\draw [line width=0.8pt] (0.833,1.611) -- (1.167,1.389);
\draw (-2.2,0.35) node {$x$};
\draw (0.95,2.1) node {$x$};
\draw (1.95,1.85) node {$y$};
\draw (2.6,0.6) node {$y$};
%\draw (0.55,0.3) node {$z$};
%\begin{scriptsize}
\draw [fill=qqqqff] (0.,0.) circle (1.2pt);
\draw[color=qqqqff] (-0.24,-0.28) node {$P$};
\draw [fill=qqqqff] (2.,3.) circle (1.2pt);
\draw[color=qqqqff] (2.25,3.3) node {$B$};
\draw [fill=qqqqff] (-3.60,0.) circle (1.2pt);
\draw[color=qqqqff] (-3.9,0) node {$A$};
\draw [fill=qqqqff] (3.60,0.) circle (1.2pt);
\draw[color=qqqqff] (3.90,0) node {$C$};
%\end{scriptsize}
\end{scope}
\end{tikzpicture}
\end{image}


\item Natalia sees that $\triangle APB$ and $\triangle BPC$ are $\answer[format=string]{isosceles}$ triangles, so she marks the figure to show angles that must congruent. (Note: Do we need a statement or citation of the theorem?)

\item In order to do some algebra with these congruent angles, Natalia labels their measures $x$ and $y$, as shown in the picture on the right.  

\item She writes an equation for the sum of the angles of $\triangle ABC$: 

\[
\answer{x+(x+y)+y} = 180^\circ
\]

(Note: Need a question about dividing the equation by 2.)  
\item Since $m\angle B = \answer{x+y}$, she concludes that $m\angle B = 90^\circ$.  (Note: Should call it $\angle ABC$ because of the new segment, or maybe note this earlier.)
\end{enumerate}
\end{problem}

\begin{problem}

New problem about relationship between inscribed angle and central angle. 

\begin{image}
\definecolor{qqwuqq}{rgb}{0.,0.3926,0.}
\definecolor{qqqqff}{rgb}{0.,0.,1.}
\begin{tikzpicture}[line cap=round,line join=round,>=triangle 45,x=1.0cm,y=1.0cm]
\clip(-4.5,-3.8) rectangle (4.5,3.8);
\draw [shift={(-3.6056,0.)},line width=0.8pt,color=qqwuqq,fill=qqwuqq,fill opacity=0.1] (0,0) -- (0.:1.115) arc (0.:28.155:1.115) -- cycle;
\draw [shift={(2.,3.)},line width=0.8pt,color=qqwuqq,fill=qqwuqq,fill opacity=0.1] (0,0) -- (-151.845:1.115) arc (-151.845:-123.690:1.115) -- cycle;
\draw [shift={(2.,3.)},line width=0.8pt,color=qqwuqq,fill=qqwuqq,fill opacity=0.1] (0,0) -- (-123.690:0.892) arc (-123.690:-61.845:0.892) -- cycle;
\draw [shift={(2.,3.)},line width=0.8pt,color=qqwuqq] (-123.690:0.669) arc (-123.690:-61.85:0.669);
\draw [shift={(3.6056,0.)},line width=0.8pt,color=qqwuqq,fill=qqwuqq,fill opacity=0.1] (0,0) -- (118.155:0.892) arc (118.155:180.:0.892) -- cycle;
\draw [shift={(3.6056,0.)},line width=0.8pt,color=qqwuqq] (118.155:0.669) arc (118.155:180.:0.669);
\draw [line width=0.8pt] (0.,0.) circle (3.606cm);
\draw [line width=0.8pt] (-3.6056,0.)-- (2.,3.);
\draw [line width=0.8pt] (2.,3.)-- (3.6056,0.);
\draw [line width=0.8pt] (-3.6056,0.)-- (0.,0.);
\draw [line width=0.8pt] (-1.8028,0.2006) -- (-1.8028,-0.2006);
\draw [line width=0.8pt] (0.,0.)-- (3.6056,0.);
\draw [line width=0.8pt] (1.8028,0.2006) -- (1.8028,-0.2006);
\draw [line width=0.8pt] (0.,0.)-- (2.,3.);
\draw [line width=0.8pt] (0.833,1.611) -- (1.167,1.389);
\draw (-2.2,0.35) node {$x$};
\draw (0.95,2.1) node {$x$};
\draw (1.95,1.85) node {$y$};
\draw (2.6,0.6) node {$y$};
\draw (0.55,0.3) node {$z$};
%\begin{scriptsize}
\draw [fill=qqqqff] (0.,0.) circle (1.2pt);
\draw[color=qqqqff] (-0.24,-0.28) node {$P$};
\draw [fill=qqqqff] (2.,3.) circle (1.2pt);
\draw[color=qqqqff] (2.25,3.3) node {$B$};
\draw [fill=qqqqff] (-3.60,0.) circle (1.2pt);
\draw[color=qqqqff] (-3.9,0) node {$A$};
\draw [fill=qqqqff] (3.60,0.) circle (1.2pt);
\draw[color=qqqqff] (3.90,0) node {$C$};
%\end{scriptsize}
\end{tikzpicture}
\end{image}

\end{problem}

\end{document}
