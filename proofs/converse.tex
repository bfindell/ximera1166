\documentclass[nooutcomes]{ximera}
%\documentclass[space,handout,nooutcomes]{ximera}

% For preamble materials

\graphicspath{
  {./}
  {chapter1/}
  {chapter2/}
  {chapter4/}
  {math1/}
  {math2/}
}

\usepackage{pgf,tikz}
\usepackage{mathrsfs}
\usetikzlibrary{arrows}
\pgfplotsset{compat=1.16}


\newcommand{\N}{\mathbb N}
\newcommand{\W}{\mathbb W}
\newcommand{\C}{\mathbb C}
\newcommand{\Z}{\mathbb Z}
\newcommand{\Q}{\mathbb Q}
\newcommand{\R}{\mathbb R}




\title{Converse}
\author{Brad Findell}
\begin{document}
\begin{abstract}
About the converse of a statement. 
\end{abstract}
\maketitle

% How to translate part of a TikZ image
%
%\begin{scope}[shift={(2,0)}]
%  ... insert graphic here
%\end{scope}
%

\begin{problem}
When a statement is of the form ``If A, then B,'' its converse is ``If $\answer{B}$, then $\answer{A}$.  

\end{problem}

\begin{problem}
Many mathematical theorems can be restated in this If-then form, making it easier to state a converse. 

Statement: All squares are rectangles.  

Restatement: If a quadrilateral is a $\answer[format=string]{square}$, then it is a $\answer[format=string]{rectangle}$.  

Converse: If a quadrilateral is a $\answer[format=string]{rectangle}$, then it is a $\answer[format=string]{square}$.  


A statement of the form ``If A and B, then C,'' its converse is more challenging to think about.  
\end{problem}


\end{document}
