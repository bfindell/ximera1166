% Section 2.2 Anatomy of Figures

\documentclass[nooutcomes]{ximera}
%\documentclass[space,handout,nooutcomes]{ximera}

% For preamble materials

\usepackage{pgf,tikz}
\usepackage{mathrsfs}
\usetikzlibrary{arrows}
\usepackage{framed}
\usepackage{amsmath}
\pgfplotsset{compat=1.17}

\def\fixnote#1{\begin{framed}{\textcolor{red}{Fix note: #1}}\end{framed}}  % Allows insertion of red notes about needed edits
%\def\fixnote#1{}

\def\detail#1{{\textcolor{blue}{Detail: #1}}}   

\pdfOnly{\renewenvironment{image}[1][]{\begin{center}}{\end{center}}}

\graphicspath{
  {./}
  {chapter1/}
  {chapter2/}
  {chapter4/}
  {proofs/}
  {graphics/}
  {../graphics/}
}

\newenvironment{sectionOutcomes}{}{}


%%% This set of code is all of our user defined commands
\newcommand{\bysame}{\mbox{\rule{3em}{.4pt}}\,}
\newcommand{\N}{\mathbb N}
\newcommand{\C}{\mathbb C}
\newcommand{\W}{\mathbb W}
\newcommand{\Z}{\mathbb Z}
\newcommand{\Q}{\mathbb Q}
\newcommand{\R}{\mathbb R}
\newcommand{\A}{\mathbb A}
\newcommand{\D}{\mathcal D}
\newcommand{\F}{\mathcal F}
\newcommand{\ph}{\varphi}
\newcommand{\ep}{\varepsilon}
\newcommand{\aph}{\alpha}
\newcommand{\QM}{\begin{center}{\huge\textbf{?}}\end{center}}

\renewcommand{\le}{\leqslant}
\renewcommand{\ge}{\geqslant}
\renewcommand{\a}{\wedge}
\renewcommand{\v}{\vee}
\renewcommand{\l}{\ell}
\newcommand{\mat}{\mathsf}
\renewcommand{\vec}{\mathbf}
\renewcommand{\subset}{\subseteq}
\renewcommand{\supset}{\supseteq}
%\renewcommand{\emptyset}{\varnothing}
%\newcommand{\xto}{\xrightarrow}
%\renewcommand{\qedsymbol}{$\blacksquare$}
%\newcommand{\bibname}{References and Further Reading}
%\renewcommand{\bar}{\protect\overline}
%\renewcommand{\hat}{\protect\widehat}
%\renewcommand{\tilde}{\widetilde}
%\newcommand{\tri}{\triangle}
%\newcommand{\minipad}{\vspace{1ex}}
%\newcommand{\leftexp}[2]{{\vphantom{#2}}^{#1}{#2}}

%% More user defined commands
\renewcommand{\epsilon}{\varepsilon}
\renewcommand{\theta}{\vartheta} %% only for kmath
\renewcommand{\l}{\ell}
\renewcommand{\d}{\, d}
\newcommand{\ddx}{\frac{d}{dx}}
\newcommand{\dydx}{\frac{dy}{dx}}


\usepackage{bigstrut}


\title{Vocabulary Review}
\author{Bart Snapp and Brad Findell}
\begin{document}
\begin{abstract}
Short-answer, multiple-choice, and select-all questions about key vocabulary.
\end{abstract}
\maketitle

%Useful questions: 
%
%What is regular quadrilateral? 
%Definition of ? 
%Write the Pythagorean theorem. 
%Measure angles. 
%Angle sum in a triangle. 
%Triangulate a figure 


\begin{question}
Equiangular means that all $\answer[format=string]{angles}$ are the same \wordChoice{\choice{length}\choice{width}\choice{height}\choice{weight}\choice[correct]{measure}\choice{volume}}.  
Equilateral means that all $\answer[format=string]{sides}$ are the same $\answer[format=string]{length}$.  A polygon that is both equiangular and equlateral is said to be $\answer[format=string]{regular}$.  
\end{question}

\begin{question}  
An \textbf{equilateral quadrilateral} is called a $\answer[format=string]{rhombus}$.
\end{question}

\begin{question}  
An \textbf{equiangular quadrilateral} is called a $\answer[format=string]{rectangle}$. 
\end{question}

\begin{question}  
An \textbf{regular quadrilateral} is called a $\answer[format=string]{square}$. 
\end{question}

%\begin{question}  
%An line segment between two points on a circle is called a $\answer[format=string]{chord}$ of the circle.  
%\end{question}

%\begin{question}  
%A $\answer[format=string]{straight angle}$ measures $180^\circ$.  (Hint: Answer with two words.)
%\end{question}
%
%\begin{question}  
%Two angles whose measures sum to $180^\circ$ are said to be $\answer[format=string]{supplementary}$.  
%\end{question}
%
%\begin{question}  
%Two angles whose measures sum to $90^\circ$ are said to be $\answer[format=string]{complementary}$.  
%\end{question}

\begin{question}
Consider the following diagram: 
\begin{image}
\definecolor{qqwuqq}{rgb}{0.,0.39215686274509803,0.}
\definecolor{uuuuuu}{rgb}{0.26666666666666666,0.26666666666666666,0.26666666666666666}
\definecolor{xdxdff}{rgb}{0.49019607843137253,0.49019607843137253,1.}
\definecolor{qqqqff}{rgb}{0.,0.,1.}
\begin{tikzpicture}[line cap=round,line join=round,>=triangle 45,x=1.0cm,y=1.0cm]
%\clip(-1.6,7.9) rectangle (8.1,13.8);
\clip(-3,7.9) rectangle (9,13.8);
\draw[color=white] (-3,8) circle (0.2pt);
\draw[color=white] (9,8) circle (0.2pt);
\draw[line width=0.8pt,color=qqwuqq,fill=qqwuqq,fill opacity=0.10000000149011612] (1.8735088935932647,9.752982212813471) -- (1.620526680779794,9.626491106406736) -- (1.7470177871865293,9.373508893593264) -- (2.,9.5) -- cycle; 
\draw[line width=0.8pt,color=qqwuqq,fill=qqwuqq,fill opacity=0.10000000149011612] (4.716982212813471,10.858491106406735) -- (4.590491106406736,11.111473319220206) -- (4.337508893593265,10.98498221281347) -- (4.464,10.732) -- cycle; 
\draw [line width=0.8pt,dash pattern=on 2pt off 2pt,domain=-1.6:8.1] plot(\x,{(--17.--1.*\x)/2.});
\draw [line width=0.8pt,domain=-1.6:8.1] plot(\x,{(-13.5--2.*\x)/-1.});
\draw [line width=0.8pt,domain=-1.6:8.1] plot(\x,{(-19.66--2.*\x)/-1.});
\draw [line width=0.8pt] (1.,9.)-- (2.,9.5);
\draw [line width=0.8pt] (1.4597507764050037,9.330498447189992) -- (1.540249223594996,9.169501552810008);
\draw [line width=0.8pt] (2.,9.5)-- (3.,10.);
\draw [line width=0.8pt] (2.4597507764050035,9.830498447189992) -- (2.540249223594996,9.669501552810008);
\draw [line width=0.8pt] (1.256,10.988)-- (4.464,10.732);
\draw [line width=0.8pt] (2.,9.5)-- (3.328,13.004);
\begin{scriptsize}
\draw [fill=qqqqff] (1.,9.) circle (1.0pt);
\draw[color=qqqqff] (1.08,8.85) node {$A$};
\draw [fill=qqqqff] (3.,10.) circle (1.0pt);
\draw[color=qqqqff] (3.02,9.81) node {$B$};
\draw [fill=xdxdff] (4.464,10.732) circle (1.0pt);
\draw[color=xdxdff] (4.68,10.67) node {$C$};
\draw [fill=uuuuuu] (2.,9.5) circle (1.0pt);
\draw[color=uuuuuu] (1.92,9.27) node {$D$};
\draw [fill=xdxdff] (3.328,13.004) circle (1.0pt);
\draw[color=xdxdff] (3.,13.27) node {$E$};
\draw [fill=xdxdff] (1.256,10.988) circle (1.0pt);
\draw[color=xdxdff] (1.,11.05) node {$F$};
\end{scriptsize}
\end{tikzpicture}
\end{image}

Name a segment that is bisected in the above diagram: $\answer{AB}$

\begin{question}
Correct!  
\begin{enumerate}
\item Name a segment that is a perpendicular bisector of $\overline{AB}$: $\answer{FD}$
\item Name a different segment that bisects $\overline{AB}$: $\answer{ED}$
\item Name a different segment that is perpendicular to $\overleftrightarrow{AB}$: $\answer{EC}$
\end{enumerate}

\end{question}
\end{question}

\begin{question}
Consider the following diagram regarding $\triangle ABC$.  
\begin{image}
\definecolor{qqwuqq}{rgb}{0.,0.39215686274509803,0.}
\definecolor{uuuuuu}{rgb}{0.26666666666666666,0.26666666666666666,0.26666666666666666}
\definecolor{qqqqff}{rgb}{0.,0.,1.}
\begin{tikzpicture}[line cap=round,line join=round,>=triangle 45,x=1.0cm,y=1.0cm]
\clip(-1,7) rectangle (10,12.5);
\draw[color=white] (-1,8) circle (0.2pt);
\draw[color=white] (10,8) circle (0.2pt);
\draw[line width=0.8pt,color=qqwuqq,fill=qqwuqq,fill opacity=0.10000000149011612] (3.7,9.3) -- (3.9,9.5) -- (3.7,9.7) -- (3.5,9.5) -- cycle; 
\draw[line width=0.8pt,color=qqwuqq,fill=qqwuqq,fill opacity=0.10000000149011612] (5.868328157299975,9.289442719099991) -- (5.778885438199983,9.557770876399966) -- (5.510557280900008,9.468328157299975) -- (5.6,9.2) -- cycle; 
\draw [shift={(5.,11.)},line width=0.8pt,color=qqwuqq,fill=qqwuqq,fill opacity=0.10000000149011612] (0,0) -- (-135.:0.6) arc (-135.:-112.5:0.6) -- cycle;
\draw [shift={(5.,11.)},line width=0.8pt,color=qqwuqq,fill=qqwuqq,fill opacity=0.10000000149011612] (0,0) -- (-112.5:0.6) arc (-112.5:-90.:0.6) -- cycle;
\draw [line width=1.2pt] (5.,11.)-- (5.,9.);
\draw [line width=1.2pt] (5.,9.)-- (2.,8.);
\draw [line width=0.8pt,dash pattern=on 2pt off 2pt,domain=-0.5:10.] plot(\x,{(--22.--1.*\x)/3.});
\draw [line width=0.8pt,dash pattern=on 2pt off 2pt,domain=-0.5:10.] plot(\x,{(-26.--3.*\x)/-1.});
\draw [line width=0.8pt,dash pattern=on 2pt off 2pt,domain=-0.5:10.] (5.,9.)-- (3.5,9.5);
\draw [line width=1.2pt] (2.,8.)-- (3.5,9.5);
\draw [line width=0.8pt] (2.6863603896932102,8.813639610306788) -- (2.813639610306789,8.686360389693212);
\draw [line width=1.2pt] (3.5,9.5)-- (5.,11.);
\draw [line width=0.8pt] (4.186360389693211,10.313639610306788) -- (4.313639610306789,10.186360389693212);
\draw [line width=0.8pt,dash pattern=on 2pt off 2pt,domain=-0.5:10.] plot(\x,{(--0.40987990654044726-0.9238795325112867*\x)/-0.3826834323650897});
\draw [line width=0.8pt,dash pattern=on 2pt off 2pt,domain=-0.5:10.] plot(\x,{(-39.--3.*\x)/-3.});
\draw [shift={(5.,11.)},line width=0.8pt,color=qqwuqq] (-135.:0.6) arc (-135.:-112.5:0.6);
\draw[line width=0.8pt,color=qqwuqq] (4.699992074169415,10.551006409356624) -- (4.633323646207061,10.451230055880318);
\draw [shift={(5.,11.)},line width=0.8pt,color=qqwuqq] (-112.5:0.6) arc (-112.5:-90.:0.6);
\draw[line width=0.8pt,color=qqwuqq] (4.89465122611129,10.470375948582253) -- (4.871240387469355,10.352681714933867);
\begin{scriptsize}
\draw [fill=qqqqff] (2.,8.) circle (1.0pt);
\draw[color=qqqqff] (1.82,8.27) node {$A$};
\draw [fill=qqqqff] (5.,11.) circle (1.0pt);
\draw[color=qqqqff] (5.22,11.09) node {$B$};
\draw [fill=qqqqff] (5.,9.) circle (1.0pt);
\draw[color=qqqqff] (5.18,8.89) node {$C$};
\draw[color=black] (0.98,7.55) node {$k$};
\draw[color=black] (4.52,11.93) node {$j$};
\draw [fill=uuuuuu] (5.6,9.2) circle (1.0pt);
\draw[color=uuuuuu] (5.84,9.11) node {$D$};
\draw [fill=uuuuuu] (3.5,9.5) circle (1.0pt);
\draw[color=uuuuuu] (3.52,9.85) node {$E$};
\draw[color=black] (5.58,11.97) node {$m$};
\draw[color=black] (1.64,11.13) node {$n$};
\end{scriptsize}
\end{tikzpicture}
\end{image}

\begin{enumerate}
\item line $n$ is a(n) $\answer[format=string]{perpendicular bisector}$, 
\item line $m$ is a(n) $\answer[format=string]{angle bisector}$,
\item segment $\overline{BD}$ is a(n) $\answer[format=string]{altitude}$, and
\item segment $\overline{EC}$ is a(n) $\answer[format=string]{median}$. 
\end{enumerate}
\end{question}




\end{document}

