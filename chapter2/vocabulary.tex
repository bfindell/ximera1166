% Section 2.2 Anatomy of Figures

\documentclass[nooutcomes]{ximera}
%\documentclass[space,handout,nooutcomes]{ximera}

% For preamble materials

\graphicspath{
  {./}
  {chapter1/}
  {chapter2/}
  {chapter4/}
  {math1/}
  {math2/}
}

\usepackage{pgf,tikz}
\usepackage{mathrsfs}
\usetikzlibrary{arrows}
\pgfplotsset{compat=1.16}


\newcommand{\N}{\mathbb N}
\newcommand{\W}{\mathbb W}
\newcommand{\C}{\mathbb C}
\newcommand{\Z}{\mathbb Z}
\newcommand{\Q}{\mathbb Q}
\newcommand{\R}{\mathbb R}




\title{Vocabulary Review}
\author{Bart Snapp and Brad Findell}
\begin{document}
\begin{abstract}
Short-answer, multiple-choice, and select-all questions about key vocabulary. 
\end{abstract}
\maketitle

%Useful questions: 
%
%What is regular quadrilateral? 
%Definition of ? 
%Write the Pythagorean theorem. 
%Measure angles. 
%Angle sum in a triangle. 
%Triangulate a figure 

\begin{question}  
Three (or more) points that lie on the same line are said to be $\answer[format=string]{collinear}$.  
\end{question}

\begin{question}  
Three (or more) lines that lie on the same point are said to be $\answer[format=string]{concurrent}$.  
\end{question}

%\begin{question}  
%When three (or more) points all lie on the same line, we say they are \dots
%\begin{multipleChoice}  
%\choice{coplanar.}  
%\choice[correct]{collinear.}  
%\choice{conjoined.}
%\choice{concurrent.}  
%\choice{none of these.}
%\end{multipleChoice}  
%\end{question}
%
%\begin{question}  
%When three (or more) lines all lie on the same point, we say they are \dots
%\begin{multipleChoice}  
%\choice{coplanar.}  
%\choice{collinear.}  
%\choice{conjoined.}
%\choice[correct]{concurrent.}  
%\choice{none of these.}
%\end{multipleChoice}  
%\end{question}

%\begin{question}  
%In a circle, the measure of an $\answer[format=string]{inscribed}$ angle is $\answer[format=string]{half}$ the measure of the corresponding central angle.  (Hint: For the second blank, answer with a word.)
%\end{question}


\begin{question}  
A \textbf{median} in a triangle always \dots
\begin{selectAll}  
\choice[correct]{contains the midpoint of a side of the triangle}
\choice{is perpendicular to a side of the triangle}
\choice[correct]{contains a vertex of the triangle}
\choice{is perpendicular to the line containing a side of the triangle}  
\choice{bisects an angle of the triangle}  
\end{selectAll}  
\end{question}

\begin{question}  
An \textbf{altitude} in a triangle always \dots
\begin{selectAll}  
\choice{contains the midpoint of a side of the triangle}
\choice{is perpendicular to a side of the triangle}
\choice[correct]{contains a vertex of the triangle}
\choice[correct]{is perpendicular to the line containing a side of the triangle}  
\choice{bisects an angle of the triangle}  
\end{selectAll}  
\end{question}

\begin{question}  
An \textbf{angle bisector} in a triangle always \dots
\begin{selectAll}  
\choice{contains the midpoint of a side of the triangle}
\choice{is perpendicular to a side of the triangle}
\choice[correct]{contains a vertex of the triangle}
\choice{is perpendicular to the line containing a side of the triangle}  
\choice[correct]{bisects an angle of the triangle}  
\end{selectAll}  
\end{question}

\begin{question}  
A \textbf{perpendicular bisector} in a triangle always \dots
\begin{selectAll}  
\choice[correct]{contains the midpoint of a side of the triangle}
\choice[correct]{is perpendicular to a side of the triangle}
\choice{contains a vertex of the triangle}
\choice{is perpendicular to the line containing a side of the triangle}  
\choice{bisects an angle of the triangle}  
\end{selectAll}  
\end{question}


\begin{question}  
The \textbf{circumcenter} of a triangle is \dots [select all]
  \begin{selectAll}  
    \choice{the point of concurrency of the medians.}  
    \choice{the point of concurrency of the angle bisectors.}  
    \choice[correct]{the point of concurrency of the perpendicular bisectors.}  
    \choice{the point of concurrency of the altitudes.}  
    \choice{the balance point for the triangle.}
    \choice{the center in the triangle.}
    \choice{the center of the incircle.}
    \choice[correct]{the center of the circumcircle.}
    \choice{equidistant from the sides of the triangle.}
    \choice[correct]{equidistant from the vertices of the triangle.}    
  \end{selectAll}  
\end{question}

\begin{question}  
  The \textbf{incenter} of a triangle is \dots [select all]
  \begin{selectAll}  
    \choice{the point of concurrency of the medians.}  
    \choice[correct]{the point of concurrency of the angle bisectors.}  
    \choice{the point of concurrency of the perpendicular bisectors.}  
    \choice{the point of concurrency of the altitudes.}  
    \choice{the balance point for the triangle.}
    \choice{the center in the triangle.}
    \choice[correct]{the center of the incircle.}
    \choice{the center of the circumcircle.}
    \choice[correct]{equidistant from the sides of the triangle.}
    \choice{equidistant from the vertices of the triangle.}    
  \end{selectAll}  
\end{question}

\begin{question}  
  The \textbf{centroid} of a triangle is \dots [select all]
  \begin{selectAll}  
    \choice[correct]{the point of concurrency of the medians.}  
    \choice{the point of concurrency of the angle bisectors.}  
    \choice{the point of concurrency of the perpendicular bisectors.}  
    \choice{the point of concurrency of the altitudes.}  
    \choice[correct]{the balance point for the triangle.}
    \choice{the center in the triangle.}
    \choice{the center of the incircle.}
    \choice{the center of the circumcircle.}
    \choice{equidistant from the sides of the triangle.}
    \choice{equidistant from the vertices of the triangle.}    
  \end{selectAll}  
\end{question}

\begin{question}  
  The \textbf{orthocenter} of a triangle is \dots [select all]
  \begin{selectAll}  
    \choice{the point of concurrency of the medians.}  
    \choice{the point of concurrency of the angle bisectors.}  
    \choice{the point of concurrency of the perpendicular bisectors.}  
    \choice[correct]{the point of concurrency of the altitudes.}  
    \choice{the balance point for the triangle.}
    \choice{the center in the triangle.}
    \choice{the center of the incircle.}
    \choice{the center of the circumcircle.}
    \choice{equidistant from the sides of the triangle.}
    \choice{equidistant from the vertices of the triangle.}    
  \end{selectAll}  
\end{question}

%\begin{question}  
%  A \textbf{midsegment} in a triangle is \dots [select all]
%  \begin{selectAll}  
%    \choice{a segment in the middle.}
%    \choice[correct]{a segment connecting the midpoints of two sides.}
%    \choice[correct]{parallel to a side of the triangle.}
%    \choice{perpendicular to a side of the triangle.}
%    \choice{also called a median.}
%  \end{selectAll}  
%\end{question}


\end{document}

%  \begin{explanation}\hfil
%    \begin{itemize}
%    \item Curve $A$ is defined by an
%      \wordChoice{\choice{even}\choice[correct]{odd}} degree
%      polynomial with a \wordChoice{\choice[correct]{positive}\choice{negative}}
%      leading term.
%    \item Curve $B$ is defined by an
%      \wordChoice{\choice{even}\choice[correct]{odd}} degree
%      polynomial with a
%      \wordChoice{\choice{positive}\choice[correct]{negative}} leading
%      term.
%    \item Curve $C$ is defined by an
%      \wordChoice{\choice[correct]{even}\choice{odd}} degree
%      polynomial with a \wordChoice{\choice[correct]{positive}\choice{negative}}
%      leading term.
%    \item Curve $D$ is defined by an
%      \wordChoice{\choice[correct]{even}\choice{odd}} degree
%      polynomial with a \wordChoice{\choice{positive}\choice[correct]{negative}}
%      leading term.
%    \end{itemize}
%  \end{explanation}
