% Section 2.2 Anatomy of Figures

\documentclass[nooutcomes]{ximera}
%\documentclass[space,handout,nooutcomes]{ximera}

% For preamble materials

\graphicspath{
  {./}
  {chapter1/}
  {chapter2/}
  {chapter4/}
  {math1/}
  {math2/}
}

\usepackage{pgf,tikz}
\usepackage{mathrsfs}
\usetikzlibrary{arrows}
\pgfplotsset{compat=1.16}


\newcommand{\N}{\mathbb N}
\newcommand{\W}{\mathbb W}
\newcommand{\C}{\mathbb C}
\newcommand{\Z}{\mathbb Z}
\newcommand{\Q}{\mathbb Q}
\newcommand{\R}{\mathbb R}




\title{Important Definitions}
\author{Bart Snapp and Brad Findell}
\begin{document}
\begin{abstract}
Multiple-choice and select-all questions about definitions. 
\end{abstract}
\maketitle

%What is an equilateral quadrilateral? 
%What is regular quadrilateral? 
%Definition of ? 
%Write the Pythagorean theorem. 
%Measure angles. 
%Angle sum in a triangle. 
%Triangulate a figure 

\begin{question}  
An \textbf{equilateral quadrilateral} is called \dots
\begin{multipleChoice}  
\choice{a square.}  
\choice{a rectangle.}  
\choice[correct]{a rhombus.}
\choice{a trapezoid.}  
\choice{none of these.}
\end{multipleChoice}  
\end{question}

\begin{question}  
An \textbf{equiangular quadrilateral} is called \dots
\begin{multipleChoice}  
\choice{a square.}  
\choice[correct]{a rectangle.}  
\choice{a rhombus.}
\choice{a trapezoid.}  
\choice{none of these.}
\end{multipleChoice}  
\end{question}

\begin{question}  
When three (or more) points all lie on the same line, we say they are \dots
\begin{multipleChoice}  
\choice{coplanar.}  
\choice[correct]{collinear.}  
\choice{conjoined.}
\choice{concurrent.}  
\choice{none of these.}
\end{multipleChoice}  
\end{question}

\begin{question}  
When three (or more) lines all lie on the same point, we say they are \dots
\begin{multipleChoice}  
\choice{coplanar.}  
\choice{collinear.}  
\choice{conjoined.}
\choice[correct]{concurrent.}  
\choice{none of these.}
\end{multipleChoice}  
\end{question}

\begin{question}  
An \textbf{altitude} in a triangle \dots
\begin{multipleChoice}  
\choice{contains the midpoint of the side of a triangle and is perpendicular to that side.}
\choice[correct]{contains a vertex of a triangle and is perpendicular to the line containing the other side.}  
\choice{contains a vertex of a triangle and the midpoint of the opposite side.}  
\choice{contains a vertex and bisects that angle.}  
\choice{none of these.}
\end{multipleChoice}  
\end{question}

\begin{question}  
An \textbf{median} in a triangle \dots
\begin{multipleChoice}  
\choice{contains the midpoint of the side of a triangle and is perpendicular to that side.}
\choice{contains a vertex of a triangle and is perpendicular to the line containing the other side.}  
\choice[correct]{contains a vertex of a triangle and the midpoint of the opposite side.}  
\choice{contains a vertex and bisects that angle.}  
\choice{none of these.}
\end{multipleChoice}  
\end{question}

\begin{question}  
  The \textbf{circumcenter} of a triangle is \dots 
  \begin{selectAll}  
    \choice{the point of concurrency of the medians.}  
    \choice{the point of concurrency of the angle bisectors.}  
    \choice[correct]{the point of concurrency of the perpendicular bisectors.}  
    \choice{the point of concurrency of the altitudes.}  
    \choice{the balance point for the triangle.}
    \choice{the center in the triangle.}
    \choice{the center of the incircle.}
    \choice[correct]{the center of the circumcircle.}
    \choice{equidistant from the sides of the triangle.}
    \choice[correct]{equidistant from the vertices of the triangle.}    
  \end{selectAll}  
\end{question}

\begin{question}  
  The \textbf{incenter} of a triangle is \dots 
  \begin{selectAll}  
    \choice{the point of concurrency of the medians.}  
    \choice[correct]{the point of concurrency of the angle bisectors.}  
    \choice{the point of concurrency of the perpendicular bisectors.}  
    \choice{the point of concurrency of the altitudes.}  
    \choice{the balance point for the triangle.}
    \choice{the center in the triangle.}
    \choice[correct]{the center of the incircle.}
    \choice{the center of the circumcircle.}
    \choice[correct]{equidistant from the sides of the triangle.}
    \choice{equidistant from the vertices of the triangle.}    
  \end{selectAll}  
\end{question}

\begin{question}  
  The \textbf{centroid} of a triangle is \dots 
  \begin{selectAll}  
    \choice[correct]{the point of concurrency of the medians.}  
    \choice{the point of concurrency of the angle bisectors.}  
    \choice{the point of concurrency of the perpendicular bisectors.}  
    \choice{the point of concurrency of the altitudes.}  
    \choice[correct]{the balance point for the triangle.}
    \choice{the center in the triangle.}
    \choice{the center of the incircle.}
    \choice{the center of the circumcircle.}
    \choice{equidistant from the sides of the triangle.}
    \choice{equidistant from the vertices of the triangle.}    
  \end{selectAll}  
\end{question}

\begin{question}  
  The \textbf{orthocenter} of a triangle is \dots 
  \begin{selectAll}  
    \choice{the point of concurrency of the medians.}  
    \choice{the point of concurrency of the angle bisectors.}  
    \choice{the point of concurrency of the perpendicular bisectors.}  
    \choice[correct]{the point of concurrency of the altitudes.}  
    \choice{the balance point for the triangle.}
    \choice{the center in the triangle.}
    \choice{the center of the incircle.}
    \choice{the center of the circumcircle.}
    \choice{equidistant from the sides of the triangle.}
    \choice{equidistant from the vertices of the triangle.}    
  \end{selectAll}  
\end{question}

\begin{question}  
  A \textbf{midsegment} in a triangle is \dots 
  \begin{selectAll}  
    \choice{a segment in the middle.}
    \choice[correct]{a segment connecting the midpoints of two sides.}
    \choice[correct]{parallel to a side of the triangle.}
    \choice{perpendicular to a side of the triangle.}
    \choice{also called a median.}
  \end{selectAll}  
\end{question}


\end{document}