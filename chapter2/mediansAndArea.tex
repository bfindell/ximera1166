\documentclass[handout,nooutcomes]{ximera}
%\documentclass[space,handout,nooutcomes]{ximera}

% For preamble materials

\graphicspath{
  {./}
  {chapter1/}
  {chapter2/}
  {chapter4/}
  {math1/}
  {math2/}
}

\usepackage{pgf,tikz}
\usepackage{mathrsfs}
\usetikzlibrary{arrows}
\pgfplotsset{compat=1.16}


\newcommand{\N}{\mathbb N}
\newcommand{\W}{\mathbb W}
\newcommand{\C}{\mathbb C}
\newcommand{\Z}{\mathbb Z}
\newcommand{\Q}{\mathbb Q}
\newcommand{\R}{\mathbb R}




\title{Medians and Areas}
\author{Brad Findell}
\begin{document}
\begin{abstract}
Exploring areas and medians. 
\end{abstract}
\maketitle




\begin{problem}
Explore the following GeoGebra sketch.  
\begin{center}  
\geogebra{gehma87x}{720}{460}  
\end{center}
In the sketch above, points $D$, $E$, and $F$ are $\answer[format=string]{midpoints}$ of segments $AB$, $BC$, $CA$, respectively.  Point $G$ is an arbitrary point used to form six triangles with the bisected sides of $\triangle ABC$.  

What invariants do you notice as you move points $A$, $B$, $C$, and $G$?  

\begin{multipleChoice}
\choice{The sum of the areas is always the same.}
\choice[correct]{The six areas are equal in pairs.}
\choice{The six areas are equal in triples.}
\choice{None of the above.}
\end{multipleChoice}

\begin{problem}
Correct! 

This makes sense from the way the six triangle are formed.  For example, $t_1 = t_2$ because segment $\answer{GD}$ is a $\answer[format=string]{median}$ of $\triangle \answer{AGB}$.
\begin{problem}
Correct!  A similar argument can be made for why $t_3 = t_4$ and $t_5 = t_6$.

Furthermore, when $G$ is near the $\answer[format=string]{centroid}$, which is defined as the point of concurrency of the medians, all six areas appear to be almost 
\wordChoice{\choice{constant}\choice{integers}\choice[correct]{equal}\choice{unbounded}}, which suggests a theorem about the areas of the triangles formed by the medians of a given triangle. 

\end{problem}
\end{problem}
\end{problem}


\end{document}
