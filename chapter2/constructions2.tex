\documentclass[nooutcomes]{ximera}
%\documentclass[space,handout,nooutcomes]{ximera}

% For preamble materials

\usepackage{pgf,tikz}
\usepackage{mathrsfs}
\usetikzlibrary{arrows}
\usepackage{framed}
\usepackage{amsmath}
\pgfplotsset{compat=1.17}

\def\fixnote#1{\begin{framed}{\textcolor{red}{Fix note: #1}}\end{framed}}  % Allows insertion of red notes about needed edits
%\def\fixnote#1{}

\def\detail#1{{\textcolor{blue}{Detail: #1}}}   

\pdfOnly{\renewenvironment{image}[1][]{\begin{center}}{\end{center}}}

\graphicspath{
  {./}
  {chapter1/}
  {chapter2/}
  {chapter4/}
  {proofs/}
  {graphics/}
  {../graphics/}
}

\newenvironment{sectionOutcomes}{}{}


%%% This set of code is all of our user defined commands
\newcommand{\bysame}{\mbox{\rule{3em}{.4pt}}\,}
\newcommand{\N}{\mathbb N}
\newcommand{\C}{\mathbb C}
\newcommand{\W}{\mathbb W}
\newcommand{\Z}{\mathbb Z}
\newcommand{\Q}{\mathbb Q}
\newcommand{\R}{\mathbb R}
\newcommand{\A}{\mathbb A}
\newcommand{\D}{\mathcal D}
\newcommand{\F}{\mathcal F}
\newcommand{\ph}{\varphi}
\newcommand{\ep}{\varepsilon}
\newcommand{\aph}{\alpha}
\newcommand{\QM}{\begin{center}{\huge\textbf{?}}\end{center}}

\renewcommand{\le}{\leqslant}
\renewcommand{\ge}{\geqslant}
\renewcommand{\a}{\wedge}
\renewcommand{\v}{\vee}
\renewcommand{\l}{\ell}
\newcommand{\mat}{\mathsf}
\renewcommand{\vec}{\mathbf}
\renewcommand{\subset}{\subseteq}
\renewcommand{\supset}{\supseteq}
%\renewcommand{\emptyset}{\varnothing}
%\newcommand{\xto}{\xrightarrow}
%\renewcommand{\qedsymbol}{$\blacksquare$}
%\newcommand{\bibname}{References and Further Reading}
%\renewcommand{\bar}{\protect\overline}
%\renewcommand{\hat}{\protect\widehat}
%\renewcommand{\tilde}{\widetilde}
%\newcommand{\tri}{\triangle}
%\newcommand{\minipad}{\vspace{1ex}}
%\newcommand{\leftexp}[2]{{\vphantom{#2}}^{#1}{#2}}

%% More user defined commands
\renewcommand{\epsilon}{\varepsilon}
\renewcommand{\theta}{\vartheta} %% only for kmath
\renewcommand{\l}{\ell}
\renewcommand{\d}{\, d}
\newcommand{\ddx}{\frac{d}{dx}}
\newcommand{\dydx}{\frac{dy}{dx}}


\usepackage{bigstrut}


\title{Parallel Construction}
\author{Bart Snapp and Brad Findell}
\begin{document}
\begin{abstract}
Short-answer questions about constructions.
\end{abstract}
\maketitle

%\usepackage{pgfplots}
%\pgfplotsset{compat=1.15}
%\usepackage{mathrsfs}
%\usetikzlibrary{arrows}
%\pagestyle{empty}

\begin{problem}
In the diagram below, points $A$ and $B$ and the line through $A$ are given.    
\begin{image}
\definecolor{uuuuuu}{rgb}{0.26666666666666666,0.26666666666666666,0.26666666666666666}
\definecolor{qqqqff}{rgb}{0.,0.,1.}
\begin{tikzpicture}[scale=0.8,line cap=round,line join=round,>=triangle 45,x=1.0cm,y=1.0cm]
\clip(-2.5,-1) rectangle (12.5,7.1);
\draw [line width=0.8pt,domain=-2.5:13.] plot(\x,{(--1.-0.*\x)/1.});
\draw [line width=0.8pt,domain=0.0:13.0,color=lightgray] plot(\x,{(--3.--2.*\x)/3.});
\draw [line width=0.8pt,color=lightgray] (3.,3.) circle (3.6055512754639896cm);
\draw [line width=0.8pt,color=lightgray] (6.,1.) circle (3.6055512754639896cm);
\draw [line width=0.8pt,color=lightgray] (6.,5.) circle (3.6055512754639896cm);
\draw [line width=0.8pt,color=lightgray] (3.,3.) -- (6.,1.);
\draw [line width=0.8pt,color=lightgray] (3.,3.) -- (3.,1.);
\draw [line width=0.8pt,domain=-2.5:13.,color=lightgray] plot(\x,{(--18.-0.*\x)/6.});
\begin{scriptsize}
\draw [fill=qqqqff] (0.,1.) circle (1.5pt);
\draw[color=qqqqff] (0.1,1.4) node {$A$};
\draw [fill=qqqqff] (3.,3.) circle (1.5pt);
\draw[color=qqqqff] (2.7,3.2) node {$B$};
\draw [fill=,color=lightgray] (6.,1.) circle (1.5pt);
\draw[color=uuuuuu] (6.1,0.8) node {$C$};
\draw [fill=,color=lightgray] (6.,5.) circle (1.5pt);
\draw[color=uuuuuu] (6.0,5.4) node {$D$};
\draw [fill=,color=lightgray] (9.,3.) circle (1.5pt);
\draw[color=uuuuuu] (9.3,3.2) node {$E$};
\end{scriptsize}
\end{tikzpicture}
\end{image}
Question: What is the purpose of the construction shown in gray?  
Answer: This is a \wordChoice{\choice{incorrect}\choice[correct]{correct}} attempt at a(n)
\wordChoice{\choice{angle bisector}\choice{perpendicular bisector}\choice[correct]{parallel line}{perpendicular line}}. 
\begin{problem}
Correct!  And $\overleftrightarrow{BE}$ is parallel to the given line through $A$ by the following steps: 
\begin{enumerate}
\item Construct ray $\answer{AB}$. 
\item Construct circle with center $\answer{B}$ through point $\answer{A}$, marking intersections $\answer{C}$ and $D$. 
\item Bisect $\angle CBD$ as follows: 
\begin{enumerate}
\item Construct circle with center $C$ through point $\answer{B}$. 
\item Construct circle with center $\answer{D}$ through point $\answer{B}$.
\item Mark the intersection $\answer{E}$.
\item Construct line $\answer{BE}$. 
\end{enumerate}
\end{enumerate}

\begin{problem}
To prove that the construction is valid, consider the diagram below.  Note that the circles have been hidden, but their congruent radii have been marked.  
\begin{image}
\definecolor{qqwuqq}{rgb}{0.,0.39215686274509803,0.}
\definecolor{uuuuuu}{rgb}{0.26666666666666666,0.26666666666666666,0.26666666666666666}
\definecolor{qqqqff}{rgb}{0.,0.,1.}
\begin{tikzpicture}[line cap=round,line join=round,>=triangle 45,x=1.0cm,y=1.0cm]
\clip(-1.4,0.4) rectangle (12.2,5.6);
\draw[line width=0.8pt,color=qqwuqq,fill=qqwuqq,fill opacity=0.10000000149011612] (3.,1.310597556069689) -- (2.689402443930311,1.3105975560696892) -- (2.689402443930311,1.) -- (3.,1.) -- cycle; 
\draw [shift={(3.,3.)},line width=0.8pt,color=qqwuqq,fill=qqwuqq,fill opacity=0.10000000149011612] (0,0) -- (-33.690067525979785:0.43925127623369215) arc (-33.690067525979785:0.:0.43925127623369215) -- cycle;
\draw [shift={(3.,3.)},line width=0.8pt,color=qqwuqq,fill=qqwuqq,fill opacity=0.10000000149011612] (0,0) -- (0.:0.43925127623369215) arc (0.:33.690067525979785:0.43925127623369215) -- cycle;
\draw [shift={(3.,3.)},line width=0.8pt,color=qqwuqq,fill=qqwuqq,fill opacity=0.10000000149011612] (0,0) -- (-146.30993247402023:0.43925127623369215) arc (-146.30993247402023:-90.:0.43925127623369215) -- cycle;
\draw [shift={(3.,3.)},line width=0.8pt,color=qqwuqq,fill=qqwuqq,fill opacity=0.10000000149011612] (0,0) -- (-90.:0.43925127623369215) arc (-90.:-33.690067525979785:0.43925127623369215) -- cycle;
\draw [line width=0.8pt,domain=-1.4:12.2] plot(\x,{(--1.-0.*\x)/1.});
\draw [line width=0.8pt,domain=-1.4:12.2] plot(\x,{(--18.-0.*\x)/6.});
\draw [line width=0.8pt] (0.,1.)-- (3.,3.);
\draw [line width=0.8pt] (1.4451781269482222,2.0822328095776665) -- (1.554821873051778,1.9177671904223341);
\draw [line width=0.8pt] (3.,3.)-- (6.,5.);
\draw [line width=0.8pt] (4.445178126948223,4.0822328095776665) -- (4.554821873051777,3.917767190422334);
\draw [line width=0.8pt] (6.,5.)-- (9.,3.);
\draw [line width=0.8pt] (7.554821873051778,4.0822328095776665) -- (7.445178126948224,3.917767190422334);
\draw [line width=0.8pt] (9.,3.)-- (6.,1.);
\draw [line width=0.8pt] (7.554821873051778,1.9177671904223341) -- (7.445178126948224,2.0822328095776665);
\draw [line width=0.8pt] (6.,1.)-- (3.,3.);
\draw [line width=0.8pt] (4.445178126948223,1.9177671904223341) -- (4.554821873051777,2.0822328095776665);
\draw [line width=0.8pt] (3.,3.)-- (3.,1.);
\draw [shift={(3.,3.)},line width=0.8pt,color=qqwuqq] (-33.690067525979785:0.43925127623369215) arc (-33.690067525979785:0.:0.43925127623369215);
\draw[line width=0.8pt,color=qqwuqq] (3.3630839558560943,2.912991911562316) -- (3.491231234393539,2.882283174466663);
\draw[line width=0.8pt,color=qqwuqq] (3.350367516068977,2.8709923640347403) -- (3.47402663938744,2.825460257223473);
\draw [shift={(3.,3.)},line width=0.8pt,color=qqwuqq] (0.:0.43925127623369215) arc (0.:33.690067525979785:0.43925127623369215);
\draw[line width=0.8pt,color=qqwuqq] (3.350367516068977,3.12900763596526) -- (3.47402663938744,3.1745397427765276);
\draw[line width=0.8pt,color=qqwuqq] (3.3630839558560943,3.0870080884376843) -- (3.491231234393539,3.1177168255333374);
\draw [shift={(3.,3.)},line width=0.8pt,color=qqwuqq] (-146.30993247402023:0.43925127623369215) arc (-146.30993247402023:-90.:0.43925127623369215);
\draw[line width=0.8pt,color=qqwuqq] (2.823825433407715,2.6708148111781) -- (2.7616461746104375,2.5546318033586055);
\draw [shift={(3.,3.)},line width=0.8pt,color=qqwuqq] (-90.:0.43925127623369215) arc (-90.:-33.690067525979785:0.43925127623369215);
\draw[line width=0.8pt,color=qqwuqq] (3.1761745665922856,2.6708148111781) -- (3.238353825389563,2.5546318033586055);
\begin{scriptsize}
\draw [fill=qqqqff] (0.,1.) circle (1.5pt);
\draw[color=qqqqff] (0.1,1.4) node {$A$};
\draw [fill=qqqqff] (3.,3.) circle (1.5pt);
\draw[color=qqqqff] (2.7,3.2) node {$B$};
\draw [fill=uuuuuu] (6.,1.) circle (1.5pt);
\draw[color=uuuuuu] (6.1,0.8) node {$C$};
\draw [fill=uuuuuu] (6.,5.) circle (1.5pt);
\draw[color=uuuuuu] (6.0,5.4) node {$D$};
\draw [fill=uuuuuu] (9.,3.) circle (1.5pt);
\draw[color=uuuuuu] (9.3,3.2) node {$E$};
\draw [fill=uuuuuu] (3.,1.) circle (1.5pt);
\draw[color=uuuuuu] (3.0,0.8) node {$F$};
\draw[color=qqwuqq] (3.8,2.8) node {$\beta$};
\draw[color=qqwuqq] (3.8,3.24) node {$\beta$};
\draw[color=qqwuqq] (2.7,2.4) node {$\alpha$};
\draw[color=qqwuqq] (3.4,2.4) node {$\alpha$};
\end{scriptsize}
\end{tikzpicture}
\end{image}
In the answer boxes, type \verb|\alpha| for $\alpha$ and \verb|\beta| for $\beta$. 
\begin{enumerate}
\item We know $\overline{BE}$ bisects $\angle DBC$ because quadrilateral $BDEC$ is a $\answer[format=string]{rhombus}$, in which the diagonals bisect the interior angles.  The congruent angles are marked with $\answer{\beta}$.  
\item We know that altitude $\overline{BF}$ bisects $\angle ABC$ because in $\triangle ABC$ is $\answer[format=string]{isosceles}$, in which an altitude to the base also bisects the vertex angle.  The congruent angles are marked with $\answer{\alpha}$. 
\item Because point $D$ lies on $\overrightarrow{AB}$, we see $\angle ABD$ is a $\answer[format=string]{straight angle}$ (two words).  
\item Using the marked angles of measures $\alpha$ and $\beta$, we see that $\angle ABD = \answer{2\alpha + 2\beta} = 180$ degrees.  
\item This equation simplifies to $\alpha + \beta = \answer{90}$ degrees, which means that $\alpha$ and $\beta$ are $\answer[format=string]{complementary}$. 
\item Because $\triangle ABC$ is $\answer[format=string]{right}$, we see that $m\angle BAC = 90 - \alpha = \answer{\beta} = m\angle DBE$.
\item And finally $\overleftrightarrow{BE}$ is parallel to  $\overleftrightarrow{AC}$ as desired, because $\angle BAF$ and $\angle DBE$, 
are congruent $\answer[format=string]{corresponding}$ angles.  
\end{enumerate}

\end{problem}
\end{problem}
\end{problem}


\end{document}
