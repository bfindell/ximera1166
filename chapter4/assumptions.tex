\documentclass[nooutcomes]{ximera}
%\documentclass[space,handout,nooutcomes]{ximera}

% For preamble materials

\usepackage{pgf,tikz}
\usepackage{mathrsfs}
\usetikzlibrary{arrows}
\usepackage{framed}
\usepackage{amsmath}
\pgfplotsset{compat=1.17}

\def\fixnote#1{\begin{framed}{\textcolor{red}{Fix note: #1}}\end{framed}}  % Allows insertion of red notes about needed edits
%\def\fixnote#1{}

\def\detail#1{{\textcolor{blue}{Detail: #1}}}   

\pdfOnly{\renewenvironment{image}[1][]{\begin{center}}{\end{center}}}

\graphicspath{
  {./}
  {chapter1/}
  {chapter2/}
  {chapter4/}
  {proofs/}
  {graphics/}
  {../graphics/}
}

\newenvironment{sectionOutcomes}{}{}


%%% This set of code is all of our user defined commands
\newcommand{\bysame}{\mbox{\rule{3em}{.4pt}}\,}
\newcommand{\N}{\mathbb N}
\newcommand{\C}{\mathbb C}
\newcommand{\W}{\mathbb W}
\newcommand{\Z}{\mathbb Z}
\newcommand{\Q}{\mathbb Q}
\newcommand{\R}{\mathbb R}
\newcommand{\A}{\mathbb A}
\newcommand{\D}{\mathcal D}
\newcommand{\F}{\mathcal F}
\newcommand{\ph}{\varphi}
\newcommand{\ep}{\varepsilon}
\newcommand{\aph}{\alpha}
\newcommand{\QM}{\begin{center}{\huge\textbf{?}}\end{center}}

\renewcommand{\le}{\leqslant}
\renewcommand{\ge}{\geqslant}
\renewcommand{\a}{\wedge}
\renewcommand{\v}{\vee}
\renewcommand{\l}{\ell}
\newcommand{\mat}{\mathsf}
\renewcommand{\vec}{\mathbf}
\renewcommand{\subset}{\subseteq}
\renewcommand{\supset}{\supseteq}
%\renewcommand{\emptyset}{\varnothing}
%\newcommand{\xto}{\xrightarrow}
%\renewcommand{\qedsymbol}{$\blacksquare$}
%\newcommand{\bibname}{References and Further Reading}
%\renewcommand{\bar}{\protect\overline}
%\renewcommand{\hat}{\protect\widehat}
%\renewcommand{\tilde}{\widetilde}
%\newcommand{\tri}{\triangle}
%\newcommand{\minipad}{\vspace{1ex}}
%\newcommand{\leftexp}[2]{{\vphantom{#2}}^{#1}{#2}}

%% More user defined commands
\renewcommand{\epsilon}{\varepsilon}
\renewcommand{\theta}{\vartheta} %% only for kmath
\renewcommand{\l}{\ell}
\renewcommand{\d}{\, d}
\newcommand{\ddx}{\frac{d}{dx}}
\newcommand{\dydx}{\frac{dy}{dx}}


\usepackage{bigstrut}


%\usepackage{tikz}


\title{Assumptions}
\author{Bart Snapp and Brad Findell}
\begin{document}
\begin{abstract}
Short-answer problems about assumptions in various geometries.
\end{abstract}
\maketitle

In our approach to Euclidean (i.e., flat) geometry, we make the following assumptions:
\begin{itemize}
\item \textbf{(A1)} Through two distinct points passes a unique line.
\item \textbf{(A2)} Given a line and a point not on the line, there is exactly one line passing through the point which is parallel to the given line (Parallel postulate).
\item \textbf{(A3)} The points on a line can be placed in one-to-one correspondence with the real numbers so that differences measure distances (Ruler postulate).  
\item \textbf{(A4)} The rays with a common endpoint can be numbered so that differences measure angles and so that straight angles measure $180^\circ$ (Protractor postulate). 
\item \textbf{(A5)} Every basic rigid motion (rotation, reflection, or translation) has the following properties:
\begin{enumerate}
\item It maps a line to a line, a ray to a ray, and a segment to a segment.
\item It preserves distance and angle measure.
\end{enumerate}
\item \textbf{(A6)} Areas of geometric figures have the following properties: 
\begin{enumerate}
\item Congruent figures enclose equal areas.
\item Area is additive, i.e., the area of the union of two regions that overlap only at their boundaries is the sum of their areas. 
%\item Area is measured by tiling a region with a two-dimensional unit (such as a square) and parts of the unit, without gaps or overlaps. 
\item A rectangle with side-lengths $a$ and $b$ has area $ab$, where $a$ and $b$ can be any non-negative real numbers.
\end{enumerate}
\end{itemize}

In the following problems, consider whether these assumptions hold in spherical and hyperbolic geometry. 



\begin{problem}
\textbf{(A1)} Through two distinct points passes a unique line.

Does this assumption hold in spherical and hyperbolic geometry? 
$\answer[format=string]{no}$

\begin{problem}
In \textbf{hyperbolic} geometry, this assumption holds. 

In \textbf{spherical} geometry, if the points are on $\answer[format=string]{opposite}$ ends of a diameter, many lines pass through both.
\begin{feedback}
\textbf{Correct!} The North and South Poles are easy examples.  
\end{feedback}
\end{problem}
\end{problem}

\begin{problem}
\textbf{(A2)} Given a line and a point not on the line, there is exactly one line passing through the point which is parallel to the given line (Parallel postulate).

Does this assumption hold in spherical and hyperbolic geometry? 
$\answer[format=string]{no}$
\begin{problem}
In $\answer[format=string]{spherical}$ geometry, there are no parallels. 

In $\answer[format=string]{hyperbolic}$ geometry, there is more than one parallel through a point not on a line. 
\end{problem}
\end{problem}

\begin{problem}
\textbf{(A3)} The points on a line can be placed in one-to-one correspondence with the real numbers so that differences measure distances (Ruler postulate).  

Does this assumption hold in spherical and hyperbolic geometry? 
$\answer[format=string]{no}$

\begin{problem}
In $\answer[format=string]{hyperbolic}$ geometry, this assumption holds. 

In $\answer[format=string]{spherical}$ geometry, lines are finite in length.  Attempting to place the real number line on a great circle on the sphere requires ``wrapping it around'' so that many real numbers land on any point on the great circle.  
\end{problem}
\end{problem}

\begin{problem}
\textbf{(A4)} The rays with a common endpoint can be numbered so that differences measure angles and so that straight angles measure $180^\circ$ (Protractor postulate). 

Does this assumption hold in spherical and hyperbolic geometry? 
$\answer[format=string]{yes}$
\end{problem}

\begin{problem}
\textbf{(A5)} Every basic rigid motion (rotation, reflection, or translation) has the following properties:
\begin{enumerate}
\item It maps a line to a line, a ray to a ray, and a segment to a segment.
\item It preserves distance and angle measure.
\end{enumerate}

Does this assumption hold in spherical and hyperbolic geometry? 
$\answer[format=string]{yes}$
\begin{feedback}
\textbf{Correct!} But in spherical geometry, translations along a line are actually rotations about the poles of that line.  (Think about moving a spherical transparency on a globe.) 
\end{feedback}
\end{problem}

\begin{problem}
\textbf{(A6)} Areas of geometric figures have the following properties: 
\begin{enumerate}
\item Congruent figures enclose equal areas.
\item Area is additive, i.e., the area of the union of two regions that overlap only at their boundaries is the sum of their areas. 
\item A rectangle with side-lengths $a$ and $b$ has area $ab$, where $a$ and $b$ can be any non-negative real numbers.
\end{enumerate}

Does this assumption hold in spherical and hyperbolic geometry? 
$\answer[format=string]{no}$
\begin{problem}
Rectangles (i.e., quadrilaterals with four $\answer[format=string]{right}$ angles) do not exist in spherical geometry.  Neither do they exist in hyperbolic geometry.  

In Euclidean geometry, area can be based on tiling with unit squares.  But in spherical and hyperbolic geometry, squares don't exist, and regular quadrilaterals don't tile the plane.
\end{problem}
\end{problem}

% Betweenness
% Angle sum in a triangle.
% Angle sum in a quadrilateral. 

\begin{problem}
\begin{enumerate}
\item In Euclidean geometry, the sum of the interior angles of a triangle is 
\wordChoice{\choice{less than}\choice[correct]{equal to}\choice{greater than}} 
$\answer{180}$ degrees.
\item In spherical geometry, the sum of the interior angles of a triangle is 
\wordChoice{\choice{less than}\choice{equal to}\choice[correct]{greater than}} 
$\answer{180}$ degrees.
\item In hyperbolic geometry, the sum of the interior angles of a triangle is 
\wordChoice{\choice[correct]{less than}\choice{equal to}\choice{greater than}} 
$\answer{180}$ degrees.
\end{enumerate}
\end{problem}

\begin{problem}
\begin{enumerate}
\item In Euclidean geometry, the sum of the interior angles of a quadrilateral is 
\wordChoice{\choice{less than}\choice[correct]{equal to}\choice{greater than}} 
$\answer{360}$ degrees.
\item In spherical geometry, the sum of the interior angles of a quadrilateral is 
\wordChoice{\choice{less than}\choice{equal to}\choice[correct]{greater than}} 
$\answer{360}$ degrees.
\item In hyperbolic geometry, the sum of the interior angles of a quadrilateral is 
\wordChoice{\choice[correct]{less than}\choice{equal to}\choice{greater than}} 
$\answer{360}$ degrees.
\end{enumerate}
\end{problem}

\begin{problem}
In Euclidean geometry, if three distinct points are collinear, then exactly one must lie between the other two.  Does this statement hold in spherical and hyperbolic geometry? 
$\answer[format=string]{no}$

\begin{problem}
In $\answer[format=string]{hyperbolic}$ geometry, this assumption holds. 

In $\answer[format=string]{spherical}$ geometry, lines are great circles.  So any of the three points can be seen as ``between'' the other two.  
\end{problem}
\end{problem}

\end{document}
