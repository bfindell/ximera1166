\documentclass[nooutcomes]{ximera}
%\documentclass[space,handout,nooutcomes]{ximera}

% For preamble materials

\graphicspath{
  {./}
  {chapter1/}
  {chapter2/}
  {chapter4/}
  {math1/}
  {math2/}
}

\usepackage{pgf,tikz}
\usepackage{mathrsfs}
\usetikzlibrary{arrows}
\pgfplotsset{compat=1.16}


\newcommand{\N}{\mathbb N}
\newcommand{\W}{\mathbb W}
\newcommand{\C}{\mathbb C}
\newcommand{\Z}{\mathbb Z}
\newcommand{\Q}{\mathbb Q}
\newcommand{\R}{\mathbb R}




%\usepackage{tikz}


\title{Transformations}
\author{Bart Snapp and Brad Findell}
\begin{document}
\begin{abstract}
Short-answer problems about the definitions of basic rigid motions and dilations.
\end{abstract}
\maketitle

%\begin{question}
%What is required to specify a translation?  
%\begin{freeResponse}
%\begin{hint}
%A vector.  Or (equivalently) a magnitude and a direction.  
%\end{hint}
%\end{freeResponse}
%\end{question}

%\begin{question}
%What is required to specify a rotation? 
%\begin{freeResponse}
%\begin{hint}
%A center and an angle (assuming an agreement about the direction of rotation).  
%\end{hint}
%\end{freeResponse}
%\end{question}

%\begin{question}
%What is required to specify a reflection?  
%\begin{freeResponse}
%\begin{hint}
%A line.  
%\end{hint}
%\end{freeResponse}
%\end{question}

To define a translation, rotation, or reflection, we specify the image of a generic point $P$ under the transformation as specified.  

\begin{question}
Under a \textbf{rotation} about center $O$ counterclockwise by an angle $\theta$, the image of $P$ is 
a point $Q$ so that $OQ = OP$ and $m\angle POQ = \theta$, measured counterclockwise 
from $\overrightarrow{OP}$ to $\overrightarrow{OQ}$ if $\theta>0$ and measured clockwise if $\theta<0$.  If $P = O$ then $Q=O=P$.  
\end{question}

\begin{question}
Under a \textbf{translation} by the distance and direction from $A$ to $B$, the image of $P$ is a point $Q$ so that $\overline{PQ} \parallel \overline{AB}$, $PQ=AB$, and $\overrightarrow{PQ}$ is in the same direction as $\overrightarrow{AB}$.  
\end{question}

\begin{question}
Under a \textbf{reflection} about a line $\ell$, if $P$ is on $\ell$, then $P$ is mapped to itself.  If $P$ is not on $\ell$, then the image of $P$ is a point $Q$ so that $\ell$ is the 
perpendicular bisector of $\overline{PQ}$.  
\end{question}

\begin{question}
Under a \textbf{dilation} about center $O$ and scale factor $r>0$, the image of $P$ is 
a point $Q$ so that $Q$ lies on $\overrightarrow{OP}$ and $OQ = rOP$.  The image of $O$ is $O$. 
\end{question}



\end{document}
