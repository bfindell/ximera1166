\documentclass[nooutcomes]{ximera}
%\documentclass[space,handout,nooutcomes]{ximera}

% For preamble materials

\graphicspath{
  {./}
  {chapter1/}
  {chapter2/}
  {chapter4/}
  {math1/}
  {math2/}
}

\usepackage{pgf,tikz}
\usepackage{mathrsfs}
\usetikzlibrary{arrows}
\pgfplotsset{compat=1.16}


\newcommand{\N}{\mathbb N}
\newcommand{\W}{\mathbb W}
\newcommand{\C}{\mathbb C}
\newcommand{\Z}{\mathbb Z}
\newcommand{\Q}{\mathbb Q}
\newcommand{\R}{\mathbb R}




%\usepackage{tikz}


\title{Betweenness}
\author{Bart Snapp and Brad Findell}
\begin{document}
\begin{abstract}
Short-answer problems about betweenness. 
\end{abstract}
\maketitle


\begin{question}
Can the two distinct lines have more than one point in common?  Use the above axioms to explain your reasoning.  
\end{question}
%\QM

%Lemma 2. If three lines L1, L2, and L3 have the property that L1 || L2 and L2 || L3,
%then L1 || L3.

The ruler postulate supports a definition of betweenness, which will allow us to to define line segment and ray.    

\begin{definition}
If points $A$, $X$, and $B$ are on a line $\l$, we say that $X$ is \emph{between} $A$ and $B$ if $AX + XB = AB$.
\end{definition}

\begin{question}
In the figure below, $C$, $M$, and $Y$ are on $\overleftrightarrow{AB}$, but $X$ is not.  Suppose $AB=5$ cm, $AC=4$ cm, $AY=1$ cm, and $M$ is the midpoint of $AB$.  Use the definition 

Then $AM=\answer{5/2}$ cm, $BM=\answer{5/2}$ cm, $BC=\answer{1}$ cm, and $AA=\answer{0}$ cm.  
\end{question}

\definecolor{xdxdff}{rgb}{0.49019607843137253,0.49019607843137253,1.}
\definecolor{uuuuuu}{rgb}{0.26666666666666666,0.26666666666666666,0.26666666666666666}
\definecolor{qqqqff}{rgb}{0.,0.,1.}
\begin{tikzpicture}[line cap=round,line join=round,>=triangle 45,x=1.0cm,y=1.0cm]
%\clip(-1.44,-3.9) rectangle (13.2,5.56);
\draw [line width=0.8pt] (1.,1.)-- (6.,1.);
\begin{scriptsize}
\draw [fill=qqqqff] (1.,1.) circle (1.5pt);
\draw[color=qqqqff] (1.14,1.29) node {$A$};
\draw [fill=qqqqff] (6.,1.) circle (1.5pt);
\draw[color=qqqqff] (6.14,1.29) node {$B$};
\draw [fill=uuuuuu] (3.5,1.) circle (1.5pt);
\draw[color=uuuuuu] (3.64,1.29) node {$M$};
\draw [fill=xdxdff] (5.,1.) circle (1.5pt);
\draw[color=xdxdff] (5.14,1.29) node {$C$};
\draw [fill=qqqqff] (3.5,2.) circle (1.5pt);
\draw[color=qqqqff] (3.64,2.29) node {$X$};
\draw [fill=qqqqff] (0.,1.) circle (1.5pt);
\draw[color=qqqqff] (0.14,1.29) node {$Y$};
\draw [fill=white] (-1.,1) circle (0.5pt);
\draw [fill=white] (7.,1) circle (0.5pt);
\end{scriptsize}
\end{tikzpicture}


$M$ is between $A$ and $B$ because 

We can manipulate lengths as quantities.  


\begin{question}
Use the concept of betweenness to define line segment $\overline{AB}$.  Now use the concept of betweenness to 
define ray $\overrightarrow{AB}$. 

\begin{solution}
Solution here. 
\end{solution}
\end{question}
%\QM

\begin{question}
Use the protractor postulate to provide a definition of adjacent angles, analogous to betweenness for distances.  
\end{question}
%\QM



\end{document}
