% Section 1.1 Anatomy of Figures

\documentclass[nooutcomes]{ximera}
%\documentclass[space,handout,nooutcomes]{ximera}

% For preamble materials

\usepackage{pgf,tikz}
\usepackage{mathrsfs}
\usetikzlibrary{arrows}
\usepackage{framed}
\usepackage{amsmath}
\pgfplotsset{compat=1.17}

\def\fixnote#1{\begin{framed}{\textcolor{red}{Fix note: #1}}\end{framed}}  % Allows insertion of red notes about needed edits
%\def\fixnote#1{}

\def\detail#1{{\textcolor{blue}{Detail: #1}}}   

\pdfOnly{\renewenvironment{image}[1][]{\begin{center}}{\end{center}}}

\graphicspath{
  {./}
  {chapter1/}
  {chapter2/}
  {chapter4/}
  {proofs/}
  {graphics/}
  {../graphics/}
}

\newenvironment{sectionOutcomes}{}{}


%%% This set of code is all of our user defined commands
\newcommand{\bysame}{\mbox{\rule{3em}{.4pt}}\,}
\newcommand{\N}{\mathbb N}
\newcommand{\C}{\mathbb C}
\newcommand{\W}{\mathbb W}
\newcommand{\Z}{\mathbb Z}
\newcommand{\Q}{\mathbb Q}
\newcommand{\R}{\mathbb R}
\newcommand{\A}{\mathbb A}
\newcommand{\D}{\mathcal D}
\newcommand{\F}{\mathcal F}
\newcommand{\ph}{\varphi}
\newcommand{\ep}{\varepsilon}
\newcommand{\aph}{\alpha}
\newcommand{\QM}{\begin{center}{\huge\textbf{?}}\end{center}}

\renewcommand{\le}{\leqslant}
\renewcommand{\ge}{\geqslant}
\renewcommand{\a}{\wedge}
\renewcommand{\v}{\vee}
\renewcommand{\l}{\ell}
\newcommand{\mat}{\mathsf}
\renewcommand{\vec}{\mathbf}
\renewcommand{\subset}{\subseteq}
\renewcommand{\supset}{\supseteq}
%\renewcommand{\emptyset}{\varnothing}
%\newcommand{\xto}{\xrightarrow}
%\renewcommand{\qedsymbol}{$\blacksquare$}
%\newcommand{\bibname}{References and Further Reading}
%\renewcommand{\bar}{\protect\overline}
%\renewcommand{\hat}{\protect\widehat}
%\renewcommand{\tilde}{\widetilde}
%\newcommand{\tri}{\triangle}
%\newcommand{\minipad}{\vspace{1ex}}
%\newcommand{\leftexp}[2]{{\vphantom{#2}}^{#1}{#2}}

%% More user defined commands
\renewcommand{\epsilon}{\varepsilon}
\renewcommand{\theta}{\vartheta} %% only for kmath
\renewcommand{\l}{\ell}
\renewcommand{\d}{\, d}
\newcommand{\ddx}{\frac{d}{dx}}
\newcommand{\dydx}{\frac{dy}{dx}}


\usepackage{bigstrut}


\title{Exterior Angle}
\author{Brad Findell}
\begin{document}
\begin{abstract}
Short-answer questions involving exterior angles in triangles. 
\end{abstract}
\maketitle


In the sketch below, explore the relationship between an exterior angle and its remote interior angles. 
% Exterior Angle
\begin{center}  
\geogebra{npuywgsd}{680}{420}  
\end{center}

\begin{problem}
In the above sketch, the exterior angle is 
\wordChoice{\choice{$\angle CBA$}\choice[correct]{$\angle DBC$}\choice{$\angle A$}\choice{$\angle C$}}, and its (corresponding) remote interior angles are \wordChoice{\choice{$\angle CBA$}\choice{$\angle DBC$}\choice[correct]{$\angle A$}\choice{$\angle C$}} and \wordChoice{\choice{$\angle CBA$}\choice{$\angle DBC$}\choice{$\angle A$}\choice[correct]{$\angle C$}}. 

\begin{feedback}[correct]
\textbf{Correct!} We say ``corresponding'' because a different exterior angle would correspond to different remote interior angles.
\end{feedback}
\begin{problem}
We call $\angle A$ and $\angle C$ ``remote'' interior angles corresponding to $\angle DBC$ because, compared to the other interior angle, $\angle CBA$, they are 
\wordChoice{\choice{next to}\choice[correct]{further away from}\choice{above}\choice{better than}} $\angle DBC$.
\end{problem}
\end{problem}

\begin{problem}
We observe in the sketch that the measure of the exterior angle is equal to the 
\wordChoice{\choice{product}\choice[correct]{sum}\choice{difference}} of the measures of its corresponding remote interior angles.  

This make sense because 
\[
m\angle DBC + m\angle CBA = \answer{180}\textrm{ degrees},
\]
as a linear pair, and also 
\[
m\angle A + m\angle \answer{C} + m\angle CBA = 180\textrm{ degrees}, 
\]
as the sum of the interior angles of $\triangle \answer{ABC}$.  

By comparing the two equations (or with a little algebra), it follows that 
\[
m\angle DBC = m\angle \answer{A} + m\angle C.
\]
\end{problem}


\end{document}

