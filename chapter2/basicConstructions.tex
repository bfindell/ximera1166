\documentclass[nooutcomes]{ximera}
%\documentclass[space,handout,nooutcomes]{ximera}

% For preamble materials

\usepackage{pgf,tikz}
\usepackage{mathrsfs}
\usetikzlibrary{arrows}
\usepackage{framed}
\usepackage{amsmath}
\pgfplotsset{compat=1.17}

\def\fixnote#1{\begin{framed}{\textcolor{red}{Fix note: #1}}\end{framed}}  % Allows insertion of red notes about needed edits
%\def\fixnote#1{}

\def\detail#1{{\textcolor{blue}{Detail: #1}}}   

\pdfOnly{\renewenvironment{image}[1][]{\begin{center}}{\end{center}}}

\graphicspath{
  {./}
  {chapter1/}
  {chapter2/}
  {chapter4/}
  {proofs/}
  {graphics/}
  {../graphics/}
}

\newenvironment{sectionOutcomes}{}{}


%%% This set of code is all of our user defined commands
\newcommand{\bysame}{\mbox{\rule{3em}{.4pt}}\,}
\newcommand{\N}{\mathbb N}
\newcommand{\C}{\mathbb C}
\newcommand{\W}{\mathbb W}
\newcommand{\Z}{\mathbb Z}
\newcommand{\Q}{\mathbb Q}
\newcommand{\R}{\mathbb R}
\newcommand{\A}{\mathbb A}
\newcommand{\D}{\mathcal D}
\newcommand{\F}{\mathcal F}
\newcommand{\ph}{\varphi}
\newcommand{\ep}{\varepsilon}
\newcommand{\aph}{\alpha}
\newcommand{\QM}{\begin{center}{\huge\textbf{?}}\end{center}}

\renewcommand{\le}{\leqslant}
\renewcommand{\ge}{\geqslant}
\renewcommand{\a}{\wedge}
\renewcommand{\v}{\vee}
\renewcommand{\l}{\ell}
\newcommand{\mat}{\mathsf}
\renewcommand{\vec}{\mathbf}
\renewcommand{\subset}{\subseteq}
\renewcommand{\supset}{\supseteq}
%\renewcommand{\emptyset}{\varnothing}
%\newcommand{\xto}{\xrightarrow}
%\renewcommand{\qedsymbol}{$\blacksquare$}
%\newcommand{\bibname}{References and Further Reading}
%\renewcommand{\bar}{\protect\overline}
%\renewcommand{\hat}{\protect\widehat}
%\renewcommand{\tilde}{\widetilde}
%\newcommand{\tri}{\triangle}
%\newcommand{\minipad}{\vspace{1ex}}
%\newcommand{\leftexp}[2]{{\vphantom{#2}}^{#1}{#2}}

%% More user defined commands
\renewcommand{\epsilon}{\varepsilon}
\renewcommand{\theta}{\vartheta} %% only for kmath
\renewcommand{\l}{\ell}
\renewcommand{\d}{\, d}
\newcommand{\ddx}{\frac{d}{dx}}
\newcommand{\dydx}{\frac{dy}{dx}}


\usepackage{bigstrut}


\title{Basic Constructions}
\author{Bart Snapp and Brad Findell}
\begin{document}
\begin{abstract}
Short-answer questions about basic constructions. 
\end{abstract}
\maketitle


\begin{warning}
The sketch below uses a ``custom toolbar'' that includes only the constructions that can be done directly with compass and straightedge.  
%If the toolbar doesn't appear, click 
%\link[here]{https://www.geogebra.org/m/ajnpdkmj} to open the sketch as a full page.
% After xake bake, the href for the previous link needs:    target="_blank"
\end{warning}
\begin{problem}
In the GeoGebra sketch below, carry out the constructions (exactly), and use the GeoGebra tool to measure the requested distances (to three decimal places).  
\begin{center}  
\geogebra[stb=true,sri=true]{ajnpdkmj}{800}{440}  
\end{center}
% After xake bake, the GeoGebra call needs to be edited as follows: 
%
%<p class="noindent"><iframe scrolling="no" src="https://tube.geogebra.org/material/iframe/id/ajnpdkmj/width/800/height/440/border/888888/rc/false/ai/false/sdz/false/smb/false/stb/true/stbh/false/ld/false/sri/true/at/auto" width="800px" height="440px" style="border:0px;" showToolBar="true" showResetIcon="true"> </iframe></p>
\begin{enumerate}
\item The length of the median to side $\overline{AB}$ 
is $\answer{7.37}$.  

\textbf{Note: It helps to reset the figure between parts.}  

\item The bisector of $\angle C$ intersects side $\overline{AB}$ at $H$.  
Then $BH =  \answer{1.931}$, and $CH = \answer{7.124}$.  

\item The altitude from vertex $A$ intersects side $\overline{BC}$ at $G$.
Then $AG = \answer{1.517}$, and 
$CG = \answer{5.884}$.  

\item Point $V$ is on the same side of $\overleftrightarrow{AC}$ as $B$ and is the vertex of an equilateral triangle with base $\overline{AC}$.  
$VB = \answer{6.265}$. 

\item The perpendicular bisector of $\overline{AB}$ intersects $\overline{CB}$ at $F$.  
Then $BF = \answer{1.845}$.
\end{enumerate}

\end{problem}


\end{document}
