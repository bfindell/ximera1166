% Section 1.1 Anatomy of Figures

\documentclass[nooutcomes]{ximera}
%\documentclass[space,handout,nooutcomes]{ximera}

% For preamble materials

\usepackage{pgf,tikz}
\usepackage{mathrsfs}
\usetikzlibrary{arrows}
\usepackage{framed}
\usepackage{amsmath}
\pgfplotsset{compat=1.17}

\def\fixnote#1{\begin{framed}{\textcolor{red}{Fix note: #1}}\end{framed}}  % Allows insertion of red notes about needed edits
%\def\fixnote#1{}

\def\detail#1{{\textcolor{blue}{Detail: #1}}}   

\pdfOnly{\renewenvironment{image}[1][]{\begin{center}}{\end{center}}}

\graphicspath{
  {./}
  {chapter1/}
  {chapter2/}
  {chapter4/}
  {proofs/}
  {graphics/}
  {../graphics/}
}

\newenvironment{sectionOutcomes}{}{}


%%% This set of code is all of our user defined commands
\newcommand{\bysame}{\mbox{\rule{3em}{.4pt}}\,}
\newcommand{\N}{\mathbb N}
\newcommand{\C}{\mathbb C}
\newcommand{\W}{\mathbb W}
\newcommand{\Z}{\mathbb Z}
\newcommand{\Q}{\mathbb Q}
\newcommand{\R}{\mathbb R}
\newcommand{\A}{\mathbb A}
\newcommand{\D}{\mathcal D}
\newcommand{\F}{\mathcal F}
\newcommand{\ph}{\varphi}
\newcommand{\ep}{\varepsilon}
\newcommand{\aph}{\alpha}
\newcommand{\QM}{\begin{center}{\huge\textbf{?}}\end{center}}

\renewcommand{\le}{\leqslant}
\renewcommand{\ge}{\geqslant}
\renewcommand{\a}{\wedge}
\renewcommand{\v}{\vee}
\renewcommand{\l}{\ell}
\newcommand{\mat}{\mathsf}
\renewcommand{\vec}{\mathbf}
\renewcommand{\subset}{\subseteq}
\renewcommand{\supset}{\supseteq}
%\renewcommand{\emptyset}{\varnothing}
%\newcommand{\xto}{\xrightarrow}
%\renewcommand{\qedsymbol}{$\blacksquare$}
%\newcommand{\bibname}{References and Further Reading}
%\renewcommand{\bar}{\protect\overline}
%\renewcommand{\hat}{\protect\widehat}
%\renewcommand{\tilde}{\widetilde}
%\newcommand{\tri}{\triangle}
%\newcommand{\minipad}{\vspace{1ex}}
%\newcommand{\leftexp}[2]{{\vphantom{#2}}^{#1}{#2}}

%% More user defined commands
\renewcommand{\epsilon}{\varepsilon}
\renewcommand{\theta}{\vartheta} %% only for kmath
\renewcommand{\l}{\ell}
\renewcommand{\d}{\, d}
\newcommand{\ddx}{\frac{d}{dx}}
\newcommand{\dydx}{\frac{dy}{dx}}


\usepackage{bigstrut}


\title{Lines in a Triangle}
\author{Brad Findell}
\begin{document}
\begin{abstract}
Short-answer questions about lines in a triangle.
\end{abstract}
\maketitle

Adjust the figures to fit the given conditions within \textbf{eyeball accuracy}.  Then enter the requested measurements.  


% Lines in a triangle
\begin{problem}
\begin{center}  
\geogebra{q32gyaud}{500}{390}  
\end{center}
In $\triangle ABC$ above, move point $D$ to make the following measurements.  \textbf{Enter -1 if it is not possible.}   
\begin{enumerate}
\item When $\overline{BD}$ is a median, $AD=\answer[tolerance=.08]{2.25}$.
\begin{hint}
A median is drawn from a vertex to the midpoint of the opposite side.
\end{hint}
\item When $\overline{BD}$ is a angle bisector, $AD=\answer[tolerance=.08]{2.78}$.
\begin{hint}
An angle bisector cuts an angle in half.  Focus near the vertex of the angle rather than near $D$.
\end{hint}
\item When $\overline{BD}$ is a perpendicular bisector, $AD=\answer{-1}$.
\begin{hint}
An perpendicular bisector cuts an segment in half and is perpendicular to it. \textbf{Enter -1 if it is not possible.} 
\end{hint}
\item When $\overline{BD}$ is an altitude, $AD=\answer[tolerance=.11]{6.46}$.
\begin{hint}
An altitude contains a vertex and is perpendicular to the line containing the opposite side. \textbf{Enter -1 if it is not possible.} 
\end{hint}
\end{enumerate}
\end{problem}



\end{document}

