\documentclass[nooutcomes]{ximera}
%\documentclass[space,handout,nooutcomes]{ximera}

% For preamble materials

\usepackage{pgf,tikz}
\usepackage{mathrsfs}
\usetikzlibrary{arrows}
\usepackage{framed}
\usepackage{amsmath}
\pgfplotsset{compat=1.17}

\def\fixnote#1{\begin{framed}{\textcolor{red}{Fix note: #1}}\end{framed}}  % Allows insertion of red notes about needed edits
%\def\fixnote#1{}

\def\detail#1{{\textcolor{blue}{Detail: #1}}}   

\pdfOnly{\renewenvironment{image}[1][]{\begin{center}}{\end{center}}}

\graphicspath{
  {./}
  {chapter1/}
  {chapter2/}
  {chapter4/}
  {proofs/}
  {graphics/}
  {../graphics/}
}

\newenvironment{sectionOutcomes}{}{}


%%% This set of code is all of our user defined commands
\newcommand{\bysame}{\mbox{\rule{3em}{.4pt}}\,}
\newcommand{\N}{\mathbb N}
\newcommand{\C}{\mathbb C}
\newcommand{\W}{\mathbb W}
\newcommand{\Z}{\mathbb Z}
\newcommand{\Q}{\mathbb Q}
\newcommand{\R}{\mathbb R}
\newcommand{\A}{\mathbb A}
\newcommand{\D}{\mathcal D}
\newcommand{\F}{\mathcal F}
\newcommand{\ph}{\varphi}
\newcommand{\ep}{\varepsilon}
\newcommand{\aph}{\alpha}
\newcommand{\QM}{\begin{center}{\huge\textbf{?}}\end{center}}

\renewcommand{\le}{\leqslant}
\renewcommand{\ge}{\geqslant}
\renewcommand{\a}{\wedge}
\renewcommand{\v}{\vee}
\renewcommand{\l}{\ell}
\newcommand{\mat}{\mathsf}
\renewcommand{\vec}{\mathbf}
\renewcommand{\subset}{\subseteq}
\renewcommand{\supset}{\supseteq}
%\renewcommand{\emptyset}{\varnothing}
%\newcommand{\xto}{\xrightarrow}
%\renewcommand{\qedsymbol}{$\blacksquare$}
%\newcommand{\bibname}{References and Further Reading}
%\renewcommand{\bar}{\protect\overline}
%\renewcommand{\hat}{\protect\widehat}
%\renewcommand{\tilde}{\widetilde}
%\newcommand{\tri}{\triangle}
%\newcommand{\minipad}{\vspace{1ex}}
%\newcommand{\leftexp}[2]{{\vphantom{#2}}^{#1}{#2}}

%% More user defined commands
\renewcommand{\epsilon}{\varepsilon}
\renewcommand{\theta}{\vartheta} %% only for kmath
\renewcommand{\l}{\ell}
\renewcommand{\d}{\, d}
\newcommand{\ddx}{\frac{d}{dx}}
\newcommand{\dydx}{\frac{dy}{dx}}


\usepackage{bigstrut}


\title{Quadrilateral Symmetry}
\author{Brad Findell}
\begin{document}
\begin{abstract}
Proof. 
\end{abstract}
\maketitle


\begin{problem}
Use symmetry to prove properties of parallelograms. 
\begin{image}
% Parallel markings have been commented out
%
\definecolor{uuuuuu}{rgb}{0.267,0.267,0.267}
\definecolor{qqqqff}{rgb}{0.,0.,1.}
\begin{tikzpicture}[line width=0.8pt,line cap=round,line join=round,>=triangle 45,x=1.0cm,y=1.0cm]
\clip(-0.4,-0.45) rectangle (5.4,2.45);
\draw (0.,0.)-- (1.,2.);
%\draw (0.5939,1.1878) -- (0.6677,1.0335);
%\draw (0.5939,1.1878) -- (0.4262,1.1542);
%\draw (0.5,1.) -- (0.5738,0.8457);
%\draw (0.5,1.) -- (0.3323,0.9665);
\draw (1.,2.)-- (5.,2.);
%\draw (3.105,2.) -- (3.,1.865);
%\draw (3.105,2.) -- (3.,2.135);
\draw (5.,2.)-- (4.,0.);
%\draw (4.4061,0.8122) -- (4.332,0.9665);
%\draw (4.4061,0.8122) -- (4.5738,0.8457);
%\draw (4.5,1.) -- (4.4262,1.1543);
%\draw (4.5,1.) -- (4.6677,1.0335);
\draw (4.,0.)-- (0.,0.);
%\draw (1.895,0.) -- (2.,0.135);
%\draw (1.895,0.) -- (2.,-0.135);
\draw (0.,0.)-- (5.,2.);
%\begin{scriptsize}
\draw [fill=qqqqff] (0.,0.) circle (1.2pt);
\draw[color=qqqqff] (-0.16,-0.2) node {$A$};
\draw [fill=qqqqff] (4.,0.) circle (1.2pt);
\draw[color=qqqqff] (4.1,-0.2) node {$B$};
\draw [fill=qqqqff] (1.,2.) circle (1.2pt);
\draw[color=qqqqff] (1.14,2.29) node {$D$};
\draw [fill=qqqqff] (5.,2.) circle (1.2pt);
\draw[color=qqqqff] (5.14,2.29) node {$C$};
\draw [fill=qqqqff] (2.5,1.) circle (1.2pt);
\draw[color=qqqqff] (2.6,1.3) node {$M$};
%\end{scriptsize}
\end{tikzpicture}
\end{image}

Consider a $180^\circ$ rotation about $M$, the midpoint of diagonal $\overline{AC}$.  Show that this rotation maps the parallelogram onto itself.  
\begin{warning}
The following proof is quite elegant, but some of the details are subtle.
\end{warning}
\begin{enumerate}
\item The rotation maps $A$ to $\answer{C}$ and $C$ to $\answer{A}$ because a $180^\circ$ rotation about a point on a line maps the line onto itself and preserves lengths.
%\item Now a $180^\circ$ rotation about $M$ takes lines not containing $M$ to parallel lines.  
\item The $180^\circ$ rotation maps $\overleftrightarrow{AB}$ to a parallel line through $\answer{C}$ (the image of $A$), which by the uniqueness of parallels must 
be the line $\answer{CD}$.  Similarly, the rotation maps 
\begin{itemize}
\item $\overleftrightarrow{CD}$ to the line $\answer{AB}$, 
\item $\overleftrightarrow{AD}$ to the line $\answer{CB}$, and
\item $\overleftrightarrow{CB}$ to the line $\answer{AD}$.
\end{itemize}

\item Furthermore, the intersection of $\overleftrightarrow{AB}$ and $\overleftrightarrow{CB}$, which is $\answer{B}$, must map to the intersection of their images, lines 
$\answer{CD}$ and $\answer{AD}$, and that intersection is $\answer{D}$.  Likewise, the rotation must map $D$ to $\answer{B}$.

\item We have shown that vertices are mapped to opposite vertices.  From this, we can conclude that sides are mapped to opposite $\answer[format=string]{sides}$, angles are mapped to 
opposite $\answer[format=string]{angles}$, and thus the parallelogram is mapped onto itself.  
\end{enumerate}

\begin{problem}
Now this symmetry proves the following properties---almost \emph{for free}!  
\begin{itemize}
\item opposite sides are $\answer[format=string]{congruent}$ (sides are mapped to opposite sides), 
\item opposite angles are $\answer[format=string]{congruent}$ (angles are mapped to opposite angles), and 
\item the diagonals $\answer[format=string]{bisect}$ each other.
\begin{quote}
Detail: The $180^\circ$ rotation about $M$ swaps $B$ and $\answer{D}$, so $\overrightarrow{MB}$ 
and ray $\answer{MD}$ must be opposite rays, and therefore $B$, $M$, and $\answer{D}$ are collinear. \\ Because the rotation preserves lengths, $MB=\answer{MD}$, so that $M$ is also the $\answer[format=string]{midpoint}$ of $\overline{BD}$, which means that the diagonals $\answer[format=string]{bisect}$ each other.
\end{quote}
\end{itemize}
\end{problem}
\end{problem}


\end{document}
