% Section 1.1 Anatomy of Figures

\documentclass[nooutcomes]{ximera}
%\documentclass[space,handout,nooutcomes]{ximera}

% For preamble materials

\usepackage{pgf,tikz}
\usepackage{mathrsfs}
\usetikzlibrary{arrows}
\usepackage{framed}
\usepackage{amsmath}
\pgfplotsset{compat=1.17}

\def\fixnote#1{\begin{framed}{\textcolor{red}{Fix note: #1}}\end{framed}}  % Allows insertion of red notes about needed edits
%\def\fixnote#1{}

\def\detail#1{{\textcolor{blue}{Detail: #1}}}   

\pdfOnly{\renewenvironment{image}[1][]{\begin{center}}{\end{center}}}

\graphicspath{
  {./}
  {chapter1/}
  {chapter2/}
  {chapter4/}
  {proofs/}
  {graphics/}
  {../graphics/}
}

\newenvironment{sectionOutcomes}{}{}


%%% This set of code is all of our user defined commands
\newcommand{\bysame}{\mbox{\rule{3em}{.4pt}}\,}
\newcommand{\N}{\mathbb N}
\newcommand{\C}{\mathbb C}
\newcommand{\W}{\mathbb W}
\newcommand{\Z}{\mathbb Z}
\newcommand{\Q}{\mathbb Q}
\newcommand{\R}{\mathbb R}
\newcommand{\A}{\mathbb A}
\newcommand{\D}{\mathcal D}
\newcommand{\F}{\mathcal F}
\newcommand{\ph}{\varphi}
\newcommand{\ep}{\varepsilon}
\newcommand{\aph}{\alpha}
\newcommand{\QM}{\begin{center}{\huge\textbf{?}}\end{center}}

\renewcommand{\le}{\leqslant}
\renewcommand{\ge}{\geqslant}
\renewcommand{\a}{\wedge}
\renewcommand{\v}{\vee}
\renewcommand{\l}{\ell}
\newcommand{\mat}{\mathsf}
\renewcommand{\vec}{\mathbf}
\renewcommand{\subset}{\subseteq}
\renewcommand{\supset}{\supseteq}
%\renewcommand{\emptyset}{\varnothing}
%\newcommand{\xto}{\xrightarrow}
%\renewcommand{\qedsymbol}{$\blacksquare$}
%\newcommand{\bibname}{References and Further Reading}
%\renewcommand{\bar}{\protect\overline}
%\renewcommand{\hat}{\protect\widehat}
%\renewcommand{\tilde}{\widetilde}
%\newcommand{\tri}{\triangle}
%\newcommand{\minipad}{\vspace{1ex}}
%\newcommand{\leftexp}[2]{{\vphantom{#2}}^{#1}{#2}}

%% More user defined commands
\renewcommand{\epsilon}{\varepsilon}
\renewcommand{\theta}{\vartheta} %% only for kmath
\renewcommand{\l}{\ell}
\renewcommand{\d}{\, d}
\newcommand{\ddx}{\frac{d}{dx}}
\newcommand{\dydx}{\frac{dy}{dx}}


\usepackage{bigstrut}


\title{Measuring Interior Angles}
\author{Brad Findell}
\begin{document}
\begin{abstract}
Short-answer questions involving angles in triangles. 
\end{abstract}
\maketitle

% Nonconvex Quadrilateral
\begin{center}  
\geogebra{zrapvzpz}{800}{460}  
\end{center}
\begin{problem}
Measure the interior angles of quadrilateral $ABCD$ above.  
\begin{enumerate}
\item $m\angle A = \answer[tolerance=1.3]{31}$ degrees.
\item $m\angle B = \answer[tolerance=1.3]{26.74}$ degrees.
\item $m\angle C = \answer[tolerance=1.3]{281}$ degrees.
\item $m\angle D = \answer[tolerance=1.3]{21.25}$ degrees.
\item $m\angle A + m\angle B + m\angle C + m\angle D = \answer[tolerance=2]{360}$ degrees.
\end{enumerate}
\begin{hint}
Be sure to measure interior angle as an amount of turning between the two sides of the angle.   
\end{hint}

\end{problem}

\begin{problem}
Use the measurements from the previous problem to answer the following questions: 
%\begin{image}
%\definecolor{qqwuqq}{rgb}{0.,0.3921,0.}
%\definecolor{uuuuuu}{rgb}{0.2667,0.2667,0.2667}
%\definecolor{qqqqff}{rgb}{0.,0.,1.}
%\begin{tikzpicture}[line width=0.8pt,line cap=round,line join=round,>=triangle 45,x=1.0cm,y=1.0cm]
%%\clip(0.245,6.95) rectangle (8.0,12.83);
%\draw [shift={(4.4,8.7)},color=qqwuqq,fill=qqwuqq,fill opacity=0.10] (0,0) -- (-21.25:0.399) arc (-21.25:57.75:0.3997) -- cycle;
%\draw (1.5,8.)-- (5.46,10.38);
%\draw (5.46,10.38)-- (4.4,8.7);
%\draw (4.4,8.7)-- (6.2,8.);
%\draw (6.2,8.)-- (1.5,8.);
%%\draw [dash pattern=on 2pt off 2pt] (4.4,8.7)-- (3.958,8.);
%%\draw [dash pattern=on 2pt off 2pt] (1.5,8.)-- (4.4,8.7);
%\draw [shift={(4.4,8.7)},color=qqwuqq] (-21.25:0.399) arc (-21.25:57.75:0.399);
%\draw [shift={(4.4,8.7)},color=qqwuqq] (-21.25:0.332) arc (-21.25:57.75:0.332);
%\draw [shift={(4.4,8.7)},color=qqwuqq] (-21.25:0.266) arc (-21.25:57.75:0.266);
%\begin{scriptsize}
%\draw [fill=qqqqff] (1.5,8.) circle (1.0pt);
%\draw[color=qqqqff] (1.40,8.22) node {$A$};
%\draw [fill=qqqqff] (5.46,10.38) circle (1.0pt);
%\draw[color=qqqqff] (5.55,10.54) node {$B$};
%\draw [fill=qqqqff] (4.4,8.7) circle (1.0pt);
%\draw[color=qqqqff] (4.3,8.9) node {$C$};
%\draw [fill=qqqqff] (6.2,8.) circle (1.0pt);
%\draw[color=qqqqff] (6.35,8.16) node {$D$};
%%\draw [fill=uuuuuu] (3.95,8.) circle (1.0pt);
%%\draw[color=uuuuuu] (3.90,7.8) node {$E$};
%\draw [color=gray] (-2.,9.) circle (0.2pt);
%\draw [color=gray] (10.,9.) circle (0.2pt);
%\end{scriptsize}
%\end{tikzpicture}
\end{image}

\begin{enumerate}
\item The marked angle should measure $\answer[tolerance=1.3]{79}$ degrees.  
\item $m\angle A + m\angle B + m\angle D = \answer[tolerance=1.3]{79}$ degrees.  
\item What do you notice?  
\begin{freeResponse}
\begin{hint}
They should be the same because, in both cases, adding the interior angle at $C$ should give $360^\circ$.  
\end{hint}
\end{freeResponse}
\end{enumerate}

\end{problem}

\begin{problem}
In order to reason about the sum of the interior angles, Bart and Brad each \emph{triangulated} the figure as shown below.  

%\begin{image}
%\definecolor{qqwuqq}{rgb}{0.,0.3921,0.}
%\definecolor{uuuuuu}{rgb}{0.2667,0.2667,0.2667}
%\definecolor{qqqqff}{rgb}{0.,0.,1.}
%\begin{tikzpicture}[line width=0.8pt,line cap=round,line join=round,>=triangle 45,x=1.0cm,y=1.0cm]
%%\clip(0.245,6.95) rectangle (8.0,12.83);
%%\draw [shift={(4.4,8.7)},color=qqwuqq,fill=qqwuqq,fill opacity=0.10] (0,0) -- (-21.25:0.399) arc (-21.25:57.75:0.3997) -- cycle;
%\draw (1.5,8.)-- (5.46,10.38);
%\draw (5.46,10.38)-- (4.4,8.7);
%\draw (4.4,8.7)-- (6.2,8.);
%\draw (6.2,8.)-- (1.5,8.);
%\draw [dash pattern=on 2pt off 2pt] (4.4,8.7)-- (3.958,8.);
%%\draw [dash pattern=on 2pt off 2pt] (1.5,8.)-- (4.4,8.7);
%%\draw [shift={(4.4,8.7)},color=qqwuqq] (-21.25:0.399) arc (-21.25:57.75:0.399);
%%\draw [shift={(4.4,8.7)},color=qqwuqq] (-21.25:0.332) arc (-21.25:57.75:0.332);
%%\draw [shift={(4.4,8.7)},color=qqwuqq] (-21.25:0.266) arc (-21.25:57.75:0.266);
%\begin{scriptsize}
%\draw [fill=qqqqff] (1.5,8.) circle (1.0pt);
%\draw[color=qqqqff] (1.40,8.22) node {$A$};
%\draw [fill=qqqqff] (5.46,10.38) circle (1.0pt);
%\draw[color=qqqqff] (5.55,10.54) node {$B$};
%\draw [fill=qqqqff] (4.4,8.7) circle (1.0pt);
%\draw[color=qqqqff] (4.3,8.9) node {$C$};
%\draw [fill=qqqqff] (6.2,8.) circle (1.0pt);
%\draw[color=qqqqff] (6.35,8.16) node {$D$};
%%\draw [fill=uuuuuu] (3.95,8.) circle (1.0pt);
%%\draw[color=uuuuuu] (3.90,7.8) node {$E$};
%\draw (3.9,7.5) node[align=center] {Brad's triangulation};
%\end{scriptsize}
%
%\begin{scope}[shift={(6,0)}]
%%\clip(0.245,6.95) rectangle (8.0,12.83);
%%\draw [shift={(4.4,8.7)},color=qqwuqq,fill=qqwuqq,fill opacity=0.10] (0,0) -- (-21.25:0.399) arc (-21.25:57.75:0.3997) -- cycle;
%\draw (1.5,8.)-- (5.46,10.38);
%\draw (5.46,10.38)-- (4.4,8.7);
%\draw (4.4,8.7)-- (6.2,8.);
%\draw (6.2,8.)-- (1.5,8.);
%%\draw [dash pattern=on 2pt off 2pt] (4.4,8.7)-- (3.958,8.);
%\draw [dash pattern=on 2pt off 2pt] (1.5,8.)-- (4.4,8.7);
%%\draw [shift={(4.4,8.7)},color=qqwuqq] (-21.25:0.399) arc (-21.25:57.75:0.399);
%%\draw [shift={(4.4,8.7)},color=qqwuqq] (-21.25:0.332) arc (-21.25:57.75:0.332);
%%\draw [shift={(4.4,8.7)},color=qqwuqq] (-21.25:0.266) arc (-21.25:57.75:0.266);
%\begin{scriptsize}
%\draw [fill=qqqqff] (1.5,8.) circle (1.0pt);
%\draw[color=qqqqff] (1.40,8.22) node {$A$};
%\draw [fill=qqqqff] (5.46,10.38) circle (1.0pt);
%\draw[color=qqqqff] (5.55,10.54) node {$B$};
%\draw [fill=qqqqff] (4.4,8.7) circle (1.0pt);
%\draw[color=qqqqff] (4.3,8.9) node {$C$};
%\draw [fill=qqqqff] (6.2,8.) circle (1.0pt);
%\draw[color=qqqqff] (6.35,8.16) node {$D$};
%%\draw [fill=uuuuuu] (3.95,8.) circle (1.0pt);
%%\draw[color=uuuuuu] (3.90,7.8) node {$E$};
%\draw (3.9,7.5) node[align=center] {Bart's triangulation};
%\end{scriptsize}
%\end{scope}
%\end{tikzpicture}
%\end{image}

Both Bart and Brad claim that because in a triangle the sum of the interior angles is $\answer{180}$ degrees, and this quadrilateral is cut into $\answer{2}$ triangles, the angle sum in this quadrilateral should be $\answer{360}$ degrees.  
What is your judgment about their reasoning?   
\begin{multipleChoice}
\choice{Both are reasoning correctly.}
\choice{Only Brad is reasoning correctly.}
\choice[correct]{Only Bart is reasoning correctly.}
\choice{Neither of them are reasoning correctly.}
\end{multipleChoice}

Explain your reasoning.
\begin{freeResponse}
\begin{hint}
In Bart's triangulation, the interior angles of the quadrilateral are composed only of interior angles of the triangles.  But in Brad's triangulation, a new angle has been created with a vertex between $A$ and $D$, and part of interior angle $C$ has been lost.  
\end{hint}
\end{freeResponse}
\end{problem}



\end{document}

