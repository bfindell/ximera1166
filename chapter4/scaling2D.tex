\documentclass[nooutcomes]{ximera}
%\documentclass[space,handout,nooutcomes]{ximera}

% For preamble materials

\usepackage{pgf,tikz}
\usepackage{mathrsfs}
\usetikzlibrary{arrows}
\usepackage{framed}
\usepackage{amsmath}
\pgfplotsset{compat=1.17}

\def\fixnote#1{\begin{framed}{\textcolor{red}{Fix note: #1}}\end{framed}}  % Allows insertion of red notes about needed edits
%\def\fixnote#1{}

\def\detail#1{{\textcolor{blue}{Detail: #1}}}   

\pdfOnly{\renewenvironment{image}[1][]{\begin{center}}{\end{center}}}

\graphicspath{
  {./}
  {chapter1/}
  {chapter2/}
  {chapter4/}
  {proofs/}
  {graphics/}
  {../graphics/}
}

\newenvironment{sectionOutcomes}{}{}


%%% This set of code is all of our user defined commands
\newcommand{\bysame}{\mbox{\rule{3em}{.4pt}}\,}
\newcommand{\N}{\mathbb N}
\newcommand{\C}{\mathbb C}
\newcommand{\W}{\mathbb W}
\newcommand{\Z}{\mathbb Z}
\newcommand{\Q}{\mathbb Q}
\newcommand{\R}{\mathbb R}
\newcommand{\A}{\mathbb A}
\newcommand{\D}{\mathcal D}
\newcommand{\F}{\mathcal F}
\newcommand{\ph}{\varphi}
\newcommand{\ep}{\varepsilon}
\newcommand{\aph}{\alpha}
\newcommand{\QM}{\begin{center}{\huge\textbf{?}}\end{center}}

\renewcommand{\le}{\leqslant}
\renewcommand{\ge}{\geqslant}
\renewcommand{\a}{\wedge}
\renewcommand{\v}{\vee}
\renewcommand{\l}{\ell}
\newcommand{\mat}{\mathsf}
\renewcommand{\vec}{\mathbf}
\renewcommand{\subset}{\subseteq}
\renewcommand{\supset}{\supseteq}
%\renewcommand{\emptyset}{\varnothing}
%\newcommand{\xto}{\xrightarrow}
%\renewcommand{\qedsymbol}{$\blacksquare$}
%\newcommand{\bibname}{References and Further Reading}
%\renewcommand{\bar}{\protect\overline}
%\renewcommand{\hat}{\protect\widehat}
%\renewcommand{\tilde}{\widetilde}
%\newcommand{\tri}{\triangle}
%\newcommand{\minipad}{\vspace{1ex}}
%\newcommand{\leftexp}[2]{{\vphantom{#2}}^{#1}{#2}}

%% More user defined commands
\renewcommand{\epsilon}{\varepsilon}
\renewcommand{\theta}{\vartheta} %% only for kmath
\renewcommand{\l}{\ell}
\renewcommand{\d}{\, d}
\newcommand{\ddx}{\frac{d}{dx}}
\newcommand{\dydx}{\frac{dy}{dx}}


\usepackage{bigstrut}


%\usepackage{tikz}


\title{Scaling}
\author{Brad Findell}
\begin{document}
\begin{abstract}
Short-answer problems about scaling. 
\end{abstract}
\maketitle



%\textbf{Definition.} Under a \textbf{dilation} about center $O$ and scale factor $r>0$, the image of $P$ is 
%a point $Q$ so that $Q$ lies on \wordChoice{\choice{segment}\choice[correct]{ray}\choice{line}} 
%$\answer[format=string]{OP}$ % $\overrightarrow{OP}$ 
%and $OQ=\answer[format=string]{rOP}$.  The image of $O$ is $\answer[format=string]{O}$. 

\section{Length and Area Under Scaling}
In this section, we explore what happens to length, area, and other measures under scaling.  
 
In a previous section, we defined similarity in terms of basic rigid motions and dilations, and we showed that this definition leads to well-known results about similarity, such as the AA criterion for triangle similarity and consistent ratios of lengths between and within similar figures. A key feature of this discussion was the notion of ``scale factor,'' which describes what happens to lengths under a dilation.  From the definition of a dilation, it is clear that segments on lines through the center of dilation scale by the scale factor.  We used the side-splitter theorems to show that other segments are scaled by the same scale factor.  

\begin{question}
Two students claims that a $3\times 5$ rectangle and a $4\times 6$ rectangle are similar.  

Fred says that that they are similar because the angles are the same.  How do you respond?  
\begin{freeResponse}
\begin{hint}
Angles are enough to determine similarity of triangles.  But similarity requires a consistent scale factor.  For these rectangles the height is scaled by $4/3$ whereas the base is scaled by $6/5$.  
\end{hint}
\end{freeResponse}

Ned says that they are similar because you can do the same thing (i.e., add 1) to ``both sides'' of the $3\times 5$ rectangle to get the $4\times 6$ rectangle.  How do you respond?  
\begin{freeResponse}
\begin{hint}
Similarity requires consistent scaling, which is a multiplicative (not additive) relationship.  
\end{hint}
\end{freeResponse}
\end{question}


\begin{question}
Complete the following sentences using words such as filling, falling, covering, wrapping, hiding, surrounding, or traveling:  

Area vs. perimeter can be thought of as $\answer[format=string]{covering}$ vs. $\answer[format=string]{surrounding}$, respectively. 

Volume vs. surface area can be thought of as $\answer[format=string]{filling}$ vs. $\answer[format=string]{wrapping}$, respectively. 
\end{question}


To explore how measures of figures change under scaling and non-scaling transformations, here are some useful strategies: 
\begin{itemize}
\item Cutting the figures and rearranging the pieces.  
\item Using ``rep-tiles.'' 
\item Using known formulas for perimeters, areas, volumes, or surface areas. 
\item Approximating with segments, squares, or cubes.  
\end{itemize}

% What about angles, perimeter, surface area, weight, and temperature
% Scaling versus stretching (other examples of non-scaling)
% Aspect ratio


\begin{question}
To estimate the length of a curve, imagine approximating it with many small segments.  Now apply a similarity transformation with a scale factor of $k$.  Each segment will scale by $\answer{k}$, so the length of the curve will be $\answer{k}$ times the original length.  
\end{question}

\begin{question}
To estimate the area of a non-polygonal region, imagine covering it approximately with a grid of squares of side length $s$.   Piecing together partial squares, suppose you count $n$ squares.  Your area estimate is then $\answer{ns^2}$.  

Now apply a similarity transformation of scale factor $k$ to both the region and the grid.  Each square in the scaled grid will have 
area $\answer{(sk)^2}$, and piecing together partial squares there will be $\answer{n}$ squares.  Thus, we estimate the area of the scaled region to be $\answer{n(sk)^2}$, which is $\answer{k^2}$ times the area of the original region.  
\end{question}

\begin{question}
When $n$ copies of a plane figure can form a figure similar to the original, the figure is called a rep-$n$-tile.  Explain briefly why any parallelogram is a rep-4-tile and also a rep-9-tile.  Generalize.  
\begin{freeResponse}
\begin{hint}
If a parallelogram is scaled by a factor of 2, then 4 original parallelograms can make the larger parallelogram.  If a parallelogram is scaled by a factor of 3, then 9 original parallelograms can make the larger parallelogram.  (Draw pictures.)
\end{hint}
\end{freeResponse}

In general, if a parallelogram is scaled by a factor of $k$, then $\answer{k^2}$ copies of the original parallelogram can make the scaled version.  
\end{question}

\begin{question}
Use formulas to determine what happens to the perimeter and area of a rectangle when it is scaled by $k$.  

Begin with a rectangle of base $b$ and height $h$.  After scaling the rectangle by $k$, the base will be $\answer{kb}$ and the height will be $\answer{kh}$.  

The original perimeter is $\answer{2b + 2h}$.  After scaling, the perimeter will be $\answer{2bk + 2hk}$, which is precisely $\answer{k}$ times the original perimeter.  

The original area is $\answer{bh}$.  After scaling, the area will be $\answer{(kb)(kh)}$, which is precisely $\answer{k^2}$ times the original area.  
\end{question}


\begin{question}
Use formulas to determine what happens to the circumference and area of a circle when it is scaled by $k$.  

Begin with a circle of radius $r$.  After scaling the circle by $k$, its radius will be $\answer{kr}$.  

The original circumference is $\answer{2\pi r}$.  After scaling, the circumference will be $\answer{2\pi kr}$, which is precisely $\answer{k}$ times the original circumference.  

The original area is $\answer{\pi r^2}$.  After scaling, the area will be $\answer{\pi (kr)^2}$, which is precisely $\answer{k^2}$ times the original area.  
\end{question}


\end{document}
