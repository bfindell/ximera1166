\documentclass[handout,nooutcomes]{ximera}
%\documentclass[space,handout,nooutcomes]{ximera}

% For preamble materials

\graphicspath{
  {./}
  {chapter1/}
  {chapter2/}
  {chapter4/}
  {math1/}
  {math2/}
}

\usepackage{pgf,tikz}
\usepackage{mathrsfs}
\usetikzlibrary{arrows}
\pgfplotsset{compat=1.16}


\newcommand{\N}{\mathbb N}
\newcommand{\W}{\mathbb W}
\newcommand{\C}{\mathbb C}
\newcommand{\Z}{\mathbb Z}
\newcommand{\Q}{\mathbb Q}
\newcommand{\R}{\mathbb R}




\title{Inscribed Angles}
\author{Brad Findell}
\begin{document}
\begin{abstract}
Inscribed angles exploration. 
\end{abstract}
\maketitle


\begin{definition}
In a circle, a \textbf{central angle} has the center of the circle as its vertex.  
An \textbf{inscribed angle} has a point on the circle as its vertex. 
\end{definition}

\begin{definition}
An \textbf{arc} of a circle is a portion of its circumference.  An arc has both a length and a measure.  An \textbf{arc length} is a distance.  An \textbf{arc measure} indicates an amount of turning (e.g., in degrees).  A \textbf{major arc} measures more than $180^\circ$; a \textbf{minor arc} measures less than $180^\circ$.    
\end{definition}

\begin{center}  
\geogebra{kcq9bpbd}{800}{460}  
\end{center}

\begin{problem}

\begin{enumerate}

\item The arc measure is \wordChoice{\choice[correct]{equal to}\choice{half}\choice{double}\choice{unrelated to}} the measure of the corresponding central angle. 
\item The measure of an inscribed angle is \wordChoice{\choice{equal to}\choice[correct]{half}\choice{double}\choice{unrelated to}} the measure of the corresponding central angle. 
\item The measure of an inscribed angle is \wordChoice{\choice{equal to}\choice[correct]{half}\choice{double}\choice{unrelated to}} the measure of the corresponding arc. 
\item Keeping points $A$ and $C$ fixed, when point $B$ moves, $m\angle ABC$ \wordChoice{\choice{increases}\choice[correct]{stays the same}\choice{decreases}\choice{varies widely}}, as long as $A$, $B$ and $C$ remain in clockwise order on the circle.  

\end{enumerate}
\end{problem}


% Maybe add some questions about visually estimating angle measures or arc measures.  


\end{document}
