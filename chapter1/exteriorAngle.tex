% Section 1.1 Anatomy of Figures

\documentclass[nooutcomes]{ximera}
%\documentclass[space,handout,nooutcomes]{ximera}

% For preamble materials

\graphicspath{
  {./}
  {chapter1/}
  {chapter2/}
  {chapter4/}
  {math1/}
  {math2/}
}

\usepackage{pgf,tikz}
\usepackage{mathrsfs}
\usetikzlibrary{arrows}
\pgfplotsset{compat=1.16}


\newcommand{\N}{\mathbb N}
\newcommand{\W}{\mathbb W}
\newcommand{\C}{\mathbb C}
\newcommand{\Z}{\mathbb Z}
\newcommand{\Q}{\mathbb Q}
\newcommand{\R}{\mathbb R}




\title{Exterior Angle}
\author{Brad Findell}
\begin{document}
\begin{abstract}
Short-answer questions involving exterior angles in triangles. 
\end{abstract}
\maketitle


In the sketch below, explore the relationship between an exterior angle and its remote interior angles. 
% Exterior Angle
\begin{center}  
\geogebra{npuywgsd}{680}{420}  
\end{center}

\begin{problem}
In the above sketch, the exterior angle is 
\wordChoice{\choice{$\angle CBA$}\choice[correct]{$\angle DBC$}\choice{$\angle A$}\choice{$\angle C$}}, and its (corresponding) remote interior angles are \wordChoice{\choice{$\angle CBA$}\choice{$\angle DBC$}\choice[correct]{$\angle A$}\choice{$\angle C$}} and \wordChoice{\choice{$\angle CBA$}\choice{$\angle DBC$}\choice{$\angle A$}\choice[correct]{$\angle C$}}. 

\begin{feedback}[correct]
\textbf{Correct!} We say ``corresponding'' because a different exterior angle would correspond to different remote interior angles.
\end{feedback}
\begin{problem}
We call $\angle A$ and $\angle C$ ``remote'' interior angles corresponding to $\angle DBC$ because, compared to the other interior angle, $\angle CBA$, they are 
\wordChoice{\choice{next to}\choice[correct]{further away from}\choice{above}\choice{better than}} $\angle DBC$.
\end{problem}
\end{problem}

\begin{problem}
We observe in the sketch that the measure of the exterior angle is equal to the 
\wordChoice{\choice{product}\choice[correct]{sum}\choice{difference}} of the measures of its corresponding remote interior angles.  

This make sense because 
\[
m\angle DBC + m\angle CBA = \answer{180},
\]
as a linear pair, and also 
\[
m\angle A + m\angle \answer{C} + m\angle CBA = 180, 
\]
as the sum of the interior angles of $\triangle \answer{ABC}$.  

By comparing the two equations (or with a little algebra), it follows that 
\[
m\angle DBC = m\angle \answer{A} + m\angle C.
\]
\end{problem}


\end{document}

