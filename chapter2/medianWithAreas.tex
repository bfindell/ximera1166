\documentclass[handout,nooutcomes]{ximera}
%\documentclass[space,handout,nooutcomes]{ximera}

% For preamble materials

\graphicspath{
  {./}
  {chapter1/}
  {chapter2/}
  {chapter4/}
  {math1/}
  {math2/}
}

\usepackage{pgf,tikz}
\usepackage{mathrsfs}
\usetikzlibrary{arrows}
\pgfplotsset{compat=1.16}


\newcommand{\N}{\mathbb N}
\newcommand{\W}{\mathbb W}
\newcommand{\C}{\mathbb C}
\newcommand{\Z}{\mathbb Z}
\newcommand{\Q}{\mathbb Q}
\newcommand{\R}{\mathbb R}




\title{Median with Areas}
\author{Brad Findell}
\begin{document}
\begin{abstract}
Inscribed angles exploration. 
\end{abstract}
\maketitle




\begin{problem}
Explore the follow GeoGebra sketch.  
\begin{center}  
\geogebra{usprmzfe}{800}{460}  
\end{center}
In a triangle, a median cuts the triangle into two triangles whose areas are \wordChoice{\choice{different}\choice{variable}\choice[correct]{equal}\choice{unrelated}}.  

This makes sense because the two triangles have the same $\answer[format=string]{base}$, which is 
$\answer[format=string]{half}$ the base of the original triangle.  Furthermore, the two triangles have the same $\answer[format=string]{height}$ as the original triangle.  By the triangle area formula, ``$\frac{1}{2} \textrm{base} \times \textrm{height}$,'' the areas must then be \wordChoice{\choice{different}\choice{variable}\choice[correct]{equal}\choice{random}}.

\end{problem}


\end{document}
