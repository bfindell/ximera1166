%\documentclass[handout]{ximera}
\documentclass[nooutcomes,noauthor]{ximera}

% For preamble materials

\usepackage{pgf,tikz}
\usepackage{mathrsfs}
\usetikzlibrary{arrows}
\usepackage{framed}
\usepackage{amsmath}
\pgfplotsset{compat=1.17}

\def\fixnote#1{\begin{framed}{\textcolor{red}{Fix note: #1}}\end{framed}}  % Allows insertion of red notes about needed edits
%\def\fixnote#1{}

\def\detail#1{{\textcolor{blue}{Detail: #1}}}   

\pdfOnly{\renewenvironment{image}[1][]{\begin{center}}{\end{center}}}

\graphicspath{
  {./}
  {chapter1/}
  {chapter2/}
  {chapter4/}
  {proofs/}
  {graphics/}
  {../graphics/}
}

\newenvironment{sectionOutcomes}{}{}


%%% This set of code is all of our user defined commands
\newcommand{\bysame}{\mbox{\rule{3em}{.4pt}}\,}
\newcommand{\N}{\mathbb N}
\newcommand{\C}{\mathbb C}
\newcommand{\W}{\mathbb W}
\newcommand{\Z}{\mathbb Z}
\newcommand{\Q}{\mathbb Q}
\newcommand{\R}{\mathbb R}
\newcommand{\A}{\mathbb A}
\newcommand{\D}{\mathcal D}
\newcommand{\F}{\mathcal F}
\newcommand{\ph}{\varphi}
\newcommand{\ep}{\varepsilon}
\newcommand{\aph}{\alpha}
\newcommand{\QM}{\begin{center}{\huge\textbf{?}}\end{center}}

\renewcommand{\le}{\leqslant}
\renewcommand{\ge}{\geqslant}
\renewcommand{\a}{\wedge}
\renewcommand{\v}{\vee}
\renewcommand{\l}{\ell}
\newcommand{\mat}{\mathsf}
\renewcommand{\vec}{\mathbf}
\renewcommand{\subset}{\subseteq}
\renewcommand{\supset}{\supseteq}
%\renewcommand{\emptyset}{\varnothing}
%\newcommand{\xto}{\xrightarrow}
%\renewcommand{\qedsymbol}{$\blacksquare$}
%\newcommand{\bibname}{References and Further Reading}
%\renewcommand{\bar}{\protect\overline}
%\renewcommand{\hat}{\protect\widehat}
%\renewcommand{\tilde}{\widetilde}
%\newcommand{\tri}{\triangle}
%\newcommand{\minipad}{\vspace{1ex}}
%\newcommand{\leftexp}[2]{{\vphantom{#2}}^{#1}{#2}}

%% More user defined commands
\renewcommand{\epsilon}{\varepsilon}
\renewcommand{\theta}{\vartheta} %% only for kmath
\renewcommand{\l}{\ell}
\renewcommand{\d}{\, d}
\newcommand{\ddx}{\frac{d}{dx}}
\newcommand{\dydx}{\frac{dy}{dx}}


\usepackage{bigstrut}


\title{Walking and Turning}
\author{Bart Snapp and Brad Findell}

\outcome{Learning outcome goes here.}

\begin{document}
\begin{abstract}
We reason about interior angles of a triangle.
\end{abstract}
\maketitle


\begin{problem}
In this problem, we aim to prove the following: 

\begin{theorem}
The \wordChoice{\choice{amount}\choice[correct]{sum}\choice{difference}\choice{product}} of the measures of the 
\wordChoice{\choice{adjacent}\choice{alternate}\choice{exterior}\choice[correct]{interior}\choice{vertical}} 
angles of a triangle is $\answer{180}$ degrees.
\end{theorem}

\begin{problem}
Consider $\triangle ABC$ shown below.  Let $a$, $b$, and $c$, respectively, be the measures of the interior angles of the triangle.  

\begin{center}  
\geogebra{gfx5sugj}{800}{660}  
\end{center}

%\begin{image}
%\definecolor{uuuuuu}{rgb}{0.26666666666666666,0.26666666666666666,0.26666666666666666}
%\definecolor{ududff}{rgb}{0.30196078431372547,0.30196078431372547,1}
%\begin{tikzpicture}[line cap=round,line join=round,>=triangle 45,x=1cm,y=1cm]
%\draw [line width=1.2pt] (0,4)-- (7,0);
%\draw [line width=1.2pt] (6,3)-- (0,4);
%\draw [line width=1.2pt] (6,3)-- (7,0);
%\begin{scriptsize}
%\draw [-latex,line width=1pt] (6,4) arc (0:270:4mm);
%\draw [fill=ududff] (6,4) circle (1pt);
%\draw [color=ududff] (6,3.8) node {Turning};
%\draw [fill=uuuuuu] (6.5,1.5) circle (1.5pt);
%\draw[color=ududff] (6.25,1.5) node {$M$};
%\draw [-latex,line width=1pt] (6.8,1.6) -- (6.58,2.2);
%\draw[color=ududff] (7.3,1.8) node {Walking};
%\draw[color=ududff] (0.6,3.7) node {$b$};
%\draw[color=ududff] (6.7,0.4) node {$c$};
%\draw[color=ududff] (5.8,2.8) node {$a$};
%\end{scriptsize}
%\end{tikzpicture}
%\end{image}
%\vfill


Beginning at point labeled Start, on a side of the triangle, imagine walking around the triangle starting in the direction indicated by the arrow.  At each vertex, turn counterclockwise (viewed from above).  Your journey ends when you return to the starting point.  
%\footnote{You will find it helpful to actually walk around a triangle outlined on the floor with masking tape, for example.}

Before reasoning about the general case, make some measurements on this specific triangle.  
\begin{enumerate}
\item Use the angle sticks to measure each of the interior angles: $a=\answer[tolerance=0.6]{20.2}$, $b=\answer[tolerance=0.6]{41.7}$, and $c=\answer[tolerance=0.6]{118.1}$.
\begin{hint}
At the vertices of the triangle, there are several possible angles to measure.  Be sure you measure the one you intend. 
\end{hint}
\item Extend the sides of the triangle.  At each vertex measure the angle through which the walker turns at that vertex:  
\begin{itemize}
\item At $A$, turn $\answer[tolerance=0.6]{180-20.2}$ degrees;  
\item at $B$, turn $\answer[tolerance=0.6]{180-41.7}$ degrees; and 
\item at $C$, turn $\answer[tolerance=0.6]{180-118.1}$ degrees.
\end{itemize}
% Add a question about exterior angles and how they relate to interior angles.  
\item According to your measurements, how much does the walker turn during the whole journey?  $\answer[tolerance=1.8]{360}$ degrees.
\item Theoretically, exactly how much should the walker turn during the whole journey?  $\answer{360}$ degrees.
\end{enumerate}

\begin{problem}
At each vertex, you found that the walker turns through a(n) 
\wordChoice{\choice{adjacent}\choice{alternate}\choice[correct]{exterior}\choice{interior}\choice{vertical}} angle, which is 
\wordChoice{\choice{complementary}\choice[correct]{supplementary}\choice{opposite}\choice{vertical}} to the interior angle at that vertex.  

To reason about the general case, think of the angle measures $a$, $b$, and $c$ as variables.  

% We want to prove (in general) that $\answer{a+b+c}=\answer{180}$ degrees. 

Now make use of your previous observations and reasoning:  
\[
\left(\text{Turning at A}\right) + \left(\text{Turning at B}\right) + \left(\text{Turning at C}\right) = \left(\text{Total amount of turning}\right).
\]
Algebraically, we can build this equation with expressions in $a$, $b$, and $c$ (in the same order as above, for clarity):  
\[
\left(\answer{180-a}\right) + \left(\answer{180-b}\right) + \left(\answer{180-c}\right)  = \answer{360} \text{ degrees.}
\]
After some algebra, we get the desired result about the sum of the interior angles:  
\[
\answer{a+b+c}=\answer{180} \text{ degrees}. 
\]

\end{problem}
\end{problem}
\end{problem}

%\newpage
%
%\begin{problem}
%Now repeat the previous problem, this time turning clockwise at each vertex and walking backwards along a side, as needed.  Again, prove what you can about $a$, $b$, and $c$.  
%\begin{enumerate}
%\item What do you want to prove about $a$, $b$, and $c$?  
%\item Extend the sides of the triangle.  At each vertex mark the angle through which the walker turns at that vertex.  
%\item How much does the walker turn during the whole journey?  
%\item Based upon your ``walking and turning'' journey, write a proof of your claim from part (a).  
%\begin{image}
%\definecolor{uuuuuu}{rgb}{0.26666666666666666,0.26666666666666666,0.26666666666666666}
%\definecolor{ududff}{rgb}{0.30196078431372547,0.30196078431372547,1}
%\begin{tikzpicture}[line cap=round,line join=round,>=triangle 45,x=1cm,y=1cm]
%\draw [line width=1.2pt] (0,4)-- (7,0);
%\draw [line width=1.2pt] (6,3)-- (0,4);
%\draw [line width=1.2pt] (6,3)-- (7,0);
%\begin{scriptsize}
%\draw [-latex,line width=1pt] (6,4) arc (180:-90:4mm);
%\draw [fill=ududff] (6,4) circle (1pt);
%\draw [color=ududff] (6,3.8) node {Turning};
%\draw [fill=uuuuuu] (6.5,1.5) circle (1.5pt);
%\draw[color=ududff] (6.25,1.5) node {$M$};
%\draw [-latex,line width=1pt] (6.8,1.6) -- (6.58,2.2);
%\draw[color=ududff] (7.3,1.8) node {Walking};
%\draw[color=ududff] (0.6,3.7) node {$b$};
%\draw[color=ududff] (6.7,0.4) node {$c$};
%\draw[color=ududff] (5.8,2.8) node {$a$};
%\end{scriptsize}
%\end{tikzpicture}
%\end{image}
%\vfill
%\end{enumerate}
%\end{problem}

\end{document}