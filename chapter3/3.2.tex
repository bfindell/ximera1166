\documentclass[nooutcomes]{ximera}
%\documentclass[space,handout,nooutcomes]{ximera}

% For preamble materials

\usepackage{pgf,tikz}
\usepackage{mathrsfs}
\usetikzlibrary{arrows}
\usepackage{framed}
\usepackage{amsmath}
\pgfplotsset{compat=1.17}

\def\fixnote#1{\begin{framed}{\textcolor{red}{Fix note: #1}}\end{framed}}  % Allows insertion of red notes about needed edits
%\def\fixnote#1{}

\def\detail#1{{\textcolor{blue}{Detail: #1}}}   

\pdfOnly{\renewenvironment{image}[1][]{\begin{center}}{\end{center}}}

\graphicspath{
  {./}
  {chapter1/}
  {chapter2/}
  {chapter4/}
  {proofs/}
  {graphics/}
  {../graphics/}
}

\newenvironment{sectionOutcomes}{}{}


%%% This set of code is all of our user defined commands
\newcommand{\bysame}{\mbox{\rule{3em}{.4pt}}\,}
\newcommand{\N}{\mathbb N}
\newcommand{\C}{\mathbb C}
\newcommand{\W}{\mathbb W}
\newcommand{\Z}{\mathbb Z}
\newcommand{\Q}{\mathbb Q}
\newcommand{\R}{\mathbb R}
\newcommand{\A}{\mathbb A}
\newcommand{\D}{\mathcal D}
\newcommand{\F}{\mathcal F}
\newcommand{\ph}{\varphi}
\newcommand{\ep}{\varepsilon}
\newcommand{\aph}{\alpha}
\newcommand{\QM}{\begin{center}{\huge\textbf{?}}\end{center}}

\renewcommand{\le}{\leqslant}
\renewcommand{\ge}{\geqslant}
\renewcommand{\a}{\wedge}
\renewcommand{\v}{\vee}
\renewcommand{\l}{\ell}
\newcommand{\mat}{\mathsf}
\renewcommand{\vec}{\mathbf}
\renewcommand{\subset}{\subseteq}
\renewcommand{\supset}{\supseteq}
%\renewcommand{\emptyset}{\varnothing}
%\newcommand{\xto}{\xrightarrow}
%\renewcommand{\qedsymbol}{$\blacksquare$}
%\newcommand{\bibname}{References and Further Reading}
%\renewcommand{\bar}{\protect\overline}
%\renewcommand{\hat}{\protect\widehat}
%\renewcommand{\tilde}{\widetilde}
%\newcommand{\tri}{\triangle}
%\newcommand{\minipad}{\vspace{1ex}}
%\newcommand{\leftexp}[2]{{\vphantom{#2}}^{#1}{#2}}

%% More user defined commands
\renewcommand{\epsilon}{\varepsilon}
\renewcommand{\theta}{\vartheta} %% only for kmath
\renewcommand{\l}{\ell}
\renewcommand{\d}{\, d}
\newcommand{\ddx}{\frac{d}{dx}}
\newcommand{\dydx}{\frac{dy}{dx}}


\usepackage{bigstrut}


\title{Proof by Picture}
\author{Bart Snapp and Brad Findell}
\begin{document}
\begin{abstract}
Short-answer proofs by pictures. 
\end{abstract}
\maketitle


%\begin{problem}
%In regards to folding and tracing constructions, what is a \textit{circle}?
%  Compare and contrast this to a naive notion of a circle.
%\end{problem}

\begin{problem}
Explain how a perpendicular bisector is different from an
  altitude. Use folding and tracing to illustrate the difference.
\end{problem}

\begin{problem}
Explain how a median different from an angle bisector.  Use
  folding and tracing to illustrate the difference.
\end{problem}

\begin{problem}
Given a triangle, use folding and tracing to construct the
  circumcenter. Explain the steps in your
  construction.\index{circumcenter}
\end{problem}

\begin{problem}
Given a triangle, use folding and tracing to construct the
  orthocenter. Explain the steps in your
  construction.\index{orthocenter}
\end{problem}

\begin{problem}
Given a triangle, use folding and tracing to construct the incenter. Explain
  the steps in your construction.\index{incenter}
\end{problem}

\begin{problem}
Given a triangle, use folding and tracing to construct the centroid. Explain
  the steps in your construction.\index{centroid}
\end{problem}

%\begin{problem}
%Could the circumcenter be outside the triangle? If so explain
%  how and use folding and tracing to give an example. If not, explain why not
%  using folding and tracing to illustrate your ideas.
%\end{problem}
%
%\begin{problem}
%Could the orthocenter be outside the triangle? If so explain how and
%  use folding and tracing to give an example. If not, explain why not using
%  folding and tracing to illustrate your ideas.
%\end{problem}
%
%\begin{problem}
%Could the incenter be outside the triangle? If so explain how
%  and use folding and tracing to give an example. If not, explain why not using
%  folding and tracing to illustrate your ideas.
%\end{problem}
%
%\begin{problem}
%Could the centroid be outside the triangle? If so explain how
%  and use folding and tracing to give an example. If not, explain why not using
%  folding and tracing to illustrate your ideas.
%\end{problem}
%
%\begin{problem}
%Where is the circumcenter of a right triangle? Explain your
%  reasoning and illustrate your ideas with folding and tracing.
%\end{problem}
%
%\begin{problem}
%Where is the orthocenter of a right triangle?  Explain your
%  reasoning and illustrate your ideas with folding and tracing.
%\end{problem}
%
%\begin{problem}
%The following picture shows a triangle that has been folded
%  along the dotted lines:
%\[
%\includegraphics{../graphics/origamiPBPTri.pdf}
%\]
%Explain how the picture ``proves'' the following statements:
%\begin{enumerate}
%\item The interior angles of a triangle sum to $180^\circ$. 
%\item The area of a triangle is given by $bh/2$. 
%\end{enumerate}
%\end{problem}

%\begin{problem}
%Use folding and tracing to construct a triangle given the length of one
%  side, the length of the the median to that side, and the length of
%  the altitude of the opposite angle. Explain the steps in your
%  construction.
%\end{problem}
%
%\begin{problem}
%Use folding and tracing to construct a triangle given one angle, the length
%  of an adjacent side and the altitude to that side. Explain the steps
%  in your construction.
%\end{problem}
%
%\begin{problem}
%Use folding and tracing to construct a triangle given one angle and the
%  altitudes to the other two angles. Explain the steps in your
%  construction.
%\end{problem}
%
%\begin{problem}
%Use folding and tracing to construct a triangle given two sides and the
%  altitude to the third side. Explain the steps in your construction.
%\end{problem}

\end{document}
