\documentclass[nooutcomes]{ximera}
%\documentclass[space,handout,nooutcomes]{ximera}

% For preamble materials

\graphicspath{
  {./}
  {chapter1/}
  {chapter2/}
  {chapter4/}
  {math1/}
  {math2/}
}

\usepackage{pgf,tikz}
\usepackage{mathrsfs}
\usetikzlibrary{arrows}
\pgfplotsset{compat=1.16}


\newcommand{\N}{\mathbb N}
\newcommand{\W}{\mathbb W}
\newcommand{\C}{\mathbb C}
\newcommand{\Z}{\mathbb Z}
\newcommand{\Q}{\mathbb Q}
\newcommand{\R}{\mathbb R}




%\usepackage{tikz}


\title{Transformations}
\author{Bart Snapp and Brad Findell}
\begin{document}
\begin{abstract}
Short-answer problems about the definitions of basic rigid motions.
\end{abstract}
\maketitle

To define a translation, reflection, or rotation, we specify the image of a generic point $P$ under the transformation.  

For background, first complete the activities at \link[Euclid: The Game]{http://euclid.findell.org/Level29}.

\begin{question}
Under a \textbf{translation} by the distance and direction from $A$ to $B$, the image of $P$ is a point $Q$ so that $\overline{PQ} \parallel \overline{AB}$, $PQ=\answer{AB}$, and $\overrightarrow{PQ}$ is in the same direction as $\overrightarrow{AB}$.  
\end{question}

\begin{question}
Under a \textbf{reflection} about a line $\ell$, if $P$ is on $\ell$, then $P$ is mapped to itself.  If $P$ is not on $\ell$, then the image of $P$ is a point $Q$ so that $\ell$ is the 
$\answer[format=string]{perpendicular bisector}$ of segment $\answer{PQ}$.  
\end{question}

\begin{question}
Under a \textbf{rotation} about center $O$ counterclockwise by an angle $\theta$, the image of $P$ is 
a point $Q$ so that $OQ = \answer{OP}$ and $m\angle POQ = \theta$, measured counterclockwise 
from $\overrightarrow{OP}$ to $\overrightarrow{OQ}$, assuming $\theta>0$.  If $\theta<0$, rotate $\answer[format=string]{clockwise}$ by $|\theta|$.  

Special cases:  If $P = O$ then $Q=O=P$.  And if $\theta=0$, do nothing. 
\end{question}

%\begin{question}
%Under a \textbf{dilation} about center $O$ and scale factor $r>0$, the image of $P$ is 
%a point $Q$ so that $Q$ lies on $\overrightarrow{OP}$ and $OQ = \answer{rOP}$.  The image of $O$ is $\answer{O}$. 
%\end{question}



\end{document}
