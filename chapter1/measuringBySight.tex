% Section 1.1 Anatomy of Figures

\documentclass[nooutcomes]{ximera}
%\documentclass[space,handout,nooutcomes]{ximera}

% For preamble materials

\graphicspath{
  {./}
  {chapter1/}
  {chapter2/}
  {chapter4/}
  {math1/}
  {math2/}
}

\usepackage{pgf,tikz}
\usepackage{mathrsfs}
\usetikzlibrary{arrows}
\pgfplotsset{compat=1.16}


\newcommand{\N}{\mathbb N}
\newcommand{\W}{\mathbb W}
\newcommand{\C}{\mathbb C}
\newcommand{\Z}{\mathbb Z}
\newcommand{\Q}{\mathbb Q}
\newcommand{\R}{\mathbb R}




\title{Measuring by Sight}
\author{Brad Findell}
\begin{document}
\begin{abstract}
Short-answer questions involving measuring. 
\end{abstract}
\maketitle

\section{Careful Measurement by Sight}
Adjust the figures to fit the given conditions within \textbf{eyeball accuracy}.  Enter the requested measurements.  

% Perpendicular
\begin{problem}
\begin{center}  
\geogebra{gjf28er6}{450}{250}  
\end{center}
In figure above, when point $C$ is adjusted so that $\overline{BC}$ is perpendicular to $\overline{AC}$, $AC=\answer[tolerance=.07]{2.09}$.
\begin{hint}
When two lines are \emph{perpendicular}, they cross to create four congruent angles. 
\end{hint}
\begin{hint}
Use the corner of a piece of paper.
\end{hint}
\end{problem}

% Triangle Height, v2
\begin{problem}
\begin{center}  
\geogebra{a888zyw2}{600}{220}  
\end{center}
In $\triangle ABC$ above, the height to base $\overline{AC}$ is $\answer[tolerance=.003]{3.585}$.
\begin{hint}
You may move point $D$.  A height is the length of an altitude, which must be perpendicular to the line containing the chosen base.  
\end{hint}
\end{problem}

% Triangle Height, v1
\begin{problem}
\begin{center}  
\geogebra{kta9hbuf}{500}{300}  
\end{center}
In $\triangle ABC$ above, the height to base $\overline{AC}$ is $\answer[tolerance=.003]{3.511}$.
\begin{hint}
You may move point $D$.  A height is the length of an altitude, which must be perpendicular to the line containing the chosen base.  
\end{hint}
\end{problem}





\end{document}

