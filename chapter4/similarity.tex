\documentclass[nooutcomes]{ximera}
%\documentclass[space,handout,nooutcomes]{ximera}

% For preamble materials

\usepackage{pgf,tikz}
\usepackage{mathrsfs}
\usetikzlibrary{arrows}
\usepackage{framed}
\usepackage{amsmath}
\pgfplotsset{compat=1.17}

\def\fixnote#1{\begin{framed}{\textcolor{red}{Fix note: #1}}\end{framed}}  % Allows insertion of red notes about needed edits
%\def\fixnote#1{}

\def\detail#1{{\textcolor{blue}{Detail: #1}}}   

\pdfOnly{\renewenvironment{image}[1][]{\begin{center}}{\end{center}}}

\graphicspath{
  {./}
  {chapter1/}
  {chapter2/}
  {chapter4/}
  {proofs/}
  {graphics/}
  {../graphics/}
}

\newenvironment{sectionOutcomes}{}{}


%%% This set of code is all of our user defined commands
\newcommand{\bysame}{\mbox{\rule{3em}{.4pt}}\,}
\newcommand{\N}{\mathbb N}
\newcommand{\C}{\mathbb C}
\newcommand{\W}{\mathbb W}
\newcommand{\Z}{\mathbb Z}
\newcommand{\Q}{\mathbb Q}
\newcommand{\R}{\mathbb R}
\newcommand{\A}{\mathbb A}
\newcommand{\D}{\mathcal D}
\newcommand{\F}{\mathcal F}
\newcommand{\ph}{\varphi}
\newcommand{\ep}{\varepsilon}
\newcommand{\aph}{\alpha}
\newcommand{\QM}{\begin{center}{\huge\textbf{?}}\end{center}}

\renewcommand{\le}{\leqslant}
\renewcommand{\ge}{\geqslant}
\renewcommand{\a}{\wedge}
\renewcommand{\v}{\vee}
\renewcommand{\l}{\ell}
\newcommand{\mat}{\mathsf}
\renewcommand{\vec}{\mathbf}
\renewcommand{\subset}{\subseteq}
\renewcommand{\supset}{\supseteq}
%\renewcommand{\emptyset}{\varnothing}
%\newcommand{\xto}{\xrightarrow}
%\renewcommand{\qedsymbol}{$\blacksquare$}
%\newcommand{\bibname}{References and Further Reading}
%\renewcommand{\bar}{\protect\overline}
%\renewcommand{\hat}{\protect\widehat}
%\renewcommand{\tilde}{\widetilde}
%\newcommand{\tri}{\triangle}
%\newcommand{\minipad}{\vspace{1ex}}
%\newcommand{\leftexp}[2]{{\vphantom{#2}}^{#1}{#2}}

%% More user defined commands
\renewcommand{\epsilon}{\varepsilon}
\renewcommand{\theta}{\vartheta} %% only for kmath
\renewcommand{\l}{\ell}
\renewcommand{\d}{\, d}
\newcommand{\ddx}{\frac{d}{dx}}
\newcommand{\dydx}{\frac{dy}{dx}}


\usepackage{bigstrut}


%\usepackage{tikz}


\title{Similarity}
\author{Bart Snapp and Brad Findell}
\begin{document}
\begin{abstract}
Short-answer problems about similarity. 
\end{abstract}
\maketitle




%To define a translation, rotation, or reflection, we indicate the image of a generic point $P$ under the transformation as specified.  
%
%Under a rotation about center $O$ counterclockwise by an angle $\theta$, the image of $P$ is 
%a point $Q$ so that $OQ = OP$ and $m\angle POQ = \theta$, measured counterclockwise 
%from $\overrightarrow{OP}$ to $\overrightarrow{OQ}$.  If $P = O$ then $Q=O=P$.  % if $\theta<0$ 
%
%Under a translation by the distance and direction from $A$ to $B$, the image of $P$ is a point $Q$ so 
%that $\overline{PQ} \parallel \overline{AB}$, $PQ=AB$, and $\overrightarrow{PQ}$ is in the same direction as $\overrightarrow{AB}$.  
%
%Under a reflection about a line $\ell$, if $P$ is on $\ell$, then $P$ is mapped to itself.  If $P$ is not on $\ell$, then the image of $P$ is a point $Q$ so that $\ell$ is the perpendicular bisector of $\overline{PQ}$.  
%

\begin{question}
\textbf{Definition.} Under a \textbf{dilation} about center $O$ and scale factor $r>0$, the image of $P$ is 
a point $Q$ so that $Q$ lies on \wordChoice{\choice{segment}\choice[correct]{ray}\choice{line}} 
$\answer[format=string]{OP}$ % $\overrightarrow{OP}$ 
and $OQ=\answer[format=string]{rOP}$.  The image of $O$ is $\answer[format=string]{O}$. 
\end{question}


\begin{question}
%Describe, both informally and formally, what it means to say two figures are congruent.
Informally, if one figure can be ``placed upon'' another so that the figures match exactly, the figures are said to be $\answer[format=string]{congruent}$.  This is called the ``principle of superposition.''  
\begin{question}
Formally, two figures are \textbf{congruent} if there is a sequence of $\answer[format=string]{basic rigid motions}$ (three words) that maps one onto the other.  
\end{question}
\end{question}

\begin{question}
%Describe, both informally and formally, what it means to say two figures are similar.
Informally, figures that have the ``same shape'' are said to be $\answer[format=string]{similar}$.  With moving and scaling, one figure can be placed on another so that the figures match exactly.    
\begin{question}
Formally, two figures are \textbf{similar} if there is a sequence of basic rigid motions and $\answer[format=string]{dilations}$ that maps one onto the other.  
\end{question}
\end{question}


\begin{question}
Compare and contrast the ideas of \textit{equal triangles},
  \textit{congruent triangles}, and \textit{similar triangles}.
\begin{freeResponse}
\end{freeResponse}
\begin{hint}
Two triangles are equal if they are the same sets of points.  For example, you could write $\triangle ABC = \triangle BCA$ because they are the same triangle, but this notation is misleading and rarely used.  

Congruence and similarity are defined above.  Two triangles that are congruent have the same angles measures and side lengths.  Two triangles that are similar, it turns out, have the same angle measures.  
\end{hint}
\end{question}

\begin{question}
A student says that any two rectangles are similar because all the angles are the same.  Is the student correct?  
\wordChoice{\choice{Yes.}\choice[correct]{No.}\choice{Not enough information.}} Explain. 
\begin{question}
Two rectangles are similar only if there is a single $\answer[format=string]{scale factor}$ (two words) that works for both the length and width.  Angles alone are not enough for quadrilaterals.  
\end{question}
\end{question}

\begin{question}When figures are similar, ratios of corresponding lengths must be equal.  We distinguish \textbf{``within figure''} and \textbf{``between figure''} ratios.  
 
Suppose $\triangle ABC \sim \triangle XYZ$.  Complete the following equation relating ratios \textbf{within} the figures: 
\[
\frac{AB}{\answer{BC}} = \frac{\answer{XY}}{YZ}.  
\]
Complete the following equation relating ratios \textbf{between} the figures: 
\[
\frac{AB}{\answer{XY}} = \frac{\answer{BC}}{YZ}.  
\]
A ratio between (or across) corresponding lengths from two similar figures is 
called a $\answer[format=string]{scale}$ factor.  
\end{question}


\begin{question}
Are all equilateral triangles are similar to each other?  
\wordChoice{\choice[correct]{Yes.}\choice{No.}\choice{It depends.}} Explain.  
\begin{question}
Equilateral triangles have interior angles that all measure $\answer{60}$ degrees.  So all equilateral triangles are similar to one another by AAA similarity.  
\end{question}
\end{question}

\begin{question}Are all isosceles right triangles similar to each other? 
\wordChoice{\choice[correct]{Yes.}\choice{No.}\choice{It depends.}} Explain.  
\begin{question}
Isosceles right triangles have interior angles that measure $90$, $\answer{45}$, and $\answer{45}$ degrees.  So all isosceles right triangles are similar to one another by AAA similarity.  
\end{question}
\end{question}


\end{document}
