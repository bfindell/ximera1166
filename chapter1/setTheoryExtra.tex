\documentclass[nooutcomes]{ximera}
%\documentclass[space,handout,nooutcomes]{ximera}

% For preamble materials

\graphicspath{
  {./}
  {chapter1/}
  {chapter2/}
  {chapter4/}
  {math1/}
  {math2/}
}

\usepackage{pgf,tikz}
\usepackage{mathrsfs}
\usetikzlibrary{arrows}
\pgfplotsset{compat=1.16}


\newcommand{\N}{\mathbb N}
\newcommand{\W}{\mathbb W}
\newcommand{\C}{\mathbb C}
\newcommand{\Z}{\mathbb Z}
\newcommand{\Q}{\mathbb Q}
\newcommand{\R}{\mathbb R}




\renewcommand{\Z}{\mathbb Z}

\title{Set Theory Problems}
\author{Bart Snapp and Brad Findell}
\begin{document}
\begin{abstract}
Extra problems about sets. 
\end{abstract}
\maketitle

\section{Reminders}
\begin{itemize}
\item Sets are collections of objects such as numbers or points.  The objects are called \emph{elements} of the set, and the order elements are listed is not important.  

\item The notation $\{7, 3\}$ means ``The set containing 7 and 3.''  

\item Note that $\{8\}$ is not the same as the number 8 but rather is a set that contains one element that happens to be a number. 

\item The set containing zero elements, sometimes call the \emph{empty set} is denoted $\{\}$ or $\emptyset$.  

\item The elements of a set can themselves be sets.  

\end{itemize}


\begin{problem}
Indicate the number of elements in each set: 
\begin{enumerate}
\item The set $\{3, 5, 6, 9, 10\}$ has $\answer{5}$ element(s).
\item The set $\{ \{3,2,7\}, \{4,5\}, \{2\}, \emptyset \}$ has $\answer{4}$ element(s).
\item The set $\{ \{ \} \}$ has $\answer{1}$ element(s).
\item The set $\{\}$ has $\answer{0}$ element(s).
\item The set $\emptyset$ has $\answer{0}$ element(s).
\item The set $\{ \emptyset \}$ has $\answer{1}$ element(s).
\end{enumerate}

\end{problem}

\begin{problem}
Indicate whether each statement is true or false: 
\begin{enumerate}
\item $2\in \{3, 2, 5\}$. \wordChoice{\choice[correct]{True}\choice{False}}
\item $2\subseteq \{3, 2, 5\}$. \wordChoice{\choice{True}\choice[correct]{False}}
\item $\{2\}\in \{3, 2, 5\}$.  \wordChoice{\choice{True}\choice[correct]{False}}
\item $\{2\}\subseteq \{3, 2, 5\}$.  \wordChoice{\choice[correct]{True}\choice{False}}
\item $\emptyset = \{ \}$. \wordChoice{\choice[correct]{True}\choice{False}}
\item $\emptyset = \{ \emptyset \} $. \wordChoice{\choice{True}\choice[correct]{False}}
\item $\{ \emptyset \} = \{\{\}\}$. \wordChoice{\choice[correct]{True}\choice{False}}
\item $\emptyset \in \{ \emptyset \} $. \wordChoice{\choice[correct]{True}\choice{False}}
\item $\emptyset \subset \{ \emptyset \} $. \wordChoice{\choice[correct]{True}\choice{False}}
\item $2 \in \{ \{3,2,7\}, \{4,5\}, \{2\}, \emptyset \}$. \wordChoice{\choice{True}\choice[correct]{False}}
\item $2 \subseteq \{ \{3,2,7\}, \{4,5\}, \{2\}, \emptyset \}$. \wordChoice{\choice{True}\choice[correct]{False}}
\item $\{2\} \in \{ \{3,2,7\}, \{4,5\}, \{2\}, \emptyset \}$. \wordChoice{\choice[correct]{True}\choice{False}}
\item $\{2\} \subseteq \{ \{3,2,7\}, \{4,5\}, \{2\}, \emptyset \}$. \wordChoice{\choice{True}\choice[correct]{False}}
\item $\{\{2\}\} \in \{ \{3,2,7\}, \{4,5\}, \{2\}, \emptyset \}$. \wordChoice{\choice{True}\choice[correct]{False}}
\item $\{\{2\}\} \subseteq \{ \{3,2,7\}, \{4,5\}, \{2\}, \emptyset \}$. \wordChoice{\choice[correct]{True}\choice{False}}
\end{enumerate}

\end{problem}

\begin{problem}
Explain the difference between the symbols $\in$ and $\subseteq$.
\begin{freeResponse}
\begin{hint}
The symbol $\in$ means ``is an element of,'' whereas $\subseteq$ means ``is a subset of.'' 
The notation $X \in Y$ means that $X$ is a single element in the set $Y$.  In this case, $X$ is typically not a set.  The notation $X \subseteq Y$, in contrast, requires that both $X$ and $Y$ are sets and, furthermore, that every element of $X$ is also in $Y$.
\end{hint}
\end{freeResponse}
\end{problem}

\begin{problem}
How is $\{\emptyset\}$ different from $\emptyset$?  
\begin{freeResponse}
\begin{hint}
The empty set, $\emptyset$, is a set that contains no elements.  That is, $\emptyset = \{\}$.  The set $\{\emptyset\}$ contains one element that is itself a set---and that element happens to be the empty set.  We could instead write $\{\{\}\}$, but that looks ugly.
\end{hint}
\end{freeResponse}
\end{problem}

%
%\begin{problem}
%Given two sets $X$ and $Y$, $X\cup Y$ is the set of elements that are
%\begin{multipleChoice}
%\choice{in $X$ or in $Y$ (but not in both).}
%\choice[correct]{in $X$ or in $Y$ (or both, as the ``or'' is inclusive).}  
%\choice{in $X$ and in $Y$.}
%\choice{in $X$ but not in $Y$.}
%\choice{in $Y$ but not in $X$.} 
%\end{multipleChoice}
%\end{problem}
%
%\begin{problem}
%Given two sets $X$ and $Y$, $X\cap Y$ is the set of elements that are 
%\begin{multipleChoice}
%\choice{in $X$ or in $Y$ (but not in both).}
%\choice{in $X$ or in $Y$ (or both, as the ``or'' is inclusive).}  
%\choice[correct]{in $X$ and in $Y$.}
%\choice{in $X$ but not in $Y$.}
%\choice{in $Y$ but not in $X$.} 
%\end{multipleChoice}
%\end{problem}
%
%\begin{problem}
%Given two sets $X$ and $Y$, $X - Y$ is the set of elements that are 
%\begin{multipleChoice}
%\choice{in $X$ or in $Y$ (but not in both).}
%\choice{in $X$ or in $Y$ (or both, as the ``or'' is inclusive).}  
%\choice{in $X$ and in $Y$.}
%\choice[correct]{in $X$ but not in $Y$.}
%\choice{in $Y$ but not in $X$.} 
%\end{multipleChoice}
%\end{problem}
%
%\begin{problem}
%Explain the difference between the symbols $\in$ and $\subseteq$.
%\begin{freeResponse}
%\begin{hint}
%The symbol $\in$ means ``is an element of,'' whereas $\subseteq$ means ``is a subset of.'' 
%The notation $X \in Y$ means that $X$ is a single element in the set $Y$.  In this case, $X$ is typically not a set.  The notation $X \subseteq Y$, in contrast, requires that both $X$ and $Y$ are sets and, furthermore, that every element of $X$ is also in $Y$.
%\end{hint}
%\end{freeResponse}
%\end{problem}
%
%\begin{problem}
%How is $\{\emptyset\}$ different from $\emptyset$?  
%\begin{freeResponse}
%\begin{hint}
%The empty set, $\emptyset$, is a set that contains no elements.  That is, $\emptyset = \{\}$.  The set $\{\emptyset\}$ contains one element that is itself a set---and that element happens to be the empty set.  We could instead write $\{\{\}\}$, but that looks ugly.
%\end{hint}
%\end{freeResponse}
%\end{problem}
%
%\begin{problem}
%Consider the following Venn diagrams:  
%
%\definecolor{ffffff}{rgb}{1.,1.,1.}
%\definecolor{cqcqcq}{rgb}{0.75,0.75,0.75}
%\begin{tikzpicture}[line width=0.8pt,line cap=round,line join=round,>=triangle 45,x=1.0cm,y=1.0cm,scale=0.4]
%%\clip(-7,-4.2) rectangle (10,4.2);
%\clip(-7,-14.2) rectangle (27,4.2);
%\draw [color=cqcqcq,fill=cqcqcq,fill opacity=1.0] (0.,0.) circle (3.cm);
%\draw [color=cqcqcq,fill=cqcqcq,fill opacity=1.0] (4.,0.) circle (3.cm);
%\draw [shift={(0.,0.)},color=ffffff,fill=ffffff,fill opacity=1.0]  plot[domain=-0.842:0.842,variable=\t]({3.*cos(\t r)},{3.*sin(\t r)});
%\draw [shift={(4.,0.)},color=ffffff,fill=ffffff,fill opacity=1.0]  plot[domain=2.301:3.983,variable=\t]({3.*cos(\t r)},{3.*sin(\t r)});
%%\draw [shift={(0.,0.)},color=cqcqcq,fill=cqcqcq,fill opacity=1.0]  plot[domain=-0.842:0.842,variable=\t]({3.*cos(\t r)},{3.*sin(\t r)});
%%\draw [shift={(4.,0.)},color=cqcqcq,fill=cqcqcq,fill opacity=1.0]  plot[domain=2.301:3.983,variable=\t]({3.*cos(\t r)},{3.*sin(\t r)});
%\draw (0.,0.) circle (3.cm);
%\draw (4.,0.) circle (3.cm);
%\draw (-5.,4.)-- (-5.,-4.);
%\draw (-5.,-4.)-- (9.,-4.);
%\draw (9.,-4.)-- (9.,4.);
%\draw (9.,4.)-- (-5.,4.);
%\draw (-2,2.2) node[anchor=north west] {X};
%\draw (5.1,2.2) node[anchor=north west] {Y};
%\draw (-7,4) node[anchor=north west] {(A)};
%
%\begin{scope}[shift={(17,0)}]
%\draw [color=cqcqcq,fill=cqcqcq,fill opacity=1.0] (0.,0.) circle (3.cm);
%\draw [color=cqcqcq,fill=cqcqcq,fill opacity=1.0] (4.,0.) circle (3.cm);
%%\draw [shift={(0.,0.)},color=ffffff,fill=ffffff,fill opacity=1.0]  plot[domain=-0.842:0.842,variable=\t]({3.*cos(\t r)},{3.*sin(\t r)});
%%\draw [shift={(4.,0.)},color=ffffff,fill=ffffff,fill opacity=1.0]  plot[domain=2.301:3.983,variable=\t]({3.*cos(\t r)},{3.*sin(\t r)});
%%\draw [shift={(0.,0.)},color=cqcqcq,fill=cqcqcq,fill opacity=1.0]  plot[domain=-0.842:0.842,variable=\t]({3.*cos(\t r)},{3.*sin(\t r)});
%%\draw [shift={(4.,0.)},color=cqcqcq,fill=cqcqcq,fill opacity=1.0]  plot[domain=2.301:3.983,variable=\t]({3.*cos(\t r)},{3.*sin(\t r)});
%\draw (0.,0.) circle (3.cm);
%\draw (4.,0.) circle (3.cm);
%\draw (-5.,4.)-- (-5.,-4.);
%\draw (-5.,-4.)-- (9.,-4.);
%\draw (9.,-4.)-- (9.,4.);
%\draw (9.,4.)-- (-5.,4.);
%\draw (-2,2.2) node[anchor=north west] {X};
%\draw (5.1,2.2) node[anchor=north west] {Y};
%\draw (-7,4) node[anchor=north west] {(B)};
%\end{scope}
%
%\begin{scope}[shift={(0,-10)}]
%%\draw [color=cqcqcq,fill=cqcqcq,fill opacity=1.0] (0.,0.) circle (3.cm);
%%\draw [color=cqcqcq,fill=cqcqcq,fill opacity=1.0] (4.,0.) circle (3.cm);
%%\draw [shift={(0.,0.)},color=ffffff,fill=ffffff,fill opacity=1.0]  plot[domain=-0.842:0.842,variable=\t]({3.*cos(\t r)},{3.*sin(\t r)});
%%\draw [shift={(4.,0.)},color=ffffff,fill=ffffff,fill opacity=1.0]  plot[domain=2.301:3.983,variable=\t]({3.*cos(\t r)},{3.*sin(\t r)});
%\draw [shift={(0.,0.)},color=cqcqcq,fill=cqcqcq,fill opacity=1.0]  plot[domain=-0.842:0.842,variable=\t]({3.*cos(\t r)},{3.*sin(\t r)});
%\draw [shift={(4.,0.)},color=cqcqcq,fill=cqcqcq,fill opacity=1.0]  plot[domain=2.301:3.983,variable=\t]({3.*cos(\t r)},{3.*sin(\t r)});
%\draw (0.,0.) circle (3.cm);
%\draw (4.,0.) circle (3.cm);
%\draw (-5.,4.)-- (-5.,-4.);
%\draw (-5.,-4.)-- (9.,-4.);
%\draw (9.,-4.)-- (9.,4.);
%\draw (9.,4.)-- (-5.,4.);
%\draw (-2,2.2) node[anchor=north west] {X};
%\draw (5.1,2.2) node[anchor=north west] {Y};
%\draw (-7,4) node[anchor=north west] {(C)};
%\end{scope}
%
%\begin{scope}[shift={(17,-10)}]
%\draw [color=cqcqcq,fill=cqcqcq,fill opacity=1.0] (0.,0.) circle (3.cm);
%%\draw [color=cqcqcq,fill=cqcqcq,fill opacity=1.0] (4.,0.) circle (3.cm);
%\draw [shift={(0.,0.)},color=ffffff,fill=ffffff,fill opacity=1.0]  plot[domain=-0.842:0.842,variable=\t]({3.*cos(\t r)},{3.*sin(\t r)});
%\draw [shift={(4.,0.)},color=ffffff,fill=ffffff,fill opacity=1.0]  plot[domain=2.301:3.983,variable=\t]({3.*cos(\t r)},{3.*sin(\t r)});
%%\draw [shift={(0.,0.)},color=cqcqcq,fill=cqcqcq,fill opacity=1.0]  plot[domain=-0.842:0.842,variable=\t]({3.*cos(\t r)},{3.*sin(\t r)});
%%\draw [shift={(4.,0.)},color=cqcqcq,fill=cqcqcq,fill opacity=1.0]  plot[domain=2.301:3.983,variable=\t]({3.*cos(\t r)},{3.*sin(\t r)});
%\draw (0.,0.) circle (3.cm);
%\draw (4.,0.) circle (3.cm);
%\draw (-5.,4.)-- (-5.,-4.);
%\draw (-5.,-4.)-- (9.,-4.);
%\draw (9.,-4.)-- (9.,4.);
%\draw (9.,4.)-- (-5.,4.);
%\draw (-2,2.2) node[anchor=north west] {X};
%\draw (5.1,2.2) node[anchor=north west] {Y};
%\draw (-7,4) node[anchor=north west] {(D)};
%\end{scope}
%
%\end{tikzpicture}
%
%For each set expression below, identify the Venn diagram above in which the corresponding region is shaded: 
%\begin{enumerate}
%\item $X\cap Y$ is figure $\answer{C}$
%\item $X\cup Y$ is figure $\answer{B}$. 
%\item $X - Y$ is figure $\answer{D}$
%\end{enumerate}
%
%\end{problem}
%
%
%\begin{problem}
%Draw a Venn diagram for the set of elements that are in $X$ or $Y$ \emph{but not both}. 
%How does it differ from the Venn diagram for $X\cup Y$?  
%\begin{freeResponse}
%\begin{hint}
%A Venn diagram for elements in $X$ or $Y$ \emph{but not both} is shown in figure (A) from the previous problem.  
%% Same as the Venn diagram for $X\cup Y$, except that the $X\cap Y$ part is not shaded.  
%\end{hint}
%\end{freeResponse}
%\end{problem}
%
%\begin{problem}
%Consider the following Venn diagrams: 
%
%\definecolor{ffffff}{rgb}{1.,1.,1.}
%\definecolor{cqcqcq}{rgb}{0.75,0.75,0.75}
%\begin{tikzpicture}[line width=0.8pt,line cap=round,line join=round,>=triangle 45,x=1.0cm,y=1.0cm,scale=0.4]
%%\clip(-7,-4.2) rectangle (10,4.2);
%\clip(-7,-4.2) rectangle (27,4.2);
%\draw (-1.,0.) circle (2.5cm);
%\draw (5,0.) circle (2.5cm);
%\draw (-5.,4.)-- (-5.,-4.);
%\draw (-5.,-4.)-- (9.,-4.);
%\draw (9.,-4.)-- (9.,4.);
%\draw (9.,4.)-- (-5.,4.);
%\draw (-2.2,2.2) node[anchor=north west] {X};
%\draw (5.1,2.2) node[anchor=north west] {Y};
%\draw (-7,4) node[anchor=north west] {(A)};
%
%\begin{scope}[shift={(17,0)}]
%\draw (3.3,-0.7) circle (1.6cm);
%\draw (4.,0.) circle (3.cm);
%\draw (-5.,4.)-- (-5.,-4.);
%\draw (-5.,-4.)-- (9.,-4.);
%\draw (9.,-4.)-- (9.,4.);
%\draw (9.,4.)-- (-5.,4.);
%\draw (2.1,0.5) node[anchor=north west] {X};
%\draw (5.1,2.2) node[anchor=north west] {Y};
%\draw (-7,4) node[anchor=north west] {(B)};
%\end{scope}
%
%\end{tikzpicture}
%
%%\begin{image}
%%  \includegraphics{set4.png}
%%\end{image}
%\begin{enumerate}
%\item If Venn diagram (A) above shows the relationship between sets $X$ and $Y$, then $X\cap Y =$ \wordChoice{\choice{$0$}\choice[correct]{$\emptyset$}\choice{$X\cup Y$}} and the sets are said to be 
%\wordChoice{\choice[correct]{disjoint}\choice{empty}\choice{subsets}}.  
%
%\item If Venn diagram (B) above shows the relationship between sets $X$ and $Y$, then we say that \wordChoice{\choice{$X$ and $Y$ are disjoint}\choice[correct]{$X\subseteq Y$}\choice{$Y\subseteq X$}}.
%
%\item If we let $X$ be the set of ``right triangles'' and we let $Y$ be the set of ``equilateral triangles'' which diagram above shows the relationship between these two sets?
%\begin{multipleChoice}
%\choice[correct]{Diagram (A).}
%\choice{Diagram (B).}
%\choice{Neither of these.}
%\choice{Not enough information.}
%\end{multipleChoice}
%
%Explain your reasoning.
%\begin{freeResponse}
%\begin{hint}
%Diagram (A) is accurate because no right triangles are also equilateral triangles.  
%\end{hint}
%\end{freeResponse}
%\end{enumerate}
%\end{problem}
%
%
%\begin{problem}
%If $X = \{1,2,3,4,5\}$ and $Y = \{3,4,5,6\}$ find the following: (List elements in ascending order, separated by commas, with no spaces.)
%\begin{enumerate}
%\item $X\cup Y = \{\answer[format=string]{1,2,3,4,5,6}\}$
%\item $X\cap Y = \{\answer[format=string]{3,4,5}\}$
%\item $X-Y = \{\answer[format=string]{1,2}\}$
%\item $Y-X = \{\answer[format=string]{6}\}$
%\end{enumerate}
%\end{problem}
%
%\begin{problem}
%Let $n\mathbb Z$ represent the integer multiples of $n$. So for example:
%\[
%3\mathbb Z = \{\dots,-12,-9,-6,-3,0,3,6,9,12,\dots\}
%\]
%Compute the following (use capital $Z$ for $\mathbb Z$):
%\begin{enumerate}
%\item $3\mathbb Z\cap 4\mathbb Z = \answer{12Z}$ 
%\item $2\mathbb Z\cap 5\mathbb Z = \answer{10Z}$
%\item $3\mathbb Z\cap 6\mathbb Z = \answer{6Z}$
%\item $4\mathbb Z\cap 6\mathbb Z = \answer{12Z}$
%\item $4\mathbb Z\cap 10\mathbb Z = \answer{20Z}$
%\end{enumerate}
%\end{problem}
%
%%\begin{problem}
%%Let $n\Z$ represent the integer multiples of $n$. So for example:
%%\[
%%3\Z = \{\dots,-12,-9,-6,-3,0,3,6,9,12,\dots\}
%%\]
%%Compute the following (use capital $Z$ for $\Z$):
%%\begin{enumerate}
%%\item $3\Z\cap 4\Z = \answer{12Z}$ 
%%\item $2\Z\cap 5\Z = \answer{10Z}$
%%\item $3\Z\cap 6\Z = \answer{6Z}$
%%\item $4\Z\cap 6\Z = \answer{12Z}$
%%\item $4\Z\cap 10\Z = \answer{20Z}$
%%\end{enumerate}
%%\end{problem}
%
%\begin{problem}
%Make a general rule for intersecting sets of the form $n\mathbb Z$ and
%  $m\mathbb Z$. Explain why your rule works.
%\begin{freeResponse}
%\begin{hint}
%The intersection of two sets is what they have in \emph{common}.  The intersection of the set of multiples of $n$ and the set of multiples of $m$ are called \emph{common multiples} (surprise!), and they are all multiples of the least common multiple of $n$ and $m$.  
%\end{hint}
%\end{freeResponse}
%\end{problem}
%
%\begin{problem}
%If $X\cup Y = X$, what can we say about the relationship between the sets $X$ and $Y$? Explain your reasoning.
%
%% $Y\subseteq X$ because every element of $Y$ must already be in $X$.  
%\wordChoice{\choice{$X\subseteq Y$}\choice{$X=Y$}\choice[correct]{$Y\subseteq X$}\choice{$X=\emptyset$}}
%because every element of \wordChoice{\choice{$X$}\choice[correct]{$Y$}} must be in \wordChoice{\choice[correct]{$X$}\choice{$Y$}}.
%
%\end{problem}
%
%\begin{problem}
%If $X\cap Y = X$, what can we say about the relationship between the sets $X$ and $Y$? Explain your reasoning.
%
%% $X\subseteq Y$ because every element of $X$ must already be in $Y$. 
%\wordChoice{\choice[correct]{$X\subseteq Y$}\choice{$X=Y$}\choice{$Y\subseteq X$}\choice{$X=\emptyset$}}
%because every element of \wordChoice{\choice[correct]{$X$}\choice{$Y$}} must be in \wordChoice{\choice{$X$}\choice[correct]{$Y$}}.
%\end{problem}
%
%\begin{problem}
%If $X-Y =\emptyset$, what can we say about the relationship between the sets $X$ and $Y$? Explain your reasoning.
%
%% $X\subseteq Y$ because that would mean $X$ contains no elements that are not also in $Y$.
%\wordChoice{\choice[correct]{$X\subseteq Y$}\choice{$X=Y$}\choice{$Y\subseteq X$}\choice{$X=\emptyset$}}
%because every element of \wordChoice{\choice[correct]{$X$}\choice{$Y$}} must be in \wordChoice{\choice{$X$}\choice[correct]{$Y$}}.
%\end{problem}
%


\end{document}
