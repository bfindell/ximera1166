\documentclass[nooutcomes,space,handout]{ximera}
%\documentclass[space,handout,nooutcomes]{ximera}

% For preamble materials

\graphicspath{
  {./}
  {chapter1/}
  {chapter2/}
  {chapter4/}
  {math1/}
  {math2/}
}

\usepackage{pgf,tikz}
\usepackage{mathrsfs}
\usetikzlibrary{arrows}
\pgfplotsset{compat=1.16}


\newcommand{\N}{\mathbb N}
\newcommand{\W}{\mathbb W}
\newcommand{\C}{\mathbb C}
\newcommand{\Z}{\mathbb Z}
\newcommand{\Q}{\mathbb Q}
\newcommand{\R}{\mathbb R}




\title{Proof by Picture 1}
\author{Bart Snapp and Brad Findell}
\begin{document}
\begin{abstract}
Short-answer proofs about triangle area. 
\end{abstract}
\maketitle

% No change?

\begin{problem}
Explain how the following picture ``proves'' that
  the area of a right triangle is half the base times the height.
%\begin{image}
%%\includegraphics{pbpAreaRight.png}
%\definecolor{qqwuqq}{rgb}{0.,0.392,0.}
%\begin{tikzpicture}[line cap=round,line join=round,>=triangle 45,x=1.0cm,y=1.0cm]
%\draw[line width=0.8pt,color=qqwuqq,fill=qqwuqq,fill opacity=0.10] (0.28,0.) -- (0.28,0.28) -- (0.,0.28) -- (0.,0.) -- cycle; 
%\draw [line width=0.8pt] (0.,3.)-- (0.,0.) -- (4.,0.) -- cycle;
%\draw [line width=0.8pt,dash pattern=on 2pt off 2pt] (0.,3.)-- (4.,3.) -- (4.,0.);
%\draw [color=white] (-4.,0.) circle (0.2pt);
%\draw [color=white] (8.,0.) circle (0.2pt);
%\end{tikzpicture}
%\end{image}
%
%\begin{freeResponse}
%\end{freeResponse}
%
%I \wordChoice{\choice[correct]{have}\choice{have not}} entered my own answer in the box above.  
%
\begin{problem}
The area of the rectangle is base times height.  The rectangle is made up of two congruent right triangles.  Because congruent triangles have the same area, the area of each right triangle is \wordChoice{\choice{equal to}\choice[correct]{half}\choice{double}} the area of the rectangle.  
\end{problem}

\end{problem}
%
%\begin{problem}
%Suppose you know that the area of a \textbf{right} triangle is
%  half the base times the height. Explain how the following picture
%  ``proves'' that the area of \textbf{every} triangle is half the base times the
%  height.
%\begin{image}
%%\includegraphics{pbpDisTri.png}
%\definecolor{qqwuqq}{rgb}{0.,0.392,0.}
%\begin{tikzpicture}[line cap=round,line join=round,>=triangle 45,x=1.0cm,y=1.0cm]
%\draw[line width=0.8pt,color=qqwuqq,fill=qqwuqq,fill opacity=0.1] (3.,0.20) -- (2.80,0.20) -- (2.80,0.) -- (3.,0.) -- cycle; 
%\draw[line width=0.8pt,color=qqwuqq,fill=qqwuqq,fill opacity=0.1] (3.20,0.) -- (3.20,0.20) -- (3.,0.20) -- (3.,0.) -- cycle; 
%\draw [line width=0.8pt] (0.,0.)-- (3.,1.8)-- (4.5,0.)-- cycle;
%\draw [line width=0.8pt,dash pattern=on 2pt off 2pt] (3.,0.)-- (3.,1.8);
%\draw [color=white] (-4.,0.) circle (0.1pt);
%\draw [color=white] (8.,0.) circle (0.1pt);
%\end{tikzpicture}
%\end{image}
%
%\begin{freeResponse}
%\end{freeResponse}
%
%I \wordChoice{\choice[correct]{have}\choice{have not}} entered my own answer in the box above.  
%
%\begin{problem}
%Explanation 1: Surround the triangle with a rectangle as shown.    
%\begin{image}
%%\includegraphics{pbpDisTri.png}
%\definecolor{qqwuqq}{rgb}{0.,0.392,0.}
%\begin{tikzpicture}[line cap=round,line join=round,>=triangle 45,x=1.0cm,y=1.0cm]
%\draw[line width=0.8pt,color=qqwuqq,fill=qqwuqq,fill opacity=0.1] (3.,0.20) -- (2.80,0.20) -- (2.80,0.) -- (3.,0.) -- cycle; 
%\draw[line width=0.8pt,color=qqwuqq,fill=qqwuqq,fill opacity=0.1] (3.20,0.) -- (3.20,0.20) -- (3.,0.20) -- (3.,0.) -- cycle; 
%\draw [line width=0.8pt] (0.,0.)-- (3.,1.8) -- (4.5,0.) -- cycle;
%\draw [line width=0.8pt,dash pattern=on 2pt off 2pt] (3.,0.)-- (3.,1.8);
%\draw [line width=0.8pt,dash pattern=on 2pt off 2pt] (0.,0.)-- (0.,1.8) -- (4.5,1.8)-- (4.5,0.);
%\draw [color=white] (-4.,0.) circle (0.1pt);
%\draw [color=white] (8.,0.) circle (0.1pt);
%\end{tikzpicture}
%\end{image}
%Then the solid triangle and the rectangle have the same $\answer[format=string]{base}$ and the same $\answer[format=string]{height}$.  And again the area of the triangle is $\answer[format=string]{half}$ the area of the rectangle.
%\end{problem}
%
%\begin{problem}
%Explanation 2: The whole triangle is made up of two right triangles.  Call the bases of the small right triangles $c$ and $d$, and call the base of the large (combined) triangle $b$.  
%\begin{image}
%%\includegraphics{pbpDisTri.png}
%\definecolor{qqwuqq}{rgb}{0.,0.392,0.}
%\begin{tikzpicture}
%\draw[line width=0.8pt,color=qqwuqq,fill=qqwuqq,fill opacity=0.1] (3.,0.20) -- (2.80,0.20) -- (2.80,0.) -- (3.,0.) -- cycle; 
%\draw[line width=0.8pt,color=qqwuqq,fill=qqwuqq,fill opacity=0.1] (3.20,0.) -- (3.20,0.20) -- (3.,0.20) -- (3.,0.) -- cycle; 
%\draw [line width=0.8pt] (0.,0.)-- (3.,1.8) -- (4.5,0.) -- cycle;
%\draw [line width=0.8pt,dash pattern=on 2pt off 2pt] (3.,0.)-- (3.,1.8);
%\draw [color=white] (-4.,0.) circle (0.1pt);
%\draw [color=white] (8.,0.) circle (0.1pt);
%\draw (1.6,-0.2) node {$c$};
%\draw (3.8,-0.2) node {$d$};
%\draw (2.8,1.0) node {$h$};
%\end{tikzpicture}
%\end{image}
%
%Then the base of the whole $b=\answer{c+d}$, and the three triangles all have the same height, $\answer{h}$.   The area of the whole triangle is the sum of the areas of the two right triangles:  
%\[
%\mathrm{Area} =  \answer{\frac{hc}{2}}+ \answer{\frac{hd}{2}}= \frac{h(c + d)}{2} = \answer{bh/2}
%\]
%because $b=c+d$.
%\end{problem}
%
%\end{problem}
%
%\begin{problem}
%Now suppose that \textit{Geometry Giorgio} attempts to
%solve a similar problem. Again knowing that the area of a right
%triangle is half the base times the height, he draws the following
%picture:
%\begin{image}
%%\includegraphics{pbpDisTriGio.png}
%\begin{tikzpicture}[line cap=round,line join=round,>=triangle 45,x=1.0cm,y=1.0cm]
%\draw [line width=0.8pt,dash pattern=on 2pt off 2pt] (0.,0.)-- (0.,2.5) -- (5.,2.5) -- (5.,0.) -- (2.,0.);
%\draw [line width=0.8pt] (0.,0.)-- (2.,0.) -- (5.,2.5) -- cycle;
%\draw [color=white] (-4.,0.) circle (0.2pt);
%\draw [color=white] (8.,0.) circle (0.2pt);
%\end{tikzpicture}
%\end{image}
%
%\textit{Geometry Giorgio} states that the diagonal line cuts the
%rectangle in half, and thus the area of the triangle is half the base
%times the height. Is this correct reasoning? 
%
%%If so, give a complete
%%explanation. If not, give correct reasoning based on \textit{Geometry
%%  Giorgio's} picture.
%
%Response: The area of the solid triangle is
%\wordChoice{\choice{greater than}\choice{equal to}\choice[correct]{less than}} half the area of the rectangle.  Furthermore, the rectangle 
%\wordChoice{\choice[correct]{does not have}\choice{has}} the same base as the triangle, so ``half the base times height'' is 
%\wordChoice{\choice[correct]{unclear}\choice{correct}\choice{incorrect}}.  
%
%What would you recommend to Giorgio?
%\begin{multipleChoice}
%\choice{The reasoning from the previous problem still works for this triangle.  Go back to that.} 
%\choice{The formula ``half base times height'' does not work for this kind of triangle.}
%\choice[correct]{Different reasoning is needed, but the same formula can work.}
%\end{multipleChoice}
%
%
%\begin{problem}
%To get a better sense of Giorgio's triangle, try some numbers.  
%\begin{image}
%\begin{tikzpicture}[line cap=round,line join=round,>=triangle 45,x=1.0cm,y=1.0cm]
%\draw [line width=0.8pt,dash pattern=on 2pt off 2pt] (0.,0.)-- (0.,2.5) -- (5.,2.5) -- (5.,0.) -- (2.,0.);
%\draw [line width=0.8pt] (0.,0.)-- (2.,0.) -- (5.,2.5) -- cycle;
%\draw (1.1,-0.2) node {$4$};
%\draw (3.5,-0.2) node {$6$};
%\draw (5.2,1.3) node {$5$};
%\draw (1.5,1.7) node {$B$};
%\draw (2.1,.6) node {$A$};
%\draw (4,.8) node {$C$};
%\draw [color=white] (-4.,0.) circle (0.2pt);
%\draw [color=white] (8.,0.) circle (0.2pt);
%\end{tikzpicture}
%\end{image}
%To find the area of this triangle (which is much less than half the rectangle), we use the following variables:   
%\begin{align*}
%A &= \textrm{Area of solid triangle} \\
%B &= \textrm{Area of large right triangle} \\
%C &= \textrm{Area of small right triangle} \\
%D &= \textrm{Area of rectangle} = A + B + C.
%\end{align*}
%
%We find $A$, the area of the solid triangle as follows:  
%\begin{align*}
%A  &= D - B  - C \\
%      &= \answer{50} - \answer{25} - \answer{15} \\
%      &= \answer{10}
%\end{align*}
%
%
%%\begin{problem}
%%Maybe a grid will help.    
%%\begin{image}
%%\begin{tikzpicture}
%%\draw [line width=0.8pt,dash pattern=on 2pt off 2pt] (0.,0.)-- (0.,2.5) -- (5.,2.5) -- (5.,0.) -- (2.,0.);
%%\draw [line width=0.8pt] (0.,0.)-- (2.,0.) -- (5.,2.5) -- cycle;
%%\draw (1.1,-0.2) node {$4$};
%%\draw (3.5,-0.2) node {$6$};
%%\draw (5.2,1.3) node {$5$};
%%%% Dot grid
%%%\foreach \x in {0,0.5,...,5} {
%%%    \foreach \y in {0,0.5,...,2.5} {
%%%        \fill[color=gray] (\x,\y) circle (0.04);
%%%    }
%%% }
%%\draw[step=0.5,gray,thin,xshift=0.5,yshift=0.5] (0.0,0.0) grid (5.0,2.5);
%%\draw [color=white] (-4.,0.) circle (0.2pt);
%%\draw [color=white] (8.,0.) circle (0.2pt);
%%\end{tikzpicture}
%%\end{image}
%%\end{problem}
%
%\end{problem}
%
%\begin{problem}
%The following argument abbreviates this approach:
%\begin{image}
%\begin{tikzpicture}[line cap=round,line join=round,>=triangle 45,x=1.0cm,y=1.0cm]
%%\draw [line width=0.8pt,dash pattern=on 2pt off 2pt] (0.,0.)-- (0.,2.5) -- (5.,2.5) -- (5.,0.) -- (2.,0.);
%\draw [line width=0.8pt,dash pattern=on 2pt off 2pt] (5.,2.5) -- (5.,0.) -- (2.,0.);
%\draw [line width=0.8pt] (0.,0.)-- (2.,0.) -- (5.,2.5) -- cycle;
%\draw (1.1,-0.2) node {$b$};
%\draw (3.5,-0.2) node {$x$};
%\draw (5.2,1.3) node {$h$};
%%\draw (1.5,1.7) node {$B$};
%\draw (2.1,.6) node {$A$};
%\draw (4,.8) node {$C$};
%\draw [color=white] (-4.,0.) circle (0.2pt);
%\draw [color=white] (8.,0.) circle (0.2pt);
%\end{tikzpicture}
%\end{image}
%
%We find $A$, the area of the solid triangle as follows: Compute the area ($A+C$) of the large right triangle (with base $\answer{b+x}$) and then 
%$\answer[format=string]{subtract}$ the area ($C$) of the small right triangle (with base $\answer{x}$): 
%\[
%\mathrm{Area} = A =  \answer{\frac{h(b+x)}{2}} - \answer{\frac{hx}{2}}= \frac{hb+hx}{2} - \frac{hx}{2}
%= \left(\frac{hb}{2} + \frac{hx}{2}\right) - \frac{hx}{2} = \answer{\frac{hb}{2}}
%\]
%\end{problem}
\end{problem}

\end{document}
