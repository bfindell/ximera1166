\documentclass[nooutcomes]{ximera}
%\documentclass[space,handout,nooutcomes]{ximera}

% For preamble materials

\usepackage{pgf,tikz}
\usepackage{mathrsfs}
\usetikzlibrary{arrows}
\usepackage{framed}
\usepackage{amsmath}
\pgfplotsset{compat=1.17}

\def\fixnote#1{\begin{framed}{\textcolor{red}{Fix note: #1}}\end{framed}}  % Allows insertion of red notes about needed edits
%\def\fixnote#1{}

\def\detail#1{{\textcolor{blue}{Detail: #1}}}   

\pdfOnly{\renewenvironment{image}[1][]{\begin{center}}{\end{center}}}

\graphicspath{
  {./}
  {chapter1/}
  {chapter2/}
  {chapter4/}
  {proofs/}
  {graphics/}
  {../graphics/}
}

\newenvironment{sectionOutcomes}{}{}


%%% This set of code is all of our user defined commands
\newcommand{\bysame}{\mbox{\rule{3em}{.4pt}}\,}
\newcommand{\N}{\mathbb N}
\newcommand{\C}{\mathbb C}
\newcommand{\W}{\mathbb W}
\newcommand{\Z}{\mathbb Z}
\newcommand{\Q}{\mathbb Q}
\newcommand{\R}{\mathbb R}
\newcommand{\A}{\mathbb A}
\newcommand{\D}{\mathcal D}
\newcommand{\F}{\mathcal F}
\newcommand{\ph}{\varphi}
\newcommand{\ep}{\varepsilon}
\newcommand{\aph}{\alpha}
\newcommand{\QM}{\begin{center}{\huge\textbf{?}}\end{center}}

\renewcommand{\le}{\leqslant}
\renewcommand{\ge}{\geqslant}
\renewcommand{\a}{\wedge}
\renewcommand{\v}{\vee}
\renewcommand{\l}{\ell}
\newcommand{\mat}{\mathsf}
\renewcommand{\vec}{\mathbf}
\renewcommand{\subset}{\subseteq}
\renewcommand{\supset}{\supseteq}
%\renewcommand{\emptyset}{\varnothing}
%\newcommand{\xto}{\xrightarrow}
%\renewcommand{\qedsymbol}{$\blacksquare$}
%\newcommand{\bibname}{References and Further Reading}
%\renewcommand{\bar}{\protect\overline}
%\renewcommand{\hat}{\protect\widehat}
%\renewcommand{\tilde}{\widetilde}
%\newcommand{\tri}{\triangle}
%\newcommand{\minipad}{\vspace{1ex}}
%\newcommand{\leftexp}[2]{{\vphantom{#2}}^{#1}{#2}}

%% More user defined commands
\renewcommand{\epsilon}{\varepsilon}
\renewcommand{\theta}{\vartheta} %% only for kmath
\renewcommand{\l}{\ell}
\renewcommand{\d}{\, d}
\newcommand{\ddx}{\frac{d}{dx}}
\newcommand{\dydx}{\frac{dy}{dx}}


\usepackage{bigstrut}


\title{Folding and Tracing}
\author{Bart Snapp and Brad Findell}
\begin{document}
\begin{abstract}
Short-answer questions about folding and tracing. 
\end{abstract}
\maketitle

%
%\begin{problem}
%What are the rules for folding and tracing constructions?
%\end{problem}

\definecolor{ududff}{rgb}{0.30196078431372547,0.30196078431372547,1}
\definecolor{xdxdff}{rgb}{0.49019607843137253,0.49019607843137253,1}

\begin{problem}
Use folding and tracing to bisect a given line segment. 
\begin{image}
\begin{tikzpicture}[line cap=round,line join=round,>=triangle 45,x=1cm,y=1cm]
\clip(-3,-1) rectangle (7,2);
\draw [line width=1pt] (1,0)-- (3,1);
\begin{scriptsize}
\draw [fill=xdxdff] (1,0) circle (1.5pt);
\draw [fill=ududff] (3,1) circle (1.5pt);
\draw[color=white] (-2,0) circle (0.2pt);
\draw[color=white] (6,0) circle (0.2pt);
\end{scriptsize}
\end{tikzpicture}
\end{image}
Which of the following \textbf{best} describes how to \textbf{begin} the process?  
\begin{multipleChoice}
\choice{Fold the segment in half.} 
\choice{Fold the segment onto itself.}
\choice{Fold along the segment.}
\choice[correct]{Fold one endpoint onto the other.}
\choice{Fold the paper in half.}
\end{multipleChoice}
\begin{problem}
Correct!  Fold one endpoint of the segment onto the other.  The fold intersects the segment at its $\answer[format=string]{midpoint}$.  
Furthermore, the fold will be the $\answer[format=string]{perpendicular bisector}$ (two words) of the segment.  
\end{problem}
\end{problem}


\begin{problem}
Given a line and a point \textbf{not on} the line, use folding and tracing to construct a line perpendicular to the segment and passing through the given point. 
\begin{image}
\begin{tikzpicture}[line cap=round,line join=round,>=triangle 45,x=1cm,y=1cm]
\clip(-1,-1) rectangle (9,3);
\draw [line width=1pt] (1,0)-- (7,1);
\begin{scriptsize}
\draw [fill=xdxdff] (4,2) circle (1.5pt);
\draw[color=white] (-1,0) circle (0.2pt);
\draw[color=white] (9,0) circle (0.2pt);
\end{scriptsize}
\end{tikzpicture}
\end{image}
Which of the following \textbf{best} describes how to \textbf{begin} the process?  
\begin{multipleChoice}
\choice{Fold the line in half.} 
\choice[correct]{Fold the line onto itself.}
\choice{Fold along the line.}
\choice{Fold one endpoint onto the other.}
\choice{Fold the paper in half.}
\end{multipleChoice}
\begin{problem}
Correct!  Fold the line onto itself so that the $\answer[format=string]{fold}$ goes through the given $\answer[format=string]{point}$.

Will the process also work when the point is \textbf{on} the line?  (Yes or no.) $\answer[format=string]{yes}$
\end{problem}
\end{problem}


\begin{problem}
Use folding and tracing to bisect a given angle.
\begin{image}
\begin{tikzpicture}[line cap=round,line join=round,>=triangle 45,x=1cm,y=1cm]
\clip(-1,-2) rectangle (9,3);
\draw [line width=1pt] (1,-0.5)-- (5,-1);
\draw [line width=1pt] (1,-0.5)-- (5,2);
\draw[color=white] (-1,0) circle (0.2pt);
\draw[color=white] (9,0) circle (0.2pt);
\begin{scriptsize}
\draw [fill=xdxdff] (1,-0.5) circle (1.5pt);
\end{scriptsize}
\end{tikzpicture}
\end{image}

Which of the following \textbf{best} describes how to \textbf{begin} the process?  
\begin{multipleChoice}
\choice{Fold the angle in half.} 
\choice{Fold the angle onto itself.}
\choice{Fold along the angle.}
\choice[correct]{Fold one ray onto the other.}
\choice{Fold the paper in half.}
\end{multipleChoice}
\begin{problem}
Correct!  Fold one side of the $\answer[format=string]{angle}$ onto the other so that the fold goes through the $\answer[format=string]{vertex}$ of the angle. 
\end{problem}
\end{problem}



\begin{problem}
Given a point and line, use folding and tracing to construct a line parallel to the given line and passing through the given point. 
\begin{image}
\begin{tikzpicture}[line cap=round,line join=round,>=triangle 45,x=1cm,y=1cm]
\clip(-1,-1) rectangle (9,3);
\draw [line width=1pt] (1,0)-- (7,1);
\begin{scriptsize}
\draw [fill=xdxdff] (4,2) circle (1.5pt);
\draw[color=white] (-1,0) circle (0.2pt);
\draw[color=white] (9,0) circle (0.2pt);
\end{scriptsize}
\end{tikzpicture}
\end{image}
% Construct a perpendicular to a perpendicular as follows:  
% (1) Fold the line onto itself so that the fold goes through the given point. 
% (2) Fold the new fold onto itself so that the second fold goes through the given point.
Which of the following \textbf{best} describes how to \textbf{begin} the process?  
\begin{multipleChoice}
\choice{Fold the point onto the line.} 
\choice{Fold the line onto the point.}
\choice{Fold along the line.}
\choice[correct]{Fold the line onto itself.}
\choice{Fold the paper in half.}
\end{multipleChoice}
\begin{problem}
Correct!  First, fold the line onto itself so that the $\answer[format=string]{fold}$ goes through the given $\answer[format=string]{point}$.

This first fold is then $\answer[format=string]{perpendicular}$ to the line and through the given point.  

Second, fold the first fold onto $\answer[format=string]{itself}$ so that the second fold goes through the given $\answer[format=string]{point}$.

The second fold is $\answer[format=string]{parallel}$ to the original line because it is $\answer[format=string]{perpendicular}$ to a perpendicular!  
\end{problem}
\end{problem}


%\begin{problem}
%Given a circle (a center and a point on the circle) and line,
%  use folding and tracing to construct the intersection. Explain the steps in your
%  construction.
%\end{problem}
%
%\begin{problem}
%Given a line segment, use folding and tracing to construct an equilateral
%  triangle whose edge has the length of the given segment. Explain the
%  steps in your construction.
%\end{problem}
%
%\begin{problem}
%Explain how to use folding and tracing to transfer a segment.
%\end{problem}
%
%\begin{problem}
%Given an angle and some point, use folding and tracing to copy the angle so
%  that the new angle has as its vertex the given point. Explain the
%  steps in your construction.
%\end{problem}
%
%\begin{problem}
%Explain how to use folding and tracing to construct envelope of tangents for
%  a parabola.
%\end{problem}
%
%\begin{problem}
%Explain how to use folding and tracing to trisect a given angle.
%\end{problem}
%
%\begin{problem}
%Use folding and tracing to construct a square. Explain the steps in your construction.
%\end{problem}
%
%\begin{problem}
%Use folding and tracing to construct a regular hexagon. Explain the steps in
%  your construction.
%\end{problem}

%\begin{problem}
%Morley's Theorem states:\index{Morley's Theorem}
%If you trisect the angles of any triangle with lines, then those lines
%form a new equilateral triangle inside the original triangle.\index{equilateral triangle}\index{trisecting the angle}
%\[
%\includegraphics[scale=.5]{../graphics/morley.pdf}
%\]
%Give a folding and tracing construction illustrating Morley's Theorem. Explain the
%steps in your construction.
%\end{problem}

%\begin{problem}
%Given a length of $1$, construct a triangle whose perimeter is a
%  multiple of $6$. Explain the steps in your construction.
%\end{problem}
%
%\begin{problem}
%Construct a $30$-$60$-$90$ right triangle. Explain the steps in your
%  construction.
%\end{problem}
%
%\begin{problem}
%Given a length of $1$, construct a triangle with a perimeter of
%  $3 + \sqrt{5}$. Explain the steps in your construction.
%\end{problem}


\end{document}
