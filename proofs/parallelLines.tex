\documentclass[nooutcomes]{ximera}
%\documentclass[space,handout,nooutcomes]{ximera}

% For preamble materials

\usepackage{pgf,tikz}
\usepackage{mathrsfs}
\usetikzlibrary{arrows}
\usepackage{framed}
\usepackage{amsmath}
\pgfplotsset{compat=1.17}

\def\fixnote#1{\begin{framed}{\textcolor{red}{Fix note: #1}}\end{framed}}  % Allows insertion of red notes about needed edits
%\def\fixnote#1{}

\def\detail#1{{\textcolor{blue}{Detail: #1}}}   

\pdfOnly{\renewenvironment{image}[1][]{\begin{center}}{\end{center}}}

\graphicspath{
  {./}
  {chapter1/}
  {chapter2/}
  {chapter4/}
  {proofs/}
  {graphics/}
  {../graphics/}
}

\newenvironment{sectionOutcomes}{}{}


%%% This set of code is all of our user defined commands
\newcommand{\bysame}{\mbox{\rule{3em}{.4pt}}\,}
\newcommand{\N}{\mathbb N}
\newcommand{\C}{\mathbb C}
\newcommand{\W}{\mathbb W}
\newcommand{\Z}{\mathbb Z}
\newcommand{\Q}{\mathbb Q}
\newcommand{\R}{\mathbb R}
\newcommand{\A}{\mathbb A}
\newcommand{\D}{\mathcal D}
\newcommand{\F}{\mathcal F}
\newcommand{\ph}{\varphi}
\newcommand{\ep}{\varepsilon}
\newcommand{\aph}{\alpha}
\newcommand{\QM}{\begin{center}{\huge\textbf{?}}\end{center}}

\renewcommand{\le}{\leqslant}
\renewcommand{\ge}{\geqslant}
\renewcommand{\a}{\wedge}
\renewcommand{\v}{\vee}
\renewcommand{\l}{\ell}
\newcommand{\mat}{\mathsf}
\renewcommand{\vec}{\mathbf}
\renewcommand{\subset}{\subseteq}
\renewcommand{\supset}{\supseteq}
%\renewcommand{\emptyset}{\varnothing}
%\newcommand{\xto}{\xrightarrow}
%\renewcommand{\qedsymbol}{$\blacksquare$}
%\newcommand{\bibname}{References and Further Reading}
%\renewcommand{\bar}{\protect\overline}
%\renewcommand{\hat}{\protect\widehat}
%\renewcommand{\tilde}{\widetilde}
%\newcommand{\tri}{\triangle}
%\newcommand{\minipad}{\vspace{1ex}}
%\newcommand{\leftexp}[2]{{\vphantom{#2}}^{#1}{#2}}

%% More user defined commands
\renewcommand{\epsilon}{\varepsilon}
\renewcommand{\theta}{\vartheta} %% only for kmath
\renewcommand{\l}{\ell}
\renewcommand{\d}{\, d}
\newcommand{\ddx}{\frac{d}{dx}}
\newcommand{\dydx}{\frac{dy}{dx}}


\usepackage{bigstrut}


\title{Parallel Lines}
\author{Brad Findell}
\begin{document}
\begin{abstract}
Proofs updated. 
\end{abstract}
\maketitle

Parallel postulate (uniqueness of parallels):  Given a line and a point not on the line, there is exactly one line through the given point parallel to the given line.  

Theorems to prove: 
\begin{enumerate}
\item A $180^\circ$ rotation about a point on a line takes the line to itself. 
\item A $180^\circ$ rotation about a point not on a line takes the line to a parallel line.
\item If two parallel lines are cut by a transversal alternate interior (and corresponding angles) are congruent.
\item If two lines are cut by a transversal so that alternate interior (or  corresponding) angles are congruent, then the lines are parallel. 
\end{enumerate}


\begin{problem}
Given that parallel lines $m$ and $n$ are cut by transversal $k$, prove that alternate interior angles are congruent.  

\begin{image}
\definecolor{uuuuuu}{rgb}{0.2667,0.2667,0.2667}
\definecolor{qqqqff}{rgb}{0.,0.,1.}
\begin{tikzpicture}[line cap=round,line join=round,>=triangle 45,x=1.0cm,y=1.0cm]
\clip(-0.2,1.7) rectangle (7.8,6.9);
\draw [line width=0.8pt,domain=-0.22:7.8] plot(\x,{(-0.--4.*\x)/2.});
\draw [line width=0.8pt,domain=-0.22:7.8] plot(\x,{(-10.--4.*\x)/2.});
\draw [line width=0.8pt,domain=-0.22:7.8] plot(\x,{(--10.--1.*\x)/3.});
\begin{scriptsize}
\draw [fill=qqqqff] (0.,0.) circle (1.5pt);
\draw [fill=qqqqff] (2.,4.) circle (1.5pt);
\draw[color=qqqqff] (2.1,3.8) node {$B$};
\draw[color=qqqqff] (1.7,3.7) node {$1$};
\draw[color=qqqqff] (2.35,4.3) node {$2$};
\draw[color=black] (0.77,2.21) node {$m$};
\draw [fill=qqqqff] (5.,5.) circle (1.5pt);
\draw[color=qqqqff] (5.1,4.8) node {$C$};
\draw[color=qqqqff] (4.65,4.7) node {$3$};
\draw[color=qqqqff] (4.8,5.2) node {$4$};
\draw[color=black] (3.8,2.19) node {$n$};
\draw[color=black] (0.82,3.91) node {$k$};
\draw [fill=qqqqff] (3.5,4.5) circle (1.5pt);
\draw[color=qqqqff] (3.64,4.79) node {$P$};
\end{scriptsize}
\end{tikzpicture}
\end{image}

\begin{enumerate}
\item Let $B$ and $C$ be the intersections of transversal $k$ with lines $m$ and $n$, respectively. Let $P$ be the midpoint of $\overline{BC}$
\item Rotate $180^\circ$ about $P$, which takes $k$ to itself.  
\item The rotation maps $B$ to $C$ and $C$ to $B$ because distances are preserved.  
\item The rotation maps $m$ to a parallel line through $C$, which must be \wordChoice{\choice{$k$}\choice{$m$}\choice[correct]{$n$}} by the uniqueness of parallels.
\item The rotation maps $n$ to \wordChoice{\choice{$k$}\choice[correct]{$m$}\choice{$n$}} by the same reasoning.
\item The rotation swaps $\angle 2$ and \wordChoice{\choice{$\angle 1$}\choice{$\angle 2$}\choice[correct]{$\angle 3$}\choice{$\angle 4$}}.  These alternate interior angles must be congruent because the rotation preserves angle measures. 
\end{enumerate}
\end{problem}


\begin{problem}
Given that parallel lines $m$ and $n$ are cut by transversal $k$, prove that corresponding angles are congruent.  

\begin{image}
\definecolor{uuuuuu}{rgb}{0.2667,0.2667,0.2667}
\definecolor{qqqqff}{rgb}{0.,0.,1.}
\begin{tikzpicture}[line cap=round,line join=round,>=triangle 45,x=1.0cm,y=1.0cm]
\clip(-0.2,1.7) rectangle (7.8,6.9);
\draw [line width=0.8pt,domain=-0.22:7.8] plot(\x,{(-0.--4.*\x)/2.});
\draw [line width=0.8pt,domain=-0.22:7.8] plot(\x,{(-10.--4.*\x)/2.});
\draw [line width=0.8pt,domain=-0.22:7.8] plot(\x,{(--10.--1.*\x)/3.});
\begin{scriptsize}
\draw [fill=qqqqff] (0.,0.) circle (1.5pt);
\draw [fill=qqqqff] (2.,4.) circle (1.5pt);
\draw[color=qqqqff] (2.1,3.8) node {$B$};
\draw[color=qqqqff] (1.7,3.7) node {$1$};
\draw[color=qqqqff] (2.35,4.3) node {$2$};
\draw[color=black] (0.77,2.21) node {$m$};
\draw [fill=qqqqff] (5.,5.) circle (1.5pt);
\draw[color=qqqqff] (5.1,4.8) node {$C$};
\draw[color=qqqqff] (4.65,4.7) node {$3$};
\draw[color=qqqqff] (4.8,5.2) node {$4$};
\draw[color=black] (3.8,2.19) node {$n$};
\draw[color=black] (0.82,3.91) node {$k$};
%\draw [fill=qqqqff] (3.5,4.5) circle (1.5pt);
%\draw[color=qqqqff] (3.64,4.79) node {$P$};
\end{scriptsize}
\end{tikzpicture}
\end{image}

\begin{enumerate}
\item Let $B$ and $C$ be the intersections of transversal $k$ with lines $m$ and $n$, respectively. 
\item Translate to the right along line $k$ by distance $BC$, which takes $k$ to itself.
\item The translation maps $B$ to $C$, and it maps $m$ to \wordChoice{\choice{$k$}\choice{$m$}\choice[correct]{$n$}} because the translation maintains parallels, and there is a unique parallel to $m$ through $C$.
\item The translation maps $\angle 1$ to \wordChoice{\choice{$\angle 1$}\choice{$\angle 2$}\choice[correct]{$\angle 3$}\choice{$\angle 4$}}.  These corresponding angles must be congruent because the translation preserves angle measures. 
\end{enumerate}

\end{problem}


\end{document}
