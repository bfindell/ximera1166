\documentclass[nooutcomes]{ximera}
%\documentclass[space,handout,nooutcomes]{ximera}

% For preamble materials

\usepackage{pgf,tikz}
\usepackage{mathrsfs}
\usetikzlibrary{arrows}
\usepackage{framed}
\usepackage{amsmath}
\pgfplotsset{compat=1.17}

\def\fixnote#1{\begin{framed}{\textcolor{red}{Fix note: #1}}\end{framed}}  % Allows insertion of red notes about needed edits
%\def\fixnote#1{}

\def\detail#1{{\textcolor{blue}{Detail: #1}}}   

\pdfOnly{\renewenvironment{image}[1][]{\begin{center}}{\end{center}}}

\graphicspath{
  {./}
  {chapter1/}
  {chapter2/}
  {chapter4/}
  {proofs/}
  {graphics/}
  {../graphics/}
}

\newenvironment{sectionOutcomes}{}{}


%%% This set of code is all of our user defined commands
\newcommand{\bysame}{\mbox{\rule{3em}{.4pt}}\,}
\newcommand{\N}{\mathbb N}
\newcommand{\C}{\mathbb C}
\newcommand{\W}{\mathbb W}
\newcommand{\Z}{\mathbb Z}
\newcommand{\Q}{\mathbb Q}
\newcommand{\R}{\mathbb R}
\newcommand{\A}{\mathbb A}
\newcommand{\D}{\mathcal D}
\newcommand{\F}{\mathcal F}
\newcommand{\ph}{\varphi}
\newcommand{\ep}{\varepsilon}
\newcommand{\aph}{\alpha}
\newcommand{\QM}{\begin{center}{\huge\textbf{?}}\end{center}}

\renewcommand{\le}{\leqslant}
\renewcommand{\ge}{\geqslant}
\renewcommand{\a}{\wedge}
\renewcommand{\v}{\vee}
\renewcommand{\l}{\ell}
\newcommand{\mat}{\mathsf}
\renewcommand{\vec}{\mathbf}
\renewcommand{\subset}{\subseteq}
\renewcommand{\supset}{\supseteq}
%\renewcommand{\emptyset}{\varnothing}
%\newcommand{\xto}{\xrightarrow}
%\renewcommand{\qedsymbol}{$\blacksquare$}
%\newcommand{\bibname}{References and Further Reading}
%\renewcommand{\bar}{\protect\overline}
%\renewcommand{\hat}{\protect\widehat}
%\renewcommand{\tilde}{\widetilde}
%\newcommand{\tri}{\triangle}
%\newcommand{\minipad}{\vspace{1ex}}
%\newcommand{\leftexp}[2]{{\vphantom{#2}}^{#1}{#2}}

%% More user defined commands
\renewcommand{\epsilon}{\varepsilon}
\renewcommand{\theta}{\vartheta} %% only for kmath
\renewcommand{\l}{\ell}
\renewcommand{\d}{\, d}
\newcommand{\ddx}{\frac{d}{dx}}
\newcommand{\dydx}{\frac{dy}{dx}}


\usepackage{bigstrut}


\title{Vertical Angles}
\author{Brad Findell}
\begin{document}
\begin{abstract}
Proofs updated. 
\end{abstract}
\maketitle

\textbf{Below are three different proofs that vertical angles are congruent.  Please consider them separately.}

\begin{problem}
Point P is the intersection of lines $m$ and $n$.  Prove that $\angle 1\cong \angle 3$.  

\begin{image}
\definecolor{qqqqff}{rgb}{0.,0.,1.}
\begin{tikzpicture}[line cap=round,line join=round,>=triangle 45,x=1.0cm,y=1.0cm]
\clip(-2.5,-1.3) rectangle (7.4,3.4);
\draw [line width=0.8pt,domain=-2.4:7.3] plot(\x,{(-0.-1.*\x)/-2.});
\draw [line width=0.8pt,domain=-2.4:7.3] plot(\x,{(--5.-1.*\x)/3.});
%\draw [line width=0.8pt,dash pattern=on 2pt off 2pt,domain=-2.4:7.3] plot(\x,{(--2.066-0.997*\x)/0.071});
\draw (1.2,1.2) node[anchor=north west] {$1$};
\draw (1.8,1.6) node[anchor=north west] {$2$};
\draw (2.45,1.3) node[anchor=north west] {$3$};
%\draw (2.,0.9) node[anchor=north west] {$4$};
%\begin{scriptsize}
\draw [fill=qqqqff] (2.,1.) circle (1.5pt);
\draw[color=qqqqff] (1.85,0.65) node {$P$};
%\draw [fill=qqqqff] (0.,0.) circle (1.5pt);
\draw[color=black] (-2.,-0.7) node {$m$};
%\draw [fill=qqqqff] (5.,0.) circle (1.5pt);
\draw[color=black] (-1.8,2.5) node {$n$};
%\draw[color=black] (2.1,3.) node {$k$};
%\draw [fill=qqqqff] (4.,2.) circle (1.5pt);
%\draw [fill=qqqqff] (-1.,2.) circle (1.5pt);
%\end{scriptsize}
\end{tikzpicture}
\end{image}

\begin{proof}
Using adjacent angles, $\angle 1\cong \angle 3$ because they are both \wordChoice{\choice{complementary}\choice[correct]{supplementary}\choice{opposite}\choice{congruent}} to $\angle 2$.
\end{proof}

\begin{problem}
Additional detail: First write down equations about linear pairs of angles: 
\[
m\angle 1 + m\angle 2 = \answer{180} \textrm{ degrees}
\]
\[
m\angle 3 + m\angle 2 = \answer{180} \textrm{ degrees}
\]
By comparing the two equations, one might see that $m\angle 1=m\angle 3$.   
Alternatively, one may do some algebra to conclude that $m\angle 1 = 180^\circ - m\angle 2 = m\angle 3$, which is essentially what the one-sentence proof says.

{\textcolor{red}{Editorial note: Students often write two equations for non-overlapping linear pairs---which doesn't lead to a proof.  The figure above is intended to help avoid that dead end, but it would be worthwhile to discuss it anyway.}}

\end{problem}

\end{problem}


\begin{problem}
Point P is the intersection of lines $m$ and $n$.  Prove that $\angle 1\cong \angle 3$.  

\begin{image}
\definecolor{qqqqff}{rgb}{0.,0.,1.}
\begin{tikzpicture}[line cap=round,line join=round,>=triangle 45,x=1.0cm,y=1.0cm]
\clip(-2.5,-1.3) rectangle (7.4,3.4);
\draw [line width=0.8pt,domain=-2.4:7.3] plot(\x,{(-0.-1.*\x)/-2.});
\draw [line width=0.8pt,domain=-2.4:7.3] plot(\x,{(--5.-1.*\x)/3.});
%\draw [line width=0.8pt,dash pattern=on 2pt off 2pt,domain=-2.4:7.3] plot(\x,{(--2.066-0.997*\x)/0.071});
\draw (1.2,1.2) node[anchor=north west] {$1$};
\draw (1.8,1.6) node[anchor=north west] {$2$};
\draw (2.45,1.3) node[anchor=north west] {$3$};
%\draw (2.,0.9) node[anchor=north west] {$4$};
%\begin{scriptsize}
\draw [fill=qqqqff] (2.,1.) circle (1.5pt);
\draw[color=qqqqff] (1.85,0.65) node {$P$};
%\draw [fill=qqqqff] (0.,0.) circle (1.5pt);
\draw[color=black] (-2.,-0.7) node {$m$};
%\draw [fill=qqqqff] (5.,0.) circle (1.5pt);
\draw[color=black] (-1.8,2.5) node {$n$};
%\draw[color=black] (2.1,3.) node {$k$};
%\draw [fill=qqqqff] (4.,2.) circle (1.5pt);
%\draw [fill=qqqqff] (-1.,2.) circle (1.5pt);
%\end{scriptsize}
\end{tikzpicture}
\end{image}

\begin{proof}
A rotation of \wordChoice{\choice{$90^\circ$}\choice[correct]{$180^\circ$}\choice{$360^\circ$}} about $P$ maps $m$ onto itself, maps $n$ onto itself, and 
swaps $\angle 1$ and \wordChoice{\choice{$\angle 1$}\choice{$\angle 2$}\choice[correct]{$\angle 3$}}.  Because rotations preserve angle measures, 
it must be that $\angle 1\cong \angle 3$.  
\end{proof}

\begin{problem}
Additional detail: Line $m$ is the union of two opposite $\answer[format=string]{rays}$ with endpoint $P$.  The rotation about $P$  swaps these opposite rays, and the same idea holds for line $n$.   That rotation maps the sides of $\angle 1$ onto the sides of $\angle 3$ and vice versa.
\end{problem}

\end{problem}


\begin{problem}
Point P is the intersection of lines $m$ and $n$.  Prove that $\angle 1\cong \angle 3$.  

\begin{image}
\definecolor{qqqqff}{rgb}{0.,0.,1.}
\begin{tikzpicture}[line cap=round,line join=round,>=triangle 45,x=1.0cm,y=1.0cm]
\clip(-2.5,-1.3) rectangle (7.4,3.4);
\draw [line width=0.8pt,domain=-2.4:7.3] plot(\x,{(-0.-1.*\x)/-2.});
\draw [line width=0.8pt,domain=-2.4:7.3] plot(\x,{(--5.-1.*\x)/3.});
\draw [line width=0.8pt,dash pattern=on 2pt off 2pt,domain=-2.4:7.3] plot(\x,{(--2.066-0.997*\x)/0.071});
\draw (1.2,1.2) node[anchor=north west] {$1$};
\draw (1.8,1.6) node[anchor=north west] {$2$};
\draw (2.45,1.3) node[anchor=north west] {$3$};
%\draw (2.,0.9) node[anchor=north west] {$4$};
%\begin{scriptsize}
\draw [fill=qqqqff] (2.,1.) circle (1.5pt);
\draw[color=qqqqff] (1.85,0.65) node {$P$};
%\draw [fill=qqqqff] (0.,0.) circle (1.5pt);
\draw[color=black] (-2.,-0.7) node {$m$};
%\draw [fill=qqqqff] (5.,0.) circle (1.5pt);
\draw[color=black] (-1.8,2.5) node {$n$};
\draw[color=black] (2.1,3.) node {$k$};
%\draw [fill=qqqqff] (4.,2.) circle (1.5pt);
%\draw [fill=qqqqff] (-1.,2.) circle (1.5pt);
%\end{scriptsize}
\end{tikzpicture}
\end{image}

\begin{proof}
Reflecting about the \wordChoice{\choice[correct]{bisector}\choice{supplement}\choice{opposite}} of $\angle 2$ swaps the sides of $\angle 2$ and therefore lines $m$ and $n$.  Thus, that reflection swaps $\angle 1$ and \wordChoice{\choice{$\angle 1$}\choice{$\angle 2$}\choice[correct]{$\angle 3$}}.  Because reflections preserve angle measures, it follows that $\angle 1\cong \angle 3$. 
\end{proof}

\begin{problem}
Additional detail: Because reflections take lines 
to $\answer[format=string]{lines}$, the reflection that swaps the sides of $\angle 2$ must swap not just the rays but lines $m$ and $n$, which contain the rays.
\end{problem}

\end{problem}


\end{document}
