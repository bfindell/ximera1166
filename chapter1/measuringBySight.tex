% Section 1.1 Anatomy of Figures

\documentclass[nooutcomes]{ximera}
%\documentclass[space,handout,nooutcomes]{ximera}

% For preamble materials

\usepackage{pgf,tikz}
\usepackage{mathrsfs}
\usetikzlibrary{arrows}
\usepackage{framed}
\usepackage{amsmath}
\pgfplotsset{compat=1.17}

\def\fixnote#1{\begin{framed}{\textcolor{red}{Fix note: #1}}\end{framed}}  % Allows insertion of red notes about needed edits
%\def\fixnote#1{}

\def\detail#1{{\textcolor{blue}{Detail: #1}}}   

\pdfOnly{\renewenvironment{image}[1][]{\begin{center}}{\end{center}}}

\graphicspath{
  {./}
  {chapter1/}
  {chapter2/}
  {chapter4/}
  {proofs/}
  {graphics/}
  {../graphics/}
}

\newenvironment{sectionOutcomes}{}{}


%%% This set of code is all of our user defined commands
\newcommand{\bysame}{\mbox{\rule{3em}{.4pt}}\,}
\newcommand{\N}{\mathbb N}
\newcommand{\C}{\mathbb C}
\newcommand{\W}{\mathbb W}
\newcommand{\Z}{\mathbb Z}
\newcommand{\Q}{\mathbb Q}
\newcommand{\R}{\mathbb R}
\newcommand{\A}{\mathbb A}
\newcommand{\D}{\mathcal D}
\newcommand{\F}{\mathcal F}
\newcommand{\ph}{\varphi}
\newcommand{\ep}{\varepsilon}
\newcommand{\aph}{\alpha}
\newcommand{\QM}{\begin{center}{\huge\textbf{?}}\end{center}}

\renewcommand{\le}{\leqslant}
\renewcommand{\ge}{\geqslant}
\renewcommand{\a}{\wedge}
\renewcommand{\v}{\vee}
\renewcommand{\l}{\ell}
\newcommand{\mat}{\mathsf}
\renewcommand{\vec}{\mathbf}
\renewcommand{\subset}{\subseteq}
\renewcommand{\supset}{\supseteq}
%\renewcommand{\emptyset}{\varnothing}
%\newcommand{\xto}{\xrightarrow}
%\renewcommand{\qedsymbol}{$\blacksquare$}
%\newcommand{\bibname}{References and Further Reading}
%\renewcommand{\bar}{\protect\overline}
%\renewcommand{\hat}{\protect\widehat}
%\renewcommand{\tilde}{\widetilde}
%\newcommand{\tri}{\triangle}
%\newcommand{\minipad}{\vspace{1ex}}
%\newcommand{\leftexp}[2]{{\vphantom{#2}}^{#1}{#2}}

%% More user defined commands
\renewcommand{\epsilon}{\varepsilon}
\renewcommand{\theta}{\vartheta} %% only for kmath
\renewcommand{\l}{\ell}
\renewcommand{\d}{\, d}
\newcommand{\ddx}{\frac{d}{dx}}
\newcommand{\dydx}{\frac{dy}{dx}}


\usepackage{bigstrut}


\title{Measuring by Sight}
\author{Brad Findell}
\begin{document}
\begin{abstract}
Short-answer measuring problems. 
\end{abstract}
\maketitle

\subsection*{Instructions}
Adjust the figures to fit the given conditions within \textbf{eyeball accuracy}.  Enter the requested measurements.  

% Perpendicular
\begin{problem}
\begin{center}  
\geogebra{gjf28er6}{450}{250}  
\end{center}
In figure above, when point $C$ is adjusted so that $\overline{BC}$ is perpendicular to $\overline{AC}$, $AC=\answer[tolerance=.07]{2.09}$.
\begin{hint}
When two lines are \emph{perpendicular}, they cross to create four congruent angles. 
\end{hint}
\begin{hint}
Use the corner of a piece of paper.
\end{hint}
\end{problem}

% Triangle Height, v2
\begin{problem}
\begin{center}  
\geogebra{a888zyw2}{600}{220}  
\end{center}
In $\triangle ABC$ above, the height to base $\overline{AC}$ is $\answer[tolerance=.003]{3.585}$.
\begin{hint}
You may move point $D$.  A height is the length of an altitude, which must be perpendicular to the line containing the chosen base.  
\end{hint}
\end{problem}

% Triangle Height, v1
\begin{problem}
\begin{center}  
\geogebra{kta9hbuf}{500}{300}  
\end{center}
In $\triangle ABC$ above, the height to base $\overline{AC}$ is $\answer[tolerance=.003]{3.511}$.
\begin{hint}
You may move point $D$.  A height is the length of an altitude, which must be perpendicular to the line containing the chosen base.  
\end{hint}
\end{problem}





\end{document}

